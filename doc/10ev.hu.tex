%{\Large\bf 10 éves a CCC} 
%1996.09-től számítva

\pagetitle%
{10 éves a CCC}%
{Vermes M.}%
{2006. Szeptember}

\medskip

A változáslisták  szerint 10 éves a CCC:
Clipper leszármazott, a forráskódot C-re, 
illetve egy  C-ben implementált veremgépre fordítja.
Platformfüggetlen programnyelv és fejlesztőeszköz.

\begin{itemize}
\item   
    A név jelentése {"}Clipper to C++ Compiler".
\item 
    \href{http://ccc.comfirm.hu/ccc3/ccc-clipper-elteresek.html#NAMESPACE}{Névterek}, 
    \href{http://ccc.comfirm.hu/ccc3/ccc-clipper-elteresek.html#THREADS}{szálak}, 
    \href{http://ccc.comfirm.hu/ccc3/exception.html}{kivételek},
    \href{http://ccc.comfirm.hu/ccc3/objektum.html}{objektumok},
    \href{http://ccc.comfirm.hu/ccc3/ccc3_ujdonsagok.html}{unicode}
    \ldots
\item 
    \href{http://ccc.comfirm.hu/ccc3/xmlrpc-framework.html}{XMLRPC}, 
    \href{http://ccc.comfirm.hu/ccc3/sql2.html}{SQL}, 
    \href{http://ccc.comfirm.hu/ccc3/cccgtk.html}{GTK (Glade)},
    \href{http://ccc.comfirm.hu/ccc3/jterminal.html}{Jáva terminál} \ldots
\item
    Éles banki alkalmazások működnek vele.
\item   
    Forrásban
    \href{git://comfirm.hu/ccc3.git}{tölthető le} 
    LGPL licensz alatt.
\end{itemize}

A jelenlegi állapot  az
\href{http://ccc.comfirm.hu/ccc3/ccc-clipper-elteresek.html}%
{Eltérések a CCC és Clipper között} dokumentumból mérhető fel legjobban.
A CCC kiterjeszti a régi Clippert.
Korszerű, kiállja az összehasonlítást olyan nyelvekkel, 
mint a Python, Ruby, Pike.
Különösen hasonló a CCC és a Python hangulata,
mindkettő  {\em praktikus}, tömör, mégis 
könnyen olvasható, kerülik a Jávára jellemző tudálékosságot.

Hozzászoktam, hogy  a fórumokon ilyeneket kapok: 
{"}Minek Clipperrel foglalkozni a .NET korában?" (sting),  
{"}Időgép is kellene hozzá",
{"}Mit ér egy nyelv önmagában, osztálykönyvtár nélkül?".
Először is, van egy s más a CCC-ben.  Másodszor, a CCC-t
könnyebb C betétekkel bővíteni, mint az említett nyelveket. 

A Jáva bővítése C-vel ellenjavallt. 
Elvész a hordozhatóság, túl bonyolult,  
az átlagprogramozó nem is ért hozzá.
Hasonló a helyzet a Pythonnál.
Az {\em alkalmazáshoz} szükséges bővítéseket a 
{\em futtatókörnyezet/interpreter} módosításával kell(ene)  
megvalósítani. Ezekben a nyelvekben valóban nélkülözhetetlen 
az egész informatikai univerzumot magábafoglaló osztálykönyvtár. 

Ezzel szemben a CCC (C fordítás közbeiktatásával) natív  
binárisokat készít, ezért bármikor alámerülhetünk C-be.
A Clipperrel együtt automatikusan fordulnak a C modulok. 
Kisebb jelentősége van így  az osztálykönyvtáraknak, 
mert mindig rendelkezésre áll a C-ből elérhető infrastruktúra. 
Ez a CCC filozófiája.


