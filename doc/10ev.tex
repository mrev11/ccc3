%{\Large\bf CCC is 10 years old} 
%1996.09-től számítva

\pagetitle%
{CCC is 10 years old}%
{M. Vermes}%
{September, 2006}

\medskip

According to changelists CCC is 10 years old: It is descended from
Clipper, it translates the source code into C and a stack machine 
implemented in C. It is a program language and developing tool
independent of platforms.


\begin{itemize}
\item   
    The name means {"}Clipper to C++ Compiler".
\item 
    \href{http://ccc.comfirm.hu/ccc3/ccc-clipper-elteresek.html#NAMESPACE}{Name spaces}, 
    \href{http://ccc.comfirm.hu/ccc3/ccc-clipper-elteresek.html#THREADS}{threads}, 
    \href{http://ccc.comfirm.hu/ccc3/exception.html}{exceptions},
    \href{http://ccc.comfirm.hu/ccc3/objektum.html}{objects},
    \href{http://ccc.comfirm.hu/ccc3/ccc3_ujdonsagok.html}{unicode}
    \ldots
\item 
    \href{http://ccc.comfirm.hu/ccc3/xmlrpc-framework.html}{XMLRPC}, 
    \href{http://ccc.comfirm.hu/ccc3/sql2.html}{SQL}, 
    \href{http://ccc.comfirm.hu/ccc3/cccgtk.html}{GTK (Glade)},
    \href{http://ccc.comfirm.hu/ccc3/jterminal.html}{Java terminal} \ldots
\item
    Critical banking applications work with it.
\item   
    It can be 
    \href{http://ccc.comfirm.hu/ccc3/download/olvass.html}{downloaded} 
    in source code under LGPL.
\end{itemize}

The present condition can be appraised the best from the document 
\href{http://ccc.comfirm.hu/ccc3/ccc-clipper-elteresek.html}%
{Differences between CCC and Clipper}.
CCC extends the old Clipper.
It is modern, it bears comparison with languages like 
Python, Ruby, Pike.
The atmospheres of CCC and Python are especially similar, both are  
{\em practical}, concise, nevertheless easy to read, they avoid 
the pedantry characteristic of Java.

I am accustomed to getting messages like these in forums:  
{"}Why to deal with Clipper in the age of .NET?",  
{"}A time machine is also necessary for it.",
{"}What is a language itself good for without a class library?".
First there are a few things in CCC. Second, it can be easier extended
by inserts C, than the languages mentioned above.

It is not recommended to extend Java by C. 
Portability gets lost, it is too complicated,
and an avarage programmer is not proficient in it. 
The situation with Python is similar.
The extensions  necessary for the  {\em application} should be 
carried out by modifications of  the
{\em runtime environment/interpreter}. A class library including the whole 
universe of informatics is really essential in these languages. 

Whereas CCC (with  translation into C) produces native  
binaries, that's why we can submerge into C any time.
The C moduls are translated automatically together with Clipper.
This way the class libraries have less importance, because the infrastructure
accesible from C is always available.  
This is the philosophy of CCC.


