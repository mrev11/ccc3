
\pagetitle%
{Saját kulcskezelőre épülő táblaobjektum}%
{Dr. Vermes Mátyás\footnote{\ComFirm}}%
{2001. december}

\section{Áttekintés}
 
A dokumentum az új, BTBTX formátumú táblaobjektumot ismerteti.
Ha csupán alkalmazni akarjuk az új könyvtárat, akkor erre az írásra
nincs szükség, mivel (természetesen) a korábbi táblaobjektumokkal
megmaradt a kompatibilitás, ezek leírása pedig a 
\href{tabobj.html}{Tábla objektum referencia} dokumentumban található.
Így a továbbiakat csak háttérinformációnak szánom.

Az új formátum legfontosabb tulajdonsága, hogy {\it saját kulcskezelőn
alapszik}, így végre függetlenedni tudunk bármiféle idegen
forrástól (Btrieve, Ctree), és feltétel nélkül szabaddá (GPL)
tehetjük a kódunkat. Személyes véleményem, hogy hosszabb távon
a saját kulcskezelő megbízhatóbb is lesz, mint az említett
szoftverek.

Az alábbiakban áttekintem a BTBTX formátum fő jellegzetességeit
a korábbi formátumokkal összeveteve.

\paragraph{Btrieve}

A Btrieve-hez hasonlóan a BTBTX formátumban sincs külön index filé.
Az adatok és az indexek táblánként egy darab  bt kiterjesztésű filében 
vannak. A memófilék kezelését ugyanaz a szoftver végzi, mint a 
többi formátumban, kiterjesztésük btx. Az új formátum neve az adat- és 
memófilék kiterjesztésének összevonásából keletkezett.
A Btrieve és BTBTX formátumú táblaobjektum egyaránt tartalmazza
a mezők és indexek leírását. Mindkét formátum képes változó
hosszúságú rekordok tárolására, de ezt a képességüket
(a resource-októl eltekintve) nem használjuk.
A Btrieve rekordkezelőt tipikusan kliens/szerver üzemmódban
használják, szemben a BTBTX-re jellemző standalone programszervezéssel.

\paragraph{DATIDX}
 
A BTBTX és DATIDX formátum hasonlít abban, hogy mindig rendelkezésre
áll a tábla mezőinek és indexeinek leírása. A DATIDX formátumban
az indexek frissítését a Ctree könyvtár automatikusan végzi.
 
\paragraph{DBFCTX}

Különösen nagy hasonlóság van a BTBTX és a DBFCTX formátum kódja között.
Az indexek módosítását nem a rekordkezelő könyvtárra bízzuk (ahogy a Ctree-re
a DATIDX esetében), hanem magasabb szinten, a táblaobjektumban oldjuk meg.
Kulcs módosításakor először töröljük az indexből a régi kulcsértéket,
módosítjuk az adatrekordot, végül beírjuk az indexbe az új kulcsot. 
Talán meglepő, de mindez hatékonyan és megbízhatóan kódolható CCC-ben.
Csakhogy, míg a DBFCTX esetében az elemi kulcskezelés függvényei
a Ctree könyvtárból valók, a BTBTX saját kulcskezelőt használ.

A bt filé a mezők és indexek leírását egyaránt tartalmazza.
Az indexek leírása a kötött dbf formátumba nem volt betehető, 
ezért DBFCTX-ben az index infót a ctx tárolja,
ami azonban nem áll rendelkezésre, ha éppen letörölték.
A BTBTX formátum mentes ettől a hibától.


\section{Keletkezéstörténet}

\subsection{Kell-e ISAM filékezelő?}

Szükséges-e egyáltalán ISAM filékezeléssel bajlódni? 
A fiatalabb programozó generáció már azt sem tudja mi az az ISAM, 
SQL-en kívül mást el sem tudnak képzelni. Tekintsük azonban
a következő SQL utasítást:

\begin{verbatim}
select szamlaszam,devnem,egyenleg 
  from szamla 
  where szamlaszam<'8830100440014832' 
  order by szamlaszam desc
\end{verbatim}
 
A fenti parancsot az Oracle, Postgres és MySQL úgy hajtja végre,
hogy először kikeresi a \verb!where! feltételnek megfelelő számlákat
(a sorok száma akár többszázezer is lehet), majd {\it az így kapott táblát 
 rendezi}, és negyedóra gondolkodás után megadja az első rekordot. 
Ettől a nyilvánvalóan rossz  algoritmustól a fenti adatbáziskezelőket 
nem lehet eltántorítani. Hiába létezik index a \verb!szamlaszam! oszlopra, 
hiába adunk bámilyen hint-et az optimalizálásra vonatkozóan, 
az Oracle kitartóan rendez.
Ugyanez a lekérdezés egy ISAM filékezelővel, számlaszám szerinti
indexszel, többszázezer eredménysorral fél percen belül elkészül úgy,
hogy az első rekord egy ezredmásodpercen belül megvan. 

Megmutattam ezt egy Oracle ,,szakértőnek'', aki azt mondta,
hogy a feladat rossz, mert az Oracle-nek olyan kérdéseket kell
feltenni, aminek eredményeként kis táblákat kapunk. Irígylem azokat
a programozókat, akiknek csak jó feladatokkal kell foglalkozniuk.

Általánosságban (finoman szólva) az mondható, hogy 
az SQL lekérdezésű, tranzakciós adatbázisszervereket nem olyan feladatok 
megoldására tervezték, amik egy nagy tábla sorainak többségét érintik. 
Ilyen feladatok azonban nyilvánvalóan léteznek. Tegyük fel például, hogy 
van egymillió számlánk, és ki akarjuk számolni mindegyiken a napi kamatot, 
vagy ki akarjuk gyűjteni azokat, amikre kivonatot kell nyomtatni, 
és éppen 900 ezer ilyen van.

Említek még egy dolgot, amit gyakran még a tapasztalt informatikusok
is elfelejtenek, nevezetesen, hogy nincsenek csodák: A legvonzóbb
hirdetésekkel rendelkező SQL adatbáziskezelő sem tud nagyobb adatbázisokban
gyorsabban keresni, mint az egyszerű ISAM, minthogy erre jelenleg nem 
ismerünk jobb adatszerkezetet, mint a btree, 
ahogy azt az ISAM-ban néhány évtizede kitalálták.

No jó, legyen ISAM, akkor is mi értelme van  új kulcskezelő
szoftvert írni, amikor az ilyenekkel Dunát lehet rekeszteni?
Meglepő, de nincs is ezekből olyan sok: Btrieve, Ctree, C-ISAM,
DISAM, Berkeley-DB (SleepyCat). Ezeknek azonban egyike sem ingyenes,
és a Ctree és a Berkeley DB kivételével a forrásuk is titkos. 
Emellett ezek legalább
10-15 éves szoftverek, amik az idők során egymásra rakódott API-k
miatt szövevényessé váltak, rengeteg vitatható fontosságú szolgáltatást
nyújtanak, viszont nehéz belőlük kibányászni a kisszámú tényleg 
szükséges funkciót. A programok fejlesztési irányával sem vagyok
elégedett, pl.{} szerintem nincs szükség e termékekre épülő
gyenge minőségű, SQL interfészű adatbázisszerverek fejlesztésére.

Nekem éppenséggel olyan kulcs/indexkezelőre van szükségem,
ami az általam fontosnak tartott szolgáltatásokkal rendelkezik,
áttekinthetősége és egyszerűsége miatt kézbentartható és megbízható.
Úgy gondolom, hogy a mai PC-k teljesítménye lehetővé teszi, hogy
az algoritmusok egyszerű és közvetlen alkalmazásával megoldjunk
olyan feladatokat, amit korábban csak agyonoptimalizált programokkal
lehetett.  Méginkább így lesz ez a 64 bites rendszerek megjelenésével. 
A BTBTX egyszerű kulcskezelőjét egy szép délután portolni fogom 64 bites 
Linuxra, de mennyi ideig fog ez tartani a Ctree esetében?\footnote{
Vagy a jelent illetően: egyszerűen bevezethető a ReiserFS 
60 bites filé offseteket támogató API-ja, amivel már most
megoldható a 2G-nál nagyobb filék kezelése.}

\subsection{Fejlesztések (vagy egyszerűsítések?)}

Ha megtekintjük a \verb!man 3 dbopen! parancs után elénk táruló
információt, rájövünk, hogy a {\it Linuxon van ingyenes 
ISAM adatbáziskezelő.} Installáltam az adatbáziskezelő forrását is, 
aminek tanulmányozásával a következők derültek ki:

A manban leírt adatbáziskezelőt eredetileg a BSD UNIX-on fejlesztették 
ki, amint az a Makefile-ben lévő megjegyzésben olvasható:
%Makefile
\begin{verbatim}
# Makefile for 4.4BSD db code in GNU C library.
# This code is taken verbatim from the BSD db 1.85 package.  Only this
# Makefile and compat.h were written for GNU libc, and the header files
# moved up to this directory.
\end{verbatim}

Ugyanennek a szoftvernek van egy fejlettebb, a SleepyCat által
fejlesztett változata, ami a SuSE Linuxokon szintén
installálódik (és általában elfedi az előzőt). 
E csomag README-jében az alábbi licence információ bújik meg:
%README
\begin{verbatim}
As a special exception, when Berkeley DB is distributed along with the
GNU C Library, in any program which uses the GNU C Library in accord
with that library's distribution terms, it is also permitted for
Berkeley DB to be loaded dynamically by the GNU C Library to implement
standard ISO/IEC 9945 and Unix interface functionality.

Sleepycat Software, Inc.
\end{verbatim}

E két adatbáziskezelő változat legfőbb jellegzetessége, 
hogy szigorúan egyfelhasználósak. Ez leginkáb azon múlik, 
hogy intenzíven cache-elnek, ezért nem is értesülhetnek más 
processzek által végzett adatmódosításokról (ritkán olvasnak).

Harmadszor, a Berkeley~DB-nek létezik egy többfelhasználós,
bonyolult tranzakciókezelésre képes változata, 
amit a SleepyCat forgalmaz, forrása letölthető a hálóról, 
ám üzleti alkalmazásra nem ingyenes.\footnote{
A MySQL legújabb, tranzakciókezelést is tartalmazó változata erre az 
adatbázismotorra épül (így nyilván az sem lehet ingyenes).} 

A saját kulcskezelő írásakor a legrégibb, legegyszerűbb változatból
indultam ki. Az átalakítások folytán a szoftver egyrészt leegyszerűsödött,
másrészt lehetővé vált a konkurrens használat. Az egyszerűsítés
mértékére jellemző adatok: 
\begin{itemize}
\item
  Az eredeti Berkeley~DB forrás btree kezelő része
  130~Kbyte (ebből indultam ki). 
\item
  A SleepyCat által feldúsított változat mérete 250~Kbyte. 
\item
  Az általam módosított változat mérete konkurrencia 
  kezeléssel együtt 90 Kbyte. 
\end{itemize}  

Nézzük, milyen egyszerűsítések történtek a szoftverben:

\begin{description}
\item[Adatbázis típusok.]
  Az eredeti API háromféle (btree, hash és rekordsorszám) szervezésű
  adatbázist támogatott. A hash alapján történő keresésre nincs
  szükség, a rekordsorszámos adatbázisnak nincs önálló jelentősége,
  így csak a btree-t hagytam meg. Megjegyzem, hogy az előbb
  említett 130/250 Kbyte-os méretekbe nem számítottam be 
  a hash és rekordsorszámos adatbázis kódját. 
\item[Kulcs/adat párok.]
  Az eredeti API kulcs/adat párok tárolását tette lehetővé.
  A gyakorlatban azonban az adatok külön egységként tárolódnak,
  ezért kulcsok mellé nincs értelme adatot helyezni. 
  Ezért a mi indexeinkben kizárólag kulcsok vannak.
\item[Duplikátumok.]
  Az eredeti kód sokat vacakolt a kulcsok között előforduló
  duplikátumokkal. A táblaobjektumban azonban az egyező kulcsok
  közötti sorrend is definiált (kötelezően a rekordsorszám),
  ezért a mi kulcsaink mindig ki vannak egészítve a rekordsorszámmal,
  tehát duplikátum egyáltalán nem lehetséges. Így a duplikátumok
  kezelése el lett hagyva.
\item[Kurzorpozíció.]
  Az eredeti egyfelhasználós kód rengeteget pátyolgatta a kurzorpozíciót.
  Ha olyan műveletet végzett, ami a kurzort elrontotta, 
  akkor addig nem nyugodott, amíg az ki nem javult.
  Többfelhasználós esetben ez nem járható, ui. a kurzor
  elsősorban a többi processz ténykedése miatt romlik el,
  amit csak utólag lehet észrevenni. Ezért az új program
  tárolja a kurzorpozíció mellett a kurzor helyén lévő
  kulcsértéket is. A kurzor felhasználása előtt ellenőrzi, 
  hogy ugyanarra a kulcsra mutat-e még, ha nem, 
  akkor újrapozícionálja. Ez sokkal egyszerűbb és biztonságosabb,
  mint az eredeti megoldás.
\item[Mpool get/put interfész.]  
  Az eredeti kód az mpool\_get/put interfésszel (lásd man)
  lapozós technikával olvasta a lemezt. Ez a függvénykészlet
  intenzív bufferelést végzett, ami megakadályozta a konkurrens
  használatot. Az interfészt megtartottam, de a függvényeket
  újraírtam a konkurrens használat igényeinek megfelelően.
\end{description}
 
A bővítések:

\begin{description}
\item[Több btree.]
  Az eredeti szoftver egyetlen btree-t támogatott. Az új 
  creord, delord, renord interfész függvényekkel új indexeket lehet
  kreálni, törölni, átnevezni. A setord interfész függvénnyel választható
  ki az aktuális index (amire a hagyományos műveletek vonatkoznak).
  Az indexek leírását tartalmazó táblázatnak el kell férnie a 0-dik
  lapon, ez korlátozza a számukat.
\item[Adatlapok.]
  Az új formátum lehetővé teszi adatrekordok tárolását.
  Az adatok (lapon belül) ugyanúgy tárolódnak, mint a kulcsok, azzal a 
  különbséggel,  hogy az adatlapok nem alkotnak fastruktúrát.
  Az adatrekordok (a kulcsokhoz hasonlóan) változó hosszúságúak lehetnek,
  de nem nyúlhatnak túl a lapon, ami tipikusan 4096 byte.
  Megjegyzem, hogy a lapozós technika egyáltalán nem az i/o sebesség
  növelését szolgálja, hanem ez teszi lehetővé a szabadlista
  alkalmazását, és ezzel a törölt lapok újbóli felhasználását.
\item[Lock protokoll.]
  Egyszerű pagelock protokoll biztosítja, hogy a konkurrens felhasználók
  tevékenysége ne ütközzön. 
\end{description}
 
Terveim szerint a formátum részét fogja képezni a bytesorrend,
ennek konverzióját azonban csak legvégül célszerű beépíteni
a szoftverbe, hogy a fejlesztés alatt ne zavarja a tiszta képet.



\section{Kulcskezelő interfész}

Az alább ömlesztve felsorolt interfész részletes leírásának 
jelenleg nem volna sok értelme, így csak néhány észrevételt teszek.


\subsection{Create/open}

\begin{verbatim}
  BTREE *__bt_open           (int fd, int psize);
  int    __bt_close          (BTREE*t);
  int    __bt_fd             (BTREE*t);
  int    __bt_pagesize       (BTREE*t);
  int    __bt_lastrec        (BTREE*t);
\end{verbatim}

Az \verb!open! paramétere nem filénév, hanem korábban 
megnyitott filé descriptor, így a nyitási mód kompatibilisnak
tartható a CCC fopen-nel, és a UNIX/Windows különbségek sem
gyűrűznek tovább.
Üres (nulla hosszúságú) filé esetén  meghatározható a pagesize, 
ha ezt 0-nak adjuk meg, azt a rendszer a default pagesize-zal 
helyettesíti, ez általában 4096 byte.
A filébe írt adatrekordok számát adja meg \verb!lastrec!.
 

\subsection{Navigáció, módosítás}

\begin{verbatim}
  int    __bt_delete         (BTREE*t, DBT*key);
  int    __bt_put            (BTREE*t, DBT*key, int cursor);
  int    __bt_seq            (BTREE*t, DBT*key, int flags);
\end{verbatim}

Ezek a hagyományos dbopen függvények kis módosítással.
A \verb!delete! függvény csak konkrétan megadott kulcs
törlését támogatja (kurzort nem töröl).
A \verb!put! harmadik paraméterével az szabályozható, 
hogy pozícionálódjon-e a kurzor.
A \verb!seq! függvény haramadik paraméterének (flags) egy bitjében
előírható, hogy a filé 0-dik lapjára a keresés alatt readlockot
tegyen (konkurrencia). Az indexet módosító processzek ennek megfelelően
writelockot tesznek a 0-dik lapra, de ez nincs beépítve
egyik (alacsonyszintű) függvénybe sem, mivel egyszerre több
elemi műveletet kell lefedni lockkal, ezért a táblaobjektumba való.
 
\subsection{Sorrendek}

\begin{verbatim}
  int    __bt_creord         (BTREE*t, char*name);
  int    __bt_setord         (BTREE*t, char*name);
  int    __bt_delord         (BTREE*t, char*name);
  int    __bt_renord         (BTREE*t, char*namold, char*namnew);
\end{verbatim}

E függvényekkel sorrendeket (btree struktúrákat) lehet a filében
kreálni, aktuálisként kijelölni, törölni, átnevezni.
Az interfészcsoport függvényei az indexet név alapján azonosítják. 
A \verb!delord! által törölt egész btree egy mozdulattal a
szabadlistába kerül. 
 

\subsection{Adatlapok}

\begin{verbatim}
  RECPOS __bt_addresource    (BTREE*t, DBT*buf, indx_t x);
  RECPOS __bt_append         (BTREE*t, DBT*buf, int*recno);
  int    __bt_rewrite        (BTREE*t, DBT*buf, RECPOS*recpos);
  int    __bt_read           (BTREE*t, DBT*buf, RECPOS*recpos);
  int    __bt_read1          (BTREE*t, DBT*buf, pgno_t p, indx_t x);
\end{verbatim}

Az \verb!addresource! függvény az 1-es lapon helyez el meg nem 
határozott célú metainfót. A táblaobjektum ezt a lapot a mezők
és indexek struktúrájának leírására használja.
Új adatrekordot ír a filébe az \verb!append!, és visszaad egy
RECPOS struktúrát, amivel hivatkozni lehet a rekordra.
Az adatrekord hossza nem kötött, de rá kell férjen egy lapra.
A \verb!rewrite! függvény felülír egy létező adatrekordot.
Az új rekord hosszának kötelezően egyeznie kell a régi hosszal.
Az adatlapok módosítása automatikus lockolással védett.
\verb!read! és \verb!read1! kiolvasnak egy adatrekordot.
 

\subsection{Konkurrencia}

\begin{verbatim}
  void   __bt_pagelock       (BTREE*t, pgno_t p, int mode);
  void   __bt_pageunlock     (BTREE*t, pgno_t p);
  int    __bt_header_read    (BTREE*t, int lock);
  int    __bt_header_write   (BTREE*t);
  int    __bt_header_sync    (BTREE*t);
  int    __bt_header_release (BTREE*t);
\end{verbatim}

A header író/olvasó függvények speciális gondossággal
kezelik a konkurrenciát, többek között egy lock számlálót
használnak az egymásba skatulyázott hívások nyilvántartására.
 

\section{Táblaobjektum szint}

Az eddigiek ismertették a nyersanyagot, most következne,
hogy hogyan lehet ebből megfőzni a vacsorát, azaz a táblaobjektumot.
Idő híján csak néhány megjegyzés:

Az 1-es lap (adatlap) speciális célra, a mezők és indexek
struktúrájának tárolására szolgál. Ez két rekordot jelent,
mindkettő \_arr2chr formátumban van kiírva.  
Más nincs az 1-es lapon.

Az adatrekordokat alapvetően a RECPOS struktúra segítségével lehet
előkeresni. Ez CCC szinten egy 6~byte-os string, aminek első 4~byte-jában
a lapszám, utolsó 2~byte-jában a lapon belüli index van.

A tabAppend művelet kiír egy (törölt) üres rekordot az utolsó adatlapra,
vagy, ha ott nincs már elég hely, kezd egy új lapot. Visszaadja
az új rekord RECPOS-át, amivel hivatkozni lehet a rekordra,
és képezhetők a kulcsok. Az indexek csak az appendált rekord 
commitjában módosulnak, ha az elmarad (pl.{} tranRollback miatt),
akkor a rekordra később egyáltalán nem lehet rápozícionálni,
még goto-val sem, azaz a recno-k sora lyukas lesz.

Az 0-dik index neve mindig "recno". A recno indexben levő kulcsok 
szerkezete: 4 byte-on (big endian) rekordsorszám plusz RECPOS.
A tabGoto művelet megkeresi az adott rekordsorszámhoz tartozó
kulcsot, majd az ebben talált RECPOS alapján beolvassa a rekordot.

A többi kulcs szerkezete: szegmensek konkatenációja plusz az előbbi recno
kulcs. Ez biztosítja, hogy az azonos kulcsok  mégiscsak különbözzenek,
és recno szerint legyenek rendezve. A tabSeek művelet nagyobbegyenlőt
keres a megadott kulccsal, a visszakapott kulcsból meg lehet
határozni az eredményrekord recno-ját, és a RECPOS alapján be lehet
olvasni a rekordot.

A pack műveletben el akartam kerülni, hogy a pack sikere
függjön a recno index épségétől, ezért a következőképpen megy:
Sorban beolvassuk a filé összes lapját, miközben csak az adatlapokkal
foglalkozunk. Ezekről az összes nem törölt rekordot átmásoljuk az
új filébe (append, recbuf csere, commit). Az indexek menet közben
újraépülnek. A szabadlista alkalmazása miatt a pack felcserélheti
a rekordok eredeti sorrendjét.

A CCC szintű rekordlock minden mástól független. Egy rekord lockja
2GB-1MB-tól kezdve visszafele számolva egyetlen byte lockját jelenti.

A btree filé mindig fopen-nel nyílik, ezért az fopen-ben alkalmazott
lock protokoll érvényesül. Egy bt filé pontosan egy descriptort használ.

Az indexeket olvasó processzek readlockot (lásd \verb!__bt_pagelock!),
a módosító processzek writelockot tesznek a 0-dik lapra. A lock
protokoll biztosítja minden processz számára a megfelelő védelmet.
Emellett teljesül, hogy egy írni akaró program rövid időn belül 
akkor is megkapja a kívánt writelockot, ha az olvasó programok readlockjai 
hosszú ideig átfednék egymást (live lock).


\section{Teljesítmény}
 
Az alábbi tesztek egy 172 ezer rekordot tartalmazó Kontó számla 
állománnyal készültek.

\begin{center}
\begin{tabular}{|l|r|r|r|r|r|r|}  \hline 
Formátum & Méret  & Skip  &  Pack  & Upgrade  & Goto+Seek  & Suppindex   \\ \hline  
BTBTX    &  95 MB &   9 s &   59 s &    143 s &       44 s &       23 s  \\ \hline
DBFCTX   &  80 MB &   5 s &   20 s &    113 s &       41 s &       25 s  \\ \hline
DATIDX   &  60 MB &  13 s &   23 s &    148 s &       43 s &       29 s  \\ \hline
\end{tabular}
\end{center}
 
A DATIDX takarékos mérete a számok lebegőpontos ábrázolásán (8 byte) 
múlik. A BTBTX ugyanolyan formátumban tárolja a mezőket, mint a DBFCTX. 
A méretnövekedés oka, hogy a rekordokat lapokra szétosztva tároljuk,
ezért minden lapon lesz egy töredék kihasználatlan hely. Emellett
a BTBTX formátum rugalmasabb: lehetővé teszi változó hosszúságú
adat-{} és indexrekordok tárolását.

A ,,Skip'' oszlop az egész adattábla számlaszám sorrendben történő
végigolvasásának ideje. Meglepően jó időt produkál a DBFCTX formátum.

A ,,Pack'' oszlop a tábla packolásának idejét mutatja. A BTBTX 
formátum packolása félig optimalizált: a rekordokat egészben másolja,
de az indexeket menet közben építi. A DBFCTX packolása nagyon gondosan van
megírva, a DATIDX ,,magától'' jó.

A tabUpgrade művelet a táblát rekordonként és mezőnként másolja,
eközben az indexek folyamatosan épülnek. 

A ,,Goto+Seek'' mérésnél minden rekordra rápozícionáltunk goto-val,
majd az ott talált számlaszámra seek-eltünk.

A ,,Suppindex'' oszlopban a számla állomány főkönyvi gyűjtéskor használt,
meglehetősen összetett indexének előállítási ideje látszik.


%\gotoindex
 