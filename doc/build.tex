
\pagetitle%
{Programkészítés a Build-del}%
{Dr. Vermes Mátyás\footnote{\ComFirm}}%
{1999. július -- 2003. szeptember}

\section{Áttekintés}

A CCC prog\-ramok a Build projekt-generátorral készülnek.
Build kielemzi a forráskönyvtárakban talált programokat, megállapítja, 
hogy azokból milyen lib-eket és exe-ket lehet csinálni.
Összeveti a források, objectek, lib-ek, exe-k dátum/idejét, 
és elvégzi a szükséges fordítást, linkelést. Build ismeri és kezeli
az összes olyan forrástípust, ami a CCC környezetben eddig előfordult:
mnt, cls, msk, pge,  asm, c, cpp, y, lex, prg,
így nemcsak Clipper programok fordítására képes, 
hanem pl.\ yacc és lex programokhoz is megfelel. 
Valójában a Build olyankor is jó szolgálatot tesz, 
amikor egyáltalán nincs a projektben Clipper nyelvű program.
Az egész CCC futtatórendszer, könyvtárak, utilityk (köztük maga a Build) 
fordítását is a Build vezérli.

Kiemelem, hogy a támogatott platformokon (DOS, Windows, Linux, Solaris) 
ugyanazt a makerendszert használjuk, ráadásul platformváltáskor
nincs szükség a források konverziójára, hanem egyszerűen átmásoljuk 
a projektet tartalmazó directory struktúrát egyik gépről a másikra. 
A Build nem keveri össze különböző fordítókkal készített,
különböző platformra szánt binárisokat.


A Build jelenleg az alábbi fordítókkal képes programot csinálni:

\begin{center}
\begin{tabular}{|l|l|} \hline
%Clipper 5.2e             & DOS              \\ \hline
%Watcom  C                & Windows NT/2000  \\ \hline 
Microsoft C (Visual C)   & Windows NT/2000  \\ \hline 
Borland C (C Builder)    & Windows NT/2000  \\ \hline 
MinGW                    & Windows NT/2000  \\ \hline 
GCC                      & Linux, Solaris   \\ \hline  
\end{tabular}
\end{center}

Maga a Build a DOS korlátai miatt csak UNIX-on és NT-n futtatható.
Lehetséges azonban DOS-os fejlesztés is, ha NT-s környezetből
fordítjuk az elavult rendszerre szánt programunkat.\footnote{
A CCC 2.x-ben megszűnt a DOS-os Clipper és a Watcom fordítási
környezet karbantartása, újdonság viszont a MinGW.}


\section{Implicit fordítási szabályok}

Build a forrásprogramok elemzésével megállapítja, hogy azok
milyen összefüggésben vannak egymással (melyikből lesz object,
melyik inkludálja valamelyik másikat, stb.), és a filéidők alapján
eldönti, hogy milyen fordítási műveleteket kell végrehajtani.
Ha Build úgy látja, hogy egy msk filéből elő kell állítani a
say-t, akkor végrehajtja az msk2say.bat 
scriptet. Általában is minden filétípust a filé kiterjesztése
alapján azonosítunk, és minden fordítási művelethez tartozik
egy batch filé, aminek a nevét az előbbi példa mintájára képezzük:
prg2obj.bat, cpp2obj.bat, obj2lib.bat, stb.\footnote{
A DOS világban meghonosodott kiterjesztéseket használjuk
Linuxon is (bat, obj, lib, exe), ez azonban, hála a Linux
rugalmasságának, semmilyen problémát nem okoz.}
Ezáltal Build működése éppen olyan, mint ahogy a make utility
alkalmazza az implicit fordítási szabályokat. A szabályokhoz
tartozó tevékenységet azonban nem a makefilében adjuk meg, 
hanem az iménti batch scriptekben.

A Build programnak paraméterként kell megadni, hogy hol keresse 
az adott platform xxx2yyy.bat alakú scriptjeit. A különböző
platformokhoz különböző script készleteket csinálunk, és ezzel
elfedjük a platformok közti eltéréseket.


\section{Build programtípusok}

A Build-et általában nem indítjuk közvetlenül, hanem olyan
scripteken keresztül, amik megfelelő paraméterezéssel előkészítik
Build-et egy-egy programtípus fordítására. Ezek a scriptek
a {\tt \$CCCDIR/usr/bin/\$CCCUNAME} könyvtárban találhatók.
A Windowson használható scriptek:

\begin{tabular}{ll} 
\verb'bapp_w320.bat          ' & Win32, CCC könyvtár nélkül             \\                           \\ 
\verb'bapp_w32_.bat          ' & Win32 konzol, képernyőkezelés nélkül   \\ 
\verb'bapp_w32c.bat          ' & Win32 konzol, fullscreen képernyő      \\ 
\verb'bapp_w32c_btbtx.bat    ' & Win32 konzol, BTBTX adatbázis          \\ 
\verb'bapp_w32c_datidx.bat   ' & Win32 konzol, DATIDX adatbázis         \\ 
\verb'bapp_w32c_dbfctx.bat   ' & Win32 konzol, DBFCTX adatbázis         \\ 
\verb'bapp_w32g.bat          ' & Win32 GUI                              \\ 
\verb'bapp_w32g_btbtx.bat    ' & Win32 GUI, BTBTX adatbázis             \\ 
\verb'bapp_w32g_datidx.bat   ' & Win32 GUI, DATIDX adatbázis            \\ 
\verb'bapp_w32g_dbfctx.bat   ' & Win32 GUI, DBFCTX adatbázis            \\ 
\end{tabular}
 
A Linux, Solaris platformon használható scriptek:

\begin{tabular}{ll} 
\verb'bapp_unix0.b          ' & UNIX, CCC könyvtár nélkül              \\ 
\verb'bapp_unix_.b          ' & UNIX, képernyőkezelés nélkül           \\  
\verb'bapp_unixc.b          ' & UNIX, karakteres (fullscreen) képernyő \\   
\verb'bapp_unixc_btbtx.b    ' & UNIX, karakteres képernyő, BTBTX adatbázis  \\ 
\verb'bapp_unixc_datidx.b   ' & UNIX, karakteres képernyő, DATIDX adatbázis \\ 
\verb'bapp_unixc_dbfctx.b   ' & UNIX, karakteres képernyő, DBFCTX adatbázis \\ 
\verb'bapp_unixg.b          ' & UNIX, Fltk (X) képernyőkezelés         \\
\verb'bapp_unix.b           ' & Dinamikus megjelenítő választás        \\ 
\end{tabular}

A dinamikus megjelenítés választás azt jelenti, hogy a program
elinduláskor felderíti (tipikusan környezeti változók alapján),
hogy milyen megjelenítéssel (grafikus vagy karakteres) kell működnie.

A felsorolás közel sem teljes. A minták alapján könnyen lehet 
készíteni további variációkat, emellett a Build egy adott
programtípuson belül, alkalmazás szinten is rugalmasan paraméterezhető,
ahogy azt a következőkben látni fogjuk.


\section{Alapvető üzemmódok}


Build legegyszerűbb használata esetén az aktuális directoryban található 
forrásprogramokból készít exe, lib és so filéket. A programnak négy üzemmódja
van, amit az {\tt -x}, {\tt -l} és {\tt -s} kapcsolókkal, illetve
ezek elhagyásával lehet beállítani.

\subsection{Egyetlen exe ({\tt -x}) üzemmód\label{switchx}}

Példa: {\tt bapp\_unix.b -xpelda}

Ez a parancs arra utasítja Build-et, hogy a pelda.prg forrásprogramból
hozza létre pelda.exe-t úgy, hogy az aktuális directoryban lévő többi 
forrást szintén fordítsa le, és az objecteket linkelje az exe-hez.
Részletesebben a következők történnek.

\begin{itemize}  
\item 
  Build  megvizsgálja, hogy van-e az aktuális directoryban
  pelda.prg, és hogy az tartalmaz-e  function main-t, ha nem,
  akkor hibát jelez, ha igen, akkor beveszi a projektbe.
\item
  Build az aktuális directory összes többi prg filéjét megvizsgálja,
  és azokat, amik nem tartalmaznak main-t hozzáveszi a projekthez.
  Ezek szintén le lesznek fordítva, és hozzá fognak linkelődni 
  pelda.exe-hez.
\item
  A projektbe bevett prg-ket Build megvizsgálja abból a szempontból is, 
  hogy milyen \#include direktívákat tartalmaznak. Ha az inkludált filé 
  megtalálható, vagy az implict fordítási szabályok alkalmazásával valamely 
  más filéből előállítható, akkor azt is beveszi a projektbe, és
  hozzáveszi a modul függőségi listájához. 
\item
  Végül a projekthez hozzáveszi az összes c, cpp, asm, y, lex programot is.
  A nem prg típusú források \#include-jait Build nem vizsgálja.
\end{itemize} 
Mint látjuk Build az aktuális directory minden szóbajövő programját
felhasználja pelda.exe elkészítéséhez. 


\subsection{Könyvtár ({\tt -l}) üzemmód\label{switchl}}

Példa: {\tt bapp\_unix.b -lpelda}

Ez a parancs arra utasítja Build-et, hogy készítse el pelda.lib-et,
amibe be kell venni az aktuális directory minden olyan forrásprogramját,
ami nem tartalmaz function main-t, ezenkívül az összes olyan prg-ből,
ami main-t tartalmaz készítsen exe programot az előbbi lib hozzálinkelésével.
A prg-k \#include-jait most is vizsgálja Build, és lehetőség szerint
beveszi őket a projekt függőségi listáiba.
A {-l} üzemmód még az {-x}-nél is mohóbb, az összes forrást így vagy úgy
felhasználja, hogy programot készítsen belőle. 

\subsection{Osztott könyvtár ({\tt -s}) üzemmód\label{switchs}}

Példa: {\tt bapp\_unix.b -spelda}

Ez az üzemmód UNIX-on alkalmazható, annyival több a \verb!-l!
módnál, hogy elkészül a könyvtár osztott változata (so) is.

\subsection{Minden main-ből exe (alapértelmezett) üzemmód }

Példa: {\tt bapp\_unix.b}

Ha az előbbi {\tt -x}, {\tt -l}, {\tt -s} kapcsolók egyikét sem 
alkalmazzuk, akkor Build az aktuális directory minden main-t tartalmazó
programjából exe-t készít, amihez hozzálinkeli az összes
többi programból készített objectet anélkül, hogy azokból
könyvtárat készítene.

A projektek (könyvtárak, végrehajtható programok) tartalmát a 
források megfelelő csoportosításával kell szabályozni. 
Például, ha nem akarjuk letörölni egy forrás elavult változatát, 
de az sem jó, ha jelenlétével zavarja Build működését, 
akkor egyszerűen félre kell rakni egy alkönyvtárba.
A fent ismertetett négy üzemmód az alkalmazási programok 90\%-ában 
elegendő. A bonyolultabb esetekben segítenek az alább ismertetett
kapcsolók.


\section{Paraméter referencia}

\subsection{Kapcsolók}
\begin{description}
\item[{\tt -xExeName}] 
  ExeName az elkészítendő program neve (\href{#switchx}{lásd fenn}).

\item[{\tt -lLibName}] 
  LibName az elkészítendő könyvtár neve (\href{#switchl}{lásd fenn}).

\item[{\tt -sLibName}] 
  LibName az elkészítendő osztott könyvtár neve (\href{#switchs}{lásd fenn}).
 

\item[{\tt -dSrcDir}] 
  SrcDir azon directoryk listája, amikben Build a lefordítandó
  forrásmodulokat keresi.   
  A lista a {\tt,; } karakterekkel lehet elválasztva.
  A parancssorban több {\tt -d} paraméter is meg lehet adva, ekkor ezek
  tartalma összeadódik (additív). Ha {\tt -d} meg van adva, az aktuális 
  directoryban csak akkor keres a program, ha az explicite fel van sorolva, 
  pl. \verb'-d.;srcprg;srccpp'.
  Ha {\tt -d} nincs megadva, akkor a program csak az aktuális directoryban
  keres, ami a \verb'-d.' paraméterezéssel egyenértékű.
  A {\tt -d}-ben felsorolt directorykban Build automatikusan
  keresi az inkludálandó filéket is, ezért ezeket nem kell a {\tt -i}-ben 
  is megadni.


\item[{\tt -mMain}] 
  Az {\tt -m} kapcsolóval lehet felsorolni a főprogramot
  tartalmazó forrásmodulokat. A main modulok listája a {\tt,;} 
  karakterekkel lehet elválasztva. A kapcsoló additív.
  Az {\tt -m} kapcsolót akkor használjuk, ha a main függvény nem 
  prg  típusú forrásban van. Ezzel Build nem Clipper nyelvű programok
  fordítására is használható. A ppo2cpp program pl.\  cpp, y és lex 
  típusú forrásokból készül, azaz egyáltalán nincs is a projektben prg. 


\item[{\tt -iIncDir}] 
  Az include directoryk listája adható meg az {\tt -i} kapcsolóval.
  A C fordító saját könyvtárait nem itt szoktuk megadni, hanem 
  a fordító konfigurálásánal, ami rendszerfüggő, pl.\ DOS-on az 
  include környezeti változót szokták erre használni.  
  A lista {\tt ,;} karakterekkel szeparálható, a kapcsoló additív. 
  Nem kell olyan directoryt megadni, ami {\tt -d}-ben már szerepelt.


\item[{\tt -pLibPath}] 
  Azon directoryk listája adható meg a {\tt -p} kapcsolóval,
  ahol a linker a CCC könyvtárakat keresi. A C fordító saját
  könyvtárainak helyét nem itt szoktuk megadni, hanem a lib 
  környezeti változóban. A lista {\tt ,;} karakterekkel
  szeparálható, a kapcsoló additív.


\item[{\tt -bLibFile}] 
  A linkelendő könyvtárak nevét tartalmazó lista. 
  A lista {\tt ,;} karakterekkel szeparálható, a kapcsoló additív. 


\item[{\tt -hHelp}] 
  Listát ad az érvényes kapcsolókról.
  
  
\item[{\tt name=value}]  
  A {\tt name=value} paraméter hatására a Build futása alatt
  a name környezeti változó értéke be lesz állítva value-ra.
  Build-et egy sereg {\tt BUILD\_XXX} alakú környezeti változóval
  is paraméterezhetjük. A Build a fordítást végző xxx2yyy.bat  
  scriptekkel szintén ilyen alakú környezeti változókon 
  (és paramétereken) keresztül kommunikál. Ez a paraméter megadási 
  forma lehetővé teszi, hogy Build környezeti változóit az operációs
  rendszertől független szintaktikával állítsuk be.

\item[{\tt \$(name)}]  
  Ha Build a parancssorában \verb'$(name)' alakú kifejezést
  talál, azt helyettesíti a \verb'getenv(name)' értékkel.
  Ez lehetővé teszi, hogy a környezeti változó értékét
  az operációs rendszertől független szintaktikával kapjuk meg.  
 


\item[{\tt @parfile}] 
  Parfile egy tetszőleges paraméterfilé, amiben a fent leírt 
  paramétereket helyezhetjük el. A paramétereket a filében újsor 
  karakterrel választjuk el. A parancssor és a paraméterfilé olvasása
  folytatólagos, pl. lehet több paraméterfilé is a parancssorban.
\end{description}


\subsection{Build környezeti változók}

\begin{description}
\item[{\tt BUILD\_BAT}]  
  Implicit szabályokhoz rendelt batch scriptek directoryja.


\item[{\tt BUILD\_BSN}]  
  A Bison-nak átadott kapcsolók.


\item[{\tt BUILD\_CLS}]  
  Az objccc paraméterei, pl. \verb'build_cls=-iclassdef'.


\item[{\tt BUILD\_CPP}]  
  Egy directory, ahol a rendszer a C++ kódot gyűjti abból a célból,
  hogy letölthető installáló készletet lehessen előállítani belőle.
 

\item[{\tt BUILD\_DBG}]  
  Ha e változó értéke {\tt on}, akkor Build kiírja a projekt
  függőségi listáit, és megmutatja a filék időbeli viszonyait.


\item[{\tt BUILD\_EXE}]  
  Az exe directory specifikációja, ide teszi Build az elkészült 
  exe-ket, defaultja az aktuális directory.

\item[{\tt BUILD\_INC}]  
  A statndard include directoryk listája.

\item[{\tt BUILD\_LEX}]  
  A Flex-nek átadott kapcsolók.

\item[{\tt BUILD\_LIB}]  
  A standard libraryk listája.

\item[{\tt BUILD\_LPT}]  
  A standard library path.

\item[{\tt BUILD\_OBJ}]  
  Az aktuális object directory neve 
  (a különböző platformú binárisokat ui.\  szét kell választani).
  Főleg könyvtár specifikációban használjuk "átnyúlkálós" linkeléskor.
  Például a KLTPINFO program így linkeli a KLTP projekt könyvtárát:
   \verb'-b../kltp/$(BUILD_OBJ)/kltp'.

\item[{\tt BUILD\_OPT}]  
  A \verb!$CCCDIR/usr/options/$CCCBIN! directoryban
  található, standard C++ fordítás opcióit tartalmazó filé neve.

\item[{\tt BUILD\_PRE}]  
  A prg2ppo preprocesszornak (vagy a Clippernek) átdott paraméterek.


\item[{\tt BUILD\_SRC}]  
  Ha ebben a változóban megadunk egy directoryt, akkor Build
  a forrásdirectory paramétereket ({\tt -d}) erre a directoryra
  relatívan értelmezi.


\item[{\tt BUILD\_SYS}]  
  Subsystem azonostó. Tartalma platformfüggő, pl. MSC-vel windows/console, 
  Watcom-mal nt\_win/nt.
\end{description}
 

\subsection{Könyvtár és include környezeti változók}


Külön kell szólni a 
{\tt BUILD\_INC}, {\tt BUILD\_LIB}, {\tt BUILD\_LPT} változókról. 

Build összegyűjti az include és library paramétereket, 
és elhelyezi egy-egy környezeti változóban úgy (blank elválasztó),
hogy később batch filéből shiftekkel könnyen fel lehessen dolgozni.
A használt környezeti változók:
\begin{description}
\item[{\tt BUILD\_INC}] {\tt -d} + {\tt -i} kapcsolók tartalma + \verb'$(BUILD_INC)' + \verb'$(INCLUDE)' 
\item[{\tt BUILD\_LPT}] {\tt -p} kapcsoló tartalma + \verb'$(BUILD_LPT)' + \verb'$(LIB)'  
\item[{\tt BUILD\_LIB}] {\tt -b} kapcsoló tartalma + \verb'$(BUILD_LIB)' 
\end{description}
Ebből megállapítható a különböző szinten előírt paraméterek
közötti összefüggés.

\begin{enumerate}
\item 
  A kapcsolók szintjén adjuk meg a projekt specifikus 
  paramétereket. Ha pl. a KLTPINFO átnyúl a KLTP projekt
  OCH (include) filéiért, azt így írjuk elő: \verb'-i../kltp/classdef'.
\item 
  A \verb'BUILD_XXX' változók szintjén adjuk meg a standard 
  paramétereket. Pl. a CCC könyvtárak, a Kontó könyvtárak is
  így vannak megadva. Ezeket a paramétereket projekt szinten
  nem tanácsos piszkálni. Ugyancsak ezen a szinten állítunk
  be olyan paramétereket, amiknek a használata nem gyakori,
  pl. \verb'BUILD_EXE', 
      \verb'BUILD_DBG', 
      \verb'BUILD_LEX', 
      \verb'BUILD_BSN', 
      \verb'BUILD_PRE'.
\item 
  Végül az általános környezeti változók szintjén kell magadni 
  az aktuális fordító (Clipper, Watcom C, stb.) normál működéséhez
  szükséges include és lib directorykat.
\end{enumerate}


\section{Példa}

A \verb!CCCDIR/tools/mask/m! script tartalma a következő:
\begin{verbatim}
#!/bin/bash
bapp_unixc.b -lmask "BUILD_EXE=$CCCDIR/usr/bin/$CCCUNAME" 
\end{verbatim}

\begin{enumerate}
\item 
   UNIX-os, karakteres képernyőkezelésű programok készülnek.
\item 
   Minden main-t nem tartalmazó program lefordul, és készül belőlük
   egy \verb!mask! nevű könyvtár.
\item 
   Minden main-t tartalmazó programból lesz egy azonos nevű exe,
   amihez hozzálinkelődik az előbbi könyvtár. A konkrét példában
   öt darab exe fog készülni (ez a scriptből nem látszik,
   a Build fogja megkeresni őket).
\item
   A \verb!BUILD_EXE! változó beállításával megadjuk azt a directoryt,
   ahová a Build a friss exe-ket tenni fogja. Enélkül az aktuális
   directoryban jönnének létre az exe-k.
\end{enumerate}

Ha az \verb!m! script tartalma csak ennyi volna:
\begin{verbatim}
#!/bin/bash
bapp_unixc.b
\end{verbatim}
akkor is lefordulna az összes program, létrejönne az öt exe
(az aktuális directoryban), de nem készülne külön könyvtár.

Az előbbi script windowsos megfelelője \verb!m.bat!:

\begin{verbatim}
@echo off
bapp_w32c -lmask  "BUILD_EXE=$(CCCDIR)/usr/bin/nt"
\end{verbatim}

Nagyon hasonló a UNIX-os változathoz, azonban a UNIX
és Windows shell eltérései miatt a paraméterekben 
előforduló speciális karaktereket másképp kell védeni.
A \verb!$(CCCDIR)! makrót a Build fogja értelmezni.
A Build egyformán megérti a \verb!\!  és \verb!/! 
elválasztó karakterrel leírt útvonalakat.

Végeredményben két nagyon rövid, egymáshoz nagyon hasonló scriptet 
kell csak megírni, ezután bármely platformon  az egybetűs \verb!m! 
parancs hatására elkészül az adott direktoriban fellelhető 
összes program. Így működik a Build.

További példák sokasága található a CCC directory struktúrában,
ahol  minden projekt, beleértve a runtime libraryk fordítását, 
a prg forrást egyáltalán nem tartalmazó programok fordítását,
a Build-del készül. 
 
%\gotoindex

 
 