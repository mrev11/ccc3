

\section{Bevezetés}


A Clipper praktikus, egyszerű, általános célra használható eszköz volt. 
Nem véletlen a rendkívüli népszerűség, amire a maga idejében szert tett. 
A DOS-szal együtt azonban a Clipper kihalt,  az utána keletkezett űr 
betöltetlen maradt. A Clipper utódok, mint a Clip, FlagShip, Harbour nem 
terjedtek el széles körben, nálam egyik sem ,,nyerő'', mint ahogy az új nyelvek sem.
A Jáva termelékenység helyett tudálékos, a Python túlságosan script \ldots

A CCC-t a banki programjainkkal párhuzamosan, saját célra fejlesztettük,
azzal az elgondolással, hogy a DOS korszak után a programokat ezzel fogjuk 
életben tartani.
Volt idő, amikor egyszerre DOS-on, Windowson és UNIX-(ok)on  
futottunk. Minden platformra ugyanabból a forrásból fordítottunk,
DOS-on Clipperrel, máshol CCC-vel. Ugyanabból a forrásból,
ez nem egy verziókezelő rendszerből elővett különféle változatokat jelent, 
hanem ténylegesen ugyanazokat a forrásfájlokat.

A CCC nem tud akármilyen Clipper kódot lefordítani. A programjainkat 
eleve egy szűkített Clipperre írtuk. A legfontosabb szűkítés, 
hogy a CCC-nek (mint nyelvnek) nem része az adatbáziskezelés. 
Mindenesetre a Clipper és a CCC közös része elég volt egy komplex 
számlavezető rendszer elkészítéséhez.
Amúgy a nyelv alkalmas akármilyen adatbáziskelő szoftver 
(manapság leginkább SQL interfész) megírására. 


Az eredeti Clipper a mostani igényeknek már nem felelne meg.
Idők során a Clipperen túllépő fejlesztések kerültek a CCC-be: 
objektumok,
kivételkezelés,
névterek, 
Unicode támogatás, SSL interfész, effélék.
A CCC-t  elég modernnek és jónak tartjuk ahhoz, 
hogy az új projektjeinkhez se keressünk más nyelvet.

A CCC történetének fő állomásai:
\begin{itemize}
\item 1996 -- Már működik a CCC1. 
\item 1999 -- Először a BB-ben használják éles számlavezetésre.
\item 2002 -- Többszál támogatás (CCC2=CCC1+multithreading).
\item 2004 -- 64-bit támogatás.
\item 2005 -- \href{http://www.fsf.org/licensing/licenses/lgpl.txt}{LGPL} 
              hatálya alá kerül.
\item 2006 -- Unicode/UTF-8 támogatás (CCC3=CCC2+Unicode).
\end{itemize}

A jelen dokumentáció a CCC3-mal foglalkozik.
A dokumentációt olyan programozóknak szánom, akik szeretnek beszélgetni
programokról, programozásról. Nem kezdőknek. A dokumentum nem ,,egyenszilárd''.
Van benne Hello World, de nem szól fejezet a for ciklusról és társairól.
A Clippert meghaladó, új részekre fókuszálok,
ezen belül is elsősorban magára a nyelvre, 
és nem a nyelven megírt könyvtárakra vagy interfészekre.
A dokumentum végére belinkeltem néhány régi doksit.
A Clipper eredeti HTML leírása egyes részeiben még mindig használható.





