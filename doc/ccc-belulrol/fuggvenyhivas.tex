

\section{Függvényhívás}


\subsection{Független modulok}

Függvény és függvényhívás a Clipper/CCC legfontosabb építőkövei. 
Mielőtt azonban a kövek részleteit vizsgálnánk, nézzünk az épület egészére.

Egy projekt általában sok forrásmodulból áll. A modulok neve legyen mondjuk
\begin{verbatim} 
    code1.prg
    code2.prg
    ...
    code100.prg
\end{verbatim}
A fordító ezeket egyesével lefordítja, és kapjuk a
\begin{verbatim} 
    code1.obj
    code2.obj
    ...
    code100.obj
\end{verbatim}
object (gépi kód) fájlokat. Lehet, hogy a prg források egy részével nem találkozunk 
közvetlenül, mert egy másik projektben készülnek, és mi már csak a lefordított 
objecteket kapjuk. Utóbbi  esetben az a tipikus, hogy az objectek nem külön-külön, 
hanem könyvtárakba (statikus lib, vagy dinamikus so) összecsomagolva állnak 
rendelkezésre. Ez a helyzet a futtatórendszer alapkönyvtáraival is,
amit minden program használ. Akárhogy is, a lényeg, hogy rendelkezésünk áll 
egy rakás object fájl. Ezekből lesz a program.

A {\em linker\/} (link editor) feladata, hogy az egymástól függetlenül 
létrehozott, egymásról mit sem tudó nagy rakás object fájlból futtatható 
programot szerkesszen (linkeljen).

Tegyük fel, hogy a projektünk végterméke egy közvetlenül elindítható program
(nem pedig könyvtár). Akkor a modulok valamelyikében kell legyen \verb!main!
függvény, ahol elindul a program. Mondjuk a \verb!code1.prg!-ben.
\begin{verbatim} 
function main()
local x,a,b,c
    ...
    x:=fuggveny1(a,b,c)
    ...
\end{verbatim}
Program nem létezhet függvényhívás nélkül.
Már a \verb!main!-re is függvényhívással kerül a vezérlés 
a futtatórendszer alapkönyvtárából. A main aztán meghívhat további
függvényeket, mint a példában \verb!fuggveny1!-et, aminek a belsejében 
további függvényhívás lehet, és így tovább. 
\begin{verbatim} 
function fuggveny1(a,b,c,d)
    ...
    fuggveny2()
    ...
\end{verbatim}
A körnek végül záródnia kell, az összes meghívott függvény
kódja meg kell legyen valahol az összeszerkesztett modulokban: 
az alapkönyvtárakban, plusz a \verb!code1.obj!... modulokban.

Az előbb ,,egymásról mit sem tudó'' modulokról beszéltünk.
Ez azt jelenti, hogy a fordítóprogram semmit (a nevükön kívül semmi egyedit)
nem tud a függvényekről. A fordítóprogram a \verb!fuggveny1! fordításakor
látja, hogy kódot kell generálnia a \verb!fuggveny2! meghívására. Miközben
ezt megteszi nem tudja, hogy a \verb!fuggveny2! hol van definiálva,
definiálva van-e egyáltalán, milyen paraméterezéssel kell meghívni,
milyen visszatérési értéket ad.

Persze nem tudnánk értelmes programot írni, 
ha nem volnánk tisztában a függvények paraméterezésével, 
visszatérési értékével, de ezt nem a fordító tudja, hanem a programozó. 
A dokumentációból vagy a forrásprogram  elolvasásából. 
A fordítóprogram szempontjából minden függvényhívás ugyanolyan,
mindegyik ugyanarra a kaptafára húzható.

C-ből nézve minden Clipper függvény 
\begin{verbatim}
extern void _clp_fuggveny1(int argno); //C++ kód
\end{verbatim}
deklarációval rendelkezik, azaz C szinten nincs visszatérési értéke (void), 
és egyetlen int paramétere van, ami megmondja,  hány paramétert 
kapott Clipper szinten. Clipper szinten minden függvénynek van visszatérési
értéke, ha más nincs megadva, akkor NIL. 

Ha a Clipper/CCC programozó kap egy bináris object fájlt 
(vagy lib/so könyvtárat) és tudja, hogy mi a fájl tartalma, 
akkor minden további nélkül belinkelheti a programjába, használhatja.
A C programozók számára ez merőben szokatlan. 
C/C++-ban egy könyvtár használatához múlhatatlan szükség van az ún. header 
fájlokra (forrás típusú fájlok), amik segítik a fordítót a kódgenerálásban és 
a függvényhívások ellenőrzésében.


A gyakorlatban a változók, függvények, objectek, könyvtárak
(egymásra épülő, egymást feltételező) uniform mivolta azt eredményezi, 
hogy egymástól függetlenül fordított modulok, könyvtárak összelapátolásával 
rendkívüli termelékenységgel hozhatunk létre óriási programokat.



\subsection{Láthatóság}

Eddig csak globális láthatóságú függvényekről beszéltünk.
Természetesen nem szerencsés, ha egy belső használatra szánt
függvény mindenhonnan látszik, és ki van téve a tervezettől eltérő használatnak
vagy akár csak véletlen névütközésnek.
A függvényeinket forrásmodulon belül eldughatjuk \verb!static! definícióval:
\begin{verbatim}
static function fuggveny1()
    ...
\end{verbatim}
Egy static függvény csak abban a forrásmodulban látható, amiben definiálták. 
Clipper/CCC-ben és C-ben a \verb!static! szó ugyanazt jelenti,
változó deklarációban és függvény definícióban egyaránt.
A static függvények C-ben is static-ok:
\begin{verbatim}
static void _clp_fuggveny1(int argno); //C++ kód
\end{verbatim}


\subsection{Függvényhívás a veremgépen}

Most megnézzük a részleteket,
hogyan valósul meg a ,,uniform'' függvényhívás a CCC veremgépén.

\begin{verbatim}
local x,a,b,c,d
    ...
    x:=fuggveny1(a,b,c)
    fuggveny1()
    fuggveny1(,,c,d)       //ua. mint fuggveny1(NIL,NIL,c,d)
    ...
\end{verbatim}
A függvényeket akárhány darab és akármilyen típusú paraméterrel 
meg lehet hívni. A függvény mindig ad visszatérési értéket, amit viszont
nem kötelező felhasználni. A fenti programrészlet talán furcsának látszik,
de formailag hibátlan. Hogy jó-e, az csak futáskor dől el. 
Lehetséges, hogy \verb!fuggveny1! úgy van megírva, hogy megvizsgálja
a paramétereit, és a konkrét esettől függően csinál ezt vagy azt.

Nézzük a dolgokat a hívott függvényből.
\begin{verbatim}
function fuggveny1(a,b,c)
local x:=0
    x+=if(a==NIL,0,a)       //NIL helyett 0
    x+=if(b==NIL,0,b)
    x+=if(c==NIL,0,c)
    return x
\end{verbatim}
Itt egy példa \verb!fuggveny1! viszonylag értelmes implementációjára.
Akkor működik jól, ha NIL vagy szám típusú paramétereket kap. Összeadja
a paraméterek számértékét (a NIL-ek helyett nullát vesz), és visszaadja az összeget.

Ha egy paraméter nincs megadva, annak a futtatórendszer 
NIL értéket ad. Nincs megadva: a paraméterlista rövid vagy hézagos.
A NIL értékeket a példaprogram nullával helyettesíti.

Ha a paraméterlista hosszabb 3-nál, akkor a fölös paramétereket 
a futtatórendszer eldobja. Az előző példában \verb!d! mintha ott se lenne.

Ha a paraméterek között valamilyen más típus van, pl. logikai, 
az valamelyik összeadásban fog kiderülni. A program ,,elszáll'',
és egyúttal kiírja, hogy melyik sorban milyen értékeket nem lehetett összeadni, 
valamint kiírja az egész stacket. Ilyen a CCC hibakezelése.

Ha már a stacknél tartunk, egészítsük ki a példaprogramot, 
és nézzük meg a stacket:

\begin{verbatim}
function main()
local x,a:=1,b:=2,c:=10,d:="d"
    x:=fuggveny1(a,b+b)

function fuggveny1(a,b,c)
local x:=0
    x+=if(a==NIL,0,a)
    x+=if(b==NIL,0,b)
    x+=if(c==NIL,0,c)
    varstack()
    return x
\end{verbatim}

A \verb!varstack()! a futtatórendszer beépített függvénye,
kiírja a program összes static (most egy sincs) és local változóját.

\begin{verbatim}
-----------------------------------------------------------
 Variable Stack
-----------------------------------------------------------
***** function main
0: NIL
1: NUMBER 1
2: NUMBER 2
3: NUMBER 10
4: STRING length=1 oref=b79b6008 "d"
***** function fuggveny1
5: NUMBER 1
6: NUMBER 4
7: NIL
8: NUMBER 5
-----------------------------------------------------------
\end{verbatim}
A rendszer tudja, hogy a stacken hol vannak a függvényhívási határok.
Azt remélem, az olvasó bonyolult magyarázat nélkül is azonnal érti, amit lát.
A \verb!fuggveny1! szintjét nézve vegyük át, hogyan zajlik le a függvényhívás,
végrehajtás, visszatérés.

    A hívó program sorban egymás után a stackre rakja a 
      paramétereket. Egy paraméter akármilyen bonyolult kifejezés lehet,
      szépen kiszámolódik, a végeredmény (a kifejezés értéke) a stack 
      tetején marad.  Az 5-ös és 6-os stack elemek a paraméterek.

    Meghívódik \verb!fuggveny1!. Meg van mondva neki, 
      hogy 2 darab paramétert kapott. A stack állásából és 
      a kapott paraméterek számából tudja, hogy a saját szintje
      az 5-ös elemtől kezdődik, egyúttal megvan a paraméterek értéke.
      Azokat az argumentum változókat, amikre nem jutott paraméter, 
      NIL-re inicializálja. Így kerül \verb!c! értekeként NIL a 7-es stack elembe.
      Az esetleges fölös paramétereket (most nincs ilyen) kipucolja.

    A \verb!fuggveny1! további helyeket foglal magának a stacken
      a local változók számára (amiket inicializál is, bár ez most nem látszik).
      A 8-as stack elemben tárolódik az \verb!x! változó.

    A hívott függvény elvégzi, amit kell. 
      Eközben a stack ,,szuszog''. Részeredmények tárolódnak
      rajta, mélyül majd visszacsökken a függvényhívások szintje. 

    Elérkezik a visszatérés ideje. \verb!fuggveny1! 
      az 5-ös elem helyére beírja a visszatérési értékét,
      majd beállítja a stackpointert, hogy a stack első szabad helyeként
      a 6-os elemet mutassa. Visszatér.    

    A hívó program szempontjából a függvényhívás egy {\em kifejezés\/} volt.
      Minden kifejezés úgy működik, hogy ,,kiszámolja magát'', 
      és az értékét a stack tetején hagyja. Esetünkben a kifejezés értéke
      a visszatérési érték. Ezzel azt csinál a hívó program, amit akar.
      Ha semmit sem csinál, akkor az érték automatikusan lekerül a stackről,
      és elvész.
      

\subsection{Referencia paraméterek}

Más nyelvekről szóló dokumentációkban találhatunk ilyen kijelentést:
,,A paraméterátadás érték szerint történik''. Ez van a Clipper/CCC-ben is.

Írjuk át az előző példaprogramot.
\begin{verbatim}
function fuggveny1(a,b,c)
    if( a==NIL )
        a:=0
    end
    ...
    return a+b+c
\end{verbatim}
Így \verb!fuggveny1! működése ugyanaz marad.
Felvetődik viszont a kérdés: Az \verb!a:=0! értékadás miatt nem fordul-e elő,
hogy az \verb!a! változó értéke a hívó programban is megváltozik? Nem. 
Aki figyelmesen olvasta az előző pontot, és követte, mi történik a vermen,
annak ez nyilvánvaló.

Bonyolultabb a referenciát tartalmazó típusok esete. 
String, array, objektum, kódblokk esetén is érték szerint történik a paraméterátadás, 
csakhogy ilyenkor az értékben közvetlenül tárolt adat egy OREF 
(memóriaobjektum referencia). Emiatt a hívó és hívott program ,,közösködik''
ugyanazon a tartalmon. 

Stringek esetében ez a közösködés readonly,  csak addig tart,
amíg a hívott program nem akar változtatni a stringen. Amikor változtat,
a hívó és hívott program változóinak tartalma szétválik.

Array, objektum, kódblokk esetén a hívott program megváltoztathatja
a memóriaobjektum {\em belsejét\/} (array esetén az array hosszát is).

Az eddigi ,,normál'', és magától értetődő eseteken túl előfordul, 
hogy kifejezetten arra van szükség, 
hogy a hívott függvényben végrehajtott változtatás
a hívó program változójára is hasson. 
Ezt szolgálja a referencia szerinti paraméterátadás.

\begin{verbatim}
function main()
local x,a,b:=2,c:=10,d:="d"     // a==NIL
    x:=fuggveny1(@a,b+b)        // @a referencia
    ? a                         // kiírja: 0

function fuggveny1(a,b,c)
    if( a==NIL )
        a:=0
    end
    ? valtype(a)                // kiírja: N
    ...
    return a+b+c
\end{verbatim}

Most az \verb!a! változók \verb!main!-ben és \verb!fuggveny1!-ben együtt változnak.
A \verb!@a! paraméterátadás hatására az eredetileg NIL típusú változó referenciává
alakul, és ez a referencia adódik át a hívott függvénynek. 
A referencia típusú változók kiolvasását/értékadását a futtatórendszer
speciálisan támogatja: Automatikusan eggyel ,,tovább nyúl'' a tényleges
értékért anélkül, hogy ezt a programban jelölni kellene. Úgy képzelhetjük el, 
mint egy egy hosszúságú tömböt, aminek az indexelését a rendszer automatikusan
elvégzi.

Külön említendő, hogy a hívott program {\em nem tudja}, hogy az argumentumai
referenciák-e vagy sem. Nem tudja? Mit mond róla a \verb!valtype()!?
Amikor \verb!valtype(a)! meghívódik, akkor is működik a referencia típusok
speciális támogatása, azaz a rendszer eggyel tovább nyúlva előveszi  \verb!a!
tényleges tartalmát, ezért \verb!valtype()! úgy látja, hogy a kérdéses típus szám.

Eddig tartott a régi Clipper tudománya. 
Az eddig tárgyalt dolgokban CCC és Clipper között tökéletesnek mondható a kompatibilitás. 
A következők már a CCC újításai.

\subsection{Default értékek}

A kényelmesebb programírás érdekében a függvénydefinícióban
default értéket adhatunk az argumentum változóknak. Még mindig
az előző példánál maradva, ezt is írhatjuk:

\begin{verbatim}
function fuggveny1(a:=0,b:=0,c:=0)
    return a+b+c
\end{verbatim}

Általánosságban
\begin{verbatim}
function fuggveny(...,a:=expr,...)
    ...
\end{verbatim}
pontosan ugyanazt jelenti, mint
\begin{verbatim}
function fuggveny(...,a,...)
    if( a==NIL )
        a:=expr
    end
    ...
\end{verbatim}
ahol \verb!expr! tetszőleges olyan kifejezés, amit az adott helyen ki lehet
értékelni.



\subsection{Változó számú paraméter}

\begin{verbatim}
function fuggveny1(a,b,c)
    ...
\end{verbatim}
Ez a függvény legfeljebb 3 darab paramétert tud átvenni. 
Ha kevesebbet kap, 
akkor NIL-re inicializálja a paraméter nélkül maradt argumentum változókat, 
ha többet kap, 
akkor a fölös paramétereket (esetünkben a 3 felettieket) kipucolja 
(mintha nem is lettek volna).

Olyan függvényre is szükség van, ami előre nem ismert számú paramétert vesz át.

\begin{verbatim}
function fuggveny1(a,b,c,*)
    ...
\end{verbatim}
Az {\em utolsó\/} argumentum változó helyén szereplő \verb!*! azt jelöli,
hogy a függvény akárhány (további) paramétert átvesz. 

Minden korábbi szabály érvényes, kivéve, hogy nincsenek ,,kipucolódó''
paraméterek.  Nem tudjuk, hogy a hívó küldött-e akár csak 1 darab paramétert.
Ha igen, akkor az első, második, harmadik paraméter behelyettesítődik az
\verb!a!,\verb!b!,\verb!c! argumentum változókba. Ha nem, akkor a rendszer
NIL-re inicializálja a paraméter nélkül maradtakat. 
A hívott program a kapott paraméterek számát 3-nál nagyobbegyenlőnek fogja látni.

Nem szükséges, hogy legyenek névvel ellátott argumentum változók.
\begin{verbatim}
function fuggveny1(*)
    ...
\end{verbatim}
Ez a változat egyszerűen minden paramétert átvesz.

No jó, ez a dolog egyik oldala, de hogyan férünk hozzá az ,,átvett'', 
de változóhoz esetleg nem rendelt értékekhez? Az alábbi kifejezésekben
\begin{verbatim}
    {*}                 // array az összes paraméterből
    fuggveny2(*)        // függvényhívás
    object:method(*)    // metódushívás
\end{verbatim}
a \verb!*! helyére behelyettesítődik a függvény összes paramétere. Elsőre 
talán furcsa, de a helyzet világos lesz a következő összehasonlítás után. 
\begin{verbatim}
function fuggveny1(a,b,c)
local array1:={a,b,c}
local array2:={*}
local array3:={*,1,*}   // {a,b,c,1,a,b,c}
    ...
    fuggveny2(a,b,c)
    fuggveny2(*)        // ugyanaz
\end{verbatim}
Hogyan készül az \verb!array1! tömb? A rendszer a stackre rakja az 
\verb!a!, \verb!b!, \verb!c! változók értékét, majd meghívja a veremgép
egy primitívjét, hogy a stack tetején levő három elemből készítsen tömböt.
Ez a primitív leveszi a három elemet, elkészíti a tömböt, és az eredményt
a stack tetején hagyja. Na, pontosan ugyanígy készül \verb!array2! is,
az egyetlen különbség, hogy az elemek konkrét felsorolása helyett
csak annyit mondtunk: az összes paraméter. Ezzel meg is vagyunk.
\verb!len({*})! megadja az összes paraméter számát (jelen esetben fix 3), 
\verb!{*}[n]! megadja az \verb!n!-edik paraméter értékét (a túlindexelés hiba).

Külön szólni kell a kódblokkokról.
\begin{verbatim}
local blk1:={|p1,*|...}
local blk2:={|*|fuggvény(*)}
\end{verbatim}
A \verb!||! jeleken belüli \verb!*! a függvényekhez hasonlóan azt jelenti, 
hogy a kérdéses kódblokk minden paramétert átvesz. Itt is rendelhetünk
argumentumváltozót az elől álló paraméterekhez. A kódblokk kifejezéslistájában
szereplő (tehát a \verb!||! jeleken kívüli) \verb!*! az összes kódblokk paraméter
felsorolását jelenti.


A \verb!*! (nem kifejezés, csak) jelölés
minden olyan kifejezésben működik, ahol értelme van a paraméterek
felsorolásának. Nézzünk néhány fontosabb esetet.
\begin{verbatim}
function fuggveny1(*)
    return fuggveny2(*)
\end{verbatim}
Itt \verb!fuggveny1! továbbítja a hívást \verb!fuggveny2!-nek, 
anélkül, hogy bármit tudna annak paraméterezéséről.
A \verb!*! felsorolás függvény- és metódushívásban a referencia változók
referencia tulajdonságát megtartja, ezzel szemben a \verb!{*}! kifejezésben 
a referenciák derefelődnek.

\begin{verbatim}
    {|*|fuggveny(*)}
\end{verbatim}
A fenti kódblokk meghív egy függvényt továbbítva neki a blokk összes paraméterét.



A paraméterek felsorolásának vannak további esetei:
\begin{verbatim}
    *[x1..x2]  // paraméterek x1-től x2-ig
    *[x1..]    // paraméterek x1-től (ameddig van)
    *[..x2]    // paraméterek x2-ig (1-től)
    *[..]      // összes paraméter (*)
\end{verbatim}
A szintaktika a string szeletekhez hasonló. 
A túllógó indexek módosulnak a tényleges méretekhez alkalmazkodva.
Magyarázat helyett néhány példa:
\begin{verbatim}
function fuggveny1(a,b,c)
local array1:={a,b,c}
local array2:={*[..]}    // ua. mint array1 vagy {*}
local array3:={b,c}
local array4:={*[2..3]}  // ua. mint array3
\end{verbatim}
A függvényhívás továbbítás (esetleg kódblokkon keresztül) fontos szerepet
kap a metódushívások implementációjában.



\subsection{Névterek}

Korszerű programnyelv nem nélkülözheti a névtereket.
A forrásmodulok legelején állhat az opcionális 
\verb!namespace! utasítás, például:
\begin{verbatim}
namespace aa.bb.cc
\end{verbatim}
aminek hatására a modulban definiált összes függvény
az \verb!aa.bb.cc! (többszintű) névtérbe kerül.
A \verb!namespace! utasítás nélkül, egyedileg is névtérbe 
helyezhetünk függvénydefiníciókat a következő módon:
\begin{verbatim}
function  aa.bb.cc.f()
    ...
\end{verbatim}
A kétféle (globális és egyedi) minősítés (névtérbe helyezés) 
egyszerre is jelen lehet, ebben az esetben a hatásuk összegződik.
A modulon belül definiált függvények meghívásakor
nincs szükség teljes minősítésre.

A modul függvényeire (pl. \verb!f!-re) kívülről a {\em minősített\/} névvel,
esetünkben az \verb!aa.bb.cc.f()! formával hivatkozhatunk.
A kívülről történő hivatkozások megkönnyítésére szolgál a \verb!using! utasítás.
A \verb!using! utasítások a modul elején, közvetlenül az
esetleges \verb!namespace! után állhatnak. A \verb!using! olyan 
rövidítést vezet be, amivel elkerülhető a teljesen minősített 
függvénynevek túl sokszori kiírása.

\begin{verbatim}
// alternatív using-ok               így hívjuk meg f-et

   using aa.bb.cc=alias           // alias.f()
   using aa.bb=x                  // x.cc.f()
   using aa.bb.cc  f g            // f(), g()
   using aa.bb  cc.f              // cc.f()
\end{verbatim}

A globális (gyökér) névteret (kezdő) pont jelöli. 
Ha pl. a \verb!using aa.bb.cc  f! utasítás után 
\verb!aa.bb.cc.f! helyett a globális névtérben definiált 
\verb!f!-et akarjuk meghívni, akkor ezt kell irni: \verb!.f()!.

A CCC névterek közvetlenül C++ névterekre vannak leképezve.


\subsection{Postfix függvényhívás}

Nézzük ezt a kifejezést:
\begin{verbatim}
    padl(alltrim(str(round(x,2))),10,"0")
\end{verbatim}
\begin{itemize}
\item
    Az \verb!x! változóban egy szám van,
\item
    azt kerekítjük 2 tizedes jegyre,
\item
    a számértéket karakter stringre konvertáljuk,
\item
    levágjuk az elől/hátul esetleg  rajta levő szóközöket,
\item 
    balról kiegészítjük "0" karakterekkel 10 szélességűre.            
\end{itemize}
Rengetegszer kell hasonló kifejezésekkel küszködni. 
Az a baj, hogy a sorozatosan egymásba ágyazott függvényhívásokat 
középről (az \verb!x! változóból) kifelé haladva, hol balról, hol jobbról kell írni.
Egy ilyet visszafejteni végképp fárasztó.

Adódik a postfix függvényhívás ötlete. Értelmezzük a \verb!::! (dupla kettőspont)
operátort a következőképp:
\begin{verbatim}
    x::fuggveny1()  // ua. mint fuggveny1(x) 
\end{verbatim}
A postfix jelző arra utal, hogy a függvénynevet nem előre, hanem hátra, 
az első argumentum után írjuk. 

Nyilvánvaló, hogy a nyelv ,,tudása'' nem változik a postfix függvényhívás
bevezetésével, csak a programírás kényelme növekszik. A hagyományos (prefix)
és a postfix függvényhívásból pontosan ugyanaz a kód generálódik.

Hogy igazán használható legyen, még egy-két általánosított formára szükség van.
\begin{verbatim}
    x::fuggveny1        // üres zárójelpár elhagyva
    (x+y)::fuggveny1    // ua. mint fuggveny1(x+y)  
    x::fuggveny1(y,z)   // ua. mint fuggveny1(x,y,z) 
\end{verbatim}
Megállapodunk abban, hogy a \verb!::! oprátor precedenciája magas, 
ugyanolyan, mint a metódushívás operátoré.

Az előző kifejezésünket most így írhatjuk.
\begin{verbatim}
    x::round(2)::str::alltrim::padl(10,"0")
\end{verbatim}

Itt akár meg is állhatnánk, ha nem ismernénk a \verb!+=!, \verb!-=! 
(hozzáadó, stb.) típusú operátorokat. Ezek mintájára kézenfekvő
bevezetni a \verb!::=! operátort a következő értelmezéssel:
\begin{verbatim}
    x::=fuggveny1()     // ua. mint x:=fuggveny1(x) 
\end{verbatim}
az összes variációjával együtt. Megállapodunk abban, hogy a \verb!::=!
operátor precedenciája alacsony, ugyanolyan, mint a \verb!:=!, \verb!+=! 
(értékadás, stb.) operátoroké.
 
Ha egyszer van egy alacsony precedenciájú értékadó operátorunk,
felvetődik a kérdés,  van-e értelme zárójelezni a jobboldalon álló kifejezést.
Itt egy példa, ami mutatja, hogy van:

\begin{verbatim}
    ? x:="1"                        // kiírja: 1 (karakter)
    ? x::=(val()+1)::str::alltrim   // kiírja: 2 (karakter)
    ? x::=(val()+1)::str::alltrim   // kiírja: 3 (karakter)
    ? x::=val+1                     // kiírja: 4 (szám)
\end{verbatim}

Amikor a függvényhívás operátort (\verb!::=! vagy \verb!::!) 
nem egy egyszerű függvénynév követi, hanem egy zárójeles kifejezés,
azt még mindig értelmezhetjük függvényhívásnak, ha a zárójeles kifejezés
nyelvtani elemzőfájának bal szélén függvényhívás van. A gyakorlatban
ritkán fordul elő ez az eset, inkább csak azért foglalkozunk vele, 
hogy ne maradjon elvarratlan szál a nyelvtanban.

A nyelvtani elemző onnan ismer fel egy függvényhívást, hogy
\begin{itemize}
 \item egy szimbólumot zárójelpár között felsorolt (esetleg üres) 
       kifejezéslista követ (ez a hagyományos eset), vagy
 \item
      függvényhívás operátort egy szimbólum követ (postfix eset).
\end{itemize}
Amikor a függvényhívás operátort nem közvetlenül követi a függvénynév
(mint ahogy a példában egy balzárójel ékelődik \verb!::=! és \verb!val! közé),
akkor az üres zárójelpárt nem lehet elhagyni.

