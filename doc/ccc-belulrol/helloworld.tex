

\section{Hello World}

Készítünk egy directoryt, mondjuk \verb!$HOME/temp/hello!, és abban egy 
\verb!hello.prg! nevű fájlt.

\begin{verbatim}
function main()
    ? "Hello World"
\end{verbatim}

Kiadjuk az alábbi parancsot:

\begin{verbatim}
bapp_unix_.b
\end{verbatim}

amire létrejön (egyebek mellett) a \verb!hello.exe! végrehajtható fájl. 
Ha ezt elindítjuk, a terminálban megjelenik a "Hello World" szöveg.

Nagyszerű!


De ne elégedjünk meg ennyivel, nézzünk szét a fájlok között.
Lett egy új directory, a \verb!ppo!. Ide került a \verb!hello.ppo! (PreProcessed Output)
fájl, amit a \verb!prg2ppo.exe! makróprocesszor készített. Minden prg fájl átmegy
az előfeldolgozáson. A prg2ppo a C-ből ismerthez hasonló, de annál
bonyolultabb és nagyobb tudású előfeldolgozó. Az eredeti Clipperben szintén megvan
ez az előfeldolgozás. A mi előfeldolgozónk - amennyire csak lehetséges - kompatibilis
az eredetivel.

A \verb!ppo! directoryban találjuk még a \verb!hello.cpp! fájlt. Ezt a ppo-ból 
készíti a \verb!ppo2cpp.exe! program, a tulajdonképpeni Clipper -> C++ 
fordító. Később tanulmányozni fogjuk a generált cpp kódot.

A szintén újonnan létrejött \verb!objlin! directoryba került a C fordítás eredménye,
a \verb!hello.obj! fájl. Ha a projektben könyvtárak (lib, so) készülnének, azok is
itt, az \verb!objlin! directoryban jönnének létre.

Próbáljuk ki a következőt. Újra kiadjuk a \verb!bapp_unix_.b!
parancsot. Megjelenik a ,,CCC Program Builder 1.2.25 Copyright (C) ComFirm Bt.''
felirat, de egyébként nem történik semmi. Módosítsuk a \verb!hello.prg! fájlt,
és próbálkozzunk újra. Látjuk, hogy a program most újra lefordul. Akkor is 
beindul a fordítás, ha letöröljük az \verb!objlin! directoryt, vagy csak egyik-másik
obj fájlt. Ha az exe-t töröljük le, akkor viszont beindul a linkelés.

Amire ki akarok lyukadni: A programkészítés elég komplex művelet.
A prg fájlokat át kell hajtani az előfeldolgozáson, a C fordításon, végül össze kell
őket linkelni más objectekkel és különféle statikus és dinamikus könyvtárakkal.
A CCC fejlesztő környezet ehhez igyekszik segítséget adni. A programozónak nem
kell leírnia, hogy melyik forrásfájlon milyen műveleteket akar végrehajtani.
Nincsenek make fájlok.

Projektben gondolkodunk. Egy directoryba beömlesztjük a projekthez tartozó
forrásfájlokat. A \verb!build.exe! program átvizsgálja a forrásokat.
A kiterjesztésekből eleve tudja, hogy melyikkel milyen fordítási
műveletet lehet végezni. A fájlidőkből megállapítja, hogy mely fordítási
műveletek időszerűek. És ha már ilyen okos, akkor el is végzi a szükséges
műveleteket. 

A \verb!bapp_unix_.b! script a \verb!build.exe!-t indítja olyan 
paraméterezéssel, hogy az összes forrás lefordításával konzolos 
programot készítsen. A build széleskörűen paraméterezhető.
Meg lehet mondani neki, hogy a working directory helyett honnan vegye a forrásokat,
honnan vegye az include-okat, honnan vegye a statikus és dinamikus libeket,
ha van statikus és dinamikus lib is, akkor melyiket preferálja. Csináljon-e az 
objectekből libeket, ha készül exe, azt hova tegye. 

Akárhogy is, a build specialitása, 
hogy a forrásokat nem fájlonként, hanem directorynként
kell megadni neki. Ebből persze az következik, hogy a munkadirectory(k)ban nem
lehetnek más forrásfájlok, csak amik ténylegesen részei a projektnek. De ez nem
baj. A nem odavaló fájlokat egyszerűen félretesszük pl. egy \verb!nemkell! nevű
subdirectoryba, így legalább tiszta marad a kép. Az is jó, hogy a fordítás során
keletkező rengeteg részeredmény subdirectorykba kerül, így a sok szemét nem
borítja el a fájlokat, amikkel ténylegesen dolgozunk.

További lényeges sajátossága a build (CCC projekt manager) programnak,
hogy nem csak prg, hanem cpp (sőt Lemon és Flex) forrásokat is automatikusan
áthajt a fordítási műveleteken. Tehát, ha egy directoryba összegyűjtünk egy rakás
Clipper, C++, Lemon, Flex forrást, azokból egy mozdulattal tudunk libeket,
exeket fordítani/linkelni. Ennek filozófiai jelentősége van. Mutatja ugyanis,
hogy a CCC nem szakadt el a C/C++ gyökerektől, hanem a Clipper és a C 
projektszinten keverhető.

Végül, szokássá vált, hogy a build-et indító parancsot (script nevet) beírjuk 
egy \verb!m! nevű fájlba (scriptbe). A CCC programkészítés így az egybetűs 
\verb!m! (make) parancsra indul.








