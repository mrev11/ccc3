
\section{Telepítés}\label{INSTALL}

A CCC többféle rendszeren fut: 
Windowson MinGW vagy Microsoft-C fordítóval.
Különféle Linuxokon, BSD-ken, Solarison GCC fordítóval.
Mi most az Ubuntu (Debian) Linuxon történő telepítést vesszük át.
Más platformokon (Windowson is) analóg módon kell eljárni.

Először  installáljuk a CCC-hez szükséges infrastruktúrát
az alábbi scripttel:

\begin{verbatim}
#!/bin/bash

# Egy szokásos Ubuntu telepítés után
# ezeket a csomagokat kell felrakni,
# hogy a CCC2/CCC3 leforduljon.
# + a jt-hez kell a Sun Java.
# + az SQL2-höz kell a PostgreSQL.

sudo apt-get install \
    g++ \
    libncurses5-dev \
    libncursesw5-dev \
    manpages-dev \
    glibc-doc \
    libx11-dev \
    libxft-dev \
    libssl-dev \
    libpcre3-dev \
    libreadline6-dev \
    git-core
#    libgtk2.0-dev \
#    libpangox-1.0-dev \
\end{verbatim}

Végrehajtjuk az alábbi parancsot:
\begin{verbatim}
git clone  git://comfirm.hu/ccc3.git
\end{verbatim}
Ehhez internet kapcsolat kell. Ha a parancsot a home-unkban adtuk ki, 
akkor létrejön a \verb!$HOME/ccc3! directory, benne a CCC forrással.

A \verb!.bashrc! fájlunkba beírjuk az alábbiakat:

\begin{verbatim}
export CCCVER=3
export CCCDIR=$HOME/ccc3
export CCCBIN=lin
export CCCUNAME=linux
export CCCTERM_CONNECT="$CCCDIR/usr/bin/$CCCUNAME/terminal-xft.exe"
export CCCTERM_INHERIT=no
export CCCTERM_SIZE=80x40
export CCCTERM_XFTFONTSPEC=Monospace-11
export OREF_SIZE=50000

PATH=$CCCDIR/usr/bin/$CCCUNAME:$PATH
PATH=.:$PATH  #fontos
LD_LIBRARY_PATH=$CCCDIR/usr/lib/$CCCBIN:$LD_LIBRARY_PATH
export PATH 
export LD_LIBRARY_PATH
export LANG=en_GB.UTF-8  #valamilyen UTF-8-as locale!
\end{verbatim}

Egyéb lehetséges variációk:

FreeBSD
\begin{verbatim}
...
export CCCBIN=fre
export CCCUNAME=freebsd
...
\end{verbatim}

NetBSD
\begin{verbatim}
...
export CCCBIN=net
export CCCUNAME=netbsd
...
\end{verbatim}

BSD-ken telepíteni kell a bash-t, és átlinkelni \verb!/bin/bash!-be (hogy a
\verb@#!/bin/bash@ kezdetű scriptek fussanak). FreeBSD-n gondoskodni kell róla,
hogy legyen használható \verb!malloc.h!.


Windows, MinGW
\begin{verbatim}
...
set CCCBIN=mng
set CCCUNAME=windows
...
\end{verbatim}

Windows, Microsoft-C
\begin{verbatim}
...
set CCCBIN=msc
set CCCUNAME=windows
...
\end{verbatim}

A Debiantól különböző rendszereken természetesen másképp kell megszerezni
a CCC függőségeit (nincs apt-get), de aki ilyeneket használ, 
az legyen tájékozott a kérdésben.

Ha most elindítunk egy terminált, abban a fent beállított környezet lesz érvényben
(ellenőrizzük). Bemegyünk a CCCDIR-be, és elindítjuk az \verb!i.b! scriptet. 
Ez lefordítja az egész CCC-t (kivéve a Jáva terminált, amihez Jáva kell, 
és az SQL2 könyvtárat, amihez Postgres kell.) 
A CCC fordításának minden platformon warning mentesen kell futnia.

Ha a fordítás alapjaiban nem megy, akkor ellenőrizzük a környezet beállítását.

Ha egyes komponensek nem fordulnak, annak oka lehet, hogy hiányzik valami
függőség. Ilyenkor meg kell keresni, hogy konkrétan mi hiányzik, pótlólag telepíteni,
majd újraindítani az \verb!i.b! scriptet.

Időről időre mégis megjelennek warningok, ez amiatt van, hogy a C++ fordító 
ellenőrzéseit szigorítják. A CCC karbantartása során ezek rendszeresen javításra 
kerülnek.


% A karakteres fullscreen programoknak szüksége 
% van egy fix szélességű, Unicode kódkiosztású fontra.
% A terminálprogram a \verb!CCCTERM_FONTSPEC! környezeti változóban 
% megadott fontot használja. Példák ennek beállítására:
%
% \begin{verbatim}
%   export CCCTERM_FONTSPEC=-misc-console-medium-r-normal--16-160-72-72-c-80-iso10646-1
%   export CCCTERM_FONTSPEC=-misc-fixed-medium-r-normal--15-140-75-75-c-90-iso10646-1
% \end{verbatim}
%
% Alapesetben (ha \verb!CCCTERM_FONTSPEC! nincs megadva) a terminál a
%
% \begin{verbatim}
%   -misc-console-medium-r-normal--16-160-72-72-c-80-iso10646-1
% \end{verbatim}
%
% specifikációjú fontot tölti be. 
% Ha nincs megfelelő font, a terminál kilép, minek következtében 
% a terminálos program is futásképtelen, ilyenkor a hibaüzenet: \verb!Cannot load font!.
%
% A mellékelt \verb!console8x16.pcf.gz! fájl tartalmazza a default fontot. Telepítése:
% \begin{itemize}
% \item A \verb!console8x16.pcf.gz! fájlt bemásoljuk \verb!/usr/share/fonts/X11/misc!-be,
% \item ugyanitt rootként lefuttatjuk a \verb!mkfontdir! programot,
% \item majd újraindítjuk az X-et.
% \end{itemize}


