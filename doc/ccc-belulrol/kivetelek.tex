
\section{Kivételkezelés}

\subsection{Mikor kapunk el  kivételt?}

Képzeljünk el egy tranzakció végrehajtó programot. 
Tegyük fel, hogy a tranzakció viszonylag bonyolult, 
sok külső feltételtől függ, végrehajtható-e.
Például egy betétlekötésnél kell legyen leköthető pénz,
kell legyen kamatláb, stb..
A program elkezdi a végrehajtást, ám egyszercsak kiderül,
hogy egy feltétel nem teljesül. Hiába minden erőlködés, 
nem lehet továbbmenni.

Mit csináljon a program? 

Egy lehetőség, hogy a függvény, ami észleli a hibát kiírja 
(csak a példa kedvéért) ,,nincs pénz'', és kilép. 
\begin{verbatim}
    if( nincs_penz )
        ? "nincs pénz"
        quit
    end
\end{verbatim}
A legrosszabb. Tegyük fel, hogy egy ilyen programot kell javítanunk. 
Tudni akarjuk, hol nincs pénz, rákeresünk tehát a hiba szövegére,
és azt találjuk, a programozó következetes volt, a szöveg 10 helyen fordul 
elő a kódban. Rémálom.

Másik lehetőség, hogy a hibát észlelő függvény hibakóddal tér vissza. 
Ezzel sajnos nincs elintézve a dolog, mert bonyolult tranzakcióról lévén szó,
a kérdéses függvény akár 10-20 függvényhívási szint mélységben lehet. Tehát
a programot úgy kell megírni, hogy a hívó mindenhol felkészül a
hibakód ,,feljebb adására''. Az ilyen program zavarossá válik,
eluralkodik benne a hibakezelés, megnehezül a karbantartás.

A kényes szituáció kezelésére szolgál (egy harmadik lehetőség) a kivétel dobás. 
A CCC-ben a
\begin{verbatim}
    break(x)
\end{verbatim}
utasítással dobunk kivételt. Az \verb!x! változó típusa bármi lehet.


Más nyelveknél (pl. Jáva) ezen a helyen a kivétel elkapásával folytatódna a leírás,
én szándékosan más sorrendet választok. Azt szeretném ezzel hangsúlyozni, hogy
\begin{itemize}
 \item a kivételnek nem az a célja és értelme, hogy elkapjuk, hanem
 \item a kivétel egyszerűen azt fejezi ki, hogy a program nem futhat tovább.
\end{itemize}

Mire számíthatunk, ha nem kapunk el egy kivételt? Nézzünk egy példaprogramot:
\begin{verbatim}
function main()
    proba1()

function proba1()    
local v:="x"
    proba2()

function proba2()    
local v:="y"
    proba3()

function proba3()    
local v:="z"
    break("HOPP")
\end{verbatim}

A \verb!break("HOPP")! hatására a program ,,elszáll'' a következő hibaüzenettel:
\begin{verbatim}
default error block evaluated
errorclass: C HOPP
  called from deferror(215)
  called from _blk__2(0)
  called from proba3(14)
  called from proba2(10)
  called from proba1(6)
  called from main(2)
-----------------------------------------------------------
 Variable Stack
-----------------------------------------------------------
0: BINARY length=24 oref=b7949048 "^^^^^^^^^^^^^^^^^^^^^^^^"
1: BLOCK oref=NULL
***** function main
***** function proba1
0: STRING length=1 oref=b7949008 "x"
***** function proba2
1: STRING length=1 oref=b7949018 "y"
***** function proba3
2: STRING length=1 oref=b7949028 "z"
3: STRING length=4 oref=b7949038 "HOPP"
***** function _blk__2
4: BLOCK oref=NULL
5: STRING length=4 oref=b7949038 "HOPP"
***** function deferror
6: STRING length=4 oref=b7949038 "HOPP"
7: FLAG .T.
8: NIL
9: STRING length=0 oref=b7949078 ""
10: ARRAY length=0 oref=b7949088
11: NIL
-----------------------------------------------------------
\end{verbatim}
%A hibakezelés hasonló a Pythonhoz. 
Kiíródik a callstack, minden függvényhívási szinten feltüntetve 
az aktuális programsor száma.
Kiíródik az összes static változó (main fölött), 
majd a függvényhívási szintek megjelölésével az egész local stack.
A gyakorlott programozó számára ez az infó az esetek többségében elegendő
egy hiba azonosítására.

Térjünk vissza a tranzakcióhoz, próbálkozzunk ilyen hibakezeléssel:
\begin{verbatim}
    if( nincs_penz )
        break("nincs pénz") //quit helyett
    end
\end{verbatim}
Máris beljebb vagyunk.
Még mindig nem foglalkozunk a hiba elkapásával.
Azért vagyunk lényegesen beljebb, mert bár a program most sem fut tovább,
nem veszítjük el a hibára vonatkozó infót, mint a \verb!quit! után.
A programozó a hibalista elemzéséből debugolás nélkül is látja, hol
nem volt pénz.

Tegyük fel, hogy a tranzakció a 4. függvényhívási szinten indul,
a \verb!break("nincs pénz")! kivétel dobása a 10. függvényhívási szintről jön,
mi pedig éppen egy 7. szinten levő függvény kódjában kotorászunk.
Meghívni készülünk egy függvényt, amiről tudjuk, hogy 
\verb!break("nincs pénz")! kivételt dobhat. Mit tegyünk, kapjuk-e el a kivételt?

A legnagyobb baklövés:
\begin{verbatim}
    begin
        ...
        fuggveny8()    // break("nincs pénz") jöhet belőle
        ...
    recover hiba <C>   // elkapja
        //elkaptam a kivételt,
        //mert lelkiismeretes munkaerő vagyok,
        //és úgy tanították, hogy a hibákat le kell kezelni,
        //csak azt tudnám, most mi a fenét kezdjek vele?
        //ez a legrosszabb        
        ? hiba //kiírja: nincs pénz
        quit
    end
\end{verbatim}
A program idejekorán elkap egy kivételt, 
amivel nem tud mit kezdeni, és ezzel megsemmisíti a hiba eredeti környezetét.
Sajnos a Jáva nyelv erőteljesen tereli a programozót ebbe az irányba,
ráadásul a Jáva tankönyvek szinte kivétel nélkül a fenti értelemben
káros hibakezelést tanítják.

A helyes hozzáállás:
\begin{itemize}
 \item 
    Észlelésekor nem a hiba elnyomására kell törekedni, 
    hanem kivételt kell dobni (a program nem folytatódhat).
 \item
    Csak olyan kivételt szabad elkapni, 
        amivel direkt kezdeni akarunk valamit.
    Kevés olyan hely van a programban, 
        ahol a hiba orvosolható, következésképp 
            az esetek többségében az a jó, ha a kivételt nem kapjuk el.
    Ha a hibát idejekorán elkapjuk, azzal megakadályozzuk, 
        hogy a kivétel eljusson a hibakezelésre ténylegesen felkészített kódhoz.
    Mindezek miatt az alapállás, 
        hogy nem kapkodunk kivételek után,
          hanem engedjük a hibát eljutni a felsőbb szintekre.
 \item
   A {\em tranzakció logikájának megfelelő helyen\/} (ha egyáltalán)
   felkészülünk a kivétel kezelésére.
\end{itemize}
Ha egy kivétel sehol sincs elkapva, az nem feltétlenül baj. 
A program kilép, és pontos infót ad a kilépés okáról. 
Az esetek többségében éppen erre van szükség.
Mindez azonban nem szabály, inkább {\em szempont}.






A tranzakció logikája mellett még egy gyakorlati szempontot
tudok ajánlani kivétel elkapás (elkapjam-nekapjam) kérdésében. 
Ne kapjunk el {\em programhibát}, vagyis olyan kivételt, 
amit inkább a program kijavításával kell megszüntetni.

A CCC kivételkezelése nagyon hasonlít a Jávához. CCC-ben azonban nincs
külön throwable osztály, bármit lehet dobni, és bármit el is lehet kapni.
Logikus, hogy így van, hiszen \verb!break()! közönséges függvény, akármilyen
paraméterezéssel meg lehet hívni. Az meg végképp nem volna szép,
ha valamit dobni lehet, de elkapni nem.

Kialakult az a konvenció, hogy a programozási 
hibákból \verb!error! osztályú kivétel objektum keletkezik, amiket nem kapunk el. 
Ahol felmerülhet, hogy hasznos a kivétel elkapása, ott az \verb!apperror! osztályt
vagy annak leszármazottait dobjuk. 
Tehát valahogy így készül a kivétel:
\begin{verbatim}
    if( nincs_penz )
        error:=apperrorNew()
        error:description:="nincs pénz"
        break(error)
    end
\end{verbatim}
Ha differenciáltabb hibakezelésre van szükség,
akkor az alkalmazás definiál magának speciális \verb!apperror! 
leszármazottakat.


Az alábbi vázlat mutatja, 
hogyan gondolom a kivételkezelést összehangolni a tranzakció logikájával:
\begin{verbatim}
    while( van_meg )
        begin               
            tranzakcio()
            //sikerült
            commit()
        recover error <apperror>
            //nem sikerült
            error:list
            rollback()
        end
    end
\end{verbatim}

\subsection{A begin...recover utasítás}

A következő példán tanulmányozzuk a \verb!begin...recover! utasítás
technikai részleteit. Tanulságos a program lefordítása és próbálgatása 
különféle variációkban.
\begin{verbatim}
function main(x)
local e
    if( x==NIL )
        //NIL
    elseif( isdigit(x) )
        x:=val(x)
    elseif( x=='e'  )
        x:=errorNew()
        x:description:="próba"
    elseif( x=='a'  )
        x:=apperrorNew()
        x:description:="szerencse"
    else
        //karakter
    end
        
    begin    
        fuggveny1(x)
    recover e <N>
        ? "szám",e
    recover e <error>
        ? "error",e:description
    finally
        ? "finally-main"
    end
    ? "OK"
    ?

function fuggveny1(x)    
local e
    begin
        fuggveny2(x)    
    recover e <apperror>
        ? "apperror",e:description
    finally
        ? "finally-fuggveny1"
    end

function fuggveny2(x)    
    //errorblock({||qout("ERROR")}) //kikapcsolja a hibakezelést
    break(x)
    ? "break után"
\end{verbatim}

A program az \verb!x! argumentumtól függően különféle típusú kivételt dob. 
A \verb!break(x)!-ben \verb!x! típusa bármi lehet, mi most csak néhány variációt 
nézünk a példa kedvéért (NIL, szám, error, apperror, karakter).

A \verb!break(x)! megszakítja a végrehajtás normál sorrendjét,
a rendszer elkezdi keresni a kivételt elkapó \verb!recover! utasítást.
\begin{verbatim}
    recover e [type_expr]
\end{verbatim}
A \verb!recover! kulcsszót egy változó majd egy opcionális típuskifejezés követi.
A \verb!recover! akkor kapja el a \verb!break!-kel dobott kivételt, ha a 
\verb!break! paraméterének típusa megfelel a \verb!recover!-beli típuskifejezésnek.
Mi állhat a típuskifejezés helyén?
\begin{itemize}
\item 
    Üres. Az ilyen \verb!recover! mindent elkap.
\item 
    Tetszőleges kifejezés, ami az adott helyen kiértékelhető.
    A kifejezésnek csak a típusa számít, az értéke
    nem hozzáférhető (de mellékhatásai lehetnek).
    A kifejezés már a \verb!begin...recover! elején kiértékelődik.
    A különböző \verb!recover! kifejezések kiértékelésénék sorrendje
    nincs meghatározva. A \verb!recover! akkor kapja el a kivételt,
    ha a típusok megegyeznek.
\item
    \verb!<U>!, 
    \verb!<L>!,
    \verb!<N>!,
    \verb!<D>!,
    \verb!<P>!,
    \verb!<C>!,
    \verb!<X>!,
    \verb!<A>!,
    \verb!<B>!,
    \verb!<O>!, azaz a különféle típusok kódja hegyes zárójelek között.
    Például: \verb!<C>! jelöli a karakter típust,  
    a \verb!<C>! típusú \verb!recover! elkapja a karakter kivételeket.
\item
    \verb!<osztaly>!, vagyis osztálynév hegyes zárójelek között. 
    Az ilyen \verb!recover! azokat az objektum típusú kivételeket kapja el, 
    amiknek az osztálya megegyezik a \verb!recover! osztályával, vagy 
    annak leszármazottja.
\end{itemize}

A \verb!recover! ,,elkapja'' a kivételt: Azt jelenti, 
hogy visszaállítódik a stack,
a kivétel (\verb!break! paramétere) behelyettesítődik a \verb!recover!
változóba (a példában \verb!e!-be), és a végrehajtás a \verb!recover! utáni
sorral folytatódik. Végrehajtódik a \verb!recover! ág, ami a következő 
\verb!recover!-ig, vagy az opcionális \verb!finally!-ig, 
vagy a \verb!begin...recover!-t lezáró \verb!end!-ig tart.
Ha van \verb!finally! ág, akkor az is végrehajtódik.
A végrehajtás ezután az \verb!end!-et követő soron folytatódik.


Hol és milyen sorrendben keresi a rendszer a kivételt elkapó
\verb!recovert!? Először is teszünk egy észrevételt: 
A \verb!begin...recover! utasítások egymásba lehetnek ágyazva. 
A keresés a \verb!break!-et tartalmazó legbelső \verb!begin...recover! 
utasításban kezdődik. (A példában a \verb!fuggveny1!-beli a legbelső.)
Itt a rendszer felülről lefelé haladva keres a \verb!recover!-ek között.
Az első illeszkedő típus nyer.

Ha nem talál, akkor -- mint héjakon -- belülről kifelé haladva keres a
többi \verb!begin...recover!-ben. Például a szám típusú kivételt 
a \verb!main!-beli első \verb!recover! fogja elkapni.

Miután megvan a kivételt elkapó \verb!recover!,
a rendszer megnézi, hogy mely \verb!begin...recover! utasításokon (héjakon)
kellett átvágnia magát, és végrehajtja ezek esetleges \verb!finally! ágait. 
A sorrend természetesen belülről kifelé. Ezután a végrehajtás a nyertes 
\verb!recover! ágra kerül, a többit már ismerjük.

A \verb!begin...recover! specialitása a többi vezérlési struktúrához képest,
hogy átlépi a függvényhatárokat. Éppen ez kell a tranzakció-orientált hibakezeléshez.

De miért mondtuk, hogy csak miután megvan a \verb!recover!, azután hajtódnak 
végre az átlépett/elhagyott \verb!begin...recover!-ek \verb!finally! ágai?
Miért nem a keresés közben? Mert nem tudható előre, lesz-e egyáltalán megfelelő
\verb!recover!. Ha semmi sem kapja el a kivételt, akkor a \verb!begin...recover!-ek
mintha ott sem lennének, az eset a magában álló \verb!break!-hez hasonlít.
Ilyenkor sehova sem tevődik át a vezérlés (következésképp a \verb!finally! ágak
sem játszanak), hanem a hiba {\em eredeti környezetében\/} kiértékelődik
az \verb!errorblock!. Az eredményt már láttuk.

Hátra van még néhány speciális eset:

Ha a \verb!begin...recover!-t nem szakítja meg \verb!break!,
akkor a végrehajtás átugorja a \verb!recover! ágakat,
belemegy az esetleges \verb!finally!-ba,
majd az \verb!end! után folytatódik.

Lehetséges, hogy nincs egy \verb!recover! ág sem:
\begin{verbatim}
    begin
        ...
    finally
        ...
    end
\end{verbatim}
Az ilyen struktúra nem kap el semmit,
de végrehajtódik a \verb!finally! ág, ha nincs kivétel, 
vagy ha van, de azt elkapja egy külső \verb!begin...recover!.

Lehetséges, hogy nincs \verb!finally! sem:
\begin{verbatim}
    begin
        ...
    end
\end{verbatim}
Ez viszont egyenértékű azzal, mintha az \verb!end! előtt állna
egy mindent elkapó (üres) \verb!recover! ág.

A  \verb!begin...recover!-ből szabad kiugrani 
\verb!return!, \verb!loop!, \verb!exit! utasításokkal.
Ilyenkor végrehajtódnak az elhagyott \verb!begin...recover!-ek \verb!finally! ágai.


Kezdetleges formában már a régi Clippernek is volt \verb!begin...recover! 
utasítása. A CCC ezt kiterjesztette, és teljessé tette.


\subsection{Az errorblock}

Még egy eszköz áll a hibakezelés szolgálatában, 
ami  a régi Clipperben is megvolt, az errorblock.  
Volt ilyen mondásunk: Az el nem kapott kivétel hatására a program ,,elszáll''.
Mit jelent ez részleteiben?

A \verb!break()! próbál olyan \verb!recover!-t keresni, 
ami elkapja a kivételt, ám lehet, hogy nincs ilyen. 
Mi mást tehetne ilyenkor, valahogy be kell fejezni a programot,
kiértékeli hát az errorblockot. Az erroblock egy kódblokk,
ami direkt arra szolgál, hogy ez hajtódjon végre kezeletlen hiba esetén.

Már láttuk hogyan működik a beépített hibakezelő:
Kiírja a callstacket, varstacket, majd kilépteti a programot.
A hibakezeleő blokkot azonban az alkalmazás a saját igényei szerint 
lecserélheti, láncba fűzheti.
\begin{verbatim}
local defblk:=errorblock()
    errorblock({|x|naplo(x),eval(defblk,x)})
    ...
\end{verbatim}
Ez a program először lekérdezi a default hibakezelőt
az \verb!errorblock()! függvény paraméter nélküli hívásával,
majd ugyanezzel  beállít egy új hibakezelő kódblokkot,
amiben először naplózza a hibát, majd végrehajtja az eredeti hibakezelést.
Csak a példa kedvéért.

A \verb!begin...recover! és \verb!break! működését tanulmányozhatjuk
kikapcsolt hibakezelés mellett, ha a \verb!fuggveny2!-ben megszüntetjük
az \verb!errorblock()! hívás kikommentezését.





