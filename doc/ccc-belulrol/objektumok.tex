

\section{Objektumok}

%\tableofcontents

\subsection{Alapok}

Az alábbi példában majdnem minden együtt van.
\begin{verbatim}
class derived(base1,base2)

    attrib  a1
    attrib  a2

    method  m1              {|this,p1,p2|expr(this,p1,p2)}
    method  m2              {|*|derived.m2(*)}
    method  m3
    method  m4              {|this,*|this:a2:m(*[2..])}
    method  m5              :a2:m

    method  initialize

static function derived.m2(this)
    ...

static function derived.m3(this,a,b,c)
    ...

static function derived.initialize(this,a,b,c)
    this:(base1)initialize(a,b,c)
    this:(base2)initialize(a,b,c)
    ...
    return this  //kötelező visszaadni a this-t
\end{verbatim}

A \verb!class! kulcsszóval induló osztálydefiníciók a forrásban ugyanott állhatnak,
ahol a függvények. Az osztálydefiníció a következő \verb!class!-ig, 
\verb!function!-ig vagy a fájl végéig tart.

A példa egy \verb!derived! nevű osztályt definiál, 
ami a \verb!base1! és \verb!base2! osztályokból van származtatva.
Mindig meg kell adni legalább egy alaposztályt, amiből az új osztály
örököl, ha mást nem, akkor a mindig létező \verb!object! osztályt.


Az osztálydefiníció eredménye két függvény:
\verb!derivedClass()! és \verb!derivedNew()!,
általában az x nevű osztály esetén xClass és xNew.
A programban e két függvény képviseli az osztályt.

\begin{verbatim}
    classid:=derivedClass()
\end{verbatim}
\verb!derivedClass()! visszatérési értéke az osztályazonosító.
Maga az osztály a \verb!derivedClass()! függvény első hívásakor jön létre.
A \verb!base1! és \verb!base2! osztályoknak is megvan a maguk class 
függvénye, amik \verb!derivedClass()!-ból meghívódnak, és amik
szintén meghívják az ősosztályuk class függvényét, stb..

A \verb!derivedNew()! függvénnyel példányosítjuk az osztályt. 
\begin{verbatim}
    object:=derivedNew(p1,p2,p3)  //derived osztályú objektum
\end{verbatim}
A \verb!derivedNew()! előállít egy \verb!derived! osztályú objektumot, 
amihez szüksége van a \verb!classid!-re, meghívja tehát \verb!derivedClass()!-t.
Ha még nem létezett, akkor ennek hatására minden szükséges osztály létrejön 
az \verb!object! osztályig bezárólag.
A new függvény automatikusan végrehajtja az új objektum \verb!initialize! 
metódusát, továbbadva neki minden paraméterét.



A \verb!derived! osztály rendelkezik a
\verb!base1! és \verb!base2!-ből örökölt minden attribútummal és metódussal, 
plusz a \verb!derived! osztályban definiált \verb!a1!\ldots,\verb!m1!\ldots  
attribútumokkal és metódusokkal.

A objektum felhasználója így hivatkozhat az attribútumokra és metódusokra:
\begin{verbatim}
    object:a1           // attribútum kiolvasás
    object:a1:=x        // attribútum értékadás
    object:a1++         // értékadás minden variációban
    object:a1+=1
    object:m1(x,y)      // metódushívás
\end{verbatim}

%Visszatérve a \verb!class! utasításra. 
A \verb!class!-ban kizárólag attribútum és metódus definíciók állhatnak.
A sorrendjük lényegtelen.

Az attribútumoknak egyszerűen megadjuk a nevét az \verb!attrib! kulcsszó után. 

A \verb!method! utasítások ennél bonyolultabbak.
Alapesetben a metódus neve után egy kódblokk van,
ez a kódblokk jelenti a metódus implementációját.
A metódushívás a kódblokk kiértékelésével történik.
Nézzük részleteiben. 
A rendszer megállapítja, hogy \verb!object! osztálya \verb!derived!.
Az osztályok nyilvántartásából
előszedi a \verb!derived! osztály \verb!m1! metódusához rendelt
kódblokkot. Ezt kiértékeli úgy, hogy a blokk első (\verb!this!) paraméterébe
helyettesítődik maga az objektum, a \verb!p1!-be \verb!x!, \verb!p2!-be \verb!y!.
A metódus visszatérési értéke a kódblokk utolsó kifejezésének --
esetünkben \verb!expr(this,p1,p2)! -- értéke. 

A példában szereplő
\verb!this! nem kulcsszó, csak konvenció, akármilyen szimbólum megfelelne.
Szabály viszont, hogy mindig a kódblokk első paraméterébe kerül az 
objektum (this, self, ki hogyan szereti).


Aha, ezek szerint az \verb!m2! metódus implementációja a \verb!derived.m2! függvény.
A metódus kódblokkja minden paramétert továbbít a benne meghívott függvénynek,
és mint tudjuk, az első paraméter a \verb!this!.

Az \verb!m3! metódus sorában a kódblokk helyén semmi sincs. 
Ilyenkor a fordító úgy jár el, mint az előző esetben, odateszi (odaérti) az
\begin{verbatim}
    {|*|osztalyneve.metodusneve(*)}
\end{verbatim}
default kódblokkot, ami minden paraméter továbbadásával meghívja
az osztály nevével minősített, a metódus nevével egyező nevű függvényt.

Ezt a függvényt valahol implementálnunk kell. 
Ha a függvényt \verb!static!-nak definiáljuk, akkor az a modulon kívülről 
függvényhívással nem érhető el, csak az objektumon keresztül metódushívással. 
Ha a metódus implementációt (kényszerűségből) más modulba tesszük, akkor nem lehet 
\verb!static!, de legalább a névtér csökkenti a véletlen névütközés lehetőségét.

A függvénynek mindig van legalább egy paramétere, hiszen az első helyen megkapja 
a \verb!this!-t. Ezt ugyanúgy ki kell írni, mint minden más normális paramétert.
(Nincs olyan bosszantó kétértelműség, mint amit C++-ban a ki nem írt this okoz). 

Az \verb!m4! metódus kódblokkja kicsit bonyolultabb.
Feltételezzük, hogy az objektum \verb!a2! attribútuma egy beágyazott
objektumot tartalmaz, méghozzá olyat, aminek van \verb!m! nevű metódusa.
A kódblokk az \verb!m4! metódushívást (minden paraméter továbbadásával) 
továbbítja a beágyazott objektum \verb!m! metódusának.
Ha nem érthető, akkor az olvasónak újra át kell néznie 
a változó számú paraméterrel történő függvényhívást.

Az \verb!m5! megint új eset? Nem egészen. Ez csak az \verb!m4! kódblokk 
rövidített írásmódja. A metódushívás-továbbítás olvashatóbb formája.

Az objektumokat általában inicializáljuk.
Egy tipikus lehetőséget mutat be a példában szereplő \verb!initialize! metódus.
Mindez azonban nem szabály. Az alkalmazás konkrét körülményeiből adódik, 
hogy mikor milyen inicializálásra van szükség. A
\begin{verbatim}
    this:(base1)initialize(a,b,c)
\end{verbatim}
sor a (\verb!derived! osztályú) \verb!this! objektumra meghívja
a \verb!base1! osztályban definiált \verb!initialize! metódust.
Ez általában lehetséges, hiszen a leszármazás miatt \verb!this!
egyúttal \verb!base1! osztályú is. A speciális jelölést nevezzük 
metódus cast-nak.

Az inicializátornak kötelezően vissza kell adnia az objektumot (a \verb!this!-t).

Megjegyzendő, hogy minden objektumnak van \verb!initialize! metódusa.
Ha a saját osztályában nincs definiálva, akkor örökli valamelyik
felmenőjétől, ha máshonnan nem, akkor az \verb!object! osztálytól.
Az \verb!object! osztály inicializátora egyébként nem csinál semmit,
minthogy az \verb!object!-ben nincs definiálva egyetlen attribútum sem.


Álljunk meg ezen a ponton egy kis összegzésre. 
Azt szeretném hangsúlyozni, hogy az eddigiek nagyon egyszerűek,
mivel kevés szabály logikus alkalmazásából adódnak.
Hogyan definiálunk egy osztályt?
\begin{itemize}
\item 
    Megadjuk az osztály nevét a \verb!class! utasításban.
\item 
    Ugyanitt felsoroljuk az ősosztályok nevét.
\item 
    A \verb!attrib! utasításokkal felsoroljuk az attribútumokat.
\item 
    A \verb!method! utasításokkal felsoroljuk a metódusokat.
\item 
    Tipikus esetben a metódushoz nem mellékelünk kódblokkot,
    hanem a metódus implementációját egy \verb!osztalyneve.metodusneve()!
    függvénybe helyezzük.
\item
    A beágyazott objektumokat és a metódushívás-továbbítást
    a fordító egyszerűen és szemléletesen támogatja.
\item
    Néha tényleg annyira egyszerű a metódus, 
    hogy az egész implementáció belefér egy kódblokkba.
    Ilyenkor valósul meg az alapeset.
\end{itemize}
Nem ritka, hogy egy osztálydefiníció
az attribútumok és metódusok  puszta  felsorolásából áll.
Egyszerűbb már nem lehetne. 
Vannak azonban részletek, amik fölött eddig átsiklottunk,
ezeket vesszük sorra a következő pontokban.

\subsection{Öröklődési szabályok}

Az öröklődés többszörös, C++ terminológiát használva: public, virtual.
Ez azt jelenti, hogy az öröklés elől nem lehet eldugni az 
attribútum/metódusokat, azaz mindig minden öröklődik (public).
A metódusok közül mindig az objektum tényleges osztályának megfelelő hívódik 
meg (virtual). Nincsenek olyan nyakatekert szabályok, mint a C++-ban vagy Jávában.

A rendszer minden osztály részére létrehoz egy hashtáblát,
amiben az attribútum/metódus név a kulcs, az érték pedig
\begin{itemize}
 \item attribútum esetén egy tömbindex
       (ezen a helyen található az attribútum értéke az objektum 
       tartalmát hordozó tömbszerű memóriaobjektumban),
 \item metódus esetén egy kódblokk.
\end{itemize}
A rendszer a névhez (kulcshoz) rendelt érték típusából tudja,
hogy attribútumról vagy metódusról van-e szó.

A rendszer a kezdetben üres hashtáblát feltölti. 
Először balról jobbra haladva végigmegy az ősosztályokon,
és összegyűjti az ezekből örökölhető tagokat. Ha olyan névvel találkozik,
ami már korábban bekerült a hashbe, akkor azt kihagyja. Ebből adódnak
a következő szabályok:

Ha a \verb!derived! osztály \verb!base1!-ből és \verb!base2!-ből is
örökölhetne egy tagot, akkor \verb!base1!-ből örököl.

Ha \verb!base1!-ben és \verb!base2!-ben is van attribútum
ugyanazon a néven, akkor ezeknek egyetlen közös attribútum felel
meg \verb!derived!-ben.

Ezután a rendszer berakja a hashbe a \verb!derived!-ben definiált
attribútumokat/metódusokat, azaz a származtatott osztály új attribútumokkal
és metódusokkal bővül az ősosztályhoz képest.

Eközben is előfordul névegyezés, 
most azonban a rendszer mindig a felülírja a korábbi értéket. 
Ez megfelel annak az objektumorientált programozási eljárásnak,
miszerint a származtatott osztály felüldefiniálja az ősosztályoktól
örökölt metódusok egyikét-másikát.

Előfordulnak azonban olyan esetek is, amikor
\begin{itemize}
 \item 
    attribútum definiál felül attribútumot,
    hatástalan, az attribútum nem többszöröződik meg,
 \item
    attribútum definiál felül metódust,
    a metódus megszűnik, helyette új attribútum,
 \item
    metódus definiál felül attribútumot,
    attribútum megszűnik, helyette új metódus
    (az attribútumindexek újraszámolódnak).
\end{itemize}

Ezeknek az öröklődési szabályoknak az az előnyük, hogy egyszerűek.
Vannak nyelvek, amik bonyolult szabályrendszerrel lekezelik a többszörös 
öröklődés minden konfliktusát. Ezekkel rejtvényszerűen bonyolult
programokat lehet írni. Más nyelvek -- ettől visszariadva -- 
inkább egyszeres öröklődésre szorítkoznak.
Vannak szerzők, akik szerint már az egyszeres öröklődés is túl bonyolult,
amit nem tanácsos programozók kezébe adni (Trey Nash, C\# 2008).

A mi hozzáállásunk praktikus. Megmaradunk a többszörös öröklődésnél,
de nem bonyolítjuk túl a dolgokat. Ha a fenti szabályok nem megfelelőek
egy feladatban, akkor valószínűleg nem érdemes az öröklődéses
programozási modellt erőltetni. Ilyenkor célszerű lehet
beágyazott objektumokhoz és metódushívás-továbbításhoz folyamodni.

Végül mutatunk egy példát virtuális metódusvégrehajtásra.

\begin{verbatim}
************************************************************
function main()
    baseNew(12345):print        //kiírja: 12345
    derivedNew(12345):print     //kiírja: 12,345.00

************************************************************
class base(object)
    attrib  number
    method  initialize
    method  print
    method  format

static function base.initialize(this,x)
    this:number:=x
    return this                 //kötelező

static function base.print(this)
    ? this:format               //hogyan formattál?

static function base.format(this)
    return str(this:number)

************************************************************
class derived(base)
    method  initialize
    method  format              //felüldefiniálja

static function derived.initialize(this,x)
    this:(base)initialize(x)
    return this                 //kötelező

static function derived.format(this)
    return transform(this:number,"999,999,999.99")

************************************************************
\end{verbatim}
A példában mindkét kiírás a \verb!base!-ben definiált \verb!print!
metódussal történik. A nagy kérdés, hogyan történik a formattálás.
A \verb!base! osztályú objektum a \verb!base!-beli metódussal, 
a \verb!derived! osztályú objektum viszont a \verb!derived!-ben 
definiált metódussal formázódik. Ha a \verb!base! osztályt nem akarjuk
példányosítani, akkor a \verb!base.format! metódusra egyáltalán nincs szükség.





\subsection{Attribútum/metódus invariancia}

A fordító és a futtatórendszer szándékosan úgy van megírva,
hogy egy objektum felhasználójának ne kelljen feltétlenül tudni róla,
hogy metódushívással, attribútum kiolvasással, netán attribútum értékadással
van-e dolga.

Először is rögzítsük: Metódushíváskor ugyanolyan szabadságunk van 
a paraméterezésben (a paraméterek darabszámát és típusát illetően), mint 
a függvényhívásban. A hívó akárhány darab, akármilyen típusú paramétert küldhet. 
A hívott kód saját hatáskörben dönti el, hogy egy konkrét paraméteregyüttesre 
mit felel.


Metódushíváskor az üres zárójelpárt nem kell kiírni.
%Az előző példákban már láttunk ilyet.
\begin{verbatim}
    obj:meth            //nem kell kiírni a zárójelpárt
    obj:meth()          //ugyanaz
\end{verbatim}
Ha nincs zárójelpár, akkor formailag nincs különbség
a metódushívás és az attribútum kiértékelés között. A ,,rendszer''
belül azért tudja, miről van szó. Honnan? Onnan, hogy attrubútum esetén
egy index (szám), metódus esetén viszont egy kódblokk kerül elő az
osztály hashtáblájából.

A dolog meg is fordítható.
\begin{verbatim}
    obj:attr()          //odaírható a zárójelpár (érdektelen)
    obj:attr            //ugyanaz
\end{verbatim}
Tehát attribútumokat is elláthatunk üres zárójelpárral, mintha ott sem lenne.
Természetesen a háttérben az áll, hogy
a paraméter nélküli metódushívásnak 
és az attribútum kiolvasásnak
ugyanaz a kódja  a CCC veremgépén.

Célszerű az invarianciát teljessé tenni, kiterjeszteni
az attribútum értékadásra.
\begin{verbatim}
    obj:attr:=x
    obj:attr(x)         //ugyanaz

    obj:meth(x)
    obj:meth:=x         //ugyanaz
\end{verbatim}
Azért előnyös ez, mert így az objektum belső implementációjában
váltani lehet attribútum és metódus között anélkül, hogy a kliens kódban
mindenhol cserélgetni kellene a zárójeles paraméterlistákat és értékadásokat.

Az invarianciát kifejező legáltalánosabb forma:
\begin{verbatim}
    obj:meth(x,y,...):=z 
    obj:meth(x,y,...,z) //ugyanaz
\end{verbatim}



\subsection{Metódus cast}

A virtuális metódushívást bemutató példában láttuk, 
hogy alapesetben mindig az objektum tényleges osztályában definiált metódus
hívódik meg. Néha ehelyett valamelyik ősosztály metódusára volna szükség,
amit azonban ,,eltakar'' a származtatott osztályban definiált azonos nevű
metódus. Ilyen esetben folyamodunk a metódus-cast-hoz.
A leggyakoribb példa, amikor az inicializátor végrehajtja
az ősosztály inicializátorát.
\begin{verbatim}
static function derived.initialize(this,p1...)
    this:(base)initialize(p1...)    //a base osztály inicializátora
    ...
    return this
\end{verbatim}
A \verb!(base)initialize! jelölés mutatja, hogy a \verb!base!-ből
kell venni a metódust, nem pedig az aktuális osztályból. 
Ha a \verb!(base)! hiányozna, akkor végtelen rekurzió volna az eredmény.

További két formája van a metódus-cast-nak:
\begin{verbatim}
    obj:(super@clsname)method
\end{verbatim}
Ez a \verb!clsname! nevű osztály (valamelyik) közvetlen ősosztályában
definiált metódust hívja meg. A \verb!super! ebben a formában kulcsszó.

\begin{verbatim}
    obj:(parent@child)method
\end{verbatim}
Ez a \verb!parent! nevű osztályban definiált metódust hívja meg,
feltéve, hogy a \verb!child! nevű osztály közvetlenül \verb!parent!-ből
származik. Ha nem, akkor runtime error keletkezik.


A metódus-cast koncepciója: Az adott objektumra meghívjuk valamelyik 
{\em ősosztály metódusát}. Az ettől eltérő használat hibákhoz vezet:

 A metódus-cast-ban ne adjuk meg olyan osztályt, 
 aminek nincs köze az objektumhoz. Ha a rendszer úgy találja, hogy
 az \verb!obj:(base)method! kifejezésben \verb!base! nem felmenője
 \verb!obj! osztályának, akkor runtime errort generál: 
 \verb!"prohibited method cast"!.

 Ne alkalmazzunk metódus-cast-ot attribútumokra. 
 Sok tekintetben a metódusok és attribútumok
 egyformán viselkednek, ez azonban egy kivétel.
 Ha az \verb!obj:(base)method! kifejezés kiértékelésekor kiderül, 
 hogy \verb!base!-ben \verb!method! mégsem metódus, hanem attribútum, 
 akkor runtime errort kapunk: \verb!"prohibited attribute cast"!.
 (Legalább nem engedi tovább. Ha a rendszer továbbengedné, akkor 
 a program a \verb!derived!  osztályú objektum attribútumainak tömbjében 
 a \verb!base!-beli indexszel akarná megcímezni a keresett elemet.
 Ez mutatja, hogy a logikánk nem terjeszthető ki attribútumokra,
 legalábbis nem egyszerűen.)







\subsection{Láthatóság}

\subsubsection{Static osztályok}
A függvényekhez hasonlóan az osztályokat is definiálhatjuk \verb!static!-nak.
\begin{verbatim}
static class derived(base)
\end{verbatim}
A nem \verb!static! esethez képest az a különbség, hogy a definícióból
(belsőleg) generálódó két függvény most \verb!static! lesz,
\begin{verbatim}
static function derivedClass()
    ...
static function derivedNew()
    ...
\end{verbatim}
így ezekre nem lehet kívülről (másik modulból) hivatkozni. 

A static osztályok azonban mégsincsenek teljesen eltemetve.
A \verb!classidbyname()! függvény neve alapján előkeresi az osztályazonosítót:
\begin{verbatim}
    classid:=classidbyname("derived")
\end{verbatim}
A ,,rendszer'' semmire sem használja az osztályok nyilvántartásában
tárolt nevet. Az objektumrendszer működését ezért nem érinti, ha a nyilvántartásban
két különböző osztály esetleg azonos névvel szerepel. ,,Alkalmazások'' azonban 
alapozhatnak a \verb!classidbyname()! függvényre. Az ilyen alkalmazásoknál
ügyelni kell rá, hogy az osztálynevek ne ütközzenek. Különösen static
osztályoknál, ahol a linker nem figyelmeztet az ütközésre.


\subsubsection{Névterek és osztályok}
Hogyan kombinálódnak a névterek és az osztálydefiníciók?
Az alábbi példa választ ad a kérdésre.
\begin{verbatim}
namespace nsp

static class base(object)

class proba.szerencse.derived(nsp.base)
    method hopp

static function proba.szerencse.derived.hopp(this)
    ? "HOPP"
\end{verbatim}
A \verb!namespace! utasítás az egész modult az \verb!nsp! névtérbe teszi.
A \verb!derived! osztályban ez még tovább mélyül a \verb!proba.szerencse! 
többszintű névtérrel.
Az ősosztály megadásakor ki kell írni a minősített osztálynevet: \verb!nsp.base!.

Próbáljuk ki a programot az alábbi főprogrammal. A  \verb!namespace! 
utasítás miatt ez most szükségszerűen külön modulban kell legyen, mert a \verb!main!
nem lehet minősítve.
\begin{verbatim}
function main()
local o
    ? "1. lista"
    classListAll()
    o:=nsp.proba.szerencse.derivedNew()
    ? "2. lista"
    classListAll()
    o:hopp
    ? classidbyname("nsp.base")
\end{verbatim}
A program ezeket írja ki:
\begin{verbatim}
1. lista

2. lista
   1 object                         0   14   64
   2 nsp.base                       0   14   64
   3 nsp.proba.szerencse.derived    0   15   64

HOPP
         2
\end{verbatim}
Az osztályok akkor jönnek létre, amikor a program hivatkozik rájuk.
Az 1. lista azért üres, mert a program nem csinált még egy objektumot sem.
%Az objektumgyartó függvényt teljes minősítéssel hívtuk meg.
A 2. lista tartalma:
\begin{enumerate}
\item oszlop: osztályazonosító index (\verb!classidbyname("object")==1!)
\item oszlop: az osztály minősített neve
\item oszlop: attribútumok száma
\item oszlop: attribútumok+metódusok száma
\item oszlop: hashtábla mérete
\end{enumerate}
A \verb!base! osztályt hiába definiáltuk \verb!static!-nak, 
mégis megjelenik a listában, és az azonosítója is megkapható (2).


\subsubsection{Hivatkozás a class függvényre}

Mint látjuk, az xClass és xNew függvényekkel minden megeshet, 
ami a függvényekkel általában megesik. Névtérbe kerülhetnek, 
\verb!static!-ok lehetnek.
Az xNew (objektumgyártó) függvénynél ennek következményei
nyilvánvalók, mert érvényesek rá a függvényekre vonatkozó 
általános szabályok.

Az xClass függvényt a programok általában nem használják közvetlenül. 
Az alábbi három szituációban azonban rejtve mégis a \verb!baseClass()!
függvényre történik hivatkozás:
\begin{verbatim}
    1) class derived(base)
    2) this:(base)initialize
    3) recover err <base>
\end{verbatim}
Ha viszont így van, 
akkor itt is felvetődik a láthatóság és a névtér kérdése:

Ha a \verb!base! osztály \verb!static!-nak van definiálva,
akkor más modulból nem tudunk rá hivatkozni. Linkeléskor 
bukik ki az ilyen hiba, a linker nem találja a \verb!baseClass()! függvényt.

Ha az alaposztály névtérből van, 
akkor azt teljes útvonallal jelölni kell:
\begin{verbatim}
    1) class derived(multi.level.namespace.base)
    2) this:(multi.level.namespace.base)initialize
    3) recover err <multi.level.namespace.base>
\end{verbatim}




\subsection{Defaulttól eltérő new függvény}

\begin{verbatim}
class derived(base) new:symbol
    ...
\end{verbatim}

A \verb!new:symbol! toldalék opcionális. 
Ha hiányzik, akkor a default \verb!derivedNew! 
nevű konstruktor készül.
Ha van new toldalék, de a symbol tagja üres 
(tehát ilyen alakú \verb!new:!), akkor egyáltalán nem keletkezik 
konstruktor függvény. Teljes new toldalék esetén a 
\verb!symbol!-ban megadott névvel képzett \verb!derivedSymbol! 
konstruktort kapjuk.



\subsection{Metódushívás-továbbítás}

\begin{verbatim}
class proba(object)
    attrib  a1
    attrib  a2
    method  m1      :a1
    method  m2      :a2:b:c:m

function main()
local p:=probaNew()
    p:a1:="Próba szerencse"
    ? p:m1      //kiírja: "Próba szerencse"
\end{verbatim}

A példában \verb!m1! lényegében egy alias az \verb!a1! attribútumra,
set-get metódusokat lehet így implementálni.

Az \verb!m2! metódus feltételezi, hogy az \verb!a2! (beágyazott objektum) 
attribútumnak, van egy \verb!b! attribútuma, annak egy \verb!c! attribútuma 
és annak egy \verb!m! metódusa. Ennek a metódusnak továbbítódik a metódushívás.
Nem kell ismernünk \verb!m! implementációját, automatikusan
minden paraméter továbbításra kerül. Ráadásul a fordító ügyesen
rendezgeti a stacket, nem jönnek létre felesleges függvényhívási szintek.




\subsection{A könyvtári object osztály}

\begin{verbatim}
class template(object)
    attrib cargo        //teljesen üres is lehetne

function main()
    templateNew():liststruct
\end{verbatim}

Az objektumorientált programozás erejét mutatja, 
hogy már az egyszerű \verb!template! osztály is említésre 
méltó tudással rendelkezik, ugyanis egy csomó dolgot
örököl az \verb!object!-től. Ki tudja listázni,
hogy milyen metódusai és attribútumai vannak, ki tudja listázni
az attribútumainak az értékét, meg tudja mondani az osztályának
és a szülő osztályainak a nevét. Ezzel a képességgel
{\em minden\/} osztály rendelkezik.  
Az \verb!object!-ből öröklődő metódusokat vesszük most sorra.
\begin{description}
\item[ancestors]
  Ad egy listát az ősosztályok nevével.

\item[asarray]
  Egy tömbben visszaadja az összes attribútumot.

\item[attrnames]
  Ad egy listát az attribútumok nevével.

\item[attrvals] 
  Visszad egy array-t, melynek elemei kételemű tömbök,
  az összes attribútum nevéből és értékéből képzett pár.

\item[baseid]
  Ad agy arrayt, ami a közvetlen ősosztályok azonosítóit tartalmazza.

\item[classname] 
  Visszaadja az objektum osztályának nevét.

\item[evalmethod] 
  Név szerinti metódushívás (\verb!o:evalmethod("methname",{a,b,c})!)

\item[initialize]
  Inicializálja az objektumot.
  Valójában egy object osztályú objektumon nincs mit inicializálni,
  mert az osztályban nincs egyetlen attribútum sem, csak metódusok.

\item[isderivedfrom(clid/obj)]
  Megmondja, hogy this (osztálya) leszármazottja-e az
  osztályazonosítóval/objektumpéldánnyal megadott másik osztálynak.

\item[length]
  Megmondja az attribútumok számát.

\item[list]
  Kilistázza magát (az attribútumait) a konzolra. 

\item[liststruct]
  Kilistázza, hogy milyen attribútumai, metódusai vannak,
  és melyiket honnan örökölte.

\item[methnames]
  Ad egy listát a metódusok nevével.

\item[struct]
  Ad egy tömböt, ami azt a struktúrát tartalmazza, amit liststruct
  kilistáz.
\end{description}

Felsorolunk néhány további objektumokkal kapcsolatos
függvényt, amik azonban nem metódusai az object osztálynak:
\begin{description}
\item[classListAll()]
    Listázza a program összes osztályát.
\item[classIdByName(classname)]
    Név alapján kikeresi és visszaadja az osztályazonosító számot.
    Ha a megadott névvel nincs osztály, akkor 0-t ad.
\item[getClassId(obj)]
    Az objektumpéldányból megadja annak (szám) osztályazonosítóját.
\item[getObjectAsArray(obj)]
    Egy tömbben visszaadja az összes attribútumot.
    Ezen a függvényen alapul az asarray metódus.
\item[iniObjectFromArray(obj,arr)]
    Inicializálja az objektumot egy olyan arrayből,
    amit korábban a getobjectasarray (vagy asarray metódussal) 
    kaptunk, enélkül a getobjectasarray nem is volna értelmesen 
    használható. 
\item[objectNew(clid)]
    A függvény paramétere az osztályazonosító (szám).
    Visszaad egy megadott osztályú, új, inicializálatlan 
    objektumpéldányt. Ezen alapul minden konstruktor.
\end{description}

A példaprogram a következőket írja ki:
\begin{verbatim}
         1 ancestors                M object
         2 asarray                  M object
         3 attrnames                M object
         4 attrvals                 M object
         5 baseid                   M object
         6 classname                M object
         7 evalmethod               M object
         8 initialize               M object
         9 isderivedfrom            M object
        10 length                   M object
        11 list                     M object
        12 liststruct               M object
        13 methnames                M object
        14 struct                   M object
        15 cargo                    A template
\end{verbatim}

\subsection{Függvény interfész}

Többször volt már említve, hogy az osztálydefiníció eredménye
az xClass és xNew függvények. A programban ez a két függvény
képviseli az osztályt. Általában a fordítóprogram generálja ezeket
az osztálydefinícióból.

Talán nem meglepő, hogy ,,kézzel'' is írhatunk xClass és xNew függvényeket.
Az alábbi példa létrehozza a \verb!template! osztályt (kibővítve egy
\verb!initialize! metódussal).

\begin{verbatim}
static clid_template:=templateRegister()

static function templateRegister()
local clid:=classRegister("template",{objectClass()})
    classMethod(clid,"initialize",{|this|template.initialize(this)})
    classAttrib(clid,"cargo")
    return clid

static function template.initialize(this)
    ...
    return this

function templateClass()
    return clid_template

function templateNew()
local clid:=templateClass()
    return objectNew(clid):initialize
\end{verbatim}

A \verb!classRegister! függvény első argumentuma tartalmazza az új
osztály nevét, a második argumentum egy array, amiben az új osztály
szülő osztályait kell felsorolni. A legegyszerűbb esetben az új
osztály az objectClass-tól (minden osztály közös ősétől) származik.

A \verb!classMethod! függvény egy metódust ad a clid-vel azonosított
osztályhoz. A metódus nevét a második paraméterben adjuk át, jelen 
esetben a név ,,initialize''. A metódus végrehajtása a harmadik argumentumban 
átadott kódblokk kiértékelésével történik. A metódusblokkok első 
paramétere mindig maga az objektum, amit általában ,,this'' névvel
illetünk, de itt ez nem kulcsszó, mint a C++-ban, vagy a Jávában.

A \verb!classAttrib! függvény egy attribútumot ad a clid-vel azonosított
osztályhoz. 

Nyilvánvaló a megfelelés a \verb!class! osztálydefiníció és a
függvényinterfész között. A fordítóprogram egyébként úgy működik,
hogy a \verb!class! definícióból belsőleg előállítja a függvényinterfész 
kódot, és azt a szokásos módon lefordítja.






