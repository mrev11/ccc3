\section{Többszálúság}

\subsection{Pthread API}

A CCC2-től kezdve van multithread támogatás.
Az alább felsorolt API áll rendelkezésre
szálak indítására és szinkronizálására:
\begin{verbatim}
thread_create(codeblock,p1,...) --> threadid
thread_self() --> threadid 
thread_detach(threadid) --> status
thread_exit() --> NIL
thread_join(threadid) --> status
thread_mutex_init() --> xMutex
thread_mutex_lock(xMutex) --> status
thread_mutex_trylock(xMutex) --> status 
thread_mutex_unlock(xMutex) --> status 
thread_mutex_destroy(xMutex) --> status 
thread_cond_init() --> xCond
thread_cond_signal(xCond) --> status
thread_cond_wait(xCond,xMutex[,nMillis]) --> status 
thread_cond_destroy(xCond) --> status
\end{verbatim}

A CCC szintre kivezetett egyszerűsített pthread API CCC-ből ugyanúgy működik, 
mint C-ből. Nincs értelme most referenciaszerű leírást adni róla, mert több
ilyen található az interneten. 
\href{http://cs.pub.ro/~apc/2003/resources/pthreads/uguide/document.htm}%
{Itt van pl. egy a számos közül}.
A  man is használható a {\em pthread\/}  címszavaknál. 
Magam is a man oldalak alapján dolgozom, pl.
\begin{verbatim}
man 3 pthread_cond_signal
\end{verbatim}
Linuxon nyilvánvaló megfelelés van a pthread könyvtár és a CCC között.
Windowson ugyanez az interfész vissza van vezetve Windows API-ra, 
nem különösebben bonyolult.


Új szálakat  kódblokk végrehajtással indítunk.
A \verb!thread_create(blk,p1,...)! függvényt pontosan úgy kell meghívni, 
mint az \verb!eval()!-t, a különbség, hogy \verb!thread_create()! azonnal visszatér,
miközben az új szál futásnak indul.

A szálak ugyanabban a változótérben dolgoznak. A static változók (külsők és belsők) 
közösek, egy példányban léteznek. Minden szál külön local stackkel rendelkezik.
Bármelyik szálból kiindulhat a szemétgyűjtés.


\subsection{Egyszerű példa - mutex}

Az alábbi példa elindít 5 darab szálat.
Minden szál ugyanazt csinálja: 100,000-szer hozzáad 1-et a \verb!sum! változóhoz,
a helyes végösszeg tehát 500,000 volna. A program azonban szándékosan el van rontva,
ki van kommentezve a \verb!sum++! sort védő mutex lock. 
\begin{verbatim}
static mutex:=thread_mutex_init()
static sum:=0

function main()
local blk:={||addtosum()}
local tid:={},i

    aadd(tid,thread_create(blk))    //új szál, az azonosítót megőrzi
    aadd(tid,thread_create(blk))
    aadd(tid,thread_create(blk))
    aadd(tid,thread_create(blk))
    aadd(tid,thread_create(blk))
    
    for i:=1 to len(tid)
        ? i, tid[i]
        thread_join(tid[i])         //megvárja, hogy befejeződjön
    next
    ? "végösszeg",sum               //a helyes eredmény 500000
    ?
    
function addtosum()
local i
    for i:=1 to 100000
        //thread_mutex_lock(mutex)  //kellene!
        sum++
        //thread_mutex_unlock(mutex)
    next
\end{verbatim}

A mutex ({\em mut}ually {\em ex}clusive) olyan dolog, 
amit lockolni (zárolni, megfogni) és unlockolni (elengedni) lehet.
A lényeg, hogy egyszerre legfeljebb egy szál foghatja a mutexet.
Amíg egy szál fogva tartja, 
addig a többi szálból meghívott \verb!thread_mutex_lock! vár
a mutex felszabadulására.  A mutex mechanizmussal tehát biztosítani lehet,
hogy egy kódrész végrehajtásával egyszerre legfeljebb egy szál foglalkozzon.

Fontos észrevétel, hogy a \verb!sum++! művelet végrehajtása nem atomi.
Atomi nagyjából azt jelenti, hogy felbonthatatlan egység. 32-bites processzoron
egy 32-bites mennyiség memóriából történő kiolvasása atomi. 2 darab
32-bites mennyiség kiolvasása már nem atomi. Lehet, hogy
közben a processzor más tevékenységre vált, pl. folytatja egy másik szál
végrehajtását. \verb!sum++! ehhez képest a static változó értékét 
(eleve több mint 64 bit) átrakja a local stackre, hozzáad 1-et, 
majd az eredményt visszamásolja az eredeti helyre. Sokszorosan nem atomi.

A saját gépemen 250 ezer körüli véletlenszerű eredményt kapok.
Kikommentezett mutex lock/unlockkal a példaprogram azért működik rosszul, 
mert egyszerre több szál is kiolvashatja \verb!sum!-ból {\em ugyanazt\/} 
a pillanatnyi értéket, és ilyenkor valamelyik szál növekménye elvész.


Vigyázzunk, hogy a mutexet ne lockoljuk ugyanabból a szálból
többszörösen, ui.
\begin{verbatim}
    thread_mutex_lock(mutex)
    thread_mutex_lock(mutex)  //deadlock
\end{verbatim}
deadlockot eredményez.

Egy további észrevétel. 
A példa a mutexet külső static változóként deklarálja. 
Az világos, hogy static-nak kell lennie. Ha local volna, akkor minden szálnak 
külön példánya volna belőle, nem tudná kifejteni a működését. De lehetne-e
belső static \verb!addtosum!-ban? Az a nehézség, hogy a static változók 
(elvileg csak egyszer futó) inicializálását is szinkronizálni kell.
Szinkronizáció híján előfordul, hogy az inicializátor néha mégis többször fut.
Éppen ezért a CCC a külső static változók inicializátorát mindig 
szinkronizálja, hogy legyen biztos kiindulópont. A belső static változók 
inicializátora nincs automatikus mutex védelem alatt (ritkán kell, viszont költséges).








\subsection{Egyszerű példa - cond}


A következő program sok szálat hoz létre.
A szálak véletlen hosszú ideig, átlagosan 1 másodpercig élnek,
az egyszerre életben levő szálak száma \verb!level<=MAXTHREAD!.
A program folyamatosan listázza a létrejövő és megszűnő szálakat,
mindaddig, amíg ESC-et nem ütünk neki.
Végül megvárja az összes thread kilépését.

\begin{verbatim}
#include "inkey.ch"

#define MAXTHREAD  16

static mutex:=thread_mutex_init()
static cond:=thread_cond_init()
static level:=0
static count:=0

function main()
local th

    while( inkey(0.05)!=K_ESC  )
        th:=thread_create({|x,r|dothread(x,r)},count,rand())
        thread_detach(th)
        
        //A threadeket vagy el kell engedni (thread_detach)
        //vagy meg kell várni (thread_join), máskülönben
        //elfogynak a létrehozható threadek (Linuxon kb. 90).

        thread_mutex_lock(mutex)
        //A level változót a threadek módosítgatják,
        //ezért csak mutex védelem alatt biztonságos a kiolvasása.
        ? padl(l2hex(th),8), count, level
        ++level
        ++count
        while( level>MAXTHREAD )
            thread_cond_wait(cond,mutex)
        end
        thread_mutex_unlock(mutex)
    end

    thread_mutex_lock(mutex)
    while( level>0 )
        thread_cond_wait(cond,mutex)
        ? "wait",level
    end
    thread_mutex_unlock(mutex)

static function dothread(x,r)
    sleep(r*2000)  //átlagosan 1 másodpercet vár
    thread_mutex_lock(mutex)
    --level
    ? "quit",x,r
    thread_cond_signal(cond)
    thread_mutex_unlock(mutex)
\end{verbatim}


A példa fő tanulsága, 
hogyan várunk arra \verb!main!-ben, hogy a \verb!level! változó 
(a többi szál tevékenységének következtében)
lecsökkenjen egy kívánt értékre.
A program elején egy mutex és egy cond objektumot hoztunk létre.
Általában egy cond ({\em cond}ition) objektumot mindig egy mutexszel együtt használunk.


Az alábbi kódrészletben a \verb!thread_cond_wait(cond,mutex)!
híváskor \verb!mutex!-nek lockolva kell lennie. A \verb!thread_cond_wait!
automatikusan elengedi a mutexet, és vár, amíg valamelyik másik száltól szignált
nem kap. A várakozás alatt a szál futása fel van függesztve, semennyi CPU időt 
nem fogyaszt. Miután megjött a szignál
\verb!thread_cond_wait! újra megfogja \verb!mutex!-et, majd visszatér.
\begin{verbatim}
    thread_mutex_lock(mutex)
    while( level>0 )
        thread_cond_wait(cond,mutex)
    end
    thread_mutex_unlock(mutex)
\end{verbatim}

Eközben a többi szál minden alkalommal, 
amikor a \verb!level! változó csökken, szignált küld,
hogy a várakozó szál értesüljön az eseményről.
\begin{verbatim}
    thread_mutex_lock(mutex)
    --level
    thread_cond_signal(cond)
    thread_mutex_unlock(mutex)
\end{verbatim}
A \verb!thread_cond_signal(cond)! hívás ,,szignált küld'' a \verb!cond!
objektumnak. Ha egyetlen szál sincs, amelyik éppen  \verb!cond!-ban várakozna,
akkor a szignál hatástalan. Ha több szál is várakozik \verb!cond!-ban,
akkor ezek közül az egyik továbbindul, de hogy melyik, az nincs meghatározva.





\subsection{Thread-local storage}

Tudjuk, 
hogy a program static változói csak egy példányban léteznek,
és az az egy példány minden szálra közös. 
Ezzel szemben minden szálnak saját local stackje van, 
a stack változók ezért szálanként elkülönülnek.
De nem csupán szálanként.
A local változók csak egy-egy függvényen belül léteznek,
és minden függvényhívásban külön létrejönnek.
Hol tudunk akkor olyan adatot tárolni, ami szálanként egyedi,
de egy szálon belül közös.  Az ilyen adatok kezelésére szolgáló 
mechanizmust nevezik ,,thread-local storage''-nak.

CCC-ben a \verb!localstack! függvénnyel lehet egyszerű TLS-t csinálni.
Nézzük az alábbi programot:
\begin{verbatim}
function main(a,b,c)
local x:="x", y:="y"
    ? localstack(1)
    ? localstack(2)
    ? localstack(3)
    ? localstack(4)
    ? localstack(5)
\end{verbatim}
Ha a programot így indítjuk
\begin{verbatim}
proba.exe q w
\end{verbatim}
akkor ezt a kimenetet kapjuk:
\begin{verbatim}
q
w
NIL
x
y
\end{verbatim}
Világos, hogy \verb!localstack(x)! előveszi (akár egy tömbből) 
a local stack \verb!x!-edik elemét. Kicsit vigyázni kell vele,
de általában tudható, van-e elég érték a stacken, a példaprogramban
biztosan megvan legalább 5 darab.

Egyszerű ötlet: A thread közösnek szánt adatait tegyük be
a local stack aljára. Tulajdonképpen egyetlen érték is elég,
hiszen lehet az egy  hashtábla (vagy array, vagy objektum), 
amiben aztán annyi további adatot tárolhatunk, amennyi kell, 
és akár név szerint is hivatkozhatunk rájuk.


\begin{verbatim}
local blk:={|p1,p2|dothread(p1,p2)} 
local a,b,c
    ...
    thread_create(blk,a:=simplehashNew(),b,c)
    ...
\end{verbatim}
A szálban hívott \verb!localstack()! értékei:
\begin{verbatim}
    localstack(1)  // -->  blk
    localstack(2)  // -->  a
    localstack(3)  // -->  b
    localstack(4)  // -->  nem definiált (dothread-től függ)
    localstack(5)  // -->  nem definiált, stb.
\end{verbatim}
A kódblokkot és a blokknak átadott paramétereket lehet megkapni
a szálból bárhonnan, legfeljebb olyan számban,
ahány \verb!p1!, \verb!p2!\ldots paramétere van a blokknak.
A példában \verb!localstack(2)! egy szálanként egyedi, 
de szálon belül közös hashtáblát ad.


\subsection{Szálbiztonság}

A szálbiztonság első kritériuma: biztonság a szemétgyűjtéssel szemben. 
Minden olyan pillanatban, amikor egy másik szálból szemétgyűjtés indulhat, 
a vermeken kell legyen minden élő változó, de nem lehet ott semmi más,
pl. keletkezőben vagy megszűnőben levő változók. Ha egy változó nincs
rajta a stacken, akkor a hozzá tartozó memóriaobjektumot kitakaríthatja
a szemétgyűjtés. A keletkezőben vagy megszűnőben levő változók érvénytelen
pointereket jelentenek, amik elrontják a szemétgyűjtésben levő gráfbejárást.
Ha nem használnak saját C kódot, akkor az alkalmazásoknak ezzel nem kell 
törődniük, mert ilyen hibát  akármilyen rossz alkalmazáslogika sem okozhat. 

A szálbiztonság második kritériuma: a static-ok szinkronizálása.
A szálak által közösen használt static változókat szinkronizálni 
kell akár C++, akár CCC szinten. A CCC alapkönyvtárakben levő néhány 
static változó szinkronizálva van, a saját static-jaikat viszont
az alkalmazásoknak maguknak kell szinkronizálni.

E kritériumok teljesítése csak a ccc3 és ccc3\_ui\_ könyvtáraknál kitűzött cél. 
Árnyaltan fogalmazva ,,kitűzött'', nem pedig elért célról beszélünk.

Szálbiztos a karakteres fullscreen megjelenítő könyvtár (ccc3\_uic),
ha csak egy szál foglalkozik a képernyővel.

Szálbiztos a Jáva terminál könyvtár, 
ha csak egy szál használja a terminált, a többi szál mással foglalkozik.

Szálbiztos az sql2 könyvtár, 
ha minden szál külön adatbáziskapcsolaton keresztül dolgozik.

A különféle interfészek szálbiztonsága attól függ,
hogy az adott könyvtár (amit az interfész közzétesz) szálbiztos-e.

A szálbiztonság ellenőrzésére nincs általános módszer.
Egy lehetőség a hibahalászat. Véletlenszerűen mindenfélét csináló programokat
hagyunk futni napokig, hetekig. Ha a program hibázik, akkor megpróbáljuk
behatárolni a hibát, ami egyáltalán nem könnyű feladat.
Képzeljünk el egy olyan hibát, ami egy folyamatosan futó tesztprogramot
átlagosan  hetente akaszt meg\ldots Szerencsére ilyen eset már évek óta
nem volt. Mérget azonban nem lehet rá venni. 


\begin{comment}
Ez nem aktuális, ki lett javítva.


A hibahalászós módszerrel találtam a következő érdekes (nem könnyen kijavítható) 
hibát, amivel mégiscsak el lehet rontani a szemétgyűjtést alkalmazásszinten is.
Az alábbi program az én rendszeremen pár másodpercen belül SIGSEGV-vel elszáll.
\begin{verbatim}
function main()
    thread_create({||dothread()})
    while(.t.)
        array(100)[1]:=1
    end

function dothread()
    while(.t.)
        gc()      // szemetet gyűjt
        sleep(1)  // 1 ezredmásodpercet alszik
    end
\end{verbatim}
A program elindít egy szálat, ami lényegében folyamatosan gyűjti a szemetet 
\verb!gc()!-vel. Általában nincs értelme indítgatni a szemétgyűjtést,
elindul az magától, amikor kell, most azonban a hiba kimutatása a cél.
Az eredeti szál közben ezt ismételgeti:
\begin{verbatim}
    array(100)[1]:=1
\end{verbatim}
És itt a baj. Ui. az \verb!array()!-ből kapott tömb
nem értéke egyetlen programváltozónak sem,
és a függvény visszatérése után rögtön lekerül a stackről.
Csak ezután próbál a program értéket adni a tömbelemnek.
Az időzítéstől függően azonban egyes esetekben a tömböt már eltakarította
a szemétgyűjtés, a program a semmibe nyúl\ldots
Az a szerencse, hogy értelmetlen. Miért akarnánk
értéket adni egy nemlétező tömb akármelyik elemének?
Ha a programnak értelmet adunk, pl.
\begin{verbatim}
    a:=array(100)
    a[1]:=1
\end{verbatim}
akkor viszont a hiba is rögtön megszűnik.
Csak játék, de mutatja, milyen könnyű elrontani a szálbiztonságot,
és milyen rejtvényszerű hibákat kell néha felderíteni.

\end{comment}







