
\section{Régebbi dokumentációk}

Összegyűjtöttem és belinkeltem néhány régről meglevő dokumentációt.

\subsection{
\protect\href{http://ccc.comfirm.hu/pub/ng}{Eredeti Clipper doksi}
}

Még mindig használható az eredeti
\href{http://ccc.comfirm.hu/pub/ng}{
 Clipper 5.x dokumentációjának HTML} változata.
Persze tudni kell, mik az elavult, és a még mindig érvényes
szakaszok. 
\begin{itemize}
\item \href{http://ccc.comfirm.hu/pub/ng/Clipper-guide/ng3ca.html}{Functions}
    \par
    A dBase adatbáziskezeléssel kapcsolatos függvények a CCC-ből hiányoznak.
    Pl. az \verb!asize()! leírása érvényes, de \verb!dbdelete()! nincs.

\item \href{http://ccc.comfirm.hu/pub/ng/Clipper-guide/ng6fdf7.html}{Commands}
    \par
    A dBase adatbáziskezeléssel kapcsolatos parancsok a CCC-ből hiányoznak.
    Vannak:
    \verb!?!, 
    \verb!??!, 
    \verb!@...box!, 
    \verb!@...clear!, 
    \verb!@...get!, 
    \verb!@...say!, 
    \verb!copy file!, 
    \verb!erase!, 
    \verb!keyboard!, 
    \verb!quit!, 
    \verb!rename!, 
    \verb!run!, 
    \verb!set...!, és talán még egyebek is, nem emlékszem mindenre.

\item \href{http://ccc.comfirm.hu/pub/ng/Clipper-guide/ngb495e.html}{Classes}
    \par
    A régi Clipperben csak ez a négy előre beépített objektum létezett,
    ezek a CCC-ben is megvannak. A CCC egyik nagy vívmánya, hogy komplett 
    objektumrendszer került bele. 

\item \href{http://ccc.comfirm.hu/pub/ng/Clipper-guide/ngb4a83.html}{Statements}
    \par
    Vannak:
    \verb!begin sequence!,
    \verb!do case!,
    \verb!do while!,
    \verb!for!,
    \verb!function!,
    \verb!if!,
    \verb!local!,
    \verb!static!,
    \verb!return!.
    Innen meg lehet tanulni, hogyen kell \verb!if! szerkezetet 
    vagy \verb!for! ciklust írni. A \verb!begin [sequence]! utasítás
    (vagyis a kivételkezelés) lényegesen bővült. A CCC-ben a Jávához
    hasonló kivételkezelés van.

\item \href{http://ccc.comfirm.hu/pub/ng/Clipper-guide/ngc8ba2.html}{Operators}
    \par
    A \verb!&! (makró) és az \verb!=! kivételével minden megvan a CCC-ben is.

\item \href{http://ccc.comfirm.hu/pub/ng/Clipper-guide/ngd8973.html}{Directives}
    \par
    Az \verb!#error! és \verb!#stdout! kivételével minden.

\item \href{http://ccc.comfirm.hu/pub/ng/Clipper-guide/nge36fa.html}{Get System}
    \par
    Ezek vannak:
    \verb!getactive()!,
    \verb!getapplykey()!,
    \verb!getpostvalidate()!,
    \verb!getprevalidate()!,
    \verb!getreader()!.

\end{itemize}

\subsection{
\protect\href{http://ccc.comfirm.hu/ccc3/build.html}{CCC projekt manager}
}

\subsection{
\protect\href{http://ccc.comfirm.hu/ccc3/jterminal.html}{Jáva terminál}
}

\subsection{
\protect\href{http://ccc.comfirm.hu/ccc3/sql2.html}{SQL2 interfész}
}

\subsection{
\protect\href{http://ccc.comfirm.hu/ccc3/cccgtk.html}{GTK interfész}
}




