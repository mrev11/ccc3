
\section{Unicode}

%A CCC3-tól kezdve van Unicode támogatás.
%\tableofcontents



\subsection{Unicode támogatás}

A régi Clipperben és a CCC1-CCC2-ben nem volt megkülönböztetve
a tetszőleges (akár bináris) adatokat tartalmazó bájtsorozat
és a karaktersorozatot tartalmazó string. Az ilyen típust egységesen 
karakternek (stringnek) neveztük, a típus kódja  "C" volt.
Hasonló volt a helyzet a 2.3 előtti Pythonban is.

Az idők azonban változnak, igény támadt az egyidejűleg
többféle nyelven is értő programokra. Világossá vált, hogy
a többnyelvűség igényeit legjobban a Unicode elégíti ki,
továbbá, hogy a Unicode problémáit (az operációs
rendszerrel való kompatibilitást illetően) legjobban
az UTF-8 kódolás oldja meg. A Unicode/UTF-8 kódolás univerzálisan 
elfogadottá vált, az operációs rendszerek sorra áttértek a használatára.
A változást a CCC-vel is követnünk kellett.
A CCC3 fő újdonsága a Unicode támogatás.

Az XMLRPC esete mutatja, mennyire elkerülhetetlen a változás követése. 
Egy 1999-es XMLRPC leírás azt mondja, hogy a \verb!string!
adattípusban bámilyen adatot küldhetünk (binárisat is),
csak arra kell ügyelnünk, hogy az XML formázásban szerepet
játszó karakterek/bájtok (\verb!<!, \verb!&!) megfelelően védve legyenek. 
A pár évvel ezelőtti XML tankönyvekben fel sem vetődik a kérdés:
Miből áll az XML dokumentum?
\begin{itemize}
\item Bájtok sorozatából,
\item vagy karakterek sorozatából?
\end{itemize}
A mai XML szabvány szerint karakterekből.
Az XMLRPC string tehát nem tartalmazhat bináris adatot, 
mert akkor elbukik az XML elemzésen. Hogy mik a karakterek,
az sem triviális, pl. a 0x00--0x20 intervallumban csak a
TAB, CR, LF számít karakternek, és máshol is vannak érvénytelen
(nem karakter) kódok, amikre a mai XML elemzők kivételt dobnak.

A Unicode támogatás megvalósítására két út kínálkozott.
A Pythonban úgy jártak el, hogy bevezettek (mint új dimenziót) 
egy új típust, a Unicodeot, ami minden mást érintetlenül hagyott.
A kompatibilitás szempontjából ez tökéletes megoldás,
azonban semmivel nem visz közelebb a régebbi programok
Unicodeosításához.

A CCC-ben a Jáva mintáját követve radikálisabb utat választottunk.
Bevezettünk egy új típust a bájtsorozatok számára. Ezt bájtarraynek,
bájtsorozatnak, vagy bináris stringnek nevezzük, a típuskódja "X"
(sajnos a B betű már foglalt a kódblokkok számára). 
A binary string (X) átveszi a régi
(C) stringek szerepét, amikor azok bináris adatot tárolnának.
A bináris stringekre működnek a szokásos string kezelő függvények
és operátorok: 
\verb!at!, 
\verb!rat!, 
\verb!strtran!, 
\verb!left!, 
\verb!right!, 
\verb!padr!, 
\verb!padl!, 
\verb!substr!, 
\verb!+!, 
\verb!==!, 
\verb!$!, 
stb.

A korábbi (C) stringek értelmezése megváltozott,
a CCC3-ban Unicode karaktersorozatot jelentenek.
Természetesen az ismert string függvények ezekre is működnek.
A két string fajtát azonban nem lehet keverni, azaz nincs
feltétlen, automatikus konverzió.
Érdemes tudni, hogy az egyes karakterek C szinten \verb!wchar_t!
típusban tárolódnak, ami a mai C fordítókban 32 bites mennyiség.

Nagyon fontos megérteni a karakter string és a binary string
közötti kapcsolatot. A karakter string (Unicode vagy UCS kódok
sorozata) szöveget tud tárolni. Ha a szöveget binary stringbe 
akarom átírni, akkor előállítom a Unicode karakterek UTF-8
kódját (karakterenként a karaktertől függő hosszúságú bájtsorozat),
ezeket konkatenálom, az eredmény egy bájtsorozat, amit a 
szöveg UTF-8 kódolású bináris reprezentációjának nevezek.
Bármely szöveg (karakter string) ezen a módon infóveszteség
nélkül bináris stringre konvertálható, és a bináris reprezentációból
maradék nélkül visszanyerhető. Általában a szöveg UTF-8 reprezentációja
több bájt, mint ahány karakter van az eredeti szövegben.
Ennek oka, hogy pl. a magyar ékezetes betűk vagy a cirill
betűk UTF-8 kódja két bájt. Más karakterek még hosszabbak
lehetnek, a létező leghosszabb UTF-8 kód hat bájtos.

Ha a karakter string memóriabeli tárolását vizsgáljuk,
azt tapasztaljuk, hogy sok 0 értékű bájt van benne.
Nyilván, ui. az ASCII kódok a 0--127 intervallumba esnek,
azaz egy bájtot foglalnak el, a string azonban 32-bitet 
használ minden karakterhez. A Unicode karakter stringekre ezért
nem működnek a C könyvtár hagyományos string kezelő függvényei,
amik a 0 bájtot a string végének tekintik. Ugyanezért nem
célszerű egy Unicode stringet bájtonként kiírni egy fájlba,
vagy egy socketba. Ezzel szemben a string UTF-8 reprezentációja
rendelkezik azzal a tulajdonsággal, hogy csakis a 0 Unicodenak
felel meg benne 0 bájt. Az UTF-8 bináris string így alkalmas
arra, hogy a program ezzel a típussal adjon meg egy fájlspecifikációt
az OS számára, amire a Unicode string nem felelne meg.


E megfontolásokból adódik, hogy mikor melyik string 
fajtát érdemes/kell használni a programokban.
Alapszabály, hogy a program szövegeit karakter stringben
tároljuk, és ebben a formában manipuláljuk. 
Vannak persze esetek, amikor ettől el kell térnünk.
\begin{itemize}
\item 
    A fájlokból, socketekből általában bájtokat lehet olvasni.
\item
    Az operációs rendszer számára UTF-8 kódolással 
    (tehát bináris string formájában) kell megadni a 
    fájlspecifikációkat.
\end{itemize}
A koncepció, hogy az alkalmazási programokban minél
kevesebbet kelljen váltogatni a bináris és karakter
reprezentáció között, ehelyett a CCC könyvtár függvényei
alkalmas helyen automatikusan elvégzik a konverziót.
A \verb!memoread!-et pl. általában arra használjuk,
hogy egy szövegfájlt egy mozdulattal beemeljünk  egy karakterváltozóba.
Ezért a \verb!memoread! automatikusan karakter stringre
konvertálja, amit olvas. Néha azonban más kell, pl. amikor
egy png képfájlt olvasunk be, ezért a \verb!memoread!
kiegészült egy opcionális paraméterrel, amivel kikapcsolható
ez a konverzió. Ilyenkor a \verb!memoread! eredménye nem 
karakter, hanem binary string.
Az \verb!fopen! a filénevet UTF-8-ra konvertálva adja 
lejjebb a C szintnek. 
Azokon a helyeken, ahol a stringtípusok találkoznak,
az alkalmazásnak mindenképpen explicite kell konvertálnia, 
ezért a programok elkerülhetetlenül bonyolultabbak lesznek, 
mint a CCC2-ben voltak.

Megemlítendő, hogy a Unicode/UTF-8 kódolás a CCC3-ban kizárólagos.
Ezen azt értem, hogy nincs támogatás semmilyen más kódolásra,
pl. Latin-1-re. Ezek a (hagyományos) kódolások elavultak,
és rendkívül megbonyolódik az élet, ha különféle kódolásokat
kell egyszerre kezelni. Mindez azt jelenti, hogy a CCC3 használata 
során az ember szövegfájljai szépen átkonvertálódnak UTF-8-ra.
A jelen sorokat a CCC3-mal fordított (tehát Unicode-os) z editorral írom,
és a saját terminálomban az angol, magyar és orosz szöveget
egyformán helyesen látom (és tudom gépelni), mint ahogy helyesen
látszik a \TeX\  kimentetén és a böngészőben is. Mindehhez
nincs szükség bütykölt fontokra és billentyű driverekre.
Vannak tehát előnyök, amik kárpótolnak a bonyodalmakért. 
Bízzunk benne, hogy az UTF-8 kódolás hosszabb nyugvópont lesz 
a gyorsan változó informatikában.







\subsection{További megfontolások, kompatibilitás}

Azt szeretném itt leírni, 
miért úgy van a Unicode string használat a CCC3-ban, ahogy van,
és hogyan kell választani a karakter és bináris string között. 

A 2.4.x Python támogatja a Unicode/UTF-8 kódolást.
Próbáljuk ki ezt a Python programot:
\begin{verbatim}
#! /usr/bin/env python
# _*_ coding: UTF-8 _*_

a="öt szép szűzlány őrült írót nyúz"   # bájtsorozat
u=u"öt szép szűzlány őrült írót nyúz"  # Unicode string

print 
print a[0:1]
print a
print u[0:1]
print u
\end{verbatim}
Ha lefuttatjuk a programot, ezt a kimenetet kapjuk: 
\begin{verbatim}
?
 öt szép szűzlány őrült írót nyúz
ö 
öt szép szűzlány őrült írót nyúz
\end{verbatim}
Az első '?' valójában egy érvénytelen UTF-8 kód, egy fél 'ö' betű!
Ebből következtetek az alábbiakra:
Az \verb!a! változóban egy bájtsorozat van, ami UTF-8 kódolással
ábrázolja az "öt szép..." szöveget. Az \verb!u! változóban egy
(Unicode) karaktersorozat van, ami ugyanazt a szöveget ábrázolja.
Amikor a stringek első elemét vesszük, akkor az első esetben
az első {\em bájtot\/} a második esetben az első {\em karaktert\/}
kapjuk. Mivel az 'ö' betű UTF-8 kódolásban 2 bájton tárolódik,
azért kapjuk a fél 'ö' betűt, ami érvénytelen kód lévén '?' formában
jelenik meg. 
Ugyanez a program CCC3-ban így néz ki:
\begin{verbatim}
function main()
local a:=a"öt szép szűzlány őrült írót nyúz" //bájtsorozat
local u:="öt szép szűzlány őrült írót nyúz"  //Unicode string
    ? left(a,1)
    ? a
    ? left(u,1)
    ? u
\end{verbatim}
Itt nem a Unicode stringet jelöljük \verb!u"..."!-val,
hanem fordítva, a bájtsorozatot kell megkülönböztetni \verb!a"..."!-val.
Az eredmény egyébként ugyanaz. A Python és a CCC3 Unicode támogatása 
közötti  eltérés:

A Python a Unicode-dal új területet nyit, de nem lép be erre
az új területre. A programozóra bízza a Unicode stringek bejelölését.
A kompatibilitás szempontjából ez tökéletes megoldás, hiszen a
régi programokat egyáltalán nem érinti. Szerintem azonban döntő
hátrány, hogy nem segíti elő az UTF-8 kódolást. Ha pl. egy régi
Python programot Latin-1-ről átkonvertálunk UTF-8-ra,
szintaktikailag semmi sem változik, mégis elromlik a program,
mert mint láttuk, megjelennek a félbevágott UTF-8 kódok.
Az új programoknál a Unicode stringek használatát kellene
előnyben részesíteni, a Python megközelítésében mégis
ezekhez kell többet írni, mindig jelölgetni kell az
\verb!u"..."! stringeket.

A CCC3 készítésekor először én is hetekig a Python útján haladtam,
de egyre kevésbé tetszett a dolog. Végül úgy döntöttem,
hogy áttérek a radikálisabb megvalósításra: Nem csak létrehozom
az új területet, hanem be is lépek rá, azaz a Unicode string
lesz a default. Ugyanez van a Jávában is. A string literálok
Unicode karaktersorozatot jelentenek, emellett használható a
bájtarray, amikor arra van szükség. Utólag biztos vagyok
abban, hogy a Jáva/CCC3 megoldás a jobb, így vannak a helyükön
a dolgok, bár elismerem, hogy ez csak egy szubjektív vélemény.

Még egy fontos kérdés van: Legyen-e feltétlen és automatikus
konverzió a karaktersorozat és a bájtsorozat között?
A Pythonban van. Azt írják, hogy a két típus találkozásánál
a bájtsorozat automatikusan a ,,pontosabb'' Unicode sorozatra
konvertálódik. Úgy gondolják, hogy a Unicode 32-bites lévén
pontosabb, mint a 8-bites bájt. Ez azonban egy melléfogás.
Az UTF-8 kódolással információveszteség nélkül tárolható
bármely Unicode string, fordítva azonban nem.
Ha egy png formátumú képet tartalmazó bájtsorozatot
Unicode stringre konvertálunk, akkor kép elromlik!
Tehát a Unicode sorozat egyáltalán nem ,,pontosabb'', 
mint a bájtsorozat, hanem fordítva, de hasznosabb azt gondolni, 
hogy más. Ezért a CCC3-ban nincs feltétlen típuskonverzió.

A tapasztalat azt mutatja, hogy az a jó, ha a program 
a lehető legszélesebb körben Unicode stringeket használ,
és csak akkor tér át bájtsorozatra, amikor tényleg
bináris adatokkal dolgozik. 

Amikor olvasunk egy fájlból, azt kell feltételeznünk, hogy
bájtokat kapunk. Még ha tudjuk is, hogy a fájl mit tartalmaz,
a POSIX API bájtok olvasásához ad eszközöket. 
Ezt a szituációt úgy jellemzem, hogy a kétféle típus találkozik,
és az alkalmazásnak kell eldönteni, hogy legyen-e konverzió,
vagy ne. Ilyen a CCC2-ben nem volt, tehát a CCC3 mindenképpen
bonyolultabb lesz, sajnos.

Emlékeztetek rá, hogy a szabány szerint az XML nem bájtok
sorozatából, hanem karakterek sorozatából áll. Amikor a programok
XML dokumentumot cserélnek, akkor valahogy sorosítani kell
a dokumentumot, azaz a karaktereket bájtsorozatra kell
konvertálni, ehhez valamilyen kódolás kell. Az XML szabvány 
az UTF-8-at jelöli ki default kódolásnak. Ahogy a CCC3 használja 
a Unicode/UTF-8 kódolást, úgy minden magától a helyén van. 
A \verb!ccc3_jt! könyvtár Unicode string formájában dolgozik az XML 
szöveggel, majd közvetlenül küldés előtt az \verb!str2bin!-nel 
UTF-8-ra konvertálja. A Jáva XML elemzője a DOM felépítésekor
az UTF-8 kódolású dokumentumot Unicodera konvertálja, 
a DOM-ból már Unicode stringeket lehet kiolvasni.
Hasonló a helyzet a fordított irányban.
A \verb!ccc3_jt! portolásához ezért lényegében semmit sem 
kellett csinálni.

A GTK mindig is UTF-8 kódolással dolgozott, azaz egy editbox
szövegét UTF-8 kódolással kell beállítani, és úgy lehet megkapni.
A CCC-GTK csatolóba be van építve, hogy paraméter átadás/átvételkor
automatikusan végezze a Unicode<-->UTF-8 konverziót.
A CCC3 program tehát kényelmesen dolgozhat a Unicode
stringekkel.

Vannak függvények, amik természetüknél fogva bináris
adatokkal dolgoznak, pl. 
\verb!base64_encode!,
\verb!base64_decode!, 
\verb!crypto_md5!, 
\verb!crypto_sha1!,
\verb!crypto_rand_bytes!,
\verb!savescreen!, stb..

%És nyilván vannak kérdéses esetek, és talán ellentmondások is,
%amiket eddig nem vettem észre.

A fentiek alapján világos, 
hogy a CCC3 nem tud kompatibilis lenni a korábbi verziókkal,
ezért a CCC3 nem tudja egyszerűen leváltani a CCC2-t. 
Képtelenség minden régi programot egyszerre ,,felhozni'' az új verzióra, 
egyúttal mindent újratesztelni. Mégha ez lehetséges  volna,
az sem volna elegendő, ui. a  CCC3  kizárólag  Unicode/UTF-8
kódolással dolgozik. Ha egy bank adatai Latin-2 kódolásúak, nem valószínű, 
hogy a kedvünkért hajlandók mindent átkonvertálni UTF-8-ra.

Érdemes-e esetleg a CCC3-ba beépíteni más kódrendszerek támogatását?
A véleményem határozott nem. Az Unicode/UTF-8 ui. nem arra lett kitalálva,
hogy még eggyel több kódrendszer legyen, amik között ide-oda lehet
konvertálgatni. (Hogy ezzel is több legyen a különböző kódrendszerek
miatti szenvedés.) Az UTF-8 az az univerzális sín, aminek mindent el kell vinnie.


Az ellentmondó szempontok feloldása, hogy a CCC2-t és a CCC3-at
is fenn kell tartani, előreláthatóan még hosszú ideig. A kezelendő
adatok kódolása alapján kell választani egyiket vagy a másikat.
Természetesen új projektekhez UTF-8 kódolást és CCC3-at választunk.




\subsection{Karakter/binary string literálok}

\subsubsection{Karakter string}
\begin{verbatim}
    x1:="Kázmér füstölgő fűnyírót húz."
    x2:="Копирование и распространение"
\end{verbatim}
A fenti értékadások  szövegének
kötelezően UTF-8 kódolásúnak kell lennie, másképp fordítási hiba
keletkezik: \verb!INVALIDENCODING!. Ebből adódóan nem nélkülözhető
az UTF-8/Unicode környezet.
A programokat UTF-8 editorral kell írni (pl. a z-vel),
a régi szövegeket át kell konvertálni. A fordító maga nem konvertál,
csak hibát jelez, ha rossz a kódolás. A fordító az UTF-8 kódolású
szövegből előállítja a Unicode karakterek sorozatát, és ez a sorozat 
(vagyis a C típusú string) lesz a változók új értéke.
C++ szinten a Unicode (UCS) karakterek \verb!wchar_t! 
típusban tárolódnak, ami általában 32 bites. 


\subsubsection{Binary string}
\begin{verbatim}
    x:=a"öt szép szűzlány őrült írót nyúz"
\end{verbatim}
A fenti értékadás eredményeképpen \verb!x! típusa binary string (X),
tartalma pedig a szöveget UTF-8 kódolással reprezentáló bájtsorozat.




\subsection{Karakter/binary string függvények}

%Nem tudok most komplett referenciát adni az újdonságokról,
%mert nem emlékszem mindenre fejből. Ideiglenesen csak annyit akarok 
%leírni, ami már megfelelően érzékelteti a dolgok hangulatát és logikáját.
%A pontos infóhoz állandóan nézni kell a forrásokat, magam is ezt teszem.

Új függvények:
\begin{description}
\item[{\tt bin(code)}]
    A \verb!chr(code)! bináris párja. Egy 0--255 közé eső
    kódból egy bájt hosszú bináris stringet készít.
\item[{\tt arr2bin(a)}]
    A korábbi \verb!_arr2chr!-t pótolja. Most nyilván a karakter
    stringeket is sorosítani kell, az eredmény egy bináris string
    (bájtsorozat).
\item[{\tt bin2arr(x)}]
    \verb!arr2bin(a)! inverze.
\item[{\tt str2bin(c)}]
    Előállítja a \verb!c! karakter string UTF-8 kódolású bináris
    reprezentációját. Ebből információveszteség nélkül visszanyerhető
    az eredeti string. A bináris reprezentáció sok helyen helyettesítheti
    is \verb!c!-t.
\item[{\tt bin2str(x)}]
    \verb!str2bin(c)! inverze. Tudni kell azonban,  ha \verb!x!
    nem érvényes UTF-8 kódolású szöveget tartalmaz, akkor információ
    vész el, pl. egy png képfájl elromlik.
\item[{\tt split(v,sep)}]
    Helyettesíti a megszűnt \verb!wordlist!-et.
    Karakteres és bináris stringekre is működik.
%\item[{\tt }]
%\item[{\tt }]
%\item[{\tt }]
%\item[{\tt }]
%\item[{\tt }]
%\item[{\tt }]
%\item[{\tt }]
%\item[{\tt }]
%\item[{\tt }]
%\item[{\tt }]
%\item[{\tt }]
%\item[{\tt }]
%\item[{\tt }]
%\item[{\tt }]
%\item[{\tt }]
%\item[{\tt }]
%\item[{\tt }]
%\item[{\tt }]
%\item[{\tt }]
\end{description}

Módusult függvények.

\begin{description}
\item[{\tt chr(code)}]
    Egy 32 bites UCS kódból egy egy karakter hosszú stringet készít.
\item[{\tt asc(v)}]
    Ha \verb!v! karakter string, akkor az első karakter UCS kódját adja.
    Ha \verb!v! bináris string, akkor az első bájt értékét adja.
\item[{\tt left(v,n)}]
    Ha \verb!v! karakter string, akkor \verb!v! első \verb!n!
    karakteréből álló részstringet adja.
    Ha \verb!v! bináris string, akkor \verb!v! első \verb!n!
    bájtjából álló bináris részstringet adja.
    Utóbbi esetben, ha \verb!v! egy szöveg UTF-8 kódolású bináris 
    reprezentációja, akkor ez a tulajdonság elromolhat, amennyiben 
    a \verb!left! elvág egy több bájtos UTF-8 kódot.
\item[{\tt len(v)}]
    Ha \verb!v! karakter string, akkor a \verb!v!-ben levő karakterek
    számát adja.
    Ha \verb!v! bináris string, akkor a \verb!v!-ben levő bájtok
    számát adja. 
\item[{\tt replicate(v,n)}]
    A \verb!v! változó karakter és bináris string is lehet,
    az eredmény ennek függvényében C vagy X típusú. A rekordbuffereket
    régen a \verb!space! függvénnyel hoztuk létre. Mivel ennek
    az eredménye C típus, ez most általában nem jó, helyette
    ilyesmit írunk: \verb!replicate(x"20",n)!.
\item[{\tt fread(fd,@buf,n)}]
    Az \verb!fread! nem karaktereket, hanem bájtokat olvas,
    ezért \verb!buf!-ot X típusúra {\it kell\/} inicializálni.
\item[{\tt fwrite(fd,buf,n)}]
    Az \verb!fwrite! nem karaktereket, hanem bájtokat ír,
    ezért, ha C típusú \verb!buf!-ot adunk meg neki, azt automatikusan
    átkonvertálja X típusra \verb!str2bin!-nel.
\item[{\tt convertfspec2nativeformat(f)}]
    Az eredményét mindig átkonvertálja binárisra, ui.
    az operációs rendszernek UTF-8 kódolású fájlspecifikációkat
    lehet megadni.
\item[{\tt hashcode(v)}]
    Karakteres és bináris stringekre is működik.
\item[{\tt isalpha(v)}]
    Karakteres és bináris stringekre is működik.
    C szinten az \verb!iswalpha!, illetve az \verb!isalpha!
    hívódik meg. Karakter string esetén az ékezetes és cirill 
    betűkre is jó eredményt ad.
\item[{\tt qout(c,...)}]
    A karakter stringek kinyomtatás előtt automatikusan
    átkonvertálódnak UTF-8-ra (vagyis  binárisra), 
    ui. az operációs rendszerek ezt értik.
\item[{\tt savescreen()}]
    A képernyő bináris stringként mentődik, egy screen
    cella a korábbiaktól eltérően most 4 bájt, mert UCS
    kódokat kell tárolni. A függvénycsalád összes tagja
    ehhez alkalmazkodik.
\item[{\tt upper(v)}]
    Karakteres és bináris stringekre is működik.
    C szinten a \verb!towupper!, illetve a \verb!toupper!
    hívódik meg. Karakter string esetén az ékezetes és cirill 
    betűkre is jó eredményt ad.
\item[{\tt val(x)}]
    Karakteres és bináris stringekre is működik.
\item[{\tt valtype(v)}]
    Bináris stringre "X"-et ad.
\item[{\tt like()}]
    Karakteres és bináris stringekre is működik.
\item[{\tt memoread(fspec [,binopt])}]
    Ha a \verb!binopt! empty, akkor \verb!bin2str!-rel
    karakterre konvertálja a beolvasott fájl tartalmát.
    Ez csak akkor jó, ha a fájl UTF-8 kódolású szöveget tartalmaz.
    Ha egy png képfájlt akarunk beolvasni, akkor azt 
    \verb!binopt:=.t.!-vel tesszük, az eredmény ilyenkor
    egy bináris string.
\item[{\tt memowrit(fspec,v)}]
    Ha \verb!v! egy karakteres string, akkor azt kiírás
    előtt átkonvertálja binárisra.
\item[{\tt inkey() }]
    Az inkey kódok megváltoztak, lásd az \verb!inkey.ch!-t.
%\item[{\tt }]
%\item[{\tt }]
%\item[{\tt }]
%\item[{\tt }]
%\item[{\tt }]
%\item[{\tt }]
%\item[{\tt }]
%\item[{\tt }]
%\item[{\tt }]
%\item[{\tt }]
%\item[{\tt }]
%\item[{\tt }]
%\item[{\tt }]
%\item[{\tt }]
%\item[{\tt }]
%\item[{\tt }]
%\item[{\tt }]
%\item[{\tt }]
%\item[{\tt }]
%\item[{\tt }]
%\item[{\tt }]
\end{description}






\subsection{Internacionalizálás}

Internacionalizálásnak sok összetevője van, 
mi itt csak egy dologgal foglalkozunk: 
Hogyan lehet többnyelvű CCC programot írni, 
amiben a string konstansok egyszerűen cserélhetők különféle 
nyelvi változatokra. 
Egy működő példa található a \$CCCDIR/tutor/nlstext directoryban, 
ezt a példát magyarázom el részletesen az alábbiakban.

Az nlstext.prg program:
\begin{verbatim}
static x:=@"Some like it hot"

function main()
    nls_load_translation("nlstext")
    fun()
    ?

static function fun()
static y:=@'Gentlemen prefer blondes'
local z:=@"Star war"
    ? x
    ? y
    ? z
    ? @"Matrix"
    ?
\end{verbatim}
Először is azokat a stringeket, amiket a program különböző 
nyelvű verzióiban fordításban akarunk látni, meg kell jelölnünk.
Erre szolgál a speciális \verb!@"..."! szintaktika.
A programfordítás idejére beállítjuk az alábbi 
környezeti változót:
\begin{verbatim}
export CCC_NLSTEXT_TAB=$(pwd)/nlstext.tran
\end{verbatim}
Ennek hatására a \verb!ppo2cpp! fordító kigyűjti nekünk a 
kukaccal megjelölt stringeket egy szövegfájlba, 
esetünkben nlstext.tran-ba:
\begin{verbatim}
"Some like it hot"<<"" from  ./nlstext.prg  (21)
"Gentlemen prefer blondes"<<"" from  ./nlstext.prg  (33)
"Star war"<<"" from  ./nlstext.prg  (34)
"Matrix"<<"" from  ./nlstext.prg  (39)
\end{verbatim}
Itt soronként egy stringet találunk. A sor a lefordítandó
stringgel kezdődik, utána jön egy \verb!<<! jel, majd egy 
üres idézet, ahová a fordítást kell majd beírni.
Az eddigiek azt jelölik, hogy a bal oldali stringet 
helyettesíteni fogja a jobb oldalra írt fordítás.
A sor végén fel van tüntetve, hogy az adott string melyik
forrásfájl melyik sorából származik.  Természetesen,
ha a project sok forrásfájlból áll, akkor az egyes fájlokból
jövő járulék halmozódik, ezért egy nagyobb program esetén
ezres nagyságrendű sor lehet az eredmény.

Minden nyelvhez készítünk egy-egy directoryt, esetünkben 
\begin{verbatim}
    translation/hu
    translation/ru
\end{verbatim}
ezekbe átmásoljuk az nlstext.tran egy-egy példányát, 
ezeken fognak dolgozni a fordítók. A fordító munkájának
eredménye egy ilyen fájl:
\begin{verbatim}
"Some like it hot"<<"Несколько мужчин любят горячо" from  ./proba.prg  (21)
"Gentlemen prefer blondes"<<"Господа любят лучше блондинок" from ./proba.prg (33)
"Star war"<<"Война эвёэд" from  ./proba.prg  (34)
\end{verbatim}
Ebből a fájlból a \verb!tran2cpp! utility  C++ forrást generál,
amit lefordítunk, és  dinamikus könyvtárat linkelünk belőle.
Elvégezzük ugyanezeket a műveleteket a magyar változatra is.
A dinamikus könyvtárak neve:
\begin{verbatim}
    translation/libnlstext.hu.so
    translation/libnlstext.ru.so
\end{verbatim}
Természetesen ugyanez megy Windowson is, csak ott dll-eket kapunk.

Namost, ha az nlstext.exe programot egy ilyen scripttel indítjuk:
\begin{verbatim}
#!/bin/bash
export CCC_LANG=ru
export LD_LIBRARY_PATH=./translation:$LD_LIBRARY_PATH
nlstext.exe
\end{verbatim}
akkor a program elején található
\begin{verbatim}
    nls_load_translation("nlstext")
\end{verbatim}
függvényhívás (amiről eddig nem szóltunk) a \verb!CCC_LANG!
változó értékéből és a paraméterként kapott "nlstext" szövegből
összerak egy könyvtárnevet, és a könyvtárat megpróbálja betölteni. 
Ha ez a betöltés sikeres, akkor a program a \verb!@"..."! stringek
helyett azok fordításait fogja megjeleníteni. Ha a fordításkönyvtár
dinamikus betöltése nem sikeres, vagy a könyvtár nem tartalmaz fordítást
egyik vagy másik stringre, attól még  működni fog a program,
csak ekkor a fordítással nem rendelkező stringek eredeti szövege 
jelenik meg.

