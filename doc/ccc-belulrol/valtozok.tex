

\section{Változók}


\subsection{Deklaráció}

Minden változót deklarálni kell.
A deklaráció végrehajtható utasítás, helyet csinál a változónak, 
és végrehajtja a változó inicializátorát.
Az inicializátor tetszőleges kifejezés lehet,
ha nincs megadva explicit inicializátor, akkor a változó NIL-re inicializálódik.


Kétféle deklaráció lehetséges, \verb!local! és \verb!static!.

\begin{verbatim}
/*
namespace  ...          // itt állhatna
using      ...          // itt állhatna
*/

static k1               // külső, NIL-re inicilizálódik
static k2:=init_k2()    // a visszatérési értékkel inicializálódik

function f1()
local a, b:=1           // egyszerre több, némelyik inicializálva
static c, d             // belső 
    ? a,b,c,d,k1,k2     // kiírja az értéküket
    ...

function f2(a1,a2,a3)
    //? a,b,c,d         // ezek nem láthatók
    ? a1,a2,a3          // nem kell őket külön deklarálni (local)
    ? k1,k1             // ok, ezek láthatók
    ...
\end{verbatim}
A \verb!static! deklarációk  is kétfélék:
\begin{description}
\item[külsők] a forrásmodul elején, 
        az esetleges \verb!namespace! és \verb!using! után,
        de az első \verb!function! vagy \verb!class! utasítás előtt állhatnak,
\item[belsők] 
        függvények belsejében vannak, 
        közvetlenül a \verb!function! utasítás után, 
        esetleg keveredve a \verb!local! deklarációkkal,
        de megelőzve az összes nem deklarációs utasítást.
\end{description}

A külső static változók csak a forrásmodulon belül, de ott minden függvényből láthatók. 
A belső static változók csak az adott függvényen belül láthatók. 
Egyáltalán nincsenek az egész programra kiterjedő globális láthatóságú változók.
A static változók inicializátora a program futása során egyszer hajtódik végre.
A static változók a program futása alatt nem vesztik el értéküket
(vagyis mindig őrzik az utolsó értékadással kapott értéket).
A belső static változók a függvény első hívásakor a leírt sorrendben inicializálódnak.
A külső static változók akkor inicializálódnak, amikor a program először hivatkozik 
az értékükre. A külső static változók inicializálása szinkronizált. A szinkronizáció 
biztosítja, hogy a külső static-ok többszálú programban se inicializálódjanak többször.
A belső static-oknál a szinkronizáció az alkalmazásra van bízva.

Egy local deklaráció mindig belső. 
A local deklarációk a belső static deklarációk helyén 
(azokkal esetleg keveredve) fordulhatnak elő. 
A local változók csak az adott függvényen belül láthatók. 
A local változók a függvény minden hívásakor inicializálódnak, ha másra nem, hát NIL-re.
Egyszerűen szólva, a \verb!local! utasítás közönséges stack változókat deklarál.

Ha egy változót deklaráció nélkül próbálnánk használni, 
a fordító hibát jelez.

A függvények argumentum változóit nem kell (nem is lehet) külön deklaráni, 
ugyanúgy viselkednek mint a local változók.

Névegyezés esetén a local és a belső static változók eltakarják
a külső static változókat.

A változónevekben a kis- és nagybetűk egyformának számítanak.
Valójában az egész nyelv case insensitive, ami Clipper örökség,
és a kompatibilitás kényszere folytán nem változtatható.
A CCC-ben  megszokott, hogy túlnyomóan kisbetűket használunk.

A változó- és függvényneveket ugyanúgy képezzük, mint C-ben.
Néhány fenntartott kulcsszót 
(\verb!if!, \verb!while!, \verb!for!, \ldots) 
kerülni kell.
A változó- és függvénynevek nem zavarják egymást (nem tudnak ütközni).





\subsection{Értékek}


A Clipper/CCC-ben nem a változóknak, hanem elsősorban az {\em értékeknek\/} 
van típusa. Ez alapján persze utólag mondhatjuk, hogy egy változónak az a típusa, 
ami a benne tárolt értéknek. Az így értelmezett típus azonban minden
értékadáskor megváltozhat. A változók uniformok: Akármelyik változó képes tartalmazni 
akármilyen értéket. Ennek megfelelően a fordító nem foglalkozik a változók 
típusának ellenőrzésével. 

Nézzük, mi van egy értékben
(teljes infó a variable.h-ban):

\begin{itemize}
\item Tartalmazza az érték típusát.
\item Fix méretű adatoknál (mint szám, dátum, logikai) 
      tartalmazza magát az adatot.
\item Változó méretű adatoknál (mint string, array)
      egy memóriaobjektum referenciát (OREF).
\item Referencia változóknál egy VALUE referenciát (VREF).
\end{itemize}


Úgy kell elképzelni, hogy egy változó a számára fenntartott memóriában
tartalmazza az előbb leírt struktúrájú {\em értéket}. Amikor a program
végrehajtja az 

\begin{verbatim}
    a:=b
\end{verbatim}
{\em értékadást}, akkor a \verb!b!-hez tartozó memóriaterületen tárolt {\em érték\/}
átmásolódik az \verb!a!-hoz tartozó memóriaterületre. Vagyis \verb!a! átveszi 
\verb!b! típusát és minden adatát.  Különösen érdekes, amikor az adat egy referencia.
Ilyenkor a két változó közösködik ugyanazon a tartalmon.


\subsection{Típusok}

Akármi is van a változóban,
a \verb!valtype()! függvénnyel lekérdezhető az aktuális típusa. 
\verb!valtype()! visszatérési értéke karakter típusú. 
A következő típusok vannak: 
NIL, logikai, szám, dátum, pointer, karakter (Unicode string),
binary (bájt string), array (érték tömb), kódblokk, objektum, referencia. 
Ezeket vesszük sorra az alábbiakban.


\subsubsection{NIL}
\begin{verbatim}
    valtype(NIL)    // --> "U"
    len(NIL)        // --> 0
    empty(NIL)      // --> .t.
    NIL==NIL        // --> .t.
\end{verbatim}

Az explicite nem inicializált változók értéke NIL, típusjele U.
A \verb!return! utasítás nélkül befejeződő függvények visszatérési
értéke NIL. A NIL hossza 0, a NIL üres, a NIL bármivel (bármilyen típussal)  
összehasonlítható, de csak saját magával egyenlő. (Általában
csak azonos típusú változók hasonlíthatók össze, a NIL ebben kivételes.)
A NIL-t már a típusa azonosítja, nem tartalmaz semmilyen adatot.


\subsubsection{Logikai}
\begin{verbatim}
    valtype(.t.)    // --> "L"  (logikai true literál)
    valtype(.f.)    // --> "L"  (logikai false literál)
    empty(.t.)      // --> .f.
    empty(.f.)      // --> .t.
\end{verbatim}

Logikai értéket adnak az összehasonlító operátorokkal képzett kifejezések.
A logikai értékekkel használható operátorok:
\begin{verbatim}
    x .and. y    // --> .t., ha x==.t. és y==.t.
                 // ha x==.f., akkor y nem értékelődik ki

    x .or.  y    // --> .t., ha x==.t. vagy y==.t.
                 // ha x==.t., akkor y nem értékelődik ki
                 
    ! x          // --> .t., ha x==.f.
    .not. x      // --> .t., ha x==.f. (ugyanaz bőbeszédűen)
\end{verbatim}

A logikai műveletek precedenciája a szokásos, lehet zárójelezni.


\subsubsection{Szám}

\begin{verbatim}
    valtype(3.141592)   // --> "N"  (szám literál)
    empty(0)            // --> .t.  (csak a 0 üres)
    empty(1)            // --> .f.
\end{verbatim}

A számok 64 bites lebegőpontos formában (double) tárolódnak. 
Többféleképpen is be lehet írni a programba szám literálokat, 
de a megadás módjától függetlenül a tárolás mindig double.

\begin{verbatim}
    0xffff              // hexadecimális (mint C-ben)
    0b11111111          // bináris (255)
\end{verbatim}

A számokra működnek a szokásos összehasonlító műveletek.
\begin{verbatim}
    x == y              // --> .t., ha egyenlőek
    x != y              // --> .t., ha eltérnek
    x <  y              // kisebb
    x <= y              // kisebbegyenlő
    x >  y              // nagyobb
    x >= y              // nagyobbegyenlő
\end{verbatim}

Megvannak a szokásos aritmetikai operátorok.
\begin{verbatim}
    x +  y              // összeadás
    x -  y              // kivonás
    x *  y              // szorzás
    x /  y              // osztás
    x %  y              // moduló
    x ** y              // hatványozás
    ++x                 // növelés (mint C-ben)
    --x                 // csökkentés (mint C-ben)
    x++                 // növelés utólag (mint C-ben)
    x--                 // csökkentés utólag (mint C-ben)
\end{verbatim}
A műveletek precedenciája ugyanaz, mint C-ben, és természetesen ugyanúgy
alkalmazható a zárójelezés.

A C-hez hasonlóan az aritmetikai operátoroknak van  értékadással
kombinált változata:
\begin{verbatim}
    x += y              // x:=x+y (hozzáad)
    x -= y              // x:=x-y (levon)
    x *= y              // x:=x*y
    x /= y              // x:=x/y
    x %= y              // x:=x%y
\end{verbatim}

Vannak számokon működő matematikai függvények:
\verb!power()!,
\verb!exp()!,
\verb!log()!,
\verb!sqrt()!,
trigonometrikus függvények,
\verb!round()!,
\verb!random()!.

Van néhány függvény bitek manipulálására:
\verb!isbit()!,
\verb!setbit()!,
\verb!clearbit()!,
\verb!numand()!,
\verb!numor()!,
\verb!numxor()!,
\verb!numnot()!.

Az \verb!str()! és \verb!transform()! 
függvények számot karakter stringre konvertálnak.
A \verb!val()! függvény  a string elején álló decimális
számból kiolvassa a számértéket.



\subsubsection{Dátum}

\begin{verbatim}
    set date format "yyyy-mm-dd"
    valtype(ctod("2009-11-20"))     // --> "D"  (nincs dátum literál)
    valtype(stod("20091120"))       // --> "D"
    empty(ctod(""))                 // --> .t.
    empty(date())                   // --> .f.
\end{verbatim}

Dátumokat nem tudunk közvetlenül (literálként) beírni a programba,
hanem a \verb!ctod()! vagy \verb!stod()! konverziós függvénnyel állítjuk
elő a string alakból. (E függvényeknek az inverze is létezik: 
\verb!dtoc()!, \verb!dtos()!.)

A dátumkezelés 1900-tól 2100-ig működik jól. Ebben az időszakban
helyesen kezeli a szökőéveket.


A dátumokra értelemszerűen működnek az összehasonlító operátorok.

A dátumokon korlátozottan lehet aritmetikai műveleteket végezni.
Az alábbi példában \verb!n! szám típusú, \verb!d1! és \verb!d2!
dátum típusú:
\begin{verbatim}
    n  := d2 - d1   // --> a két dátum különbsége napokban
    d2 := d1 + n    // --> d1 dátum plusz n nap
    d2 := d1 - n    // --> d1 dátum minusz n nap
\end{verbatim}

Van egy rakás dátumokon működő függvény:
\verb!doy()!,
\verb!eom()!,
\verb!eoy()!,
\verb!bom()!,
\verb!boy()!,
\verb!day()!,
\verb!dow()!,
\verb!cdow()!,
\verb!month()!,
\verb!addmonth()!,
\verb!cmonth()!,
\verb!year()!.


\subsubsection{Pointer}

\begin{verbatim}
    valtype(p)      // --> "P"  (nincs pointer literál)
    empty(p)        // --> .t.  (csak a null pointer üres)
\end{verbatim}

A pointereket interfészekben használjuk. Arra való, hogy C könyvtárak
adatait (tipikusan egy struktúrára mutató pointert) tároljunk benne Clipper szinten.
Példa: Az Oracle minden SQL utasításhoz rendel egy ún. statement handle-t (pointer).
A program erre a statement handle-re kell hivatkozzon, amikor műveleteket
akar végezni az SQL utasítással, pl.  le akarja kérdezni egy select
eredménysorait. Egy ilyen statement handle tárolása oldható meg a P típusú
változóval.


\subsubsection{Karakter (Unicode string)}
\begin{verbatim}
    valtype("Van, aki forrón szereti")  // --> "C"
    len("")                             // --> 0
    len("123")                          // --> 3
    empty("")                           // --> .t.
    empty("  ")                         // --> .t. (csak blank)
    empty("x")                          // --> .f.
\end{verbatim}

Példák:
\begin{verbatim}
local x:='ТЕРМИНЫ И УСЛОВИЯ КОПИРОВАНИЯ'
local y:=@"Some like it hot"  //nlstext

local text:=<<tetszoleges_symbol>>
    Itt tetszőleges (UTF-8 kódolású) szöveg lehet,
    idézőjelek, tab, soremelés, akármi,
    kivéve a kezdő <<tetsz...>> (nem írhatom oda) jelet,
    mert az lezárja a string literált.
<<tetszoleges_symbol>> //lezárva
\end{verbatim}


A karakterváltozók (stringek) Unicode karakterek sorozatát tartalmazzák.
A karakterváltozók tudják a saját hosszukat. Nincs lezáró 0, nincs vizsgálva 
a karakterek érvényessége, ezért a karakterek között akármilyen kódérték 
előfordulhat. A karakterváltozók hossza csak azért van korlátozva, hogy az 
elszabadult programok ne fektessék ki a rendszert. Jelenleg egy karakterstring 
maximális mérete 64MB. 

A karakterliterálokat UTF-8 kódolással kell beírni. 
A fordító végzi el a konverziót UTF-8-ról Unicode (\verb!wchar_t!) tömbre.
Ha az  UTF-8 kódolás hibás, a fordító INVALIDENCODING hibát ad.

A stringeket aposztróf vagy macskaköröm határolja.
Aposztróffal határolt string tartalmazhat macskakörmöt, és fordítva.
A stringben nem lehet escape szekvenciákkal trükközni.

Alapesetben a stringek egysorosak. 
Hosszabb stringeket részeiből összeadással és folytatósorokkal lehet képezni.

A hosszabb (többsoros) stringek egyszerű beírását teszi lehetővé 
a \verb!<<SYMBOL>>! típusú határoló.

A \verb!@"..."! alakú stringekből (nlstext) a fordító készít egy hash táblát, 
amiben a string értéke a kulcs. A kulcsokhoz különféle nyelvű szövegeket
(fordításokat) lehet kapcsolni, amiket a program futás közben felszed,
és automatikusan behelyettesít, így ugyanaz a program különféle nyelveken 
jelenhet meg.

Bárhogy is hoztuk létre a stringet,  ugyanazokat a dolgokat lehet
csinálni vele.


Összehasonlítás:
\begin{verbatim}
    x == y      // egyenlő-e (ez az egy normális)
    x != y      // a jobboldal hosszában eltér-e
    x <= y      // a jobboldal hosszában! 
    x <  y      // a jobboldal hosszában! 
    x >= y      // a jobboldal hosszában! 
    x >  y      // a jobboldal hosszában! 
\end{verbatim}

A stringek rendezése a Unicode kódértékek szerint lexikografikusan történik.
A \verb!==! kivételével a többi összehasonlító operátor úgy működik,
hogy először a baloldalt levágja a jobboldal hosszára, és az így kapott
x-szel végzi az összehasonlítást, tehát:

\begin{verbatim}
    "abc" != "ab"   // --> .f.
    "ab"  != "abc"  // --> .t.
    "abc" >  "ab"   // --> .f.
    "abc" <= "ab"   // --> .t.
\end{verbatim}
Sajnos ezek elég zavaró dolgok, de Clipper örökség,
és a kompatibilitás kényszere miatt nem lehet rajta változtatni.


Összeadás:
\begin{verbatim}
    y:="Some like "+"it hot"
    x:="Some like "
    x+="it hot"
    x==y  // --> .t.
\end{verbatim}

Indexelés, szeletek:
\begin{verbatim}
    "Some like it hot"[1]       // --> "S", indexelhető, 1-től indul
    "Some like it hot"[2..4]    // --> "ome", szelet
    "Some like it hot"[11..]    // --> "it hot"
    "Some like it hot"[..11]    // --> "Some like i"
    "Some like it hot"[..]      // --> másolat az egészről
\end{verbatim}
A túlindexelés runtime errort okoz. 
A szeletek túllógó indexei módosulnak a tényleges méretekhez alkalmazkodva. 
Nincs trükközés, hátulról számolás, meg effélék.

Részstring tartalmazás:
\begin{verbatim}
    "like" $ "Some like it hot" // --> .t.
\end{verbatim}


A stringeken működő fontosabb függvények: 
\verb!strtran()!, 
\verb!stuff()!, 
\verb!substr()!, 
\verb!left()!, 
\verb!right()!, 
\verb!padr()!,
\verb!padl()!, 
\verb!ltrim()!, 
\verb!rtrim()!, 
\verb!alltrim()!, 
\verb!at()!, 
\verb!rat()!, 
\verb!len()!, 
\verb!replicate()!,
\verb!upper()!,
\verb!lower()!,
\verb!isalpha()!,
\verb!isdigit()!,
\verb!isalnum()!,
\verb!isupper()!,
\verb!islower()!.

\verb!upper()!, \verb!isupper()! és társaik értik a Unicodeot, 
ezért cirill betűs stringre is jól működnek.

A \verb!chr(code)! függvény visszaad egy egy hosszúságú stringet,
ami a megadott kódértékű (Unicode) karaktert tartalmazza.
Az \verb!asc(str)! függvény visszaadja a bemeneti string első karakterének
kódértékét.


A karakterstring értékekben közvetlenül tárolódó adat 
(ahogy korábban már szó esett róla) egy OREF. Az \verb!a:=b!
értékadás után az \verb!a! és \verb!b! és változó közösködik
az objektumreferencián. Ugyanaz az értékük, ráadásul a string csak egy 
példányban létezik a memóriában. Mi történik, ha az egyik változót módosítjuk?
\begin{verbatim}
    b+="próba szerencse"
\end{verbatim}
Változik-e \verb!b!-vel együtt az \verb!a!-is? Nem, a változók útja szétválik.
Lesz egy külön memóriaobjektuma \verb!a!-nak és egy másik a \verb!b!-nek.
Ez az értelme annak a más nyelvek dokumentációiban olvasható, kissé rejtélyes
kijelentésnek, hogy a ,,karakterstring (érték) nem módosítható''. 


\subsubsection{Binary (bájt string)}
\begin{verbatim}
    valtype(a"Próba szerencse")     // --> "X"
    valtype(x"0d0a")                // --> "X" (CR/LF)
    len(a"")                        // --> 0
    len(a"123")                     // --> 3
    empty(a"")                      // --> .t.
    empty(a"  ")                    // --> .t. (csak blank)
    empty(a"x")                     // --> .f.
\end{verbatim}

A binary változók (bájt stringek) bájtok sorozatát tartalmazzák.
Tudják a saját hosszukat. Nincs lezáró 0, ezért a bájtok között akármilyen érték 
előfordulhat. A binary változók hossza csak azért van korlátozva, hogy az 
elszabadult programok ne fektessék ki a rendszert. Jelenleg egy binary string
maximális mérete 64MB. 

Legegyszerűbben az \verb!a"..."! alakban írhatók be egy programba:
\begin{verbatim}
    x:=a"Van, aki 'forrón' szereti"
    y:=a'Van, aki "forrón" szereti'
\end{verbatim}

A karakterliterálokhoz képest az a különbség,
hogy most nincs UTF-8 --> Unicode konverzió, 
a tartalom nem karaktertömb, hanem bájttömb.

A stringeket aposztróf vagy macskaköröm határolja.
Aposztróffal határolt string tartalmazhat macskakörmöt, és fordítva.
Nem értelmezünk semmilyen escape szekvenciát.

Hexadecimális kódokkal is megadhatjuk a bájtsorozatot a \verb!x"..."!
formában. Ennek kötelezően páros számú betűt kell tartalmaznia 
(minden bájt egy kétjegyű hexa szám: 00-ff), a kis-nagybetű nem számít.

A binary változókra ugyanazok az operátorok és függvények működnek,
mint a karakterváltozókra, csak értelemszerűen karakterek helyett bájtokkal.
Pl. \verb!asc()! most nem az első karakter, hanem az első bájt kódértékét adja.
A \verb!bin(code)! függvény ad egy egyelemű bájtsorozatot, melyben az egyetlen
elem értéke a \verb!code!.

\begin{verbatim}
    binvar:=str2bin(chrvar)
    chrvar:=bin2str(binvar)  //visszaadja az eredetit
\end{verbatim}

Az \verb!str2bin()! függvény a bemeneti karaktersorozatból
előállít egy bájtsorozatot, ami a stringet UTF-8 kódolásban ábrázolja.

A \verb!bin2str()! függvény {\em feltételezi}, hogy a bemenete egy UTF-8 
kódolású szöveg, és UTF-8 --> Unicode konverzióval előállítja a bájtsorozatnak
megfelelő karaktersorozatot. Ha a bemenet mégsem kifogástalan UTF-8 kódolású,
akkor a hibák helyén ? karakterek jelennek meg a kimeneten.  Emiatt az 
str2bin és bin2str nem egymás inverzei. Pl. egy png képet tartalmazó 
binary változót nem lehet oda-vissza karakterre konvertálni, mert elromlik a kép.

A bájtsorozatok és karakterek között nincs automatikus konverzió, 
ezért pl. nem lehet őket összeadni vagy összehasonlítani.

A karakterekhez hasonlóan a bájt string értékekben is a közvetlenül tárolódó 
adat egy OREF. Ugyanolyan értelemben most is mondhatjuk, hogy (első közelítésben)
a bájt stringek nem módosíthatók.  Igazából természetesen minden módosítható.
Van rá API, hogy a programok dogozhassanak a bufferen belül.


\subsubsection{Array}

\begin{verbatim}
    valtype({1,"2",.f.,{}})     // --> "A"
    len({})                     // --> 0
    len(array(1000))            // --> 1000
    empty({})                   // --> .t.
    empty(array(1000))          // --> .f.
\end{verbatim}

Az array (tömb) típus Clipper értékek sorozatát tartalmazza.
A tömbelemek típusa bármi lehet, bármilyen értéket tartalmazhat
elemként, akár saját magát is (ami persze végtelen rekurziót okoz
a tömb kiprintelésekor).

\begin{verbatim}
    a:={}               // üres array
    {1,2,3,4}[3]        // --> 3, az indexek 1-től indulnak
    len({1,2,3,4})      // --> 4, tudja a saját hosszát
    array(10)           // NIL-ekkel inicializált, 10 elemű
    a[i]:=x             // értékadás egy tömbelemnek
    a[i][j]             // a i-edik elemének j-edik eleme
    aadd(a,x)           // a végéhez ad egy új elemet
    asize(a,len(a)+1)   // hossz növelése
\end{verbatim}

A futtatórendszer minden indexelést ellenőriz, 
túlindexelés esetén runtime error keletkezik. 

Az inicializálatlan tömbelemek értéke NIL.

A tömbökön használható fontosabb függvények:
\verb!array()!,
\verb!asize()!,
\verb!adel()!,
\verb!aadd()!,
\verb!atail()!,
\verb!ascan()!,
\verb!asort()!.

Az array típusú értékekben a közvetlenül tárolt adat egy OREF.
Az OREF-hez tartozó memóriaobjektumban a tömbelemek szépen, 
katonásan egymás után sorakoznak.
A stringekre azt mondtuk, hogy nem változtathatók.
A tömbök ezzel szemben igen, a különbséget az alábbi példa
szemlélteti.

\begin{verbatim}
    a:=array(10)
    b:=a
    b[5]:="HOPP"
    ? a[5]              // kiírja: HOPP
\end{verbatim}


Láttuk, hogy a tömbök mérete változhat. 
Emiatt a futtatórendszer időnként az egész tömböt 
kénytelen áthelyezni új memóriacímre.

\subsubsection{Kódblokk}

\begin{verbatim}
    valtype({||NIL})    // --> "B"
    len({||x})          // --> 1    (ref változók száma)
    empty({||NIL})      // --> .f.  (sosem üres)
\end{verbatim}

Amit más nyelvekben (Lisp, Python, Smalltalk,\ldots) úgy hívnak, hogy closure vagy
lambda-függvény, az a Clipperben a kódblokk. Általános alakja: 
\begin{verbatim}
    blk:={|p1,p2,...|expr1,expr2,...}
\end{verbatim}
A \verb!||! jelek között vannak felsorolva a kódblokk paraméterei (lehet üres).
Ezután kifejezések vesszővel elválasztott listája következik. A kifejezésekben 
a kódblokk paraméterei, plusz a kódblokk definiálásának helyéről látható, érvényes
programelemek (literálok, változók, függvények) szerepelhetnek. 

A kódblokk értelme, hogy ki lehet őt értékelni. Az előbbi \verb!blk!
változót átadjuk az \verb!eval! függvénynek:
\begin{verbatim}
    eval(blk,a1,a2,...)  // --> az utolsó kifejezés értéke
\end{verbatim}
Az a1, a2, \ldots értékek behelyettesítődnek, a p1, p2, \ldots változókba,
majd kiértékelődnek az expr1,expr2,\ldots kifejezések. Az utolsó kifejezés
értéke lesz az eval függvény visszatérési értéke.

A Clipper/CCC-ben nincs olyan, hogy függvénypointer. Ha a programban leírunk 
egy függvénynevet  zárójelpár (azaz függvényhívás operátor) nélkül, 
akkor az szándékainkkal ellentétben nem a függvényt jelenti, hanem egy változónevet, 
és hibát kapunk, ha az adott néven nincs változó deklarálva. 

A kódblokkok legegyszerűbb alkalmazása függvények paraméterként való átadása.
Nézzünk egy összeadó függvényt:
\begin{verbatim}
function osszead(a,b)
    return a+b
\end{verbatim}
Ebből készítünk egy kódblokkot:
\begin{verbatim}
    blk:={|p1,p2|osszead(p1,p2)}
\end{verbatim}
A blk változó értéke átkerül a program egy másik helyére, 
pl. visszatérési értékként, vagy függvény\-paraméterként. 
Ezen a másik helyen így értékelhető ki:
\begin{verbatim}
    eval(blk,1,2) // --> 3
\end{verbatim}

Rendkívül érdekes tulajdonsága a kódblokkoknak, hogy a blokk kifejezéslistájában
szerepelhetnek változók abból a láthatósági körből, ahol a kódblokkot
definiáltuk. A kódblokk révén ezek a változók láthatóvá válnak olyan helyen,
ahol egyébként normálisan nem volnának láthatók: Külső static változó egy másik 
modulban, vagy belső változó egy másik függvényben. Úgy szoktuk mondani,
hogy a kódblokkba ,,belerefesednek'' a változók (lásd a referencia változókat).

\begin{verbatim}
function main()
local x,blk:={|p|x+=p}

    x:="a"
    ? proba1(blk)   //kiírja: a-proba1
    ? x             //kiírja: a-proba1

    x:="b"
    ? proba1(blk)   //kiírja: b-proba1
    ? x             //kiírja: b-proba1
    ?
    
function proba1(blk)    
    return eval(blk,"-proba1")
\end{verbatim}
Most van min tűnődni:
Kétszer (ugyanúgy?) meghívtuk a \verb!proba1! függvényt,
ám a két alkalommal más-más eredményt kaptunk. 
Ráadásul, hogyan változhat meg az \verb!x! értéke,
ami a \verb!main!-en belül lokális, és így máshol nem látható?


A kódblokk egy másik tulajdonságára világít rá a következő példa:
\begin{verbatim}
function main()
local blk:=proba1()
    ? eval(blk,"HOPP")  // kiírja "Próba szerencse:HOPP"
    ?

function proba1()    
local x:="Próba szerencse:"
    return {|p|x+p}
\end{verbatim}
A nagy kérdés itt, hol tárolódik a "Próba szerencse:" érték?
A kódblokk kiértékelésekor (\verb!main!-ben) a \verb!proba1! már visszatért,
tehát az \verb!x! változó már nem létezik. A válasz: az érték a kódblokkban tárolódik.


A kódblokkban közvetlenül tárolódó adatok:
\begin{itemize}
 \item kódpointer a kifejezéslistára (erre adja a vezérlést az \verb!eval()!),
 \item OREF a kifejezéslistában szereplő változókból képzett tömbre.
\end{itemize}
A kódblokkba kerülő változók referenciák. Ez azt jelenti, hogy az eredeti
és a kódblokkban tárolt példányok együtt változnak (ez a ,,belerefesedés'').

A kódblokk a Clipper nagyon sokoldalúan használható eszköze.
Többek között a kódblokkokon alapul az objektum-metódusok implementációja.




\subsubsection{Objektum}

Vannak ún. {\em  objektum alapú\/} nyelvek.
Ezekben egy objektumot úgy képzelhetünk el, mint egy hashtáblát.
Az objektum-hashtáblában az attribútumok/metódusok neve a kulcs.
A hash a kulcs mellé az attribútum értékét, illetve a metódus
implementációját képviselő kódblokkot rendeli. A metódushívás
a kódblokk automatikus kiértékelésével történhet. Minden objektum
egyedi, bármikor bővíthető új attribútum/metódussal.

Ezzel az egyszerű megközelítéssel szemben a CCC-ben {\em osztályok\/} vannak.
Objektumok helyett  az osztályok leírására használunk hashtáblát.
A hashben most is az attribútumok/metódusok neve a kulcs,
az attribútum kulcsokhoz azonban most egy indexet rendelünk.
Ennél az indexnél található meg az érték az objektum attribútumai között.
Maga az objektum csak az attribútumaiból áll. 
A lényeg, hogy {\em megkülönböztetjük}, és elkülönítve tároljuk {\em az osztály- 
és objektum-információt}.


\begin{verbatim}
    valtype(o:=errorNew())  // --> "O"
    len(o)                  // --> 14   (attribútumok száma)
    empty(o)                // --> .f.  (len(o)==0)
    o:classname             // --> "error"
\end{verbatim}

Objektumokat nem operátorral gyártunk
(nincs new operátor), hanem objektumgyártó függvénnyel.
Ezek nevét az osztály nevéből (a példában \verb!error!) 
és a \verb!New! szóból rakjuk össze. 
Az xNew függvény visszatérési értéke az x osztályú
objektum. A régi Clipperben is ez volt a helyzet, az error objektumokat 
az \verb!errorNew()! függvény adta. A kompatibilitás érdekében megtartottuk
a sémát. Az objektumgyártó függvényt a fordító generálja az osztálydefinícióból.

Az objektumok \verb!valtype()! szerinti típusa osztályuktól 
függetlenül mindig O. 

Minden osztály az \verb!object! osztály leszármazottja, ezért mindnek
van \verb!classname! metódusa, amivel lekérdezhető az osztály neve.


A futtatórendszer statikusan tárolja az osztályokra vonatkozó infót:
\begin{itemize}
  \item a leszármazási viszonyokat (többszörös öröklődés),
  \item az osztályok szerkezetét
        (milyen metódusai, attribútumai vannak egy osztálynak, 
        mit honnan örökölt),
  \item az attribútum indexeket (melyik attribútum, hanyadik az attribútumok tömbjében),
  \item a metódusok kódját (a kód kódblokk formájában adatként tárolható).
\end{itemize}
A fentiek tehát osztályadatok. 

Az objektumokban tárolódó adatok:
\begin{itemize}
  \item osztály azonosító,
  \item OREF az objektum attribútumait tartalmazó tömbre.
\end{itemize}

Ha a fordítóprogram az \verb!obj:slot! kifejezéssel találkozik,
akkor felismeri, hogy attribútum kiértékelésről vagy metódushívásról van szó.
Elkészíti a megfelelő kódot, de nem vizsgálja \verb!obj! típusát/osztályát,
sem a \verb!slot! attribútum/metódus létezését.

Futásidőben a kifejezés kiértékelése többféle eredményre vezethet. 
Ha \verb!obj! típusa nem objektum, akkor azonnal runtime error keletkezik. 
A rendszer megkeresi \verb!obj! osztályában a \verb!slot! kulcsot.
Ha nem találja, akkor megintcsak runtime error keletkezik. 
Ha a kulcsnál attribútum indexet talál, akkor előveszi az attribútum értékét.
Ha kódblokkot talál, akkor azt automatikusan kiértékeli.

Talán már nem is kéne külön említeni, hogy a stringekkel szemben
az objektumok változékonyak (teljesen ugyanaz a helyzet, mint az array típusnál):
\begin{verbatim}
    e1:=errorNew()          // minden tag üres
    e2:=e1
    e2:description:="Próba szerencse"
    ? e1:description        // kiírja: "Próba szerencse"
\end{verbatim}

Az osztálydefiníciókkal külön fejezetben fogunk részletesen foglalkozni.







\subsubsection{Referencia változók}

Nézzük az alábbi példát:
\begin{verbatim}
function main()
local a:="A"
local b:="B"
    felkialt(@a,b)
    ? a,b               // kiírja: A! B
    
function felkialt(x,y)
    x+="!"
    y+="!"
    varstack()          // kiírja a stacket
    ? valtype(x)        // kiírja: C (x típusa)
    return NIL
\end{verbatim}


Közönséges esetben a függvényhívás értékek átadásával történik.
A hívó kód a veremre rakja a paramétereket (a-t, aztán b-t) és meghívja
a függvényt (esetünkben felkialt-ot). A hívott függvény csinál, amit csinál,
majd rendbeteszi maga után a stacket. Az összes local változóját és paraméterét
leszedi a stackről. Ekkor a stacken az első üres hely ott van, 
ahol a függvény első paramétere volt (esetünkben az x). Ezután ráteszi a stackre 
a visszatérési értékét (esetünkben NIL-t), és visszatér.

Nade, akkor mitől változik meg az \verb!a! változó értéke?
A magyarázat, hogy a tárgyalt eset nem ,,közönséges''. 
Figyeljük meg a \verb!felkialt(@a,b)! függvényhívásban az \verb!a! előtti
\verb!@! karaktert. Ennek hatására az \verb!a! változó {\em referenciává\/} alakul.

A referencia változók az értéküket nem közvetlenül tartalmazzák.
Amit közvetlenül tartalmaznak, az egy VREF (VALUE referencia).
Ez direkt azt a célt szolgálja, hogy több változó közösködni
tudjon egy értéken, és ennek az értéknek a változásával
a változók együtt változzanak.

Nézzük az előbbi példaprogram tényleges kiírásait:
\begin{verbatim}
-----------------------------------------------------------
 Variable Stack
-----------------------------------------------------------
***** function main
0: REFSTRING length=2 oref=7fd95b61b058 "A!"
1: STRING length=1 oref=7fd95b61b028 "B"
***** function felkialt
2: REFSTRING length=2 oref=7fd95b61b058 "A!"
3: STRING length=2 oref=7fd95b61b088 "B!"
-----------------------------------------------------------
C
A! B
\end{verbatim}
Gyönyörűen látszik, hogy \verb!a! és \verb!x! együtt változnak,
míg \verb!b! és \verb!y! külön életet élnek.
(Honnan? \verb!a! és \verb!x! oref-je azonos, és a \verb!main!-beli
\verb!a! már a \verb!felkialt! visszatérése előtt tartalmazza a ! jelet.
\verb!b! és \verb!y! viszont külön oref-fel rendelkeznek.)

Még egy kérdés: A \verb!varstack! kiírása szerint \verb!x! típusa 
REFSTRING, a \verb!valtype()! szerint viszont a típus C (közönséges karakter). 
Hogy is van ez? Éppen ebben nyilvánul meg, hogy a futtatórendszer 
támogatja a referencia változókat. Amikor úgy látja, hogy a változó
(aminek az értékét elő kell venni) referencia típusú, akkor nem a
közvetlenül tárolt referenciát adja, hanem ,,derefeli'', azaz eggyel tovább 
nyúl, és a valódi értéket veszi elő. Ezért a \verb!valtype()!  a ,,derefelt''
értékkel hívódik meg. 

Automatikusan kezeli a rendszer
a referenciákat az értékadások mindkét oldalán.
\begin{verbatim}
    refvar:=x
\end{verbatim}
Az \verb!x! változó (az értéket tartalmazó memóriaterület) nem
\verb!refvar! memóriaterületére másolódik, hanem oda, ahová \verb!refvar!
hivatkozik. Ezért \verb!refvar! referencia típusa megmarad.

\begin{verbatim}
    x:=refvar
\end{verbatim}
Hasonlóképp, \verb!refvar! derefelt értéke íródik \verb!x!-be,
ezért az \verb!x! változó nem válik referenciává.


\subsection{Változótér}

A következőkben egy gráfot fogunk vizsgálni.

Vegyük fel a csúcsok közé a program statikus változóinak tömbjét.
A programban nem találkozunk ilyen tömbbel, mégsem értelmetlen beszélni
róla, ui. a statikus változók szakasztott ugyanúgy tárolódnak, mint egy array:
az értékek szépen, katonásan egymás után sorakoznak a memóriában.

Vegyük fel a csúcsok közé a program stackjeit (vermeit). 
Hogyhogy több? Igen, minden szál külön stackkel rendelkezik.
Ezen tárolódnak a local változók, a függvény\-paraméterek, 
a kifejezések részeredményei. Egy stacket is tekinthetünk tömbnek, 
csak éppen a hossza változik, ahogy a stackpointer szuszog.
(A félreértések elkerülése végett: 
Nem a C program stackjéről van szó.
A futtatórendszernek saját stackje van,
amire CCC értéket tesz a push, és CCC értéket vesz le róla a pop.)


Vegyük fel a csúcsok közé mindazon memóriaobjektumokat,
amiket a program valaha létrehozott, és még nem szabadított fel:
tömbök, objektumok, kódblokkok, referencia változók, stringek.

A stringek kivételével az összes eset a tömbök mintájára tárgyalható.
Az objektumok esetében az attribútumok tömbjéről van szó.
Kódblokkoknál a blokkba refesedett változók alkotnak tömbszerű
képződményt. A referencia változók kicsit speciálisak,
de ezeket is tekinhetjük egyelemű tömböknek, amiket a futtatórendszer
automatikusan indexel. Ezek a csúcsok tehát minden esetben értéktömbök. 
Memóriaterület, amiben a korábban tárgyalt típusú {\em értékek\/} helyezkednek 
el egymás után.

A string memóriaobjektumok viszont biztosan nem tartalmaznak 
{\em értékeket}, hiszen tudjuk róluk, hogy csak bájtokat vagy Unicode
karaktereket tartalmaznak.

Látjuk tehát, hogy a csúcsaink egy részében értékek vannak, és ezek
között lehetnek referencia típusúak, amikben a közvetlenül tárolt adat egy
OREF vagy VREF.  
Vegyünk fel a gráfban irányított éleket, amik a referenciát tartalmazó csúcsokból 
(memóriaobjektumokból) a hivatkozott memóriaobjektumokba (csúcsokba) mutatnak.
 

Összefoglalva: 
A gráf csúcsai a program memóriaobjektumai, beleértve
a statikus változók tömbjét, a stackeket, a tömböket, objektumokat,
kódblokkokat, referencia változókat és stringeket. Az irányított élek
megfelelnek a hivatkozásoknak. Hivatkozik, értsd: olyan elemet tartalmaz,
ami nem közvetlenül tartalmazza az értékét, hanem egy memóriaobjektumban.
A stringek olyan csúcsok, amikből nem indul él. Az egyszerű típusok
(szám, dátum, logikai, pointer, NIL) a gráf szempontjából érdektelenek.


\subsection{Szemétgyűjtés}

Sokadszor, újra nézzük meg az értékadást.
\begin{verbatim}
local x:="Próba"+" "+"szerencse"

    //most van egy "Próba szerencse" tartalmú memóriaobjektum,
    //amire az x változó hivatkozik, következésképp a stackből 
    //(mint csúcsból) él mutat erre a memóriaobjektumra

    x:=NIL

    //az előbbi él törölve, mi lesz a memóriaobjektummal?
\end{verbatim}

A fenti példa mutatja a ,,szemét'' képződésének legegyszerűbb esetét.
A programból a továbbiakban már sehol sem látszik a \verb!"Próba szerencse"! érték,
nincs mód hozzáférni, ha megint ugyanez az érték kell, akkor újra le kell gyártani.
Az ilyen elérhetetlenné (és így feleslegessé) vált memóriaobjektumok megszüntetését
nevezik szemétgyűjtésnek.


\subsubsection{Referencia számlálás}

Az idők hajnalán a CCC-nek is volt referencia számlálással működő kísérleti változata.
Nem véletlenül. A referencia számlálás egyszerűnek látszik, ez az, ami először eszébe 
jut a dilettáns programozónak.

A változótér  minden csúcsában
nyilvántartjuk a befutó élek számát. Az előbb láttuk, hogy az éleket
a legegyszerűbben az értékadó utasítások módosítják. Egyes élek törlődnek,
mások létrejönnek. Vannak más esetek is, amikor az élek módosulnak. Akárhogy is,
megtehetjük, hogy minden módosuláskor a befutó
élek számát (referenciaszám) aktualizáljuk. Amikor ez a szám nullára csökken, 
akkor az adott memóriaobjektum törölhető. (Persze a stackeket sosem töröljük.)

Ezen az elven működik a Python szemétgyűjtése.

A referencia számlálás sajnos több sebből vérzik. 
Gondot okoznak a körök. 
A változótérben lehetnek körök, nagyon egyszerű ilyet csinálni, pl.
\begin{verbatim}
local a:={NIL}
    a[1]:=a         // ez itt egy hurok
\end{verbatim}
A körben résztvevő csúcsok referenciaszáma sosem csökken nullára. 
Ez egy elvi probléma.

Gyakorlati probléma, hogy nehézkes a referencia számlálós rendszerekben
a programozás. A referenciaszám karbantartása ui. az API részévé válik,
és sok esetben nem világos, hogy kinek a feladata a  referenciaszámot módosítani,
és milyen időzítéssel kell azt végrehajtani.

Bár előnyei is lehetnek, 
a szakemberek körében a referenciaszámlálás presztízse nem áll valami magasan,
inkább más módszereket preferálnak. Sokan a referenciaszámlálást nem is tekintik
,,igazi'' szemétgyűjtésnek.


\subsubsection{Mark and sweep}

Az igazi szemétgyűjtő algoritmusok legegyszerűbb változata a
{\em mark and sweep\/} algoritmus. A program időről időre (amikor fogytán van
a memória, amikor éppen ráér, amikor úgy gondolja, hogy már régen
nem csinálta) beindítja a szemétgyűjtést.

\begin{itemize}
\item A {\em mark\/} (=bejelöl) szakaszban a kezdő csúcsokból
    (esetünkben a static változókból és a stackekből) kiindulva bejárja
    a gráfot és útközben bejelöli, hogy mely csúcsokba (memóriaobjektumokba)
    jutott el. Ezek a program élő adatai.
\item A {\em sweep\/} (=kisöpör) szakaszban végigmegy az összes csúcson
    (memóriaobjektumon), és kitörli azokat, amiket a mark szakasz nem jelölt
    be élőnek.
\end{itemize}


Ez van a CCC-ben, ennek is a legegyszerűbb magvalósítása. 
A szemétgyűjtés bele van építve a futtatórendszerbe, 
ezért tudja, hogyan kell bejárni a gráfot,
és hogyan kell végigmenni az összes memóriaobjektumon.
A memóriafoglalás és felszabadítás a \verb!malloc! és 
\verb!free! (Windowson \verb!GlobalAlloc! és \verb!GlobalFree!)
függvényhívásokkal történik. Vagyis a memóriakezelés egyszerűen 
rá van lőcsölve az operációs rendszerre.

Igen, mielőtt először kipróbáltam, én is aggódtam, nem fogja-e a primitív
memóriakezelés túlterhelni az operációs rendszert. Nem terheli túl.
A pentiumos korszaktól kezdve a CCC gyorsan és vidáman fut.

Egy átlagos CCC program memóriafoglalása nem kirívóan nagy. 
Miközben ezt a szöveget írom, megnéztem a top-ban, mennyi a \verb!z.exe!
(CCC-ben írt editor) memóriafoglalása. A legtöbb memóriát a \verb!firefox-bin!
foglalja (55M), a \verb!z.exe! (8M) a tizedik helyen áll, 
az \verb!xfce4-menu-plugin! után és az \verb!xterm! előtt.
Ugyanakkor van CCC program, amelyik egyszerre többmillió memóriaobjektummal
rendelkezik, és gigabájtnyi helyet foglal. Mindezt nem dicsekvésnek szánom,
csak annyit mondok, hogy a modern operációs rendszerek
elég jól kezelik a memóriát, nem érdemes azt alkalmazásszinten újraírni.


A mark and sweep algoritmusnak is vannak hátrányai.
A szemétgyűjtés kampányszerűen történik, ilyenkor a program minden 
más tevékenységet felfüggeszt. Átlagos méretű programoknál a szünet 
tizedmásodpercekben mérhető, amit interaktív használatban nem lehet észrevenni.
A legnagyobb (többmillió objektumos) programokban a szünet 2-3 másodperc.
Mindenesetre lehetnek olyan feladatok, robotvezérlés, effélék, ahol semmilyen
szünet nem megengedhető, ilyesmire a CCC nem alkalmas. (Egyébként a Jáva
licencében is leírják, hogy atomerőművek és repülőgépek vezérlésére nem jó.)






