
\pagetitle%
{Eltérések a CCC és a Clipper  között}%
{Dr. Vermes Mátyás\footnote{\ComFirm}}%
{első változat: 2005. augusztus, utolsó módosítás: 2006 július}

\section{Áttekintés}

Ebben a kis pamfletben összegyűjtöttem a CCC és a Clipper közötti 
eltéréseket. Elsősorban a {\em nyelvi\/} különbségekre koncentráltam. 
Kevés szó esik a rendelkezésre álló könyvtárakról és ezzel kapcsolatban 
a nyelv {\em lehetőségeiről}. E tekintetben a UNIX-on és Windowson 
futó CCC nyilván óriási előnyben van.

Kezdem a rossz hírrel: A CCC nem alkalmas  
régi DOS-os Clipper programok változtatás nélküli lefordítására. 
A CCC  általános célú programnyelv. Az eredeti Clipperből
csak a jó részeket tartja meg, azokban viszont közel teljes 
a kompatibilitás. 
A CCC több területen kiterjeszti a Clippert 
(objektumok, szálak, névterek, kivételek). A kiterjesztés minden esetben
úgy valósul meg, hogy a CCC speciális esetként tartalmazza
a Clippper 5.x-et, miközben minimális mennyiségű szintaktikai 
újítást vezet be. 

Manapság egyre kevesebben emlékeznek a Clipperre,
érdemes ezért a CCC-t a Pythonhoz képest is pozícionálni.
A CCC a Pythonhoz hasonlóan platformfüggetlen  nyelv.
A programozás hangulata, a leírandó kód mennyisége
(saját tapasztalat alapján állítom ezt) nagyon hasonló. 
Mindkettővel pillanatok alatt meg lehet írni az egyszerű
programokat. Nagy Python projektekről nincs közvetlen élményem,
a CCC-ről azonban elmondhatom, hogy nagy és kritikus fontosságú
projektekben is bevált.\footnote{Hogy mi a kritikus szoftver?
Ha pl. a szoftver  működésképtelensége miatt a bank 
nem tudja kiszolgálni az ügyfeleit, az kritikus.}
A Python bytekódra, a CCC (C++-on keresztül) natív kódra fordít. 
A Python nagy elterjedtsége folytán több kész modullal rendelkezik, 
viszont a CCC nyitottabb a C++ irányába, ezért könnyebb 
pótolni a hiányzó komponenseket. 


\section{Mi az, ami nincs a CCC-ben?}

\subsection{Adatbáziskezelés}

A régi Clipperben a {\em nyelv részének\/} tekintették 
a dBase  adatbáziskezeléssel kapcsolatos utasításokat. 
A CCC ezzel szemben általános célú  programnyelv,
amiben tetszés szerint írhatunk adatbáziskezelő szoftvert,
ám ezeket nem a nyelv részének, hanem {\em  alkalmazásnak\/} 
tekintjük. Az alap CCC nem tartalmaz  hagyományos DBFNTX 
formátumú adatbáziskezelésre szolgáló eszközt, és nem is akarunk 
ilyen anakronisztikus  dolgokkal vesződni. 
%Indexszekvenciális fájlkezeléshez a 
%\href{tabobj.html}{táblaobjektumok} 
%adnak keretet, amihez létezik egy saját fejlesztésű 
%\href{btbtx.html}{adatbázismotorunk}.
Az új fejlesztések középpontjában az 
\href{sql2.html}{SQL2  adatbázisinterfész}
áll, ami  Oracle és Postgres adatbázisokhoz ad objektumorientált 
hozzáférést.


\subsection{Public és private változók}

Amikor megismerkedtem a Clipperrel, 
az 5.01-es verzió volt éppen  használatban.
Ennek dokumentációjában azt találtam, hogy a \verb!public! és
\verb!private! minősítésű változók elavult dBase örökség, használatuk 
nem javasolt, mivel ellentmondanak a korszerű programozási elveknek.
Azóta sem tudok többet a public és private változókról,
értelemszerűen nem kerültek be a CCC-be.




\subsection{Makrók (futásidejű fordítás)}

A régi Clipperben makrónak nevezték a forráskódot tartalmazó
string változókat.  Az ilyen forráskódot a Clipper röptében 
lefordította és végrehajtotta.  A clipperes guruk kiterjedten használták
a technikát  menük megvalósítására, futás közben beolvasott
lekérdezések végrehajtására. A makró azonban mindig a Clipper 
mocsarasabb  területei közé tartozott. Sosem tudták egyértelműen 
szabványosítani, mi az,  amire a makrófordító képes, és mi az, amire nem.
A CCC-ben körülményes volna megvalósítani a makrókat, ezért jobbnak láttam 
az egészet elhagyni.\footnote{A bytekódra fordító programoknál
(mint a Python, Jáva, régi Clipper) egyszerűbb a helyzet, a CCC
azonban C++-ra, arról pedig natív kódra fordít.}

%\subsection{Preprocesszor}
%A  Clippernek rendkívüli képességekkel bíró preprocesszora volt.
%A CCC fordító megírásakor ennek pótlása nem kis nehézséget okozott. 
%Az eredeti  preprocesszor elfogad olyan makródefiníciókat, amiknek 
%a jobboldalán (vagyis a kimenetén) egy másik makródefiníció jelenik meg. 
%Ezzel aztán hihetetlenül bonyolult szövegfeldolgozási feladatokat is meg 
%lehetett oldani. Most két nagyjából ekvivalens preprocesszorunk is van 
%(az egyik CCC-ben a másik C++-ban írva), mindkettő egyszerűbb az eredetinél.




\section{Kis eltérések és javítgatások}


\subsection{Kötelező deklarációk}

CCC-ben a változókat kötelezően deklarálni kell (local vagy static).
A régi Clipperben ez fordítási opció volt.


\subsection{Procedure helyett function}

A Clipper ismerte a \verb!procedure! utasítást, 
ami azonban semmiféle önálló jelentőséggel nem bírt.
A CCC-be nem vettük át, helyette NIL-t visszaadó
\verb!function!-t kell írni.
Hasonlóképp, nem megengedett az üres \verb!return!, 
hanem a \verb!return!-nel mindig meg kell adni egy visszatérési 
értéket, ami alkalmasint lehet NIL. Egy \verb!function!-t nem kötelező
\verb!return!-nel befejezni, ha hiányzik, az a  NIL visszatéréssel
egyenértékű.


\subsection{Kötelező main függvény}

A régi Clipper a linkernek elsőként megadott object modul
első függvényében kezdte a program végrehajtását. A CCC
programban mindig kell legyen \verb!main! függvény, a modulok
sorrendjétől függetlenül mindig ott indul a végrehajtás.


\subsection{Nincsenek rövidítések}

A CCC-ben a kulcsszavakat mindig teljes hosszukban kell kiírni, 
pl. nem írhatunk  \verb!function! helyett \verb!func!-ot.



\subsection{Értékadás és egyenlőség vizsgálat}

A régi Clipperben kétféle értékadás létezett:
\begin{itemize}
\item \verb!x=y!, 
    ez egy {\em utasítás}, ami csak önmagában állhat, nincs értéke.
\item \verb!x:=y!, 
    ez egy {\em kifejezés}, értéke a jobboldal, mellékhatásként x megkapja
    y értékét, bárhol állhat, ahol pl. az \verb!x+y! kifejezés is állhat.
\end{itemize}
Emellett a régi Clipper az '='-t logikai egyenlőség operátornak is használta.
Nincs jó véleményem az ilyesmiről.
Hogy véletlenül se lehessen keveredés, a CCC-ből teljesen száműztük 
az '=' operátort. Helyette az  \verb|x:=y|, \verb|x==y| vagy \verb|!(x!=y)| 
formákat kell használni (a régi Clipperrel megegyező jelentéssel).


\subsection{Egymásba ágyazott kódblokkok}

A régi Clipper megengedte a kódblokkok egymásba ágyazását,
hogy azután az esetek egy részében hibás kódot fordítson.
Az egymásba skatulyázott  kódblokkok kellemetlenül bonyolítanák
a fordítót, emellett feleslegesek is: A belső kódblokk helyére egy 
változót írunk, aminek értéke  a kérdéses kódblokk.


\subsection{Logikai kifejezések vágása}

A logikai kifejezések kiszámításakor, amint ismertté válik a végeredmény, 
a CCC abbahagyja a további tagok kiértékelését. Ez a Clipperben fordítási 
opció volt.


\subsection{NIL-t nem lehet pad-olni}

A CCC pad (padl, padr) függvényei nem fogadják el (blankekkel kiegészítendő) 
argumentumként a NIL értéket. A régi Clipper ilyen esetben üres stringet adott, 
ami véleményem szerint megtévesztő, azaz hibás.


\subsection{Index anomáliák}

A régi Clipperben a \verb!local a[10]! deklaráció
egy tízelemű arrayt (tömböt) hozott létre. 
A CCC-ben ehelyett azt írjuk: \verb!local a:=array(10)!,
ami a Clipperben is ismert volt,  CCC-ben és Clipperben ugyanazt 
jelenti, ráadásul még logikus is.

A régi Clipper elfogadta az \verb!a[1,2]! kifejezést
a kétdimenziós array egy elemének címzéseként.
A CCC-ben helyette ezt kell írni: \verb!a[1][2]!, 
ami megintcsak ugyanazt jelenti a CCC-ben és Clipperben, 
ráadásul logikus. A dolog logikája:  \verb![]! egy operátor,
amit a tömbökre jobbról alkalmazunk, eredménye pedig a tömbelem.


\section{Kiterjesztések}




\label{NAMESPACE}
\subsection{Névterek}

Korszerű programnyelv nem nélkülözheti a névtereket.
A forrásmodulok legelején állhat az opcionális 
\verb!namespace! utasítás, például:
\begin{verbatim}
namespace aa.bb.cc
\end{verbatim}
aminek hatására a modulban definiált összes függvény
az \verb!aa.bb.cc! (többszintű) névtérbe kerül.
A \verb!namespace! utasítás nélkül, egyedileg is névtérbe 
helyezhetünk függvénydefiníciókat a következő módon:
\begin{verbatim}
function  aa.bb.cc.f()
    ...
\end{verbatim}
A kétféle (globális és egyedi) minősítés (névtérbe helyezés) 
egyszerre is jelen lehet, ebben az esetben a hatásuk összegződik.
A modulon belül definiált függvények meghívásakor
nincs szükség teljes minősítésre.

A modul függvényeire (pl. \verb!f!-re) kívülről a mínősített névvel,
esetünkban az \verb!aa.bb.cc.f()! formával hivatkozhatunk.
A külső hivatkozások megkönnyítésére szolgál a \verb!using! utasítás.
A \verb!using! utasítások a modul elején, közvetlenül az
esetleges \verb!namespace! után állhatnak. A \verb!using! olyan 
rövidítést vezet be, amivel elkerülhető a teljesen minősített 
függvénynevek túl sokszori kiírása.

\begin{center}
\begin{tabular}{|l|l|}\hline
alternatív {\tt using} utasítások   &  hivatkozás {\tt f}-re \\ \hline \hline
\tt  using aa.bb.cc=alias           &  \tt  alias.f() \\ \hline
\tt  using aa.bb=x                  &  \tt  x.cc.f() \\ \hline
\tt  using aa.bb.cc  f g            &  \tt  f(), g() \\ \hline
\tt  using aa.bb  cc.f              &  \tt  cc.f() \\ \hline
\end{tabular}
\end{center}

A globális (gyökér) névteret (kezdő) pont jelöli. 
Ha pl. a \verb!using aa.bb.cc  f! utasítás után 
\verb!aa.bb.cc.f! helyett a globális névtérben definiált 
\verb!f!-et akarjuk meghívni, akkor ezt írjuk: \verb!.f()!.

\label{VARARG}
\subsection{Vararg API}

CCC/Clipperben a hagyományos függvénydefiníció
\verb!function fname(a,b,c,d)! alakú. 
A függvényt akárhány paraméterrel meg lehet hívni.
Ha a függvény hívása (a példában szereplő) négynél több paraméterrel 
történik,  akkor a négy felettieket a rendszer kipucolja (elvesznek), 
ha négynél kevesebbel, akkor azok az argumentum változók,
amikre nem jutott érték, NIL-re lesznek beállítva.

Előre nem ismert számú paraméter átvételére valók a
\begin{verbatim}
function fname(*)
function fname(a,b,c,*)
\end{verbatim}
formák. A '*' az utolsó helyen állhat, és jelzi, hogy a függvény
akárhány paramétert átvesz. A második példában az argumentumok
száma legalább három, ezekre név szerint is lehet hivatkozni,
és NIL-re inicializálódnak, ha a függvényhíváskor nincs  megadva
elég paraméter.

Az alábbi kifejezésekben:
\begin{verbatim}
    funcall(*)
    obj:method(*)
    {*}
\end{verbatim}
a '*' karakter helyére behelyettesítődik a függvény összes
argumentum változója. Nézzük az alábbi kódot:
\begin{verbatim}
function fname(a,b,*)
    ? {*}       //kiírja az összes argumentumot tartalmazó arrayt
    ? {*}[1]    //kiírja a-t
    ? len({*})  //ennyi argumentum van (a-t és b-t is beleértve)
    ? ff(*)     //továbbadja az összes paramétert (a-t és b-t is)
    ? o:meth(*) //továbbadja az összes paramétert (a-t és b-t is)
\end{verbatim}
A *-helyettesítés komplikáltabb formái is megengedettek, például
\verb!{1,2,*,*}! legális kifejezés, tehát '*' minden pozícióban
és többször is  használható. 
Függvény- és metódushívásban a *-helyettesítés megőrzi
a refeket, azaz refes paraméterek refesen adódnak tovább.

Ugyanez a logika működik kódblokkban is azzal a különbséggel,
hogy blokkon belül '*' a blokkargumentumokat helyettesíti.
A \verb!{|*|fname(*)}! blokk pl. minden paraméter továbbaddásával
meghívja fname-et. Megengedett minden komplikáltabb eset is, pl.
\verb!{|p1,p2,*|fname(*),p1+p2}! minden paraméterrel meghívja
fname-et, és visszaadja az első két paraméter összegét.

Külön szólni kell arról az esetről,
amikor a blokkban egyetlen függvényhívás van, és a blokk
minden paraméterét változtatás nélkül tovább akarjuk adni a függvénynek.
\begin{verbatim}
    {|*|fname(*)}
\end{verbatim}
Az ilyen blokkok nem hoznak létre külön függvényhívási szintet a CCC stacken
(nincs paraméter átpakkolás), legfontosabb alkalmazásuk az optimalizált 
metódushívás. Néhány további alkalmazás.

Konstruktor, ami minden paramétert átad az inicializálónak:
\begin{verbatim}
function clnameNew(*)
    return objectNew(clnameClass()):initialize(*)
\end{verbatim}

Függvény meghívása arrayben megadott paraméterekkel:
\begin{verbatim}
    evalarray({|*|fname(*)},{p1,p2,p3,...})
\end{verbatim}


\subsection{Objektumok}

A Clipper 5.x-ben négy előre bedrótozott osztály volt, 
újak létrehozására nem volt lehetőség. A CCC-ben természetesen 
megvan a négy régi osztály, ám ami fontosabb, az alkalmazások 
tetszés szerint definiálhatnak új osztályokat. 
A CCC-ben nem szívesen vezetünk be új szintaktikát, 
ezért sokáig osztályok készítéséhez sem volt speciális szintaktika, 
hanem  függvényhívási API szolgált a célra. 
%Megjegyzem, hogy a Pythonban és a Jávában is van ilyen runtime 
%osztálydefiniálási API, legfeljebb a kezdők nem azzal találkoznak először. 
Újabban az osztályokat a class utasítással definiáljuk, 
amihez kicsit kevesebbet kell írni. 
%A két módszer felhasználói szempontból egyenértékű. 
%A régebbi függvényhívási API nem vesztette el a jelentőségét, 
%ui. lehetővé teszi osztályok futásidőben történő létrehozását.

\subsubsection{Függvényhívási API}

\begin{verbatim}
static clid_template:=templateRegister()

static function templateRegister()
local clid:=classRegister("template",{objectClass()})
    classMethod(clid,"initialize",{|this|templateIni(this)})
    classAttrib(clid,"cargo")
    return clid

function templateClass()
    return clid_template

function templateNew()
local clid:=templateClass()
    return objectNew(clid):initialize()

function templateIni(this)
    objectIni(this)
    return this
\end{verbatim}

A  fenti példában
a \verb!template! osztály az \verb!object! osztályból örököl,
egy új metódusa (\verb!initialize!) és egy új attribútuma (\verb!cargo!) van.
A metódusokat kódblokkal adjuk meg.
%\footnote{Egy másik Clipper klón a {\em Clip\/}
% szintén kiterjesztette a Clippert szabadon definiálható objektumokkal.
% Ők új szintaktikát használnak a metódusok jelölésére. A C-ből kölcsönzött
% módon a zárójelpár nélküli függvénynév valamiféle függvénypointert jelent, 
% véleményem szerint  azonban ez  idegen a Clippertől.
% Megjegyzem, hogy a Clipben az objektum orientáltság egyszerűbb
% fajtája valósul meg, amit gyakran objektum alapúnak mondanak.}
Nincs new operátor, hanem az 
\begin{verbatim}
    o:=templateNew()
\end{verbatim}
függvényhívással jutunk új objektumpéldányhoz. Az egyszerű
API-ból ,,magától'' adódik egy sor olyan tulajdonság,
amit a C++-ban, Jávában speciális kulcsszavakkal fejeznek ki. 
Például, ha nem definiáljuk a templateNew függvényt,
akkor az osztály absztrakt lesz, azaz nem lehet példányosítani.
Ha a templateClass függvényt static-nak definiáljuk, 
akkor az osztályból nem lehet örökölni, 
azaz Jáva terminológiával \verb!final!.


Az objektumokat használó programnak nem kell számontartania,
hogy az objektumműveletek belsőleg attribútumként vagy metódusként
vannak-e implementálva. Ez annak következménye, hogy a fordító 
ugyanazt a kódot fordítja az \verb!o:initialize! vagy \verb!o:initialize()!
bemenetre. Tehát az üres zárójelpár léte/nemléte nem utal arra,
hogy attribútumról vagy metódusról van-e szó.
Hasonlóképpen, ugyanaz a kód keletkezik az \verb!o:initialize:=x! 
vagy \verb!o:initialize(x)! bemenetből. Ez biztosítja,
hogy az osztály implementációjában szabadon lehessen attribútumot metódusra,
vagy metódust attribútumra cserélni. A Jávában
elterjedt ún. set-get metódusoknak ezért kevesebb jelentősége van.

Az osztályok többszörös öröklődéssel örökölnek. 
A Pythonhoz hasonlóan minden objektum tag nyilvános.
Mindig az objektum dinamikus típusa szerinti metódusok hívódnak meg,
kivéve, ha az alábbi (kivételesen új) szintaktikával mást írunk elő:
\begin{verbatim}
    o:(object)initialize
\end{verbatim}
A fenti explicit típuskényszerítés (metódus-cast) eredményeképpen 
az \verb!o! objektum típusától  függetlenül az \verb!object! osztály 
\verb!initialize! metódusa fog meghívódni.

\subsubsection{Class utasítás}

Néhány új kulcsszó (\verb!class!, \verb!attrib!, \verb!method!)
árán kicsit csökkenthető az osztályok készítéséhez szükséges kód
terjedelme. Az alábbi szintaktikát a fordító visszavezeti
a függvényhívási API-ra, tehát az ott elmondottak most is érvényesek.

\begin{verbatim}
class derived(base1,base2,...)
    attrib a1
    attrib a2
    method m1 codeblock
    method m2 
    method m3
    ...
\end{verbatim}
A class definíció  függvények helyén állhat.
A class a következő \verb!class!-ig vagy \verb!function!-ig tart.
A  baseclass-okat zárójelek között felsorolva kell megadni.
Mindig van legalább egy baseclass.
Az osztálydefiníció  névtérbe helyezhető.
A class definícióból automatikusan keletkezik a
\verb!derivedClass! és \verb!derivedNew! függvény. 
A class függvény a szokásos módon az osztályazonosítót adja.
A new (konstruktor) függvény létrehozza a megfelelő osztályú objektumot, 
és meghívja rá az \verb!initialize! metódust.
A konstruktor minden paramétert továbbad az inicializálónak.

A \verb!class! törzsében csak attribútum és metódus deklarációk
állhatnak. Megengedett, hogy a class törzse üres legyen.

Az \verb!attrib! teljesen egyszerű: Lesz egy új (vagy
felüldefiniált) attribútum az osztályban a megadott névvel.

A \verb!method! kulcsszó és metódusnév után egy tetszőleges kódblokk írható,
ebben az esetben a metódus implementációja maga a kódblokk.
A metódus deklarációnak van egy  alternatív formája:
Ha a deklaráció csak a nevet tartalmazza,
az olyan, mintha kiírtuk volna a következő
(optimalizáltan forduló) kódblokkot: \verb!{|*|derived.m2(*)}!. 
Tehát ilyenkor a metódust úgy kell implementálni, 
hogy megírjuk a \verb!derived.m2! (közönséges) függvényt. 
Mint látjuk,  a metódusok az osztálynévből származó 
névtérbe kerülnek (automatikus prefixelés).

A CCC objektumrendszer  speciális esetként tartalmazza
a Clipper 5.x lehetőségeit. Ha a régebbi függvényhívási
API-t használjuk és nem alkalmazunk metódus-castot, 
akkor a régi fordító nem talál szintaktikaliag ismeretlen
elemet a programunkban. Így szintaktikai újítások nélkül is
teljes értékű, általánosan használható objektumrendszerünk van.
Sajnos bizonyos hátrányok is közösek: Mivel a Clipperben, Pythonban, CCC-ben
csak futásidőben kapnak típust a változók, azért csak futásidőben
derülhet ki, ha elírás folytán nemlétező metódusnévre hivatkozunk.%
\footnote{Pythonban még rosszabb a helyzet, ui. az eltévesztett nevű 
attribútum automatikusan létrejön, tehát még később derül fény a hibára.}
Az objektumokról szóló leírás teljes változata itt olvasható:
\href{objektum.html}{Objektumok használata a CCC-ben}.

\label{THREADS}
\subsection{Szálak}

A POSIX thread API-t (pthread könyvtárat) használjuk szálak kezelésére. 
Minden szál saját lokális stackkel rendelkezik a függvényargumentumok
és local változók számára. A program statikus objektumai  közösek.
A futtatórendszer gondoskodik a belsőleg definiált statikus objektumok
konkurrens használatának szinkronizálásáról. Az alkalmazások a saját
statikus változóikat saját hatáskörben kell szinkronizálják.
Bármelyik szálban beindulhat a szemétgyűjtés. Tudni kell, hogy 
nem minden CCC könyvtár szálbiztos.%
\footnote{Csak a ccc2 és ccc2\_ui\_  könyvtárakban kitűzött cél
a szálbiztonság. A szálbiztonság legnehezebben
teljesíthető kritériuma: Minden olyan pillanatban, amikor egy másik 
szálból szemétgyűjtés indulhat, a vermeken kell legyen minden 
élő változó, de nem lehet ott semmi más, pl. keletkezőben vagy megszűnőben 
levő változók. Ugyanez szükséges a korrekt szignálkezeléshez is.
Persze ez nem az alkalmazások, hanem a futtatórendszer 
felelőssége.}

A DOS-os Clipper esetében nem lehetett szó többszálúságról. 
Tudomásom szerint a Python és Ruby úgy valósítja meg a többszálúságot, 
hogy a valójában egyszálú interpreter váltogatja az interpretált kódrészt. 
A CCC fordítás végeredménye natív kód, tehát a CCC-ben natív többszálúság van.



\subsection{Kivételkezelés}

A kivételkezelést Jáva mintára kiterjesztettük:

\begin{verbatim}
function ff(x)
local e
    begin /*sequence*/
        ? "HOPP-1"
        break(x) //kivételt dob
        ? "HOPP-2"
    recover  e  <specerror>  //elkapja specerror leszármazottait
        ? "rec1", e:classname
    recover  e  <error> //elkapja error leszármazottait
        ? "rec2", e:classname
    recover  e <c> //elkapja a stringeket
        ? "rec3", upper(e)
    recover /*using*/ e  //bármit elkap (régi szintaktika)
        ? "rec4", e
    recover //ez is bármit elkapna (felesleges) 
        ? "ide nem jöhet"
    finally
        ? "lefut a begin-recover elhagyásakor"
    end  /*sequence*/
\end{verbatim}

Az újdonságok Clipper egyszintű kivételkezeléséhez képest:
\begin{itemize}
\item a sequence és using kulcsszó elhagyható (zajszó),
\item több recover lehet lineárisan felsorolva,
\item a recover változó után megadható egy típuskifejezés,
\item opcionális finally záradék.
\end{itemize}

A típuskifejezéssel bővített recover akkor kapja el a kivételt,
ha a kivétel típusa megfelel a típuskifejezésnek. Az \verb!<error>! 
típusú recover az error osztály leszármazottait kapja el.
Könnyen látható, hogy az új begin-recover speciális esetként
tartalmazza a régit: ha kiírjuk a sequence és using zajszavakat,
ha csak egy típuskifejezés nélküli (régi módon mindent elkapó) recover-t 
használunk, és nem írunk finally záradékot.

A CCC kivételkezelése mindent tud, amit a Jáva, sőt, 
a kivétel tetszőleges  CCC típus lehet. A példában
külön recover ág kapja el a \verb!<c>! (string) típusú 
kivételeket. A kivételekről szóló leírás teljes változata
megtalálható itt: \href{exception.html}{Struktúrált kivételek a CCC-ben}.


\subsection{Hosszú stringek}

Példa hosszú stringre:
\begin{verbatim}
local query_ab:=<<query>>
select
    konto.a.id as id_a,
    name,
    datum,
    flag,
    konto.a.value as val_a,
    konto.b.id as id_b,
    konto.b.value as val_b
from
    konto.a full join konto.b on konto.a.id=konto.b.id
order by
    konto.a.id -- nulls first -- csak Oracle
<<query>>
\end{verbatim}
A \verb!query_ab! változó egy SQL parancs szövegét tartalmazza.
A szövegben {\em bármi\/} előfordulhat, kivéve a \verb!<<query>>!
részstringet, mert az a szöveg végét jelzi. Persze a \verb!query! helyett
más szimbólumot is használhatunk, a lényeg, hogy a
\verb!<<symbol>>! részstring ne szerepeljen a közrefogott szövegben.


\subsection{Pointer típus}

A 64-bites rendszerek megjelenésével szükségessé vált
egy olyan adattípus, ami 64-bites mennyiségek tárolására képes.
Egy P típusú változóval CCC (Clipper) szinten nem sokat tudunk kezdeni,
lényegében csak az értékadás és egyenlőségvizsgálat működik vele.
A típus értelme, hogy a C kiterjesztésekben előadódó handlereket,
pointereket tárolni tudjuk a CCC programban. Korábban (32-biten)
gyakran az N típust használtuk ugyanerre a célra.


\subsection{Megjelenítés}

Szigorúan véve a megjelenítés nem része a 
{\em nyelvnek}, mégis érdemes pár mondatot írni róla:

A karakteres megjelenítés ugyanazt tudja, mint a Clipper, megtoldva 
egy spéci terminál protokollal. A terminálos üzemmód lehetővé teszi, 
hogy a CCC program futtatása és megjelenítése különböző (akár földrajzilag 
távoleső) gépeken történjen. Természetesen a UNIX-on futó program szót 
ért a Windowson futó terminállal, és fordítva.

Az új programjainkban a megjelenítést a 
\href{jterminal.html}{Jáva terminálra}
bízzuk, teljesen szétválasztva az alkalmazási logikát  és a megjelenítést.  
A Jáva terminál egy 300K-s jar fájl, ami bárhol fut, ahol a JRE installálva 
van. Működési elve leginkább egy szerver oldali widgetekkel dolgozó 
X szerverre hasonlít. A Jáva terminállal igényes GUI-t lehet készíteni 
ügyviteli programokhoz. Rendkívüli előnye, hogy ugyanaz a program 
használható lokálisan, intranetes és internetes környezetben.

A \href{cccgtk.html}{CCC-GTK csatoló} könyvtár szintén alkalmas 
a legigényesebb GUI-k készítésére. A \verb!ccc3_glade! modul lehetővé teszi, 
hogy a népszerű glade-del készített GUI-kat működtessük CCC-ből.
Ezzel egy grafikus GUI szerkesztő is helyet kapott CCC fegyvertárában.


\subsection{Unicode, internacionalizálás}

A Clipperben és a CCC2-ben nincs unicode támogatás.
A CCC2 bedrótozottan Latin/CWI karakterkészlettel működik, 
a stringeket és a (tetszőleges tartalmú) bytesorozatokat nem 
különbözteti meg. Ezért koreai, arab, héber, orosz, stb.\ nyelvű 
programok készítésére nem alkalmas. 

A CCC3-ban viszont megjelent a unicode támogatás.
A Jávához hasonlóan külön típus szolgál a bináris
adatok (bytesorozat) és a karakter stringek tárolására.
Az új binary string típus átveszi a régi stringek szerepét,
amikor azokat bináris adatok tárolására használtuk.
A karakter string értelmezése változott: 
az új (CCC3) rendszerben unicode karakterek sorozatát tartalmazza.
A CCC3 programokban a karakterliterálokat UTF-8 kódolással kell beírni.
Az új/megváltozott típusok miatt nincs teljes kompatibilitás,
de viszonylag kis munkával lehet portolni a programokat CCC3-ra, 
ami ímmár támogatja az internacionalizált programok készítését,
legalábbis, ami a karakterállandók különböző nyelvi változatainak 
csereberéjét illeti. 
A függvény- és változónevekben továbbra sem lehet ékezetes 
(vagy cirill) betűket használni.
További infó a témáról az
\href{ccc3_ujdonsagok.html}{Újdonságok a CCC3-ban} és
\href{ccc3_unicode_hasznalat.html}{Unicode használat a CCC3-ban}
dokumentumokban.




