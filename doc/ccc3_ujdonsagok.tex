
\pagetitle{Újdonságok a CCC3-ban}{Dr. Vermes Mátyás}{2006. május}


\section{Unicode támogatás}

A régi Clipperben és a CCC1-CCC2-ben nem volt megkülönböztetve
a tetszőleges (akár bináris) adatokat tartalmazó bytesorozat
és a karaktersorozatot tartalmazó string. Az ilyen típust egységesen 
karakternek (stringnek) neveztük, a típus kódja  "C" volt.
Hasonló volt a helyzet a 2.3 előtti Pythonban is.

Az idők azonban változnak, igény támadt az egyidejűleg
többféle nyelven is értő programokra. Világossá vált, hogy
a többnyelvűség igényeit legjobban a unicode elégíti ki,
továbbá, hogy a unicode problémáit (az operációs
rendszerrel való kompatibilitást illetően) legjobban
az UTF-8 kódolás oldja meg. A unicode/UTF-8 kódolás
ezért kezd univerzálisan elfogadottá válni, 
az operációs rendszerek sorra térnek át a használatára.
A változást a CCC-vel is követnünk kellett.
A CCC3 fő újdonsága a unicode támogatás.

Az XMLRPC esete mutatja, mennyire elkerülhetetlen a változás követése. 
Egy 1999-es XMLRPC leírás azt mondja, hogy a \verb!string!
adattípusban bámilyen adatot küldhetünk (binárisat is),
csak arra kell ügyelnünk, hogy az XML formázásban szerepet
játszó karakterek/byteok (\verb!<!, \verb!&!) megfelelően védve legyenek. 
A pár évvel ezelőtti XML tankönyvekben fel sem vetődik a kérdés:
Miből áll az XML dokumentum?
\begin{itemize}
\item Byteok sorozatából,
\item vagy karakterek sorozatából?
\end{itemize}
A mai XML szabvány szerint karakterek sorozatából.
Az XMLRPC string tehát nem tartalmazhat bináris adatot, 
mert akkor elbukik az XML elemzésen. Hogy mik a karakterek,
az sem triviális, pl. a 0x00--0x20 intervallumban csak a
TAB, CR, LF számít karakternek, és máshol is vannak érvénytelen
(nem karakter) kódok, amikre a mai XML elemzők kivételt dobnak.

A unicode támogatás megvalósítására két út kínálkozott.
A Pythonban úgy jártak el, hogy bevezettek (mint új dimenziót) 
egy új típust, a unicodeot, ami minden mást érintetlenül hagyott.
A kompatibilitás szempontjából ez tökéletes megoldás,
azonban semmivel nem visz közelebb a régebbi programok
unicodeosításához.

A CCC-ben a Jáva mintáját követve radikálisabb utat választottunk.
Bevezettünk egy új típust a bytesorozatok számára. Ezt bytearraynek,
bytesorozatnak, vagy bináris stringnek nevezzük, a típuskódja "X"
(sajnos a B betű már foglalt a kódblokkok számára). Ez a legalapvetőbb
típus, tetszőleges értékű byteokat tartalmazhat, a többi típus 
többsége tárolható benne. A binary string (X) átveszi a régi
(C) stringek szerepét, amikor azok bináris adatot tárolnának.
A bináris stringekre működnek a szokásos string kezelő függvények
és operátorok: 
\verb!at!, 
\verb!rat!, 
\verb!strtran!, 
\verb!left!, 
\verb!right!, 
\verb!padr!, 
\verb!padl!, 
\verb!substr!, 
\verb!+!, 
\verb!==!, 
\verb!$!, 
stb.

A korábbi (C) stringek értelmezése megváltozott,
a CCC3-ban unicode karaktersorozatot jelentenek.
Természetesen az ismert string függvények ezekre is működnek.
A két string fajtát azonban nem lehet keverni, azaz nincs
feltétlen, automatikus konverzió.
Érdemes tudni, hogy az egyes karakterek C szinten \verb!wchar_t!
típusban tárolódnak, ami a mai C fordítókban 32 bites mennyiség.

Nagyon fontos megérteni a karakter string és a binary string
közötti kapcsolatot. A karakter string (unicode vagy UCS kódok
sorozata) szöveget tud tárolni. Ha a szöveget binary stringbe 
akarom átírni, akkor előállítom a unicode karakterek UTF-8
kódját (karakterenként a karaktertől függő hosszúságú bytesorozat),
ezeket konkatenálom, az eredmény egy bytesorozat, amit a 
szöveg UTF-8 kódolású bináris reprezentációjának nevezek.
Bármely szöveg (karakter string) ezen a módon infóveszteség
nélkül bináris stringre konvertálható, és a bináris reprezentációból
maradék nélkül visszanyerhető. Általában a szöveg UTF-8 reprezentációja
több byte, mint ahány karakter van az eredeti szövegben.
Ennek oka, hogy pl. a magyar ékezetes betűk vagy a cirill
betűk UTF-8 kódja két byte. Más karakterek még hosszabbak
lehetnek, a létező leghosszabb UTF-8 kód hat byteos.

Ha a karakter string memóriabeli tárolását vizsgáljuk,
azt tapasztaljuk, hogy sok 0 értékű byte van benne.
Nyilván, ui. az ASCII kódok a 0--127 intervallumba esnek,
azaz egy byteot foglalnak el, a string azonban 32-bitet 
használ minden karakterhez. A unicode karakter stringekre ezért
nem működnek a C könyvtár hagyományos string kezelő függvényei,
amik a 0 byteot a string végének tekintik. Ugyanezért nem
célszerű egy unicode stringet byteonként kiírni egy fájlba,
vagy egy socketba. Ezzel szemben a string UTF-8 reprezentációja
rendelkezik azzal a tulajdonsággal, hogy csakis a 0 unicodenak
felel meg benne 0 byte. Az UTF-8 bináris string így alkalmas
arra, hogy a program ezzel a típussal adjon meg egy fájlspecifikációt
az OS számára, amire a unicode string nem felelne meg.


E megfontolásokból adódik, hogy mikor melyik string 
fajtát érdemes/kell használni a programokban.
Alapszabály, hogy a program szövegeit karakter stringben
tároljuk, és ebben a formában manipuláljuk. 
Vannak persze esetek, amikor ettől el kell térnünk.
\begin{itemize}
\item 
    A fájlokból, socketekből általában byteokat lehet olvasni.
\item
    Az operációs rendszer számára UTF-8 kódolással 
    (tehát bináris string formájában) kell megadni a 
    fájlspecifikációkat.
\end{itemize}
A koncepció, hogy az alkalmazási programokban minél
kevesebbet kelljen váltogatni a bináris és karakter
reprezentáció között, ehelyett a CCC könyvtár függvényei
alkalmas helyen automatikusan elvégzik a konverziót.
A \verb!memoread!-et pl. általában arra használjuk,
hogy egy szövegfájlt egy mozdulattal beemeljünk  egy karakterváltozóba.
Ezért a \verb!memoread! automatikusan karakter stringre
konvertálja, amit olvas. Néha azonban más kell, pl. amikor
egy png képfájlt olvasunk be, ezért a \verb!memoread!
kiegészült egy opcionális paraméterrel, amivel kikapcsolható
ez a konverzió. Ilyenkor a \verb!memoread! eredménye nem 
karakter, hanem binary string.
Az \verb!fopen! a filénevet UTF-8-ra konvertálva adja 
lejjebb a C szintnek. Jelenleg még nincs kikristályosodva,
hogy hova célszerű ezeket az automatikus konverziókat tenni.
Azokon a helyeken, ahol a stringtípusok találkoznak,
az alkalmazásnak mindenképpen explicite kell konvertálnia, 
ezért a programok elkerülhetetlenül bonyolultabbak lesznek, 
mint a CCC2-ben voltak.

Megemlítendő, hogy a unicode/UTF-8 kódolás a CCC3-ban kizárólagos.
Ezen azt értem, hogy nincs támogatás semmilyen más kódolásra,
pl. Latin-1-re. Ezek a (hagyományos) kódolások elavultak,
és rendkívül megbonyolódik az élet, ha különféle kódolásokat
kell egyszerre kezelni. Mindez azt jelenti, hogy a CCC3 használata 
során az ember szövegfájljai szépen átkonvertálódnak UTF-8-ra.
A jelen sorokat a CCC3-mal fordított (tehát unicodeos) z editorral írom,
és a saját terminálomban az angol, magyar és orosz szöveget
egyformán helyesen látom (és tudom gépelni), mint ahogy helyesen
látszik a \TeX\  kimentetén és a böngészőben is. Mindehhez
nincs szükség bütykölt fontokra és billentyű driverekre.
Vannak tehát előnyök, amik kárpótolnak a bonyodalmakért. 
Bízzunk benne, hogy az UTF-8 kódolás hosszabb nyugvópont lesz 
a gyorsan változó informatikában.


\section{String (és egyéb) literálok}

\paragraph{Karakter string}
\begin{verbatim}
    x1:="Kázmér füstölgő fűnyírót húz."
    x2:="Копирование и распространение"
\end{verbatim}
A fenti értékadások  szövegének
kötelezően UTF-8 kódolásúnak kell lennie, másképp fordítási hiba
keletkezik: \verb!INVALIDENCODING!. Ebből adódóan nem nélkülözhető
az UTF-8/unicode környezet.
A programokat UTF-8 editorral kell írni (pl. a z-vel),
a régi szövegeket át kell konvertálni. A fordító maga nem konvertál,
csak hibát jelez, ha rossz a kódolás. A fordító az UTF-8 kódolású
szövegből előállítja a unicode karakterek sorozatát, és ez a sorozat 
(vagyis a C típusú string) lesz a változók új értéke.
C++ szinten a unicode (UCS) karakterek \verb!wchar_t! 
típusban tárolódnak, ami általában 32 bites. 


\paragraph{Binary string}
\begin{verbatim}
    x:=a"öt szép szűzlány őrült írót nyúz"
\end{verbatim}
A fenti értékadás eredményeképpen \verb!x! típusa binary string (X),
tartalma pedig a szöveget UTF-8 kódolással reprezentáló byte sorozat.


\paragraph{Hexadecimálisan megadott binary string}
\begin{verbatim}
    crlf:=x"0d0a"
\end{verbatim}
A fenti értékadás után \verb!crlf! egy két byte hosszú 
bináris string (X), ami a \verb!0d! és \verb!0a! byteokat tartalmazza.
Az \verb!x"..."! stringeknél minden byteot kétjegyű hexadecimális kóddal
adunk meg. Fordítási hibát okoz, ha a string páratlan számú betűt,
vagy nem hexa számot tartalmaz. A kis/nagybetű érdektelen.

\paragraph{Hexadecimálisan megadott számok}
\begin{verbatim}
    number255:=0xff
\end{verbatim}
A C-hez hasonlóan használhatunk hexadecimális egészszámokat.

\paragraph{Kettes számrendszerben megadott számok}
\begin{verbatim}
    number255:=0b11111111
\end{verbatim}
A számokat megadhatjuk bitenként is, azaz kettes számrendszerben.

\section{Bővített függvény API}

Nem tudok most komplett referenciát adni az újdonságokról,
mert nem emlékszem mindenre fejből. Ideiglenesen csak annyit akarok 
leírni, ami már megfelelően érzékelteti a dolgok hangulatát és logikáját.
A pontos infóhoz állandóan nézni kell a forrásokat, magam is ezt teszem.

Új függvények:
\begin{description}
\item[{\tt bin(code)}]
    A \verb!chr(code)! bináris párja. Egy 0--255 közé eső
    kódból egy byte hosszú bináris stringet készít.
\item[{\tt arr2bin(a)}]
    A korábbi \verb!_arr2chr!-t pótolja. Most nyilván a karakter
    stringeket is sorosítani kell, az eredmény egy bináris string
    (bytesorozat).
\item[{\tt bin2arr(x)}]
    \verb!arr2bin(a)! inverze.
\item[{\tt str2bin(c)}]
    Előállítja a \verb!c! karakter string UTF-8 kódolású bináris
    reprezentációját. Ebből információveszteség nélkül visszanyerhető
    az eredeti string. A bináris reprezentáció sok helyen helyettesítheti
    is \verb!c!-t.
\item[{\tt bin2str(x)}]
    \verb!str2bin(c)! inverze. Tudni kell azonban,  ha \verb!x!
    nem érvényes UTF-8 kódolású szöveget tartalmaz, akkor információ
    vész el, pl. egy png képfájl elromlik.
\item[{\tt split(v,sep)}]
    Helyettesíti a megszűnt \verb!wordlist!-et.
    Karakteres és bináris stringekre is működik.
%\item[{\tt }]
%\item[{\tt }]
%\item[{\tt }]
%\item[{\tt }]
%\item[{\tt }]
%\item[{\tt }]
%\item[{\tt }]
%\item[{\tt }]
%\item[{\tt }]
%\item[{\tt }]
%\item[{\tt }]
%\item[{\tt }]
%\item[{\tt }]
%\item[{\tt }]
%\item[{\tt }]
%\item[{\tt }]
%\item[{\tt }]
%\item[{\tt }]
%\item[{\tt }]
\end{description}

Módusult függvények.

\begin{description}
\item[{\tt chr(code)}]
    Egy 32 bites UCS kódból egy egy karakter hosszú stringet készít.
\item[{\tt asc(v)}]
    Ha \verb!v! karakter string, akkor az első karakter UCS kódját adja.
    Ha \verb!v! bináris string, akkor az első byte értékét adja.
\item[{\tt left(v,n)}]
    Ha \verb!v! karakter string, akkor \verb!v! első \verb!n!
    karakteréből álló részstringet adja.
    Ha \verb!v! bináris string, akkor \verb!v! első \verb!n!
    bytejából álló bináris részstringet adja.
    Utóbbi esetben, ha \verb!v! egy szöveg UTF-8 kódolású bináris 
    reprezentációja, akkor ez a tulajdonság elromolhat, amennyiben 
    a \verb!left! elvág egy több byteos UTF-8 kódot.
\item[{\tt len(v)}]
    Ha \verb!v! karakter string, akkor a \verb!v!-ben levő karakterek
    számát adja.
    Ha \verb!v! bináris string, akkor a \verb!v!-ben levő byteok
    számát adja. 
\item[{\tt replicate(v,n)}]
    A \verb!v! változó karakter és bináris string is lehet,
    az eredmény ennek függvényében C vagy X típusú. A rekordbuffereket
    régen a \verb!space! függvénnyel hoztuk létre. Mivel ennek
    az eredménye C típus, ez most általában nem jó, helyette
    ilyesmit írunk: \verb!replicate(x"20",n)!.
\item[{\tt fread(fd,@buf,n)}]
    Az \verb!fread! nem karaktereket, hanem byteokat olvas,
    ezért \verb!buf!-ot X típusúra {\it kell\/} inicializálni.
\item[{\tt fwrite(fd,buf,n)}]
    Az \verb!fwrite! nem karaktereket, hanem byteokat ír,
    ezért, ha C típusú \verb!buf!-ot adunk meg neki, azt automatikusan
    átkonvertálja X típusra \verb!str2bin!-nel.
\item[{\tt convertfspec2nativeformat(f)}]
    Az eredményét mindig átkonvertálja binárisra, ui.
    az operációs rendszernek UTF-8 kódolású fájlspecifikációkat
    lehet megadni.
\item[{\tt hashcode(v)}]
    Karakteres és bináris stringekre is működik.
\item[{\tt isalpha(v)}]
    Karakteres és bináris stringekre is működik.
    C szinten az \verb!iswalpha!, illetve az \verb!isalpha!
    hívódik meg. Karakter string esetén az ékezetes és cirill 
    betűkre is jó eredményt ad.
\item[{\tt qout(c,...)}]
    A karakter stringek kinyomtatás előtt automatikusan
    átkonvertálódnak UTF-8-ra (vagyis  binárisra), 
    ui. az operációs rendszerek ezt értik.
\item[{\tt savescreen()}]
    A képernyő bináris stringként mentődik, egy screen
    cella a korábbiaktól eltérően most 4 byte, mert UCS
    kódokat kell tárolni. A függvénycsalád összes tagja
    ehhez alkalmazkodik.
\item[{\tt upper(v)}]
    Karakteres és bináris stringekre is működik.
    C szinten a \verb!towupper!, illetve a \verb!toupper!
    hívódik meg. Karakter string esetén az ékezetes és cirill 
    betűkre is jó eredményt ad.
\item[{\tt val(x)}]
    Karakteres és bináris stringekre is működik.
\item[{\tt valtype(v)}]
    Bináris stringre "X"-et ad.
\item[{\tt like()}]
    Karakteres és bináris stringekre is működik.
\item[{\tt memoread(fspec [,binopt])}]
    Ha a \verb!binopt! empty, akkor \verb!bin2str!-rel
    karakterre konvertálja a beolvasott fájl tartalmát.
    Ez csak akkor jó, ha a fájl UTF-8 kódolású szöveget tartalmaz.
    Ha egy png képfájlt akarunk beolvasni, akkor azt 
    \verb!binopt:=.t.!-vel tesszük, az eredmény ilyenkor
    egy bináris string.
\item[{\tt memowrit(fspec,v)}]
    Ha \verb!v! egy karakteres string, akkor azt kiírás
    előtt átkonvertálja binárisra.
\item[{\tt inkey() }]
    Az inkey kódok megváltoztak, lásd az \verb!inkey.ch!-t.
%\item[{\tt }]
%\item[{\tt }]
%\item[{\tt }]
%\item[{\tt }]
%\item[{\tt }]
%\item[{\tt }]
%\item[{\tt }]
%\item[{\tt }]
%\item[{\tt }]
%\item[{\tt }]
%\item[{\tt }]
%\item[{\tt }]
%\item[{\tt }]
%\item[{\tt }]
%\item[{\tt }]
%\item[{\tt }]
%\item[{\tt }]
%\item[{\tt }]
%\item[{\tt }]
%\item[{\tt }]
%\item[{\tt }]
\end{description}

\section{Internacionalizálás}

Internacionalizálásnak sok összetevője van, 
mi itt csak egy dologgal foglalkozunk: 
Hogyan lehet többnyelvű CCC programot írni, 
amiben a string konstansok egyszerűen cserélhetők különféle 
nyelvi változatokra. 
Egy működő példa található a \$CCCDIR/tutor/nlstext directoryban, 
ezt a példát magyarázom el részletesen az alábbiakban.

Az nlstext.prg program:
\begin{verbatim}
static x:=@"Some like hot"

function main()
    nls_load_translation("nlstext")
    fun()
    ?

static function fun()
static y:=@'Gentlemen prefer blondes'
local z:=@"Star war"
    ? x
    ? y
    ? z
    ? @"Matrix"
    ?
\end{verbatim}
Először is azokat a stringeket, amiket a program különböző 
nyelvű verzióiban fordításban akarunk látni, meg kell jelölnünk.
Erre szolgál a speciális \verb!@"..."! szintaktika.
A programfordítás idejére beállítjuk az alábbi 
környezeti változót:
\begin{verbatim}
export CCC_NLSTEXT_TAB=$(pwd)/nlstext.tran
\end{verbatim}
Ennek hatására a \verb!ppo2cpp! fordító kigyűjti nekünk a 
kukaccal megjelölt stringeket egy szövegfájlba, 
esetünkben nlstext.tran-ba:
\begin{verbatim}
"Some like hot"<<"" from  ./nlstext.prg  (21)
"Gentlemen prefer blondes"<<"" from  ./nlstext.prg  (33)
"Star war"<<"" from  ./nlstext.prg  (34)
"Matrix"<<"" from  ./nlstext.prg  (39)
\end{verbatim}
Itt soronként egy stringet találunk. A sor a lefordítandó
stringgel kezdődik, utána jön egy \verb!<<! jel, majd egy 
üres idézet, ahová a fordítást kell majd beírni.
Az eddigiek azt jelölik, hogy a bal oldali stringet 
helyettesíteni fogja a jobb oldalra írt fordítás.
A sor végén fel van tüntetve, hogy az adott string melyik
forrásfájl melyik sorából származik.  Természetesen,
ha a project sok forrásfájlból áll, akkor az egyes fájlokból
jövő járulék halmozódik, ezért egy nagyobb program esetén
ezres nagyságrendű sor lehet az eredmény.

Minden nyelvhez készítünk egy-egy directoryt, esetünkben 
\begin{verbatim}
    translation/hu
    translation/ru
\end{verbatim}
ezekbe átmásoljuk az nlstext.tran egy-egy példányát, 
ezeken fognak dolgozni a fordítók. A fordító munkájának
eredménye egy ilyen fájl:
\begin{verbatim}
"Some like hot"<<"Несколько мужчин любят горячо" from  ./proba.prg  (21)
"Gentlemen prefer blondes"<<"Господа любят лучше блондинок" from  ./proba.prg  (33)
"Star war"<<"Война эвёэд" from  ./proba.prg  (34)
\end{verbatim}
Ebből a fájlból a \verb!tran2cpp! utility  C++ forrást generál,
amit lefordítunk, és  dinamikus könyvtárat linkelünk belőle.
Elvégezzük ugyanezeket a műveleteket a magyar változatra is.
A dinamikus könyvtárak neve:
\begin{verbatim}
    translation/libnlstext.hu.so
    translation/libnlstext.ru.so
\end{verbatim}
Természetesen ugyanez megy Windowson is, csak ott dll-eket kapunk.

Namost, ha az nlstext.exe programot egy ilyen scripttel indítjuk:
\begin{verbatim}
#!/bin/bash
export CCC_LANG=ru
export LD_LIBRARY_PATH=./translation:$LD_LIBRARY_PATH
nlstext.exe
\end{verbatim}
akkor a program elején található
\begin{verbatim}
    nls_load_translation("nlstext")
\end{verbatim}
függvényhívás (amiről eddig nem szóltunk) a \verb!CCC_LANG!
változó értékéből és a paraméterként kapott "nlstext" szövegből
összerak egy könyvtárnevet, és a könyvtárat megpróbálja betölteni. 
Ha ez a betöltés sikeres, akkor a program a \verb!@"..."! stringek
helyett azok fordításait fogja megjeleníteni. Ha a fordításkönyvtár
dinamikus betöltése nem sikeres, vagy a könyvtár nem tartalmaz fordítást
egyik vagy másik stringre, attól még  működni fog a program,
csak ekkor a fordítással nem rendelkező stringek eredeti szövege 
jelenik meg.

\section{Megjelenítő könyvtárak}

A megjelenítő modulok átfogó profiltisztításon estek át.
A CCC3-ból kimaradtak az \verb!uiw!, \verb!uif!, \verb!uid! könyvtárak. 
Ha egy alkalmazás ezekre épül, akkor azt továbbra is a CCC2-vel kell 
fordítani (aminek a karbantartása folytatódik), vagy alkalmazássszinten 
kell megoldani a könyvtárak portolását (a CCC2-beli állapotukban ezek nem
támogatják az UTF-8-at). 
A CCC3 háromféle megjelenítőkönyvtárat tartalmaz:
\begin{description} 
\item[{\tt ccc3\_uic}]
    Ez a hagyományos karakteres megjelenítés,
    ami messzemenően kompatibilis az eredeti Clipperrel, 
    ám a CCC3-beli megvalósítás támogatja a unicodeot.
\item[{\tt ccc3\_gtk}]
    Interfész a GTK-hoz. A GTK egy platformfüggetlen 
    grafikus könyvtár, amihez számos nyelv tartalmaz 
    csatolót. A GTK mindig is UTF-8 kódolással dolgozott,
    most ehhez jól illeszkedik a CCC3.
    Lásd: \href{cccgtk.html}{CCC-GTK csatoló}.
\item[{\tt ccc3\_jt}]
    Saját találmány a Jáva megjelenítő modul (Jáva terminál).
    A Jáva szintén a kezdetektől támogatta a unicodeot,
    most ezt a tulajdonságát kihasználja a CCC3.
    Lásd: \href{jterminal.html}{Jáva Terminál}.
\end{description} 

Egyszerűsítések történtek az \verb!uic! könyvtáron belül is.
A CCC2-ben (UNIX-on) a karakteres megjelenítés történhetett
X-szel, az ncurses könyvtárral, illetve távoli terminálon.
A CCC3 csak az utóbbit (a távoli terminált) tartja meg.
Ha egy alkalmazást a lokális munkaállomáson futtatunk,
és ugyanitt akarjuk megjeleníteni, akkor is a terminált
kell használnunk. A terminál konfigurálásával a következő
fejezet foglalkozik.  A (távoli) terminál előnye, 
hogy az \verb!uic! könyvtárnak nincs külön UNIX-os
és windowsos változata, ui. az \verb!uic! feladatai a terminállal
való kommunikációnál véget érnek.
A tényleges megjelenítéshez persze kell egy terminál program 
UNIX-ra és egy másik Windowsra, ezek foglalják magukba a
megjelenítés platform specifikus részleteit.

\section{Karakteres terminál}

\subsection{Konnektálás}

A szerverprogram és a terminál többféle módon kapcsolódhat.
Sorravesszük a lehetőségeket:

\subsubsection{Lokális szerver és terminál}
\begin{verbatim}
    export CCCTERM_CONNECT="a_terminál_teljes_fájlspecifikációja"
\end{verbatim}
Ez a legegyszerűbb módja, hogy egy gépen CCC3 környezetben 
dolgozzunk. A környezeti változó tudatja a programokkal,
hogy el kell indítaniuk maguk számára a terminált, ezt meg is teszik,
a program és a terminál automatikusan konnektálnak.


\subsubsection{Listener a szerveren}

Elindítjuk az \verb!xstart!-ot (CCC listener) a szerveren,
az xstartnak az alábbi paraméterfájlt adjuk meg:
\begin{verbatim}
<xstart>
<item>
    <name>Test Program</name>
    <port>55000</port>
    <env>CCCTERM_CONNECT=SOCKET:$(SOCKET)</env>
    <command>proba1.exe</command>
</item>
</xstart>
\end{verbatim}
Ha most elindítjuk a terminált egy másik gépen
\begin{verbatim}
    terminal.exe host 55000
\end{verbatim}
ahol \verb!host! az xstartot futtató gép, akkor a lokálisan
futó terminálban megjelenik a \verb!host!-on automatikusan elinduló 
\verb!proba1.exe! program képernyője. Figyeljük meg, 
hogy a \verb!CCCTERM_CONNECT! környezeti változó (most egy másik 
szintaktikával) a listenertől örökölt  socket leíróját tudatja
a szerver programmal.


\subsubsection{Listener a munkaállomáson}
 
Elindítjuk az \verb!xstart!-ot lokális munkaállomáson
az alábbi paraméterfájllal:
\begin{verbatim}
<xstart>
<item>
    <name>terminal</name>
    <port>55000</port>
    <env>CCCTERM_SIZE=80x43</env>
    <command>terminal.exe --socket $(SOCKET)</command>
</item>
</xstart>
\end{verbatim}

A (távoli) szerveren beállítjuk:
\begin{verbatim}
    export CCCTERM_CONNECT=host:55000
\end{verbatim}
Ahol \verb!host! a munkaállomásunk neve (vagy IP címe), 
ahogy azt a távoli szerver ismeri. Ha ezután a szerveren
elindítunk egy tetszőleges karakteres CCC3 programot, akkor 
a lokális munkaállomáson automatikusan elindul a terminál,
és abban megjelenik a távoli gépen futó program képernyője.
Figyeljük meg, hogy a \verb!CCCTERM_CONNECT! környezeti
változó (most egy harmadik szintaktikával) azt tudatja 
a szerver programmal, hogy hova kell konnektálni a
megjelenítés érdekében.

Természetesen lehetséges, 
hogy az előző esetek akármelyikében,
a szerver és a munkaállomás ugyanaz a gép legyen.


\subsubsection{Összegzés}

A szerver program (\verb!uic! könyvtár) és a terminál 
alapvetően kétféle módon juthatnak hozzá a kommunikációs sockethez,
konnektálnak, vagy öröklik a socketet:

Ha a szerveren beállítjuk:
\begin{verbatim}
    export CCCTERM_CONNECT=host:port
\end{verbatim} akkor a program a \verb!host:port! címre próbál konnektálni,
ahol egy listenernek kell várnia a konnektálásokat, és indítania a terminált.
Ha a terminált így indítjuk:
\begin{verbatim}
    terminal.exe host port
\end{verbatim} akkor a terminál a \verb!host:port! címre próbál konnektálni,
ahol egy listenernek kell várnia a konnektálásokat, és indítania a 
szerver alkalmazásokat.

A másik oldalon, a listener úgy indítja a szerver alkalmazást,
hogy beállítja számára a 
\begin{verbatim}
    export CCCTERM_CONNECT=SOCKET:sck
\end{verbatim} környezeti változót, amiből a szerver program 
(\verb!uic! könyvtár)  értesül az örökölt socket leírójáról (\verb!sck!).
Ha a listener így indítja a terminált,
\begin{verbatim}
    terminal.exe  --socket sck
\end{verbatim} akkor a terminál tudja, hogy \verb!sck!-ban az örökölt 
socket leíróját kapta meg. 

Speciálisan kezelhető a legegyszerűbb eset, amikor az
\begin{verbatim}
    export CCCTERM_CONNECT="a_terminál_teljes_fájlspecifikációja"
\end{verbatim} beállítás hatására a szerver program automatikusan 
indítja maga számára a terminált.

\subsection{A terminál paraméterei}

A terminál egy sor további környezeti változóval paraméterezhető:
\begin{description}
\item[{\tt CCCTERM\_SIZE}]
\item[{\tt CCCTERM\_REDIR\_CONSOLE}]
\item[{\tt CCCTERM\_REDIR\_PRINTER}]
\item[{\tt CCCTERM\_REDIR\_ALTERNATE}]
\item[{\tt CCCTERM\_REDIR\_EXTRA}]
\item[{\tt CCCTERM\_REDIR\_ERROR}]
\item[{\tt CCCTERM\_CAPTURE\_PRN}]
\item[{\tt CCCTERM\_CAPTURE\_LPT1}]
\item[{\tt CCCTERM\_CAPTURE\_LPT2}]
\item[{\tt CCCTERM\_CAPTURE\_LPT3}]
\item[{\tt CCCTERM\_FONTSPEC}] %UNIX-on.
\item[{\tt CCCTERM\_FONTFACE}] %Windowson.
\item[{\tt CCCTERM\_FONTSIZE}] %Windowson.
\end{description}









