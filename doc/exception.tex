
\setcounter{secnumdepth}{1}  

\def\megj{{\bf Megjegyzés:~}}
\def\szab{{\bf Szabály:~}}
\def\tipus#1{\item {\tt #1~}}

\pagetitle{Struktúrált kivételek a CCC-ben}%
{Dr. Vermes Mátyás}%
{2005. június}

\section{Hibakezelés a régi Clipperben}

Ha fatális hibát észleltünk, végrehajtottuk az error\-blokkot.
A default error\-blokk általában értesít a hiba körülményeiről
(kiírja a stacket) majd kilépteti a programot.

Ha nem akartuk, hogy egy bizonyos hibára a program kilépjen,
kicseréltük az errorblokkot \verb!{|x|break(x)}!-re, az esetleges 
breaket pedig elkaptuk egy recoverrel. Ezzel a technikával 
az a baj, hogy  ilyenkor a recover más  hibákat is elkap, 
amivel megnehezül a program tesztelése, mivel a recover után 
már nincs meg a hiba környezete.

\megj
A (régi) break semmi mást nem csinált, mint visszaállította 
a stacket, és a legbelső recover ágnak átadta a vezérlést.
Ha a break körül nem volt begin-recover utasítás, akkor  a program 
csendben kilépett. Nem túl szerencsés ez a csendbeni kilépés, 
de ilyen volt a Clipper.

Megtehettük, hogy a hibakezelés gyanánt \verb!eval(errorblock(),x)! 
helyett rögtön \verb!break(x)!-et írtunk. Így a korábbi eszközökkel is
lehetett programozni az errorblokk cserélgetése nélkül:

\begin{verbatim}
function tranzakcio()
local e
    begin
        tranzakcio1()  //break lehet benne
    recover using e
        //hibakezelés
    end
\end{verbatim}

A probléma az, hogy a hiba keletkezésének helyén el kell dönteni, 
hogy \verb!eval(errorblock(),x)!, vagy \verb!break(x)! legyen-e a reakció. 

\subsubsection{\tt eval(errorblock(),x)}

Ha a tranzakciót hívó program nem cseréli ki az errorblokkot,
akkor a hiba környezetéről részletes jelentés készül, majd
a program kilép. Ha a tranzakciót hívó program kicseréli az 
errorblokkot, akkor átveheti a hibakezelést, de ilyenkor bajlódnia
kell a bezavaró  hibákkal is, amiknek már nincs is meg a környezete.

\subsubsection{\tt break(x)}

Csak akkor célszerű alkalmazni, ha ismeretes, hogy a tranzakciót 
hívó program recoverrel készült fel a hiba elkapására (máskülönben 
a program csendben kilép). Nehéz a nagy programokat összehangolni.

A fatális hibákat (különösen a programozási hibákat)  jobb 
a default error\-blokkal elintézni. A magasabb szintű tranzakciós 
hibákat gyakran jobb breakkel visszaadni a hívónak. A nagy 
rendszerek egyes részeinek egyet kell érteniük abban, hogy 
mikor melyik eset áll fenn.


\section{Strukturált kivételkezelés}

A struktúrált kivételkezelés eszközöket ad ahhoz, hogy a hibakezelést
az egyszerűbb, breakes módszerre alapozzuk.
Az első intézkedés:~
Az el nem kapott kivétel nem csendben kilép,
hanem  kiértékeli az errorblokkot (enyhe inkompatibilitás).
Így a recover nélküli break is teljes hibajelentést ad.

Az új break nem egyszerűen a legbelső recovert keresi meg,
hanem az egymásba skatulyázott (vagy ilyennek tekinthető)
begin-recover utasítások között belülről kifelé haladva addig keres, 
amíg megfelelő típusú recovert talál, és arra adja a vezérlést. 
Ez lehetővé teszi, hogy egy recover elkapjon egy bizonyos típusú hibát, 
miközben nem semmisíti meg más olyan hibák környezetét, amit nem tud,
vagy nem akar kezelni.


\begin{verbatim}
function ff(x)

local e

    begin /*sequence*/
        ? "HOPP-1"
        break(x)
        ? "HOPP-2"

    recover  e  <specerror>  //elkapja specerror leszármazottait
        ? "rec1", e:classname

    recover  e  <error> //elkapja error leszármazottait
        ? "rec2", e:classname

    recover  e <c> //elkapja a stringeket
        ? "rec3", upper(e)

    recover /*using*/ e  //bármit elkap (régi szintaktika)
        ? "rec4", e

    recover //ez is bármit elkapna (felesleges) 
        ? "ide nem jöhet"

    finally
        ? "lefut a begin-recover elhagyásakor"

    end 
\end{verbatim}

Új szintaktika: a sequence kulcsszó elhagyható (zajszó)!\\
Új szintaktika: a using kulcsszó elhagyható (zajszó)!\\
Új szintaktika: több recover lehet lineárisan felsorolva!\\
Új szintaktika: a recover változó után megadható egy kifejezés!\\
Új szintaktika: opcionális finally klóz!\\

A {\bf lineárisan felsorolt recovereknek} ugyanaz a hatása, mintha 
a begin-recover utasítások  egymásba volnának skatulyázva
(feltéve, hogy a recover ágakból nem indul újabb break):

\begin{verbatim}
function ff(x)

local e

    begin
        begin
            begin
                begin
                    begin
                        ? "HOPP-1"
                        break(x)
                        ? "HOPP-2"
                    recover  e specerrNew() 
                        ? "rec1", e:classname
                    end
                recover  e errorNew()
                    ? "rec2", e:classname
                end
            recover  e ""
                ? "rec3", upper(e)
            end
        recover using e
            ? "rec4", e
        end
    recover
        ? "ide nem jöhet"
    finally
        ? "lefut a begin-recover elhagyásakor"
    end
\end{verbatim}


A recover utasításban az újdonság a végére írt 
{\bf típuskifejezés}. A típuskifejezés formái a következők:
\begin{itemize}
\tipus{expr}
    Tetszőleges CCC kifejezés, mint  ami pl. a return után állhat. 
    A kifejezés még a begin előtt kiértékelődik 
    (mellékhatásai lehetnek, elszállhat).
    A végeredménynek csak a típusa érdekes, 
    ez a típus határozza meg a recover típusát.
\tipus{ <c>} A recover típusa string.
\tipus{ <d>} A recover típusa dátum.
\tipus{ <l>} A recover típusa logikai.
\tipus{ <n>} A recover típusa szám.
\tipus{ <a>} A recover típusa array.
\tipus{ <b>} A recover típusa kódblokk.
\tipus{ <other>} 
    A recover típusa az \verb!other! osztály, melynek 
    osztályazonosítóját konvenció szerint az \verb!otherClass()! 
    függvény szolgáltatja.
\end{itemize}


Tehát a recover változó után írt kifejezéssel lehet 
beállítani a recover típusát és ezáltal szűrni, hogy
milyen típusú hibákat kapjon el az adott recover.
{\bf A kivétel elkapásának szabályai}:

\begin{itemize}
\item
Ha a recover változó után egyáltalán nincs kifejezés,
vagy a kifejezés értéke NIL, 
akkor a recover utasítás minden kivételt elkap 
(ez a Clipperből ismert, régi eset).

\item
Ha a dobott kivétel és a recover típusa megegyezik,
és a típus nem object, akkor a recover elkapja a kivételt.

\item
Ha a dobott kivétel és a recover típusa is object,
a recover akkor kapja el a kivételt, ha a kivétel osztálya
leszármazottja a recover osztályának.

\item
Ha a begin-end között egyáltalán nincs (kiírt) recover, 
sem finally (lásd később), az olyan, mintha közvetlenül az end előtt 
volna  egy  mindent elkapó üres recover.

\item
Ha a begin-end között egyáltalán nincs recover,
de van benne finally ág,
akkor a begin-end semmilyen kivételt nem kap el. 
Ennek a formának az értelme a finally ág végrehajtatása
a begin utasítás elhagyásakor.

\item
A kivételt elkapó recovert a break() keresi meg. 
A break() az egymásba skatulyázott (vagy ilyennek tekinthető)
begin-recover utasításokat belülről kifelé haladva vizsgálja, 
eközben átlépi a függvényhívási határokat is.

\item
Ha nincs olyan recover, ami alkalmas a hiba elkapására
(kezeletlen kivétel), akkor a dobott kivétellel kiértékelődik
az errorblokk. Ilyenkor a vezérlés nem ugrik ki a begin
utasításból, ezért nem is hajtódnak végre az esetleges
finally ágak.
\end{itemize}



\megj Az új kivételkezelés a régi Clipper természetes kiterjesztése,
abban az értelemben, hogy  speciális esetként tartalmazza a régi Clipperből
ismert formákat. Megmarad a (majdnem teljes) kompatibilitás 
a korábbi forrásokkal, ui. az új működés új szintaktikához kapcsolódik.

\megj  A kivétel típustól függő elkapásához típusinfót
kell rendelnünk a recoverekhez, ezt szolgálják a recover változók
után írt kifejezések. Az összes ilyen kifejezés a begin utasítás előtt
hajtódik végre. Ha van mellékhatásuk, azzal a programban 
számolni kell. A kifejezések kiértékelésének sorrendje implementációfüggő,
azaz nem szabad számítani egy meghatározott sorrendre.
Alapesetben a kifejezések értéke CCC szintű változókon keresztül 
nem érhető el, éppen ezért az értékhez tartozó memóriaobjektumokat
a szemétgyűjtés bármikor (akár azonnal) megszüntetheti, de ez nem 
okoz gondot, mivel a kifejezéseknek csak a típusára van szükség,
a tényleges értékére nem. A \verb!<symbol>! alakú típuskifejezések eleve létre
sem hozzák a kérdéses memóriaobjektumokat, ezért hatékonyabbak.

Amikor a vezérlés elhagyja a begin-recover utasítást, 
végrehajtódik  a {\bf finally ág\/}. A finally ág lefut,
\begin{itemize}
\item ha a begin-recover közötti kód  return, loop, exit, break nélkül lefut,
\item ha a begin-recover utasításból a return, loop, exit
      utasítások valamelyikével kiugrunk,
\item ha a futást megszakítja egy kivétel (break),
      és azt elkapja egy recover,
      vagyis a begin-recover egy recover ággal ér véget,
\item ha a futást megszakítja egy kivétel,
      amit  ugyan az adott begin-recover utasítás egyik recovere
      sem kap el, de elkapja valamely kijjebb levő recover.
\end{itemize}

Ha a kivételt egyáltalán semmi sem kapja el, 
akkor a break eredeti környezetében kiértékelődik az  errorblock.
Ilyenkor a finally ág nem hajtódik végre, hiszen a vezérlés nem hagyta
el a begin-recover utasítást. Ha például a hibát a default errorblokk
kezeli, és az error objektumban \verb!candefault==.t.!, akkor
a break még vissza is térhet.

A régi Clipper tiltotta a 
{\bf begin-recover közül történő kiugrást} 
return, loop, exit utasításokkal.
A CCC új kivételkezelése (a Jávához hasonlóan) ezt lehetővé teszi,
és az elhagyott begin-recover utasítások finally ágait
(belülről kifelé) végrehajtja.


\section{Átállás az új kivételkezelésre}


\subsubsection{Mi van a régi programokkal?}

A régi programokat nem kötelező átalakítani. 
Például a Kontó minden változtatás nélkül is kiválóan fut.
Érdemes azonban tudni a régi és új kivételkezelés eltéréseiről
Az eltérések az alábbiakban foglalhatók össze:

\begin{enumerate}
\item Az el nem kapott break kiértékeli az errorblokkot.
Általában nincsenek a programokban ilyen breakek,
ha mégis vannak, az hiba, és ez most legalább ki fog derülni.

\item Az új CCC könyvtárak ezentúl error-on kívül más
error-ból leszármazó hibákat is dobhatnak. 
Az alkalmazási programoknak nem létfontosságú, 
hogy erről tudjanak, kezelhetik a hibákat egyszerű
error-ként is.

\item A régi stílusú hibakezelésre felkészült programok
továbbra is cserélgetik az errorblokkot (feleslegesen),
ez azonban nem fog hibát okozni.
\end{enumerate}

A következőkben megvizsgáljuk a részleteket, 
miközben eljárást dolgozunk ki arra,
hogyan lehet viszonylag mechanikusan átállni
az új kivételkezelésre.


\subsubsection{Recover nélküli break}

A recover nélküli break az eddigiektől eltérően nem csendben
kilép, hanem végrehajtja az errorblokkot. Emiatt egy el nem kapott
\verb!break(x)! hatása megegyezik \verb!eval(errorblock(),x)! hatásával.
A break még vissza is térhet, ha pl.\ a default errorblokk kezeli le, 
és a dobott errorban \verb!candefault==.t.! vagy \verb!canretry==.t.!.

\szab \verb!break(x)! majdnem ugyanaz, mint 
\verb!eval(errorblokk(),x)!, kivéve, hogy a break begin-recoverrel 
eltéríthető. Speciálisan, ha nincs recover, akkor 
\verb!break(x)! ugyanaz, mint \verb!eval(errorblock(),x)!.

\megj
Tudni kell, ha az errorblokk \verb!{|x|break(x)}!-re van cserélve,
akkor a recover nélküli break végtelen rekurziót okoz.

\megj
Ezt a működést a régi Clipperben és a vele kompatibilis (korábbi)
CCC-ben nem lehetett megvalósítani. Ha ui. a main-eket átírjuk a 
következő módon
\begin{verbatim}
function main()
local e
    begin
        main1()
    recover using e
        eval(errorblock(),e)
    end
\end{verbatim}
akkor ugyan elkaphatunk minden (eredetileg) recover nélküli breaket, 
viszont elveszítjük a hibák eredeti környezetét, azaz nem lesznek 
használható hibaüzeneteink.


\subsubsection{{ eval(errorblock(),x)} cseréje { break(x)}-re}

Szinte minden  \verb!eval(errorblock(),x)!-t \verb!break(x)!-re kell cserélni,
az alapkönyvtárban is, és az alkalmazásokban is. Ez a változtatás 
elvileg mindenhol végrehajtható a működés változása nélkül. 

Kivételesen ott változhat a működés, ahol az  errorblokk le van 
cserélve, de nem a szokásos \verb!{|x|break(x)}!-re, hanem valami másra. 
Az ilyen eseteket külön meg kell vizsgálni.

Ott is vátozhat a működés, ahol nincs lecserélve errorblokk,
viszont a prog\-ramban egy korábban semmit el nem kapó recover van.
A csere aktivizálhat egy ilyen alvó recovert.


\subsubsection{Errorblokk cserebere megszüntetése}

Azokban az esetekben, amikor az errorblokk egy tranzakció erejéig 
\verb!{|x|break(x)}!-re van cserélve, majd visszaállítva, a cserebere 
megszüntethető.  
Az egész programra kiterjedő errorblokk cseréket meg kell őrizni.
Például a Kontó a hibákat az errorblokkban naplózza. A z editor bármilyen 
hiba esetén kilépés előtt az editált szöveget menti. 

Ha a feleslegessé vált errorblokk cserét bennefelejtjük a programban,
az nem okoz hibát.


\subsubsection{A hibák minősítése}

A CCC belső hibáira és a programozási hibákra (pl. argument error)
error objektumot dobunk (most is így van). Az ilyeneket általában
nem fogjuk elkapni, kivéve, ha mindenképpen meg kell akadályozni
a program elszállását.

Azokra a hibákra, amiket esetleg el akarunk kapni error leszármazottat 
dobunk. Nem errort, azért, hogy  az alkalmazási hibák elkülönüljenek 
a programozási hibáktól.

Azoknál a hibáknál jó error-t dobni, ahol a hibát nem a kivétel
elkapásával, hanem a program kijavításával kell kezelni.


\subsubsection{Kivétel osztályok}

Ilyesmi lehetne:

\begin{verbatim}
error -> apperror

error -> apperror -> invalidoptionerror

error -> apperror -> invalidformaterror
error -> apperror -> invalidformaterror -> invalidstructerror
error -> apperror -> invalidformaterror -> xmlsyntaxerror
error -> apperror -> invalidformaterror -> xmlsyntaxerror -> xmltagerror

error -> apperror -> tabobjerror
error -> apperror -> tabobjerror -> tabindexerror
error -> apperror -> tabobjerror -> tabstructerror
error -> apperror -> tabobjerror -> memoerror
error -> apperror -> tabobjerror -> tranlogerror

error -> apperror -> ioerror
error -> apperror -> ioerror -> eoferror
error -> apperror -> ioerror -> fnferror
error -> apperror -> ioerror -> readerror
error -> apperror -> ioerror -> writeerror
error -> apperror -> ioerror -> socketerror
\end{verbatim}

Nagyon lényeges kérdés, hogy a kivételosztály hierarchia jól 
el legyen  találva.  

A Jávától eltérően az az alapállás, hogy {\em a hibákat nem kapjuk el},
hanem hagyjuk, hogy a program elszálljon, és magától kiíródjon a hiba 
környezete. Csak akkor kapunk el egy hibát, ha érdemben ki is tudjuk 
javítani, vagy mindenképpen meg kell akadályozni a program elszállását. 
A kivétel osztályokat ennek megfelelően kell megtervezni.

A könyvtárakat ki kell egészíteni olyan modulokkal, amik definiálják 
az adott könyvtár által használt kivétel osztályokat, és dokumentálni kell, 
hogy honnan, milyen kivétel jöhet, mint ahogy az a Jávában is van.


\subsubsection{A begin-recover utasítások felülvizsgálata}

Ha az eddigieket végrehajtjuk
\begin{itemize}
\item eval(errorblock(),x) --> break(x),
\item errorblokk cserebere megszüntetése,
\item hibák minősítése,
\end{itemize}
és eközben vigyázunk, hogy a csekély számú kivételes esetet 
ne rontsuk el, akkor a programok működésében nem lesz változás,
viszont még nem is használtuk ki az előnyöket.
Ezután van értelme áttérni az új begin-recover utasítás használatára, 
ami lehetővé teszi, hogy a recoverekkel válogassunk az eddig ömlesztve 
kapott hibák között, a finally ágakkal pedig kényelmesen 
takarítsunk magunk után.

