
\pagetitle{CCC2 telepítés}{Dr. Vermes Mátyás}{2004. október}



\section{Újabb infó}

Ez a dokumentáció a régebbi CCC2-ről szól.
Az új CCC3 telepítéséről rövid dokumentáció található 
\href{http://ccc.comfirm.hu/ccc3/download/olvass.html}{itt}.
Néhány részletkérdés miatt azonban érdemes lehet ezt a régebbi
dokumentációt is nézegetni.

\section{CCC installáció gyorsan}

Feltételezzük, hogy Linux/UNIX-on, vagy Windows-2K-n vagyunk,
és telepítve van gépünkön egy alkalmas C fordító. 
UNIX-on telepíteni kell a bash parancsértelmezőt,
ennek kell lennie a default shellnek, és a \verb!/bin/bash!
paranccsal kell tudni indítani.
Elkészítjük a CCCDIR directoryt, és beállítjuk az alábbi
környezeti változókat (ez itt UNIX, Windowson természetesen
windowsos szintaktikát használunk):
\begin{verbatim}
export CCCDIR=/opt/ccc2 # bárhol lehet (ahol írási jogunk van)
export CCCUNAME=linux   # lehetséges: linux,solaris,freebsd,nt
export CCCBIN=lin       # lehetséges: lin,sol,fre,mng,bor,msc
export PATH=$PATH:$CCCDIR/usr/bin/$CCCUNAME
export LD_LIBRARY_PATH=$LD_LIBRARY_PATH:$CCCDIR/usr/lib/$CCCBIN
\end{verbatim}

A CCCUNAME változó tartalmazza az OS nevét (a Windowst a régi NT 
név alapján nt jelöli). A CCCBIN változó tartalmazza a tárgykód típusát:
\begin{center}
\begin{tabular}{|l|l|}
     lin & Linux/GCC \\
     sol & Solaris/GCC \\
     fre & FreeBSD/GCC \\
     mng & Windows/MinGW (GCC) \\
     msc & Windows/Microsoft C \\
     bor & Windows/Borland C \\
\end{tabular}
\end{center}


\href{http://www.comfirm.hu/ccc2/download}{Letöltjük} az alábbi csomagokat:
\begin{center}
\begin{tabular}{|l|l|}
     Linux/UNIX                     & Windows\\
     \tt ccc2-yyyymmdd.zip          & \tt ccc2-yyyymmdd.zip \\
     \tt setup-flex-yyyymmdd.zip    & \tt setup-flex-yyyymmdd.zip \\
     \tt setup-lemon-yyyymmdd.zip   & \tt setup-lemon-yyyymmdd.zip \\
     \tt setup-unix-yyyymmdd.zip    & \tt setup-windows-yyyymmdd.zip \\
\end{tabular}
\end{center}

Azaz a négy csomagból csak a setup-unix/setup-windows különbözik.
Kibontjuk a csomagokat CCCDIR-ben.
Linux/UNIX-on elindítjuk \verb!install.b!-t, Windowson
\verb!install.bat!-ot.



\section{CCC csomagok beszerzése}
 
A CCC csomagokhoz kétféleképpen lehet hozzájutni:
\begin{itemize}
\item 
  A \href{http://www.comfirm.hu/ccc2/download}{http://www.comfirm.hu/ccc2/download}
  URL-ről a böngészővel egyenként letölthetők a csomagok.
\item 
  Ha installálva van gépünkön a Jáva Webstart, akkor használhatjuk a
  \href{http://www.comfirm.hu/jnlp/cccdown.jnlp}{CCC Letöltő}-t,
  ami letölti és/vagy frissíti a csomagokat.
  Érdekesség, hogy ez a program maga is CCC-ben készült.
\end{itemize}


A CCC installációja közel sem automatikus. 
Feltételezzük, hogy a kedves felhasználó eligazodik
a shell scriptek és C fordítók világában. 

A C fordításban törekedni kell a warning-mentességre.
E sorok írásakor ez teljesül: az összes C program, az összes
C fordítóval warning-mentesen fordul. Újabb változatú
fordítók esetén a szigorúbb ellenőrzés felszínre szokott hozni
újabb warningokat, ha azonban túl sok ilyen van, akkor
valószínűleg a fordító nincs rendesen telepítve.

 
Tisztában kell lenni bizonyos függőségi viszonyokkal:
\begin{enumerate}
\item 
  A fordító eszközök (prg2ppo, ppo2cpp) és az XML elemző
  igénylik a Flex és Lemon programokat. Az újabb CCC kiadásokban
  a CCC-vel együtt automatikusan települ a Flex és a Lemon. 
\item 
  Az Fltk megjelenítéshez előzőleg installálni kell az Fltk
  grafikus könyvtárat.
\item
  Az OpenSSL interfésznek szüksége van az openssl könyvtárra.  
\end{enumerate}
Ezek az eszközök általában nincsenek meg maguktól, 
hanem be kell őket szerezni, installálni kell őket, 
és némi gyakorlatot kell szerezni a használatukban.
 

\section{CCC környezet kialakítása}

\subsection{Windows}

Mindenképpen legalább Windows NT-re van szükség, Windows 2000 ajánlott, 
Windows 9x biztosan nem felel meg.
Szükségünk van egy C fordítóra, a támogatott fordítók:
\begin{itemize}
\item 
    MinGW, utolsó tesztelt változat MinGW320rc, 
    amiben GCC 3.4.2-es fordító van,
    letölthető a \href{http://www.mingw.org}{http://www.mingw.org}-ról.
    Ez az előnyben részesített fordító.
\item 
    Microsoft C, az utolsó tesztelt fordító változatszáma 13.10.3077.
    Letölthető a \href{http://www.microsoft.com}{http://www.microsoft.com}-ról.
    Két csomagot kell letölteni. A {\em Visual C++ Toolkit} csomagban (32~MB)
    van a fordító, a linker, a C és C++ könyvtárak, 
    a {\em Windows Server 2003 SP1 SDK} csomagban (400+~MB CD~image) pedig 
    a Windows könyvtárak, a windows.h filé, Windows API dokumentáció, stb.. 
    A két csomagot úgy installáljuk, hogy közös directoryba kerüljenek, 
    így közös bin, lib, include  directorykba kerülnek  az alkatrészeik.
\item 
    Borland C, utolsó tesztelt változat BCC-5.5.1. 
    A freecommandlinetools.exe csomagot kell keresni a
    \href{http://www.borland.com}{http://www.borland.com}-on.
    Kicsit shareware íze van a dolognak, pl. a BCC-vel fordított 
    programok nem tudnak egyszerre 50-nél több file descriptort megnyitni.
\end{itemize}


Hozzunk létre egy shell scriptet dirmap.bat néven az alábbihoz
hasonló tartalommal:
\begin{verbatim}
@echo off
set CCCDIR=d:\ccc2
set MSCDIR=d:\msc2003
set BORDIR=d:\borland\bcc55
set GNUDIR=c:\mingw320
set JAVDIR=d:\j2sdk1.4.2
\end{verbatim}
Ez a script mutatja, hogy hol vannak a dolgok a filérendszerünkben. 
Persze gépenként változó, hogy hol vannak a dolgok.
Az én scriptemben három fordító helye is meg van adva (MSC, Borland, MinGW),
de a CCC használatához egy is elég. 

A következő példa a MinGW környezet kialakítását mutatja,
hasonlóan kell eljárni a többi fordító esetén is.
Hozzunk létre egy másik shell scriptet ccc\_mng.bat 
(vagy ccc\_msc.bat, vagy ccc\_bor.bat) néven az alábbihoz
hasonló tartalommal:
\begin{verbatim}
@echo off
call dirmap
set CCCBIN=mng
set CCCUNAME=nt
set PATH=%WINDIR%;%WINDIR%\system32
set PATH=%JAVDIR%\bin;%PATH%
set PATH=%GNUDIR%\bin;%PATH%
set PATH=%CCCDIR%\usr\bin\nt;%PATH%
set INCLUDE=%GNUDIR%\include;
set LIB=%GNUDIR%\lib;
start "CCC-MinGW environment"
\end{verbatim}

A CCCBIN környezeti változó mutatja, hogy mivel fordítunk (msc, bor, mng).
A CCCUNAME változó mutatja, hogy milyen rendszeren futó programok készülnek (nt).
Ha a ccc\_xxx.bat scriptet elindítjuk, akkor feljön egy shell,
amiben éppen a nekünk szükséges beállítások lesznek érvényben.
A továbbiakban mindig ebben a shellben dolgozunk.
Hozzuk létre ténylegesen a CCC directoryt (CCCDIR) a dirmap.bat-ban 
megadott helyen. Töltsük le a 
\begin{itemize}
\item \tt ccc2-yyyymmdd.zip
\item \tt setup-flex-yyyymmdd.zip
\item \tt setup-lemon-yyyymmdd.zip
\item \tt setup-windows-yyyymmdd.zip
\end{itemize}
csomagokat, másoljuk be CCCDIR-be, és bontsuk ki. (A windowsosoknak 
nincs unzip programjuk, meg kell tehát szerezni a webről az Infozip 
csomagot, ez mindent tud, ami kell, és ingyenes.) 

A CCCDIR-ben találunk egy install.bat scriptet, ami lefordítja
az egész CCC-t. Ne indítsuk azonban el vakon, hanem némi áttekintés
megszerzése céljából nyissuk meg egy editorral, és nézzük meg, 
hogy mit fog csinálni:
 
\begin{enumerate}
\item
A setup-windows directoryból installálja a  fordítás
elkezdéséhez szükséges négy exe filét (build, prg2ppo, ppo2cpp, ccomp).

\item
A setup-flex directoryból installálja a flex-et (szintaktikai elemző
generátor).

\item
A setup-lemon directoryból installálja a lemon-t (nyelvtani
elemző generátor).

\item
Bemegy a ccclib directoryba, és lefordítja a CCC alapkönyvtárait.
Az Fltk megjelenítő könyvtár fordítása csak akkor lehet sikeres,
ha  gépünkön installálva van az Fltk, ezért
kezdetben a ccc2\_uif könyvtár fordítása ki van kommentezve.

\item
Utána bemegy a ccctools directoryba, és lefordítja a CCC fordításban
szerepet játszó programokat.  A (setup-windows.zip-ben letöltött)
build, prg2ppo, ppo2cpp programok most lefordítják saját magukat.

\item
A ccctab directoryban lefordulnak a táblaobjektum könyvtárak.
A tbdatidx és tbdbfctx könyvtárak installálásához előzőleg a ctree-nt.zip
csomagot is le kell tölteni (és CCCDIR-ben kibontani). Ha nincs
szükség CTREE adatbáziskezelésre, akkor a megfelelő sorokat 
kommentezzük ki.

\item
A terminal directorykban lefordulnak a távoli karakteres megjelenítéshez
szükséges eszközök.

\item
A tools-ban fordul néhány hasznos könyvtár és eszköz:
\begin{tabular}{ll}
\tt  crypto   & openssl interfész (előzőleg installálni kell az openssl-t) \\
\tt  dbaseiii & osztály dbf állományok olvasásához \\ 
\tt  socket   & socket könyvtár \\ 
\tt  xmldom   & XML elemző osztály \\ 
\tt  xmlrpc   & XMLRPC szerver és kliens osztály \\ 
\tt  ddict2   & adatszótár karbantartó \\ 
\tt  mask     & dialogbox editor \\ 
\tt  savex    & eszköz directory struktúrák szinkronizálásra \\ 
\tt  sql2     & interfész Oracle és Postgres adatbáziskezelőhöz \\ 
\tt  z        & text editor \\ 
\end{tabular}
\end{enumerate}


\subsection{Linux}

Installáljuk az x-fonts.zip csomagban lévő fontokat.
A pcf.gz  font filéket az /usr/lib/X11/fonts/misc directoryba 
kell bemásolni, majd ugyaninnen rootként kiadjuk a \verb!mkfontdir!  
parancsot. Az X-et újraindítjuk.
Létrehozunk egy \verb!ccc! nevű csoportot, és ezt a csoportot
állítjuk be a saját csoportunknak. A továbbiakban mindig
ebben a csoportban dolgozunk.
A saját home-unkban a .bashrc filébe írjuk be az alábbiakat
\begin{verbatim}
export CCCDIR=/opt/ccc2
export CCCBIN=lin
export CCCUNAME=linux
export CCC_XSIZE=80x36
export CCC_XFONT=ccc8x16
export OREF_SIZE=20000
export PATH=$CCCDIR/usr/bin/$CCCUNAME:$PATH
export LD_LIBRARY_PATH=$CCCDIR/usr/lib/$CCCBIN:$LD_LIBRARY_PATH
\end{verbatim}
 
Hozzuk létre ténylegesen a CCC directoryt  a .bashrc-ban 
megadott helyen, a csoportja legyen ccc,
a permission flageket állítsuk be 2775-re.
Töltsük le a 
\begin{itemize}
\item \tt ccc2-yyyymmdd.zip
\item \tt setup-flex-yyyymmdd.zip
\item \tt setup-lemon-yyyymmdd.zip
\item \tt setup-unix-yyyymmdd.zip
\end{itemize}
csomagokat, másoljuk be CCCDIR-be, és bontsuk ki.
A CCCDIR-ben találunk egy install.b scriptet, ami lefordítja
az egész CCC-t. Ne indítsuk azonban el vakon, hanem némi áttekintés
megszerzése céljából nyissuk meg egy editorral, és nézzük meg, 
hogy mit fog csinálni:
 
\begin{enumerate}
\item
Ellenőrzi, hogy jól van-e beállítva a CCCDIR környezeti
változó.

\item
A setup-unix directoryban lefordítja és installálja a fordítás
elkezdéséhez szükséges programokat (build, prg2ppo, ppo2cpp).

\item
A setup-flex directoryból installálja a flex-et (szintaktikai elemző
generátor).

\item
A setup-lemon directoryból installálja a lemon-t (nyelvtani
elemző generátor).

\item
A terminal-unix direcotryban lefordulnak a fullscreen karakteres 
(lokális és távoli) megjelenítéshez  szükséges eszközök.

\item
Bemegy a ccclib directoryba, és lefordítja a CCC alapkönyvtárait.
Az Fltk megjelenítő könyvtár fordítása csak akkor fog lefutni,
ha  gépünkön installálva van az Fltk, ezért
kezdetben a ccc2\_uif könyvtár fordítása ki van kommentezve.

\item
Utána bemegy a ccctools directoryba, és lefordítja a CCC fordításban
szerepet játszó programokat. A (setup-unix.zip-ben letöltött)
build, prg2ppo, ppo2cpp programok most lefordítják saját magukat.

\item
A ccctab directoryban lefordulnak a táblaobjektum könyvtárak.
A tbdatidx és tbdbfctx könyvtárak installálásához előzőleg a ctree-lin.zip
csomagot is le kell tölteni (és CCCDIR-ben kibontani). Ha nincs
szükség CTREE adatbáziskezelésre, akkor a megfelelő sorokat 
kommentezzük ki.

\item
A tools-ban fordul néhány hasznos könyvtár és eszköz:
\begin{tabular}{ll}
\tt  crypto   & openssl interfész (előzőleg installálni kell az openssl-t) \\
\tt  dbaseiii & osztály dbf állományok olvasásához \\ 
\tt  socket   & socket könyvtár \\ 
\tt  xmldom   & XML elemző osztály \\
\tt  xmlrpc   & XMLRPC szerver és kliens osztály \\ 
\tt  ddict2   & adatszótár karbantartó \\ 
\tt  mask     & dialogbox editor \\ 
\tt  savex    & eszköz directory struktúrák szinkronizálásra \\ 
\tt  sql2     & interfész Oracle és Postgres adatbáziskezelőhöz \\ 
\tt  z        & text editor \\ 
\end{tabular}
\end{enumerate}

Ha nincs installálva gépünkön az OpenSSL, vagy a PostgreSQL
fejlesztő környezet, akkor pótoljuk, vagy az install.b script
a megfelelő sorait kommentezzük ki (máskülönben fordításkor
egy csomó hibát kapunk).
                    

\subsection{FreeBSD}

FreeBSD-n ugyanúgy alakítunk ki  CCC környezetet, 
mint Linuxon, tehát a Linuxnál leírtak érvényesek,
itt csak az eltéréseket ismertetem.

\begin{description}
\item[Környezeti változók]
FreeBSD-n az alábbi környezeti változókat kell beállítani:
\begin{verbatim}
export CCCBIN=fre
export CCCUNAME=freebsd
\end{verbatim}
\item[bash]
Installálni kell a bash2 shell-t (magától nem települ), 
és kell csinálni egy \verb!/bin/bash! linket (magától
/usr/local/bin-be teszi). A CCC fordító scriptek
mind így kezdődnek: \verb|#!/bin/bash|, úgyhogy enélkül 
egy lépést sem lehet tenni.

\item[malloc.h]
A CCC programok számos helyen inkludálják malloc.h-t.
A FreeBSD 5.3-ban a GCC erre azt a hibaüzenetet adja, 
hogy az új C szabványban  malloc.h megszűnt, és helyette 
stdlib.h-t kell használni. Természetesen ez nem elfogadható,
tehát csinálni kell egy malloc.h-t, ami csendben inkludálja
stdlib.h-t.
\end{description}

Ezektől az apróságoktól eltekintve a Linux és FreeBSD
CCC programozói szempontból nézve meglehetősen egyforma.


\subsection{Solaris}
Solarison a GNU környezet kialakításával kezdjük.
A CCC installáció lényegében azonos a Linuxnál leírtakkal,
csak CCCBIN=sol, CCCUNAME=solaris beállításokat alkalmazunk.
Az alábbiakban egy Sun Blade 100 (SPARC 64 bit, Solaris 8, GCC 2.95.3)
rendszeren szerzett tapasztalataimat írom le:

\begin{enumerate}
\item 
    A gépen előre installálva van a Solaris,
    rákötjük a hálózatra.
    Megcsináljuk a ccc csoportot, a CCC felhasználókat,
    a felhasználók \verb!.profile! filéjét, ami majd
    beállítja a 
    \verb!CCCDIR!, 
    \verb!CCCBIN!, 
    \verb!CCCUNAME!,
    \verb!PATH!, 
    \verb!LD_LIBRARY_PATH!, 
    umask(=002) változókat.
\item
    Letöltjük a Sun-tól a  
    \href{http://wwws.sun.com/software/solaris/freeware/download.html}{GNU Companion CD-t}, 
    és  installáljuk.
    Sajnos ezt a CD-t nem frissítik, jelenleg is a GCC 2.95.3 van rajta, 
    noha Linuxon már egy ideje a 3.3.1-es változatot használjuk.
    A GNU szoftver a \verb!/opt/sfw! directoryba kerül. 
    Egyes programok neve elé raktak egy g betűt, 
    pl. make helyett gmake, 
    ld helyett gld,
    nm helyett gnm, 
    stb.
    A passwd filében beállítjuk, hogy minden felhasználó 
    shellje a bash legyen (a rooté is).
\item
    Nincs a Companion CD-n mc, ezért
    szerzünk a webről egy mc-t (régit, mert az újhoz GTK kell),
    lefordítjuk és installáljuk.
\item 
    Ugyancsak beszerezzük a webről, lefordítjuk és installáljuk 
    a legújabb     openssl-t és openssh-t (ezek sincsenek a Companion CD-n). 
    Beállítjuk, hogy az sshd automatikusan induljon a 3.\ futási szinten, 
    elindítjuk az sshd-t, és átülünk egy Linuxhoz. A továbbiakban
    ssh-val dolgozunk, mert a Solaris CDE desktop olyan, 
    amilyennek a windowsosok mondják.
\item
    Innen kezdve úgy mennek a dolgok, mint Linuxon,
    kivéve a linkelést. A fentiek szerint előkészített rendszeren
    a GNU fordító (c++) meglepetésemre a solarisos link editorral 
    akar linkelni.   Erőltettem egy darabig a GNU linkert (gld), 
    de azt tapasztaltam, hogy a gld-vel linkelt programok 
    SIGSEGV-znek. Nemcsak a CCC programok, hanem az egysoros
    "Hello World" C program is.
    Mivel ezt nem tudtam megjavítani, vissza kellett
    térnem az eredeti állapothoz, vagyis a \verb!/usr/ccs/bin/ld! 
    (solarisos) link editorhoz, ehhez viszont néhány scriptet 
    át kellett írni.
    
    A \verb!-Wl,--start-group! és \verb!-Wl,--end-group! 
    opciókat ki kellett hagyni, ez lehet hogy hiányozni fog,
    de nem találtam meg, mivel helyettesíthető.
    
    A \verb!-shared! opció helyett \verb!-G! kell.
    
    A \verb!-soname=LIBNAME!  helyett \verb!-hLIBNAME! kell.
\end{enumerate}
Tudom, hogy a kezdők nem ebből a leírásból fognak UNIX
gyakorlatot szerezni, de minek a részletes leírás, 
ha a részletek úgyis mindig változnak.
A fő nehézség, hogy a Solaris viszonylag ismeretlen terep,
nem adja könnyen magát.  Ha már megvan az mc,
és egy kézreálló Linuxról ssh-zunk, fellélegezhetünk.

Időközben teszteltem a CCC-t az ingyenesen letölthető
(inteles) Solaris-10  rendszeren is.  A helyzet kicsit javult:
\begin{itemize}
\item 
    A telepítőkészlet helyből tartalmazza a GNU szoftvereket.
\item 
    Egyből lesz OpenSSH.
\item 
    Könnyen feléleszthető az NFS szerver.
\item 
    Használható a Java Desktop (ami lényegében egy GNOME).
\end{itemize}
Mindezek miatt a CCC installálásával (ami egyébként nem változott) 
könnyebben el lehet boldogulni.


\section{A csomagok tartalma}

Az alcímek mellett zárójelben jelezzük, hogy az adott
csomag mely platformokon kap szerepet:
\begin{tabular}{ll}
\bf ~~W~~   & Windows \\
\bf ~~L~~   & Linux   \\
\bf ~~F~~   & FreeBSD \\
\bf ~~S~~   & Solaris \\
\end{tabular}

\subsection{ccc2.zip (WLFS)}
A CCC teljes forráskódja. 

\subsection{setup-flex.zip (WLFS)}

Az új CCC kiadásban a Flex programok C++ elemző objektumokat
készítenek (\verb!%option c++!). Ennek előnye, hogy az elemző reentráns
és szálbiztos, több Flex elemzőt tartalmazó rutin lehet ugyanabban
a CCC programban névütközés nélkül. 
Sajnos ez komplikációkat is okoz, ui. a különféle Linux 
disztribúciókból származó Flexek különféleképpen hibásak.
Ráadásul a Flex új fejlesztési iránya sem valami megnyugtató.
A setup-flex.zip csomagból települő flex
minden CCC-vel használt fordítóval működik, 
nem tartalmaz olyan ismert hibát, ami a használatot akadályozná.
További részletek magában a csomagban.


\subsection{setup-lemon.zip (WLFS)}

A CCC 2.1.x kiadásában a Bison/YACC-ot felváltotta a Lemon. 
A Lemon a Bisonhoz képest egyszerűbben terjeszthető és telepíthető,
mivel egyetlen C forrásfájlból, plusz egy C template fájlból áll.
A Lemon által generált kód a Bisonnal szemben reentráns és szálbiztos. 
A setup-lemon.zip csomag az általam patchelt forrásváltozatot tartalmazza. 
A változások fel vannak sorolva a lemon.c fájl elején. Az eredeti
forrás megtalálható itt: 
\href{http://www.hwaci.com/sw/lemon}{http://www.hwaci.com/sw/lemon}.

\subsection{setup-unix.zip (LFS)}
A CCC-t nehéz a semmiből kiindulva lefordítani, ui.\ a fordításhoz
CCC-ben készült eszközökre is szükség van. Ez a csomag tisztán C-ből
előállítja a build, prg2ppo, ppo2cpp programokat, amik minimálisan
szükségesek a CCC saját programkészítő rendszerének a felélesztéséhez.

\subsection{setup-windows.zip (W)}
A CCC-t nehéz a semmiből kiindulva lefordítani, ui.\ a fordításhoz
CCC-ben készült eszközökre is szükség van. Ez a csomag tartalmazza
a build, prg2ppo, ppo2cpp, ccomp programokat, amik minimálisan
szükségesek a CCC saját programkészítő rendszerének a felélesztéséhez.
A ccomp program csak a MinGW fordító esetén kell.

\subsection{ccc-fltk.zip (WLFS)}
Az Fltk könyvtár forrása Build projekt kiszerelésben.
Korábban Linuxon a telepítő csomagból származó Fltk használatát
ajánlottam, azonban belefáradtam az API változásainak követésébe,
ezért újabban Linuxon is a (befagyasztott) ccc-fltk.zip csomagot 
használjuk.

A csalódások megelőzése érdekében megjegyzem:
A CCC-ben nincs olyan általános CCC-Fltk csatoló, 
amivel tetszőleges grafikus programokat készíthetnénk.
Az Fltk-t arra használtuk, hogy {\em meglevő karakteres
programokat\/} jelenítsünk meg grafikus felületen {\em átírás nélkül}.
Ezek ablakokkal, menükkel, táblázatokkal, 
adatbeviteli mezőkkel rendelkeznek, de semmi több, 
pl. nem tudnak rajzolni. Készülni fog viszont egy
általánosan használható GTK csatoló.

\subsection{jt.zip (WLFS)}
Érdekes új eszköz a Jáva terminál.
Ez egy {\em alkalmazásfüggetlen} grafikus megjelenítő,
ami a felhasználó gépén megjeleníti a távoli gépen futó 
szerveralkalmazást. Természetesen annak sincs akadálya, hogy a terminál 
és a szerver ugyanazon a gépen fusson, ám a Jáva terminál fő erőssége 
a távoli programfuttatás, ezzel a webes programozás egy alternatíváját 
jelenti. 
A Jáva terminál elsősorban centralizálandó ügyviteli alkalmazásoknál
előnyös, mint pl. egy bolthálózat  nyilvántartása. A terminál és a 
szerver között TCP kapcsolat van (nem HTTP), és XML szintaktikájú 
üzenetekkel kommunikálnak. A kommunikáció SSL-lel titkosítható.
A terminálról bővebb infó található a 
\href{http://ok.comfirm.hu/ccc2/jterminal.html}{Jáva Terminál}
oldalon. 

\subsection{jtpython.zip (WLFS)}
Python nyelvű szerver oldali interfész könyvtár a Jáva terminálhoz.
 
\subsection{openssl-xxx.zip (W)}
Az openssl könyvtár MSC-vel fordított változata.
Legcélszerűbben akkor járunk el, ha a windowsos C fordítónkba
,,integráljuk'' az openssl-t: az include-okat bemásoljuk az include alá,
a lib-eket a lib alá, az exe-ket, dll-eket pedig betesszük a 
Windows könyvtárba. 

Megjegyzés: Az openssl-ben le van írva, hogy hogyan kell őt
lefordítani MSC-vel, Borland C-vel, és MinGW-vel, nekem azonban
csak az első (MSC) működött. Az MSC-vel kapott dll-ek és lib-ek
a MinGW-hez egy az egyben jók, a Borland C-nek pedig van eszköze hozzá,
hogy egy dll-ből import lib-et csináljon.
 

 
\subsection{x-fonts.zip (LFS)}
X-hez való fix (bitmap) fontok, amik a Latin-2 kódolású magyar 
ékezetes karakterek mellett tartalmazzák a Clipper dobozrajzoló karaktereit.
A fontfiléket rootként a /usr/lib/X11/fonts/misc-be kell másolni,
majd misc-ben végrehajtjuk: 
\begin{verbatim}
/usr/bin/X11/mkfontdir
\end{verbatim}
Utána az X-et újraindítjuk.    
 

\section{Telepítési problémák}

\subsection{Hiányzik az unistd.h} 
Előfordulhat, hogy NT-s C fordítónk nem talál egy unistd.h nevű filét.
Ennek oka, hogy a Bison és Flex kódgenerátora (elsősorban) UNIX-ra 
készíti a kimenetét, ahol az említett filé megtalálható valahol az
include pathban, NT-n viszont nincs ilyen. A hiba legegyszerűbben 
úgy hárítható el, ha létrehozunk a C fordító include directoryjában 
egy üres unistd.h-t, ezzel a fordító meg fog elégedni.

\subsection{Hiányzik malloc.h}
A CCC programok számos helyen inkludálják malloc.h-t.
Az újabb GCC-k erre azt a hibaüzenetet adják, 
hogy a C szabványban  malloc.h megszűnt, és helyette 
stdlib.h-t kell használni. Csináljunk ilyenkor egy malloc.h-t, 
ami csendben inkludálja stdlib.h-t.


\subsection{Fltk installáció} 

Kezdetben Fltk nélkül installáljuk a CCC-t.
Ha már van üzemképes CCC környezetünk, akkor le tudjuk 
fordítani a ccc-fltk.zip csomagban letölthető Build projektet. 
A jelenlegi változat az összes támogatott C fordítóval fordul 
(Windowson, Linuxon, Solarison egyaránt).

Az installáció mindössze annyit jelent, hogy a csomag
include filéit, és a lefordított lib-eket a C fordító
rendelkezésére bocsátjuk. A legegyszerűbben akkor járunk el, 
ha a  C fordítónkba ,,integráljuk'' az Fltk-t: az include-okat 
bemásoljuk az include alá, a lib-eket a lib alá, ezzel
Windowson készen is vagyunk.

UNIX/Linuxon korábban a telepítőkészletből származó Fltk könyvtárat használtuk. 
Az Fltk fejlesztői azonban állandóan változtatgatják az API-t, ezért az újabb
változatok átvételekor mindig a CCC interfész módosítására kényszerültem. 
Ezt megunván, áttértem a saját, befagyasztott ccc-fltk csomagból fordított 
Fltk-ra. Hogy a többféle csomag ne keveredjen, a korábban esetlegesen 
installált Fltk csomagot távolítsuk el.

Az eredeti Fltk függ a GL könyvtártól.
A CCC-hez azonban nincs szükség a GL-re,
ezért a külső függőségek csökkentése érdekében
a mi Fltk könyvtárunkból ki vannak operálva a GL-es hivatkozások. 
A GL könyvtár helyett egyszerűen az X11-et kell linkelni.


\subsection{Borland cfg filék} 
A Borland C használata esetén vegyük ki a ccc2 csomagból a
   \verb'usr\options\bor\bcc32.cfg' és
   \verb'usr\options\bor\ilink32.cfg' filéket,
   tartalmukat a konkrét include és lib directoryknak megfelelően 
   aktualizáljuk, és másoljuk be a Borland bin directoryba 
   (oda, ahol bcc32.exe is található). 

\subsection{Directory nevek} 

A CCC directory nevének megválasztásakor figyelembe kell venni,
hogy a C fordító egyes alkatrészei nem támogatnak minden nevet, 
amit egyébként az operációs rendszer megengedne. Így pl.\ a Borland
és a Watcom könyvtárkezelő működésképtelen, ha a library teljes
specifikációjában kötőjelet is tartalmazó directory van. 

\subsection{Nagy fájlok (Large File Support)} 

UNIX-on a CCC az 1.5.00 változattól kezdve nagy filé támogatással
fordítható. Solaris 8-on és 2.4.x kernelű ext2 filérendszerű Linuxon
az LFS gond nélkül működik. Nem működik azonban az LFS 2.2.x kernelen,
vagy 3.6-nál kisebb verziószámú ReiserFS-en. Ezért a biztonság
kedvéért a setup során Linuxon LFS nélküli CCC keletkezik.
Az LFS úgy kapcsolható be, hogy az opt filékben beélesítjük a

\begin{verbatim}
   -D_FILE_OFFSET_BITS=64
   -D_LARGEFILE_SOURCE
\end{verbatim}

fordítási opciókat, majd az egész CCC-t forrásból újrafordítjuk.
 

\subsection{GCC verziók}

Jelentős változások történtek a C++ szabványban
a GCC 2.95 és 3.x változatainak megjelenése között.
A CCC a különbségekhez a \verb!$CCCDIR/usr/options/lin/gccver.opt! 
filében definiált makrók segítségével alkalmazkodik.
Amikor a CCC-t először installáljuk (vagy egy másik C fordítóval
teljesen újrafordítjuk), akkor előzőleg le kell törölni a
gccver.opt filét. A CCC installáló scriptje észre fogja venni
gccver.opt hiányát, és automatikusan elkészíti az aktuális
fordítónak megfelelő opt filét. 

