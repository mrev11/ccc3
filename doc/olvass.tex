\pagetitle{CCC3 installáció}{2008. január}{}

(Akinek túlságosan tömör az alábbi leírás,  részletesebb 
infót talál a \href{http://wiki.hup.hu/index.php/CCC}{HUPWikin}.)

A preferált telepítési módszer forrásletöltés subversion (svn) klienssel,
majd fordítás. Emellett lehet telepíteni a zip csomagokból is, ahogy
azt a \href{webinst.b}{webinst.b} script teszi. 
A \href{ccc-dependencies}{ccc-dependencies} script mutatja,
hogy Ubuntu Linuxon milyen csomagok szükségesek a CCC lefordításához.

\paragraph{Linux}

\begin{enumerate}
\item Installálni kell az svn kliens részét.

\item Kell készíteni egy CCC3 directoryt.
    A CCC3  szimbolikus név, a konkrét hely akárhol lehet,
    pl. \verb!/opt/ccc3!, vagy \verb!$HOME/ccc3!.

\item Kezdeti letöltés. CCC3-ból kiadjuk a parancsot:
\begin{verbatim}
    svn checkout svn://ccc.comfirm.hu/ccc3/trunk
\end{verbatim}
  amire CCC3/trunk-ban létrejön a CCC3 directory struktúra.
  Ugyanide jutunk svn nélkül is, ha az összes
  zip-et kibontjuk  CCC3/trunk-ban (kivéve,
  hogy így később nem használhatjuk az svn update funkcióját).

\item Az i.b script végzi el a fordítást 
  a környezeti változók szokásos beállítása után:
\begin{verbatim}
    export CCCDIR=.../CCC3/trunk
    export CCCBIN=lin
    export CCCUNAME=linux
    export CCCTERM_CONNECT=$CCCDIR/usr/bin/$CCCUNAME/terminal.exe
    export PATH=.:$CCCDIR/usr/bin/$CCCUNAME:$PATH
    export LD_LIBRARY_PATH=$CCCDIR/usr/lib/$CCCBIN:$LD_LIBRARY_PATH
\end{verbatim}
  A jt, gtk, sql2 modulok installálását az i.b script kihagyja,
  ezeket utólag kell/lehet installálni.
  Ennek oka, hogy olyan komponensektől függnek
  (Jáva, GTK, Postgres), amik általában 
  nincsenek benne az alapértelmezett módon telepített rendszerekben.

  
\item Frissítés. CCC3/trunk-ban kiadjuk a parancsot:
\begin{verbatim}
    svn up
\end{verbatim}
  Ez csak a különbségeket hozza le. Utána megint
  az i.b scriptet kell futtatni a változások lefordítása céljából
  (illetve a külön forduló részekkel most is külön kell bibelődni).
\end{enumerate}

\paragraph{Windows}

Mindez Windowson is megy, bár az svn klienst nem próbáltam,
és Windowson viszonylag ritkán tesztelünk. Ezeket kell beállítani:
\begin{verbatim}
    set CCCDIR=...\CCC3\trunk
    set CCCBIN=mng
    set CCCUNAME=windows
    set CCCTERM_CONNECT=%CCCDIR%\usr\bin\%CCCUNAME%\terminal.exe
    set PATH=%CCCDIR%\usr\bin\%CCCUNAME%;%PATH%
\end{verbatim}
A bor fordító támogatása megszűnt, 
mert a freecommandlinetools.exe-nek 2000 óta nincs új kiadása, 
a régi pedig rosszul fordítja a unicode-ot. 
Az msc fordító rosszul fordítja a GTK-t, 
ezért csak az mng (MinGW) felel meg. 
CCCUNAME-ba a korábbi "nt" helyett "windows"-t kell írni.


\paragraph{Segítség a CCC-vel ismerkedőknek}

A CCC projektek directorykba vannak
szervezve. A directorynevekből általában ki lehet találni,
hogy miről van szó. A projektek többségében megtalálható \verb!m!
(Windowson \verb!m.bat!) script szolgál az adott projekt
lefordítására és installálására. Egyes helyeken ilyet találunk: 
\verb!mkall.b!, \verb!install.b!, \verb!i.b!, 
ezek olyan   scriptek, amik több directoryba is bemennek, 
hogy végrehajtsák   az ottani \verb!m!-et. Kockázat nélkül 
lehet az \verb!m! parancsokat próbálgatni. Persze a modulok 
fordításának  sorrendje számít (mivel egymásra épülnek),
de ezt hamar ki lehet ismerni.
  
A CCC fordítás nem ellenőrzi előre a külső függőségeket,
hanem egyszerűen belefut a hibákba. Ilyenkor fel kell ismerni,
hogy mi hiányzik neki, pótolni kell a hiányt, és újraindítani
a fordítást. Aki végképp elakad, annak segítünk. 
Néhány dolog, ami gyakran hiányozni szokott: 
X, GTK, SSL, Postgres, Jáva fejlesztői környezetek.
Ezzel szemben az Oraclehez (fordításhoz) semmi külső forrás nem kell.
A Lemont és Flexet az CCC tartalmazza, és telepíti,
a CCC-vel adott változatot kell használni.

 