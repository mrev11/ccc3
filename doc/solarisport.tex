
\iftth
\def\pcenter#1{#1}
\else
\def\pcenter#1{\parbox{\hsize}{\center #1}}
\fi

\pagetitle%
{\pcenter{A Kontó számlavezető rendszer portolása Solarisra}}%
{Dr. Vermes Mátyás\footnote{\ComFirm}}%
{2002.\ április 11.}

\section{Áttekintés}

A ComFirm Bt.\  2002.\ márciusában abban a szerencsés 
és megtisztelő helyzetben volt, hogy fő termékét, a Kontó
számlavezető rendszert portolhatta Solaris operációs
rendszerre. Kizárólag saját erejére támaszkodva $-$ kis cég lévén $-$
a ComFirm nem vállalkozhatott volna erre a munkára. 
A projektet az tette lehetővé, hogy 
a  \emph{Sun Microsystems Magyarország Kft.} és 
az \emph{Assyst Rendszerház Kft.} 
egy hónapra ingyenesen rendelkezésünkre
bocsátott egy SunFire~880 típusú gépet. 
A gép az alábbi főbb jellemzőkkel bírt:

\begin{itemize}
 \item 2 db 750MHz Ultrasparc III.\ processzor,
 \item 4 GB memória,
 \item Solaris 8 operációs  rendszer.
\end{itemize}

Jelen dokumentum röviden ismerteti a Kontó felépítését és
a fejlesztésében használt eszközöket, majd tárgyalja a portolás 
tapasztalatait és eredményeit.

\section{A Kontó architektúrája}

\subsection{DOS-Novell korszak}

A Kontó egy multidevizás, önálló főkönyvvel, zsírókapcsolattal
rendelkező, számos banki ügyletet (betét, hitel, 
értékpapír, csoportos megbízások, állandó megbízások, faktoring, stb.) 
támogató integrált banki számlavezető rendszer.  

A program első változatát, ami kb.\ tíz éve készült, 
több bank (Budapest Bank, Pogári Bank, Konzumbank) fiókrendszerként 
használta. A földrajzilag távoleső bankfiókokban (melyek száma éveken át 100
felett volt) mind egy-egy Kontó installáció működött. A bankfiókok
és a központ közötti kommunikáció sokáig csak modemkapcsolattal volt 
megoldható, így csak decentralizált rendszert lehetett működtetni. 
A rendszer DOS-Novell hálózaton futott.

A körülmények kényszere folytán a Kontó fejlesztéséhez a fent leírt
környezet tipikus eszközét, a Clippert használtuk.\footnote{Ezt lényegében
kötelezővé tették számunkra a BBRt-ben.} A Clipper használatbavételekor
azonban nyitvahagytuk azokat az utakat, amik később lehetővé tették 
a program korszerűsítését:
\begin{itemize}
\item 
  Megtiltottuk a dBase-ből örökölt, programozáselméleti szempontból 
  rossz nyelvi elemek használatát, pl.\ a public változókat.
\item 
  A dBase stílusú adatbáziskezelés helyett bevezettük a táblaobjektumokat
  (ez a relációs adatbázis navigációs modelljének egy objektum alapú
  implementációja).
\item
  Az adatbázis leírására bevezettünk egy adatszótárt.
\item
  Létrehoztunk egy sor kódgenerátort, amivel a programok mind belsőleg
  mind külsőleg egységes képet kaptak a programozó személyétől függetlenül.
\end{itemize}
E módszerekkel elértük, hogy a programjaink meglehetősen stabilak lettek,
amire szükség is volt, hiszen nagy mennyiségű programról van szó.
A Kontó többszáz önálló, de ugyanazon az adathalmazon dolgozó modulból áll, 
amiknek egymással és az adatbázissal összhangban kell működni. 
A fiókrendszernek a programozók közvetlen segítsége nélkül kellett 
életképesnek maradnia, miközben folyamatosan követni kellett a banki 
ügyletekben bekövetkezett változásokat, amik mind a programokat, 
mind az adatbázis struktúrákat érintették (és azóta is érintik).  
Hogy mást ne említsek, 1996-ban egyszerre három bank fiókhálózatában 
hajtottuk végre a számlaszámkonverziót 12-ről 24~jegyre.

\subsection{Windows/UNIX platform}

A korszerűsítésnek azt az érdekes útját választottuk, 
hogy nem az alkalmazási programokat írtuk át új platformra,
hanem a fejlesztő eszközt. Ez azt jelenti, hogy elkészítettünk
egy olyan  fordítót és futtató környezetet (ez a CCC), 
ami képes a Kontó futtatására Windows NT és UNIX operációs rendszereken.
A CCC 1996-ban már alkalmas volt arra, hogy a BBRt-nek javasoljuk
a bankkártya rendszer Windows NT-re való átállítását. 

Az addig karakteres megjelenésű programokhoz grafikus interfészt
készítettünk. Itt is megőriztük a kompatibilitást a korábbi alkalmazási 
programokkal, azaz a karakteres és grafikus interfészű program
azonos, csak a (UNIX-on dinamikusan betöltött) megjelenítő
könyvtárban mutatkozik eltérés. A kompatibilitás megtartásának ára, 
hogy a grafikus interfész szükségszerűen puritán. 

A karakteres interfésznek ugyanakkor fontos szerepe maradt, ui.\
ehhez egyszerű fullscreen terminál protokoll készíthető (CCC terminál), 
amivel megoldottuk a Kontó terminálszervereken való futtatását mind Windowson,
mind UNIX-on. A CCC terminál minden elképzelhető rendszeren fut
(DOS, Windows, Linux, böngésző), a használatához elegendő egy 14 ezres
modem, azaz a lehető legvékonyabb kliens.

Jelenleg a Kontó mindenhol terminálszerveres üzemmódban fut:
\begin{itemize}
\item 
   A Budapest Bankban a bankkártya-elszámolást végző Kontó
   Windows~NT szerveren fut, míg a karakteres terminálok 
   Windows~NT/2000 gépeken.
\item 
   Az Otthon Lakástakarék Pénztárban a Kontó Linux szerveren fut
   grafikus módban. Itt a munkaállomások is Linuxot használnak, 
   és a grafikus Kontó képernyőjét X szerver jeleníti meg.
\item 
   A Konzumbankban Linuxon fut a Kontó, az országos fiókhálózatban
   karakteres terminál van használatban, amit Windows~2000 futtat.
\end{itemize}
 

\section{Solaris port}

\subsection{CCC fejlesztői környezet}

A CCC fejlesztői környezetről részletes információ kapható
a \href{http://www.comfirm.hu/ccc2/ccc2.html}{CCC 2.x áttekintés} 
dokumentumban, sőt innen a fejlesztőeszköz le is tölthető.
Korábban a CCC-t csak Windows~NT/2000 és Linux rendszereken 
lehetett installálni, de éppen a jelen projekt eredményeként
a CCC már Solarison is elérhető.

Dióhéjban annyit érdemes megjegyezni a CCC-ről, 
hogy a  kódot C++-ra fordítja, amit továbbfordítunk
az adott platform C++ fordítójával, és linkelünk a rendszerkönyvtárakkal,
valamint a CCC futtató könyvtárral. Mivel a CCC-t \textit{nagyon könnyen}
lehet C++ rutinokkal bővíteni,  az operációs rendszer 
(és a rengeteg egyéb könyvtár) minden lehetősége ugyanúgy elérhető, 
mint a C programokból, miközben a programozás zöme magasszintű, termelékeny
és biztonságos nyelven történik.

A CCC környezet fontos eleme egy projekt menedzser
(lásd \href{http://www.comfirm.hu/ccc2/build.html}{Programkészítés a Build-del}), 
ami arra van tervezve, hogy ugyanabból a forrásállományból minden támogatott 
platformon automatikusan elkészítse a működő programot. A projekt menedzser 
elvégzi az ismert make program feladatát, ügyel a megfelelő 
karakterkészlet használatára, elkülönítve tárolja a különböző 
platformokra fordított objecteket, stb., végeredményben lehetővé teszi,
hogy a CCC projektek egyszerű másolással legyenek mozgathatók
a platformok között, bármiféle konverzió nélkül.

A fentiekből sejthető, hogy a Kontó Solarisra való portolása
lényegében ugyanaz a feladat, mint a CCC Solarisos változatának
előállítása. Valóban ez volt a helyzet. A portolás során kizárólag
a CCC fordító és futtatókönyvtár (csekély mértékű) javítgatásával
kellett foglalkozni. Amikor ez elkészült, az egész Kontó azonnal működőképes
lett anélkül, hogy az alkalmazási programokban akár egy byte
átírására is szükség lett volna. 

A következőkben kitérünk a portolási munka egyes problémáira.

\subsection{Különféle fordítók}

Először is választani kellett a szóbajövő fordítók között.
\begin{itemize}
 \item 32 bites GCC 2.95 (GNU),
 \item 64 bites GCC 3.01 (GNU), 
 \item Forte, a Solaris natív 64-bites C fordítója.
\end{itemize}

A legkézenfekvőbb választás a 32~bites~GCC, hiszen a Linuxon eddig is
ezt használtuk.\footnote{Windowson az Microsoft-C, Borland-C és Watcom-C is
támogatott.}  A fordító hátránya, hogy nem használja ki teljes 
mértékben a 64~bites processzor lehetőségeit. A 64~bitesség 
a nagy filék kezelésében, és az ún.\ file mapping-ben jelent pluszt
a 32~bithez képest. Megjegyzem, hogy a 64 bites programok általában
nem gyorsabbak a 32~biteseknél. Végül emellett a fordító mellett
maradtunk, és a nagy filék kérdését (LFS) a következő pontban ismertetett
módon oldottuk meg.

A 64~bites GCC az ismertetése szerint béta állapotban van, 
ezért nem tettünk vele komoly kísérletet.

Az \emph{Assyst Rendszerház Kft.} jóvoltából 40 napos licence-szel 
kapott Forte fordítóval biztató kísérletünk volt, de végül elakadtunk.
A C++ függvények névképzése a GCC-ben és a Fortéban eltérő. 
Amikor egy megosztott könyvtárat dinamikusan betöltünk, akkor a  programban
explicite meg kell nevezni a könyvtárban levő függvényeket, ezért az ilyen 
program nem lehet hordozható. Mivel ez viszonylag sok kódot érintett, 
le kellett mondanunk a Forte használatáról.\footnote{
Hasonló problémával kerültem szembe a Windowsos CCC Alpha gépre történő
portolásánál, ám ott csak néhány függvényről volt szó.}

\subsection{Nagy filék támogatása}

32 bites rendszereken azért nehéz a nagy filék kezelése, 
mert 32 bites \verb!long!-okban csak 2~GB-ig lehet megadni a file offseteket
az \verb!lseek! függvénynek és társainak. A szabványos LFS (Large File Support)
azonban az ilyen rendszereken is lehetővé teszi a nagy filék kezelését.

A CCC az 1.5.00 változattól LFS-sel fordítható, 
ehhez az alábbi javításokra volt szükség:

\begin{enumerate}
\item 
   Az egész rendszerben alkalmazni kell az alábbi fordítási opciókat:
   \begin{verbatim}
   -D_FILE_OFFSET_BITS=64
   -D_LARGEFILE_SOURCE
   \end{verbatim}
\item
   A \verb!fseek!, \verb!fsetlock!, \verb!funlock! függvényekben,
   az \verb!offs_t! típust megfelelő gondossággal kell továbbadni.
\item
   Az \verb!fopen!-ben alkalmazott protokoll lockot hátrább kell tenni.
\end{enumerate}

A leírtak gond nélkül működnek Solarison és 2.4.x kernelű Linuxon.
2.2.x kernelű Linuxon a CCC-t nem szabad LFS-sel fordítani, mert
az fopen protokoll lockja (2~GB fölé kerülvén) mindig sikertelen lesz.


\subsection{Adatbázis meghajtók}

A CCC-ben léteznek különféle adatbázisszerverekhez készült interfészek
(Oracle, MS-SQL Server, PostgreSQL, MySQL), most azonban nem ezekről
lesz szó, hanem azokról az index szekvenciális filékezelőkről,
amikkel a Kontót teszteltük a Solarison, ezek:

\begin{itemize}
 \item  BTBTX:   saját kulcskezelőre épülő adatbázis driver,
 \item  DATIDX:  Ctree adatbázis driver (Konzumbank), 
 \item  DBFCTX:  dbf adatfilé, Ctree index (BBRt, Otthon Lakástakarék).
\end{itemize}

A BTBTX driver az egyetlen, ami ,,mindent'' tud. Másolással 
hordozhatók az filék Solaris, Linux és Windows között. A platformok
eltérő bytesorrendjét az adatbázis driver automatikusan kezeli.
Nincs akadálya a nagy filék létrehozásának. Egy 13 milliós számla
állomány (mérete 7GB) packolása fél óráig tart. A BTBTX driver
kifejezetten alkalmas óriási adatállományok tárolására.

A DATIDX driver hibája, hogy a dat adatfilé nem hordozható másolással
az eltérő bytesorrendű platformok között. Nem sikerült a Ctree-hez
olyan fordítási opciót találni, ami ezt lehetővé tette volna,
annak ellenére, hogy a dokumentáció szerint létezik ilyen.
A Ctree könyvtár az LFS-t sem támogatja.

A DBFCTX driver esetében a dbf adatfilék triviálisan platformfüggetlenek,
és az LFS-is érvényesül. A ctx indexek nem hordozhatók, de ez nem okozhat
gondot, mivel automatikusan újragenerálódnak. A ctx-ek nem növekedhetnek
2~GB fölé.

A Ctree könyvtárra épülő utóbbi két adatbázis driver bizonyos korlátokkal
rendelkezik, ezek azonban messze felülmúlják a Kontóban jelenleg előre
látható követelményeket. A FairCom honlapjáról származó információ
szerint a Ctree újabb verzióiban ezek a határok már megszűntek, a tesztelésre
rendelkezésre álló rövid idő azonban nem lett volna elég az újabb Ctree
változat beszerzésére. Minhárom adatbázis driverrel a Kontó hibamentesen 
működött.

\subsection{Bus error}

Az olyan kifejezések, mint  \verb!long l=*(long*)p;! hibásak,
mert a processzor nem minden címről képes felszedni long típusú
változót, hanem bizonyos (nem 4 byte-ra igazított) p értékeknél 
,,bus error'' üzenet keletkezik. Ezekkel a hibákkal találkozik először 
a RISC processzorok világában újonc programozó.


\subsection{Byte sorrend}

Intel platformokon az integer típusú számok legkisebb helyértékű
byte-ja van a legkisebb memóriacímen, ez az ún. \emph{little endian}
számábrázolás. A RISC típusú processzorok többségén (a Sparcon is)
fordított a helyzet, a legkisebb memóriacímen a legnagyobb 
helyiértékű byte van, ez a \emph{big endian} számábrázolás. 

A CCC kódjában megbújt néhány olyan rész, ami a számábrázolás
eltérése miatt nem volt hordozható az Intel és Sparc között.
Ezeket a hibákat fel kellett kutatni, és ki kellett javítani. 
Különös figyelmet kellett fordítani a hálózaton át továbbított integer
mennyiségek byte sorrendjére.


\section{Futási eredmények értékelése}

A programok futási sebességéről tömören azt mondhatjuk, hogy
az arányos a CPU órajelével függetlenül attól, hogy Intel vagy
Sparc processzorról van-e szó. Ezt az alapján állítjuk, hogy a Kontó
az 1.5~GHz-es AMD processzoron (Linuxon) kb.\ kétszer olyan gyors, 
mint a 750~MHz-es Ultrasparcon. Ezért a futási idők részletesebb mérésének
és elemzésének nem is láttuk értelmét.

A processzorteljesítmény szerepét két tipikus szituációban
érdemes vizsgálni:
\begin{itemize}
\item 
  A batch jellegű feldolgozások (napi nyitás/zárás, időszaki zárlat, stb.)
  szempontjából, ahol követelmény, hogy a feldolgozás ésszerű időn
  belül elkészüljön.
\item
  Az egyszerre futtatott processzek számát illetően az a követelmény,
  hogy a bank összes ügyintézője által egyidejűleg használt összes program
  fennakadás nélkül fusson.
\end{itemize}
 
A batch feldolgozásokra jellemző, hogy ezek futási ideje a processzorok
számának növelésével nem javítható lényegesen. A jelenlegi gyorsasági
követelményeket azonban a Kontó kényelmesen teljesíti, néhány perces, 
negyedórás futási időkről van szó, ezért a batch feldolgozások
Sparcon ugyanúgy elvégezhetők, mint Intelen.\footnote{
Nem volt azonban ez mindig így. A BBRt bankkártya fiókban 
a DOS-Novell környezetben futó régi Kontót sok munkával kellett
optimalizálni, amíg a félnapos futási időket pár órára leszorítottuk.
Megjegyzem, hogy a futási idő egyben biztonsági kérdés, ui.\
nem megengedhető, hogy egy gikszer miatt megismételt feldolgozás
akadályozza a bank kinyitását.}

Az egyszerre futó programok teljesítménye nyilvánvalóan arányos
a processzorok számával és a memória méretével. 
Ebben rejlik a Sun platform óriási előnye:
a processzorok számának emelésével és a memória méretének
növelésével\footnote{
Intelen mindkét lehetőség korlátozott. A mai inteles architektúrák
egyike sem támogat 4GB-nál nagyobb memóriát.}
arányosan növekszik a párhuzamosan futtatható programok száma. 
Ez alapján állítható (a gyakorlati ellenőrzés azért szükséges volna), 
hogy Sun platformon a Kontó (magyar viszonylatban) bármekkora fiókhálózat 
igényeit képes volna kielégíteni.


\section{Összefoglalás}

Az összességében sikeres projekt erdeményeit az alábbiakban 
sorolom fel:

\begin{itemize}
\item 
   Elkészült a CCC szabad forráskódú fejlesztő eszköz Solarisos
   változata (32 bites GCC-vel), és ez napokon belül letölthető
   lesz a webről.
\item
   Teszteltük a Kontó integrált banki számlavezető rendszert
   Solarison három adatbáziskezeléssel is (BTBTX, DATIDX, DBFCTX),
   és minden esetben megbízható, hibamentes működést tapasztaltunk.
\item
   A futásteljesítmény elemzése alapján valószínűsíthető,
   hogy megfelelően kiépített Sun számítógépen a Kontó
   magyar viszonylatban bármekkora fiókhálózat kiszolgálására képes.
\end{itemize}
 
Befejezésül köszönetet mondok 
a  \emph{Sun Microsystems Magyarország Kft.}-nek és 
az \emph{Assyst Rendszerház Kft.}-nak, 
akik kölcsönözték részünkre a portoláshoz használt számítógépet, 
és \emph{Csiszár Leventének}, aki a kódban  szükséges
javítások nagyobb részét elvégezte.

 
 