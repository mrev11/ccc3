
%#define LOG_FILE   "CCC_TRANSACTION_LOG_FILE"      //logilé neve
%#define LOG_MUTEX  "CCC_TRANSACTION_LOG_MUTEX"     //lockfilé neve
%#define LOG_CREATE "CCC_TRANSACTION_LOG_CREATE"    //=auto esetén létrehozza
%#define LOG_TABLES "CCC_TRANSACTION_LOG_TABLES"    //naplózandó táblák
%#define LOG_SIZE   "CCC_TRANSACTION_LOG_SIZE"      //logilé max mérete
 
\pagetitle%
{Módosítások naplózása a táblaobjektumokban}%
{Dr. Vermes Mátyás\footnote{\ComFirm}}%
{2003.\ április 10.}

\section{Áttekintés}

A táblaobjektumos adatbáziskezelő naplózással egészült ki.
A naplózás érdekében az alkalmazási programokon semmit sem kell változtatni:
mindazok a programok képesek a naplózásra, amik  naplózást
 támogató dinamikus táblaobjektum könyvtárral futnak.
A táblaobjektum könyvtárak a \verb!$CCCDIR/ccctab/common.src!-beli  
2003.01.07-es javítás óta  támogatják a naplózást.

A napló egy XML szintaktikájú szövegfilé,
amibe minden adatbázismódosításkor beíródik egy rekord,
ami tartalmazza 
\begin{itemize}
\item
   a módosított tábla filé specifikációját,
\item
   a módosítás jellegét (hozzáadás, módosítás, törlés),
\item
   a módosított mezők régi és új értékét,
\item
   a módosítást végző UNIX/Windows felhasználót,
\item
   a módosítás idejét,
\item
   tetszőleges program specifikus extra információt.
\end{itemize}

A naplózás bekapcsolása és paraméterezése környezeti
változókkal és konfigurációs filékkel lehetséges.
 

\section{Környezeti változók}

\begin{description}
  \item[CCC\_TRANSACTION\_LOG\_FILE]       
    Ez a környezeti változó kapcsolja be a naplózást (ha nem üres),
    egyúttal megadja azt a filéspecifikációt, amibe a napló kerül.
    A filé nem csak a lokális filérendszerben lehet, hanem pl. 
    NFS-en is.
    \par

  \item[CCC\_TRANSACTION\_LOG\_CREATE]  
    Ha ennek a környezeti változónak az értéke \verb!auto!, akkor
    a naplófilé automatikusan létrejön, ha korábban nem létezett volna.
    Ellenkező esetben az adatbázis módosítás kísérlete runtime
    errorhoz vezet.
    \par


\item[CCC\_TRANSACTION\_LOG\_MUTEX]   
    Ez egy olyan filé specifikációja, amit a naplózó rendszer mutexnek
    használ, azaz lockolva tart azalatt, amíg valamelyik program éppen
    írja a naplót. Ha nincs megadva, akkor a
    \verb!CCC_TRANSACTION_LOG_FILE!-ban megadott filé lesz a mutex.
    Külön mutex megadása akkor szükséges, ha a naplót olyan
    filérendszeren vezetjük, ami nem támogatja a lockot.
    \par

\item[CCC\_TRANSACTION\_LOG\_TABLES]   
    Egy szövegfilé specifikációja, ami a naplózandó filék
    aliasait tartalmazza (egyszerű felsorolás). Ha nincs megadva,
    akkor a naplózás hatálya minden filére kiterjed.
    \par
\end{description}
 
\section{Teljesítmény}

A totális (minden filére kiterjedő) naplózás által okozott lassulás 
kb.\ 30--100\%, attól függően, hogy a napló írásához szükséges sorbaállítás 
mennyire akadályozza a munkát. A felhasználók növekvő száma
nyilván növeli a sorbaállításkor fellépő várakozási időt.
Ugyanakkor a sok egyidejű írási művelet eleve (naplózás nélkül is) 
lassú, ezért nem kell nagyobb {\em relatív lassulásra} számítani.


\section{A naplózás korlátai}

Minden módosítás naplózásra kerül, ami a tabCommit() függvényen 
keresztül íródik ki. Így a különféle importok is benne lesznek 
a naplóban.

Nincs benne a naplóban a tabZap() és tabPack() által
végzett módosítás.

Külön kiemelendő, hogy a filék törlésével, bemásolásával, átnevezésével
okozott változások nincsenek (nem is lehetnek)
benne a naplóban.


\section{A napló hasznosítása}

Jelenleg a naplót csak arra lehet használni, hogy text editorral
keresgéljünk benne.

Viszonylag kevés munkával írható volna olyan program, ami egyes 
adattáblák mentett állapotából kiindulva, a naplóban rögzített változások
újbóli végrehajtásával rekonstruálni tudná a tábla aktuális állapotát.
Ez az eljárás alkalmas sérült táblák helyreállítására.

Ugyanez a technika alkalmazható volna egyes fontos adattáblák
realtime replikálására egy másik számítógépen.

\section{Példa}

Az alábbi rövid naplórészlet rögzítí a Kontóba való belépés,
a KDIRDD elindítása, majd kilépés mozzanatait.

\begin{verbatim}
<update><prg>setpass</prg><pid>3001</pid>
<time>2003041013:04:47</time>
<table><alias>USER</alias><path>USER.DAT</path><rec>415</rec></table>
<f><n>CHANGED</n><b>20030408</b><a>20030410</a></f>
</update>
<append><prg>executiv.exe</prg><pid>3004</pid>
<time>2003041013:04:53</time>
<table><alias>NAPLO</alias><path>NAPLO.DAT</path><rec>195014</rec></table>
<f><n>MODULE</n><v>KDIRDD.EXE</v></f>
<f><n>USER</n><v>m</v></f>
<f><n>START</n><v>13:04:53</v></f>
<f><n>DATUM</n><v>20030410</v></f>
</append>
<update><prg>kdirdd.exe</prg><pid>3009</pid>
<time>2003041013:04:56</time>
<table><alias>NAPLO</alias><path>NAPLO.DAT</path><rec>195014</rec></table>
<f><n>STOP</n><b></b><a>13:04:56</a></f>
</update>
\end{verbatim}

A naplózás odafigyelést igényel a rendszer üzemeltetőjétől, 
ui.\  a napló hamar GB-os méretekre duzzadhat, ügyelni kell tehát az 
archiválásra, nehogy a lemez beteljen.
 

%\gotoindex



 