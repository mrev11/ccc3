
\pagetitle%
{Táblaobjektum tranzakció kezeléssel}%
{Dr. Vermes Mátyás\footnote{\ComFirm}}%
{2000.\ június 1.}


\section{Példa a használatra}
 
\begin{verbatim}
     begin sequence
         tx:=tranBegin()
         ...
         1-könyvelés
         ...
         2-könyvelés
         ...
         tranCommit(tx)
     recover
         tranRollback(tx)
     end sequence
\end{verbatim}
 
Ha a könyvelések során valamilyen hiba keletkezik (pl.\ nem lockolható
egy rekord, nincs fedezet, nincs árfolyam, nem található egy számla, stb.),
akkor csak egy break-et kell végrehajtani, és a félig sikerült 
könyveléseknek nyoma sincs az adatbázisban. Akkor is ez a helyzet,
ha CTRL-C-t nyomnak, vagy, ha elszáll a program. 
A fenti tranzakció elvileg csak úgy vezethet inkonzisztens 
adatbázistartalomhoz, ha a lefagyás a könyvtári tranCommit végrehajtása 
alatt következik be. 

Az összes módosítás a memóriában tárolódik mindaddig, amíg végül tranCommit
ki nem írja az adatbázisba.  Mivel commit előtt a módosítások nem íródnak a lemezre,
azok csak a tranzakciót végző processz számára láthatók, idegen processzek
számára nem. A processz a saját tranzakciójának részeredményeit látja,
ezért a tranzakció többször is érintheti ugyanazt a rekordot, pl.{}
az egymás utáni könyvelések összegei megfelelően fognak
halmozódni a számlán.


%A dokumentum korábbi változata hibás példát tartalmazott.
%Ügyelni kell rá, hogy tranCommit nem állhat az  end sequence után, 
%mert akkor korábban rollback-elt tranzakciókra is meghívódhat,
%ami hibát eredményez.
 
 
\section{Új függvények} 


\subsection{tranBegin()}

tranBegin() egy számot ad, ami a tranzakció indexe a tranzakciókat
tároló stackben (array). A továbbiakban tranBegin() és tranCommit() között:

\begin{itemize}
\item
  A módosított rekordok visszaírása elhalasztódik. A kiírásra váró 
  rekordbuffereket a program memóriában tárolja. 

\item  
  A tranzakcióban végrehajtott változások csak a tranzakciót végző
  processz számára láthatók.

\item  
  Az unlock jellegű műveletek hatástalanok, minden egyszer már 
  megfogott lockot a rendszer megtart a tranzakció végéig 
  (teljes tranCommit, vagy tranRollback).
  
\item
  Azok a műveletek, amik a lockok megtartását és a rekordok
  azonosítását meggátolnák runtime errort okoznak, ha olyan
  táblán akarjuk végrehajtani, ami a tranzakció során változott.
  Ilyenek: close, suppindex, dropindex, pack, zap, \ldots.
  
\end{itemize}
   

\subsection{tranRollback(tx)}
 
tranRollback(tx) 
eldobja azokat a változásokat, amik a tx azonosítójú
tranzakció kezdete óta történtek. 
Ha nem marad aktív tranzakció, akkor elengedi a lockokat, 
egyébként minden lockot megtart. Ha tx==NIL, akkor a legutolsó
tranzakciót dobja el, ha tx==1, akkor minden tranzakciót eldob.
A tranzakció után pontosan azok a lockok maradnak meg,
amik a tranzakció előtt is megvoltak.
 
 
\subsection{tranCommit(tx)}
Befejezi (leveszi a stackről) a tx tranzakciót, és a tx után indított 
tranzakciókat. Ha tx==NIL, akkor a legutolsó tranzakciót fejezi be.
Ha nem marad aktív tranzakció, akkor kiírja
a függőben lévő módosításokat, és minden lockot elenged.
A tranzakció után pontosan azok a lockok maradnak meg,
amik a tranzakció előtt is megvoltak.

\subsection{tranSynchronizeRecord(table,lockflag)}

Tranzakció közben lemezre írja a table aktuális rekordját.
Olyankor használjuk, amikor egy közös számláló tartalmát, pl.{} ugyszam(),
vagy egy új egyedi kulcsot, pl.{} tabInsert(), láhatóvá akarunk tenni
a többi program számára. Ha a lockflag értéke \verb!.t.!, akkor
elengedi a szinkronizált rekord lockját (minden más lockot megtart). 
Az ügyszámot tartalmazó rekordot célszerű rögtön elengedni, máskülönben 
a tranzakció teljesen sorbaállítaná a konkurrens programokat.


\section{Navigálás, láthatóság} 

A tranzakcióban keletkezett változások tranCommit előtt még nincsenek
kiírva az adatbázisba, ezért az új adatok csak a
tranzakciót végző processz számára láthatók. Mivel az indexek sincsenek
módosítva, azért {\bf a tranzakció közbeni navigálás, mindig a 
tábla tranzakció előtti állapotának megfelelő} eredménnyel jár. 
Ennek következményeként bizonyos algoritmusok, 
amik a kiírt adat visszakeresésén alapulnak nem működnek tranzakcióban. 
A klibraryban lévő ugyszam() és a tabInsert() függvények
azonnal láthatóvá teszik az új rekordot egy tranSynchronizeRecord()
hívással.

{\small
A  tranzakciós navigálás plusz feladata, hogy fel kell ismerni
a tranzakcióban módosított rekordokat, és a legfrissebb verzióval 
helyettesíteni kell a lemezen található adatokat. Eközben
el kell igazodni az eredetileg is törölt állapotú,
a tranzakció közben törlődött és a kifilterezett rekordok között.
A navigálás algoritmusa tranzakcióban a következő:

\begin{enumerate}
\item A tranzakció kezdete előtt érvényes indexállapot szerint keres,
\item ha a talált rekord törölt, továbblép, ameddig nem törölt rekordra ér,
\item ha a rekordnak van tranzakcióban módosult verziója, azt behelyettesíti,
\item ha a (helyettesített) rekord törölt, vagy kifilterezett, akkor továbblép.
\end{enumerate}
} 
 
 
\section{Módosítások, törlés}
A rendszer rekordmódosításkor tranzakcióban, vagy azon kívül
egyformán megköveteli a rekordlockot. Ha az elmaradna, runtime
error keletkezik. Az rlock metódus tranzakcióban nem törli a
többi rekordlockot.

A törlés megegyezik egy olyan módosítással, ami a deleted mezőt true-ra
állítja, és (kompatibilitási okból) eggyel előrelép. A törölt rekord
lockja is megmarad.

 

\section{Append} 
Az append kiír a lemezre egy törölt(!) rekordot, az aktuális rekordbufferbe
pedig betesz egy normál állapotú (nem törölt) üres rekordot.
Ha ezután kitöltjük a rekordot, és később végrehajtjuk a tranCommit()-ot, 
akkor a rekord kiíródik. Ha a tranzakciót eldobjuk, akkor az adatbázisban
megmarad a törölt állapotú rekord. Az új rekord lockolva lesz, 
az esetleges többi lock megmarad.

A táblaobjektum gotop, gobottom, seek, skip metódusai számára 
a tranzakcióban appendált (és ezért törölt) rekordok láthatatlanok. 
Az új rekordokra ezért csak a goto metódussal lehet visszapozícionálni.


{\small
A legtöbb bonyodalmat az append megvalósítása okozza, 
néhány szó a részletekről: A tranzakcióban appendált rekordoknak 
láthatatlannak kell lennie a többi processz számára, valamit viszont 
mégis csak ki kell írni, hogy az új rekord recno-t kaphasson. 
A láthatatlanság két módon érhető el: 

\begin{itemize}
\item az adatrekordra nem mutat index,
\item a rekordot törölt flaggel írjuk lemezre.
\end{itemize}
 
A tranzakcióban appendált rekordokat a gotop, gobottom, seek, skip 
utasítások nem látják, az ilyen rekordokra csak a goto-val lehet 
rápozícionálni. Az egyes adatbáziskezelőkben a következő a helyzet.

\begin{description}
\item[DBFCTX]
A tranzakciós append kiír a dbf-be egy rekordot, a dbf header-ben 
a rekordszám eggyel nő. Az indexek egyáltalán nem módosulnak, ezért, 
ha az append után közvetlenül egy rollback jönne, akkor a rekordot 
többé egyáltalán nem lehetne elérni, még goto-val sem, mert a recno 
index sem mutat rá.

Ha a tranzakción belül goto-t hajtunk végre az új rekordra, 
akkor a goto megtalálja az új rekord memóriabeli változatát, 
és azt behelyettesíti.  A commit-ban ki fognak íródni az indexek,
ami után a rekord a normál navigáció számára is látható lesz.

A tranzakció közben a többi processz számára az új rekord teljesen
láthatatlan, azaz még goto-val sem tudnak rápozícionálni.

\item[DATIDX]
A tranzakciós append kiír a dat-ba egy törölt állapotú rekordot,
aminek csak recno indexe íródik ki.  A többi index kiírása szándékosan
le van tiltva azáltal, hogy a rekordot nullkey byteokkal töltjük fel.
(A Ctree-nek definiálni lehet egy nullkey byteot, ha a kulcsok helyén
ezt találja, akkor az indexet nem módosítja.) A recno index kiírását
viszont nem lehet elkerülni, mert ezzel kap recno-t az új rekord.

A recno sorrend szerinti navigálás számára azért lesz láthatatlan
a rekord, mert a lemezre törölt flaggel írjuk ki. Amikor goto-val
rápozícionálunk, akkor  behelyettesítődik a memóriában tárolt
normál (nem törölt) állapotú rekord. A commit a nem törölt
állapotot fogja lemezre írni, a commit elmaradása, 
vagy rollback esetén a rekord végleg törölt marad.

A recno kiírása miatt az idegen processzek számára az új rekord
nem teljesen láthatatlan, ui. goto-val be tudják olvasni a lemezen
törölt, kitöltetlen állapotban levő rekordot, ennek azonban
nincs nagy gyakorlati jelentősége.


%\item[Btrieve]
%A helyzet hasonló a DATIDX-hez azzal a különbséggel, hogy a törölt
%állapotú új rekord kiírásakor a Btrieve minden indexet aktualizál.
%Az üres rekordban levő üres kulcsok kiírása azonban nem hatékony, 
%mert a kitöltött rekord lemezre írásakor úgyis törölni kell őket.
%A Btrieve-ben nincs (?) módszer ennek elkerülésére.

\end{description}
} 

 

\section{Korlátok}

A rendszer a tranzakció közbeni változásokat a memóriában gyűjti
egészen a végső tranCommit-ig, ebből adódóan nem képes akármilyen nagy 
tranzakciót végrehajtani. Kerülendők az olyan tranzakciók, amik egy 
tábla minden rekordját módosítják, és gyakran 
OREF overflow\footnote{A CCC az  OREF\_SIZE
környezeti változóban megadott darabszámú objektumot (stringet,
arrayt, kódblokkot) tud tárolni, ha a program ezt túllépi,
akkor keletkezik az ismert  OREF overflow hiba.
 OREF\_SIZE deafult értéke 8192, ami kis és közepes
programokhoz elegendő, de általában legalább 20000-re van beállítva.} 
hibához vezetnek. Ugyanezért  a tranzakciókezelésből ki kellett
zárni a tömeges módosítást tartalmazó metódusokat.
Ha tranzakció közben próbáljuk meg végrehajtani az alábbi műveleteket, 
runtime errort kapunk:
\begin{itemize}
\item tabAppendFrom(),
\item tabLoadDBF(),
\item tabPack(),
\item tabZap(),
\item tabUpgrade().
\end{itemize}
 
A tranzakcióban módosult táblára tiltottak az alábbi műveletek:
\begin{itemize}
\item tabSuppindex(),
\item tabDropindex(),
\item tabOpen(),
\item tabClose().
\end{itemize}
Egy újonnan nyíló tábla nyilván még nem módosult, 
tehát nincs akadálya annak, hogy egy könyvelési tranzakció közben 
megnyissuk pl.{} a zárlati forgalmak tábláját. Ha viszont ebbe beírtunk 
egy forgalmi tételt, akkor azt a tranCommit előtt tilos lezárni 
(vagy újranyitni), mert az megakadályozna egy esetleges későbbi 
tranRollbacket. 
 

\section{Debug funkció} 

A tranzakciók különösen megkönnyítik a debugolást.
Állítsuk be a TRANCOMMIT=debug környezeti változót,
ilyenkor a rendszer minden változást listáz, így azonnal látszik,
hogy mit művel a program az adatbázisban.
 



\section{Gyakorlat}

Az itt közölt példaprogram megtalálható a ccctutor/tran directoryban.
Fordítsuk le a próba projektet (m script), majd a ccctutor/tran/test
directoryban indítsuk el s-et.

\begin{verbatim}
#include "table.ch"
#include "_proba.ch"

*****************************************************************************
function main()

local dd

    PROBA:create
    PROBA:open(OPEN_EXCLUSIVE)
    PROBA:zap
    
    //teszt adatbázis

    app("a","Van, aki forrón szereti.")
    app("b","Van, aki forrón szereti.")
    app("c","Van, aki forrón szereti.")
    app("d","Van, aki forrón szereti.")
    
    view("INDULÓ")
 
    tranBegin()
    begin sequence

      PROBA:goto(2); PROBA_TEXT:="Próba szerencse." 
      view("módos1")
      //break(NIL) 
\end{verbatim}

      Navigáció közben látjuk a 2-es rekord változását,
      ha azonban itt kilépnénk (break), akkor az adatbázis 
      eredeti (INDULÓ) állapota maradna meg.

\begin{verbatim}
      app("bb","Van, aki forrón szereti.")
      view("app1")
      rec()
\end{verbatim}

      Betettünk egy új rekordot, ám az a navigáció közben 
      nem látszik. Csak goto-val lehet rápozícionálni, ahogy
      az pl. a view() végén lévő PROBA:restore-ral történik.
      A rec() kiírás mutatja, hogy azért megvan a rekord.

\begin{verbatim}
      PROBA:control("field")  //korlátozás feloldva
 
      PROBA:goto(3);  PROBA_TEXT:="Próba szerencse." 
      PROBA:goto(4);  PROBA:delete
      app("cc","Van, aki forrón szereti.")
      tranSynchronizeRecord(PROBA:table)
      app("dd","Van, aki forrón szereti.")
      dd:=PROBA:position
      view("módos2")
\end{verbatim}

      Mindenféle módosítás.      
      A törölt rekord kikerül a navigáció látóköréből.
      A navigáció most sem mutatja az új rekordokat,
      kivéve, ha azt külön utasítással szinkronizáljuk.
      A dd kulcsú rekord pozícióját megjegyeztük későbbi
      használatra.

\begin{verbatim}
      PROBA:seek("c")
      PROBA_FIELD:="aaa" 
      view("seek")
\end{verbatim}

      A "c" kulcsot átírtuk {"}aaa"-ra. A navigáció az új kulcsot
      mutatja, de a régi kulcsnak megfelelő helyen, ui. először
      megtörténik a pozícionálás a még nem módosult indexekkel,
      majd az eredményre rátöltődnek a tranzakció közben készült
      (és a memóriában tárolt) változások.

\begin{verbatim}
      PROBA:seek("c"); rec()
      PROBA:seek("aa"); rec()
      PROBA:gobottom(); rec()
\end{verbatim}

      Ezek eredménye szintén mutatja, hogy a navigálás az index
      tranzakció előtti állapotának megfelelő eredményt ad.

\begin{verbatim}
      PROBA:goto(dd);  PROBA_TEXT:=upper(PROBA_TEXT)
      //break()
\end{verbatim}

      A tranzakcióban létrejött rekordokra goto-val lehet
      visszapozícionálni és a tartalmukat újra felhasználni.
      Ha itt break-et csinálnánk, az induló állapothoz képest
      csak a külön szinkronizált cc rekord volna az eltérés.

\begin{verbatim}
      tranCommit()
    recover
      tranRollback()
    end sequence
       
    view("VÉGSŐ")
\end{verbatim}

    Kiíródnak a módosítások, aktualizálódnak az indexek,
    így a navigáció mutatja az eddig rejtett rekordokat is.

\begin{verbatim}
    return NIL
    
*****************************************************************************
static function app(fld,txt)    
    PROBA:append
    PROBA_TEXT:=txt 
    PROBA_FIELD:=fld 
    return NIL

*****************************************************************************
static function rec()
    ? PROBA:position, PROBA_FIELD, PROBA_TEXT
    ?
    return NIL

*****************************************************************************
static function view(txt)
local state:=PROBA:save
    ? padr(txt,6),">", PROBA:position 
    PROBA:gotop
    while( !PROBA:eof )
        ? padr(txt,6),">", PROBA:position, PROBA_FIELD, PROBA_TEXT
        PROBA:skip
    end
    ?
    PROBA:restore:=state
    return NIL


*****************************************************************************
static function view0(txt)
local state:=PROBA:save, n
    ? padr(txt,6),">", PROBA:position 
    for n:=1 to PROBA:lastrec
        PROBA:goto(n)
        ? padr(txt,6),">", PROBA:position, PROBA_FIELD, PROBA_TEXT
    next
    ?
    PROBA:restore:=state
    return NIL
 
*****************************************************************************
\end{verbatim}


\section{Megjegyzések}
 
Ez a tranzakciókezelő rendszer az általunk használt táblaobjektumokkal
megvalósítható. %Jelenleg a datidx, dbfctx és Btrieve verziókban próbáltam ki.
Hangsúlyozom, hogy a program {\em kifejezetten egyszerű}
(kevés, az eredeti kódtól jól elkülönülő kiegészítésről van csak szó),
ezért bízni lehet abban, hogy sikerül stabilra megcsinálni.
Ez az egyszerű tranzakciókezelés a máshol 
(pl.\ Oracle) elérhető funkcionalitásnak
legalább 90\%-át biztosítja, ugyanakkor a hatékonyságban szinte
semmi csökkenést nem okoz.
 

%\gotoindex
 
