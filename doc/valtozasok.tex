
\begin{description}

\item[2008.02.17]
  A ppo2cpp-ben a flex elemző a hosszú stringeket soronként darabolja,
  mert az MSC fordítónak gondja van a túl hosszú stringekkel.

\begin{verbatim}
  "\n"   {raw_cat("\\n\\\n");/*egy nagy string*/}
\end{verbatim}
  
  helyett

\begin{verbatim}
  "\n"   {raw_cat("\\n\"\nL\"");/*darabol*/}
\end{verbatim}

\item[2008.02.17]
  Windowson a file API-kban áttérés a unicodera:
\begin{verbatim}
    chmod     ->    _wchmod  
    getcwd    ->    _wgetcwd 
    mkdir     ->    _wmkdir  
    rmdir     ->    _wrmdir  
    remove    ->    _wremove 
    rename    ->    _wrename 
    fopen     ->    _wfopen  
    stat      ->    _wstat   
    sopen     ->    _wsopen  

    FindFirstFile -> FindFirstFileW
    CopyFile -> CopyFileW
    stb.
      
    nincs: _wexec, _wspawn 
    rossz: _wsystem (elromlanak a nem-ASCII argumentumok)
\end{verbatim}
    
  Példa:

\begin{verbatim}
    #ifdef _UNIX_
        _retni( chmod(_parb(1),_parni(2)) );
    #else
        bin2str(base);
        _retni( _wchmod(_parc(1),_parni(2)) );
    #endif
\end{verbatim}

  Kellemetlen, hogy sok helyen megjelent ez a kettősség.
  Most a windowsos websrv kezeli a cirill betűs directorykat
  és fájlokat, de nem tud végrehajtani nem-ASCII CGI-t.

\item[2008.02.05]
  Nem lehet definiálva az MSC fordítónak a \verb!_WINDOWS_! szimbólum, 
  mert attól elromlik (ki kellett venni a compile.opt filékből).

\item[2008.01.31]
  Rossz volt Windows XP-n a socket.inherit implementáció.
  Windows 2000-en a DuplicateHandle-es megoldás jól működött.
  XP-től kezdődően viszont a DuplicateHandle nem alkalmazható socketekre,
  hanem helyette a Set/GetHandleInformation-t kell használni.

\item[2008.01.24]
  Küzdelem a \verb!char *p="stringconstant"! forma ellen,
  amire a gcc 4.2.1 warningot ad.
  Helyette \verb!const char *p="stringconstant"!-t kell írni,
  ezért egy csomó helyen meg kellett változtatni az interfészeket.

\item[2008.01.19]
  {\bf Fontos hibajavítás} az attrib -> method felüldefiniálásban:

  Korábban ilyenkor az attribútumok tömbjében lyuk maradt. 
  Az  object:attrvals metódus az attrnames és asarray
  párhuzamosnak gondolt tömbök összepárosításával működik.
  A felüldefiniálás után asarray lyukas maradt, de attrnames nem,
  azaz a tömbök mégsem párhuzamosak, így az adatok elcsúsztak.
  Ezért attrib -> method felüldefiniálás után a classMethod
  újraszámolja az attribútumok indexét, és megszünteti a lyukat.
 
  Új függvény class.prg-ben: classMethodCount.
  classMethNames a hash-beli elhelyezkedés sorrendjében adja
  a metódusneveket (a korábbi név sorrend helyett).
  
  A CCC3-beli class.prg átvive CCC2-be (backport).

\item[2007.12.17]
  A directory és findfirst függvényekben binopt opció,
  hogy hozzá lehessen férni a nonascii karaktereket tartalmazó
  filénevekhez.

\item[2007.10.30]
  direxist-ben kiegyenlítve a záró / okozta eltérés:
  ha dir létezik, akkor dir/ is létezik (Windowson is).

\item[2007.10.20]
  ppo2cpp-ben az implicit recover kódgenerálása javítva.

\item[2007.09.10]
  Windows dirname() javítva (a path D: részét ki kell hagyni).

\label{20070730}
\item[2007.07.30]
  SSL támogatás a \verb!ccc3_sslsocket! könyvtárban.
  Új osztályok: socket, sslcon, sslctx.
  A socket fd-k kaptak egy burkoló osztályt. 
  Az SSL kapcsolatok is kaptak egy burkoló osztályt.
  Az osztályok  egyforma interfésze lehetővé teszi, 
  hogy ugyanaz a program (az SSL bekapcsolásától eltekintve) 
  egyformán működjön plain és secure socketekkel.
  A websrv program át van írva az új osztályok használatára,
  tehát saját erőből (sslforwarding nélkül is) tud titkosítani és azonosítani.
  A jtlib könyvtárba is bekerült az SSL támogatás. 
  Az alkalmazásokban a kliens (jterminal) és szerver is 
  tudja hitelesíteni a másik felet.
  
  A CCC2 olyan compatibility réteget kapott, ami lehetővé teszi,
  hogy a CCC3 alá fejlesztett könyvtárak (egy része) változtatás
  nélkül leforduljon CCC2 alatt. Így fordulnak CCC2-ben 
  a socket, sql2, jtlib  könyvtárak, 
  tehát nem kell ezeket két változatban is karban tartani.

  Új UNIX spawn implementáció prg-ben.

\label{20070513}
\item[2007.05.13]
  CCC Boehm-féle szemétgyűjtéssel:  \href{boehm_gc.hu.txt}{boehm\_gc.hu.txt}.

\label{20061024}
\item[2006.10.24]
  Reguláris kifejezések: Csatoló a \verb!pcreposix! 
  könyvtárhoz \verb!$CCCDIR/tools/regex!-ben.
  Installálni kell hozzá a libpcre3 csomagot,
  Linuxon a fejlesztő (libpcre3-dev) csomagot is.
  Windowsra forrásból lehet installálni pcreposix-ot,
  az eljárás a pcre directoryban van leírva.
  Csak a POSIX interfész támogatott,
  nincs UTF-8 támogatás, 
  csak bináris stringre működik.

\label{20060918}
\item[2006.09.18]
  Elkészült a CCC3. 

  Lényegesen fejlődött a GTK interfész, van Glade modul,
  amivel CCC-ben is lehet egérhúzogatással GUI-t szerkeszteni.
  Fejlődött a Jáva terminál interfész (komaptibilis javítások).
  Utoljára készült el, egyben lényegesen fejlődött az SQL2 interfész,
  a specifikáció verziószáma 1.0-ról 1.1-re változott,   a 
  \href{sql2.html}{dokumentáció} átvizsgálva, kibővítve.
  Dinamikus tableentity objektum generálás tds-ből.
  Jeed: Jáva terminálos tableEntity EDitor.
  
  A CCC3 telepítéséről rövid dokumentáció olvasható 
  \href{http://ccc.comfirm.hu./ccc3/download/olvass.html}{itt}.
  A preferált terjesztési módszer: forrás letöltés subversion 
  (svn) klienssel.
  
  A GTK-t, Jáva terminált, SQL2-t érintő javítások/változások
  nincsenek backportolva a CCC2-be, ezért az új projekteket 
  mindenképpen CCC3 környezetben érdemes fejleszteni.

\label{20060520}
\item[2006.05.20]
  Útjára indult a CCC3.
  A fő újdonság a karakter stringek 
  és a bináris stringek megkülönböztetése,
  más szóval az unicode/UTF-8 támogatás.
  A változások részleteiről az
  \href{ccc3_ujdonsagok.html}{Újdonságok a CCC3-ban}
  dokumentumban találunk információt.
  A csomagok a 
  \href{http://www.comfirm.hu/ccc3/download}{http://www.comfirm.hu/ccc3/download}
  directoryból tölthetők le. A készültség jelenleg kb.~80\%.
  Egyelőre csak azok használják, akik teszteléssel és hibajelentéssel
  akarják segíteni a fejlesztést.


\label{20060512}
\item[2006.05.12]

  Befejeződött a CCC2 fejlesztése.
  Az új/megváltozott típusok miatt a CCC3 nem teljesen 
  kompatibilis a CCC2-vel, ezért a CCC2 nem vált feleslegessé,
  mindazontáltal a továbbiakban csak hibajavítások lesznek hozzá
  a nem portolandó programok fenntartása érdekében.

  A mostani új CCC2 csomagok tartalmazzák a korábbi delta csomagokat
  és az azóta keletkezett kisebb javításokat. A csomagnevek
  kissé változtak, hogy a CCC2 és CCC3  csomagok egymás mellett
  keveredés nélkül kezelhetők legyenek.

  Megjegyzés a GTK-hoz: A windowsos fordítók közül a Borland
  (FreeCommandLineTools, 2000-ből származik, nincs újabb) nem képes 
  lefordítani a GTK-t. Az MSC lefordítja ugyan, de egyes programok 
  (pl. appwindow) SIGSEGV-vel elszállnak. Ezért egyedül a MinGW 
  látszik alkalmasnak windowsos GTK programok készítésére.

\label{20060124}
\item[2006.01.24]
  Class utasítás, 
  vararg API, 
  qout-ba visszatéve a karakterkonverzió,
  windowos terminálok korszerűsítve, 
  CCC névtérnevek C++-beli prefixelése,
  DOSCONV alapértelmezése 0.

\label{20051001}
\item[2005.10.01]
  A futtatórendszer kisebb változásai miatt 
  az új objectek nem linkelhetők össze a régiekkel, 
  ezért átálláskor egyszerre mindent újra kell fordítani.

  \paragraph{LGPL license:}\label{LGPL}
    Az egész CCC projektet LGPL license alá helyeztük.
    Ebből adódóan el kellett hagynunk azokat a nem szabad 
    komponenseket,  amik license-elése ellentmond az LGPL-nek,
    nevezetesen a Ctree alapú táblaobjektumokat. Megjegyzés:
    a Ctree teljes mértékben helyettesíthető a BTBTX táblaobjektummal.
  
  \paragraph{Flex és Lemon telepítés:}\label{FLEX}
    A CCC telepítője automatikusan installálja a flex-et és lemon-t
    (\$CCCDIR/usr alá), ezért nem kell a telepítésükkel külön  foglalkozni.

  \paragraph{Kivételkezelés:}\label{EXCEPTION}
    Több recover, amik típus alapján
    válogatnak a kivételek között, új finally ág, mindez Jáva mintára.
    A ppo2cpp a 4.3.xx változattól kezdve fordítja az új szintaktikát.
    Recover nélküli break kilépés helyett kiértékeli az errorblockot.
    A könyvtárakban eval(errorblock(),e) helyett break(e).
    Különféle error leszármazottak syserror-ban.
    Az object osztályban új metódus: isderivedfrom.
    Prototype objectek.

    \paragraph{Táblaobjektumok:}
    A hibakezelés a fentieknek megfelelően változott,
    speciális hibaosztályok: tabStructError, tabIndexError.
    A Ctree alapú táblaobjektumok kikerültek a csomagokból.

  \paragraph{Szignál kezelés:}\label{SIGNAL}
    signal.ch-ban egységesen használt signum konstansok.
    Mutex lockok alatt a szignálok tiltva (deadlock ellen).
    Windows ctrlcblock megszűnt, setposixsignal átalakítva.
    Szignálok blokkolása Linuxon és Windowson egységesítve.
    Signalblock paraméterezése változott: eval(signalblock(),signum).
    Szálak indításakor védekezés a szignálok ellen.
    siglocklev, sigcccmask öröklődik a szálak között.

    Új szignál API: 
        signal\_description, signal\_lock, signal\_unlock,
        signal\_raise, signal\_send, signal\_pending, signal\_clear, 
        signal\_setmask, signal\_mask, signal\_unmask.

  \paragraph{Megjegyzés a szignálokhoz:}
    A szálbiztonság és a korrekt szignálkezelés legfontosabb
    kritériuma: Minden olyan pillanatban, amikor egy másik 
    szálból szemétgyűjtés indulhat, vagy a szál szignált kaphat 
    a vermeken kell legyen minden élő változó, de nem lehet 
    ott semmi más (pl. keletkezőben vagy megszűnőben levő változók). 
    E kritérium  teljesítése csak a ccc2 és ccc2\_ui\_ könyvtáraknál 
    kitűzött cél. Ha egy nem szálbiztosan megírt program 
    a szignálkezelőben  szemetet gyűjt, elszállhat.
    A CCC alap futtatórendszere (ccc2 és ccc2\_ui\_) a szignálkezelés 
    alatt is ép állapotban tartja a változóteret, és elkerüli 
    a deadlockokat.  A reentráns programoknál szokásos általános
    szabályokon kívül semmilyen tiltás nincs, elvileg  bármi
    futhat a szignálkezelőben.

    A signal\_lock(), signal\_unlock() API csak azt a szálat védi
    a szignáltól, ami a signal\_lock()-ot meghívta. Ettől még bármely
    másik szál kaphat SIGINT-et, és az egész program kiléphet.
    Windowson éppenséggel mindig ez a helyzet, ui. a SIGINT-et
    mindig egy újonnan induló szál kapja. A signal\_lock() ezért
    csak arra való, hogy a mutexeket védje deadlock ellen.
    Ha az egész programot kell védeni a SIGINT-től akkor ki kell
    cserélni a minden szálra globális signalblock()-ot. Így
    működik az új setposixsignal().

  \paragraph{Castolások:}
    Az összes (char*) cast felülvizsgálva.
    Áttérés \verb!const char*! deklarációkra, ahol az lehetővé teszi 
    a  (char*) cast elhagyását, pl. \verb!char *txt=(char*)YYText()! 
    helyett   \verb!const char *txt=YYText()!.
    A strings(), stringn() függvények char* helyett const char*-t 
    várnak, szintén a castok kerülése érdekében.
    
  \paragraph{Megjegyzés a castoláshoz:}
        \verb!char *buf="string literal";!
    régi értelmezése: inicializált karakterbuffer,
    új értelmezése: karakterkonstans (readonly szegmensbe helyezve).
    Korábban a -fwritable\_strings opcióval választható volt
    a régi értelmezés, most azonban ez az opció a GCC-ben megszűnt.
    A string literál helybeni nagybetűre konvertálása SIGSEGV-t okoz.
    Ha a C++ logikus akarna lenni, akkor a fenti sorra hibát kéne
    jeleznie. Ehelyett a C++ tervezőjének két kötetes,  1000 oldalas 
    könyvében azt olvasom: A sok régi programra való tekintettel (!) 
    a fordítók a fenti sort warning nélkül fogadják el. Következmény:
    Az inkompatibilis változás miatt, és amiatt, hogy az új C++ szabvány 
    következetlenül használja a char* és const char*  típusokat, régi 
    C programok warning-mentes fordítás után SIGSEGV-znek.
    
  \paragraph{Szemétgyűjtés, szálbiztonság:}
    A szemétgyűjtés és a VALUE értékadás szinkronizálásában
    a vitatott zsilipelés helyett szálanként privát mutexek lockolva.
    A nem szálbiztos *stack++=NIL helyett mindenhol PUSHNIL().
    
  \paragraph{Kulcsszavak:}  
    Egyes kulcsszavak (pl. \verb!next!) használhatók függvénynévként.
    A lexikai elemző a szimbólum előtt/mögött levő "." 
    (névtér határoló), illetve a szimbólumot követő "("
    karakter alapján állapítja meg, hogy nem kulcsszóról,
    hanem közönséges szimbólumról van szó. Továbbra sem
    lehet függvénynévként használni olyan kulcsszavakat,
    amiket a normál használatban zárójeles kifejezés
    követhet, ilyen pl. az \verb!if! és \verb!while!.
    Megjegyzések: 1) Metódus és osztálynév szerepben mindig is
    megengedett volt kulcsszavakat használni. 2) Változónév
    szerepben ezután sem megengedettek a kulcsszavak.

  \paragraph{Egyéb:}
    Build, prg2ppo, z, zgrep statikusan linkel.
    sread()-ből kivéve az 1 órás timeout.
    Stack kezelési hiba javítva evalarr()-ban.
    asort()-ban az összehasonlító block alatt gc engedélyezve.
    Egyes include filék több példányban tárolása megszűnt.
    Run és quit preprocesszálása megszűnt.
    Solarison javítva a waitpid() WNOHANG módja.
    A cccapi.h Windowson inkludálja windows.h-t, 
    azért felesleges azt máshol is inkludálni.
    A prg2ppo preprocesszorban áttértünk case sensitive
    filéspecifikációk használatára, ez régi programokban 
    hibákat hozhat ki.

\label{20050508}
\item[2005.05.08]
    Solaris port karbantartása. 
    Védelem veremtúlcsordulás ellen.
    
    A GCC-ben megszűnt a \verb!-fwritable-strings! opció.
    Ez az inkompatibilis változtatás bármely eddig jónak számító
    és a \verb!-fwritable-strings! nélkül is warning mentesen forduló 
    programba segfaultot vihet. Például, ha a program egy stringet 
    helyben akar nagybetűre konvertálni. A \verb!ccc-fltk-config!-ban 
    előfordult ilyen eset.

\label{FREEBSD}
\item[2005.04.11]
    Portolás FreeBSD-re. 
    A ccc2 csomag nevébe bekerült a dátum: \verb!ccc2-20050411.zip!.

\label{SQL2}
\item[2005.02.28]
    Korszerűsödött az SQL2 interfész. Az új változat 
    ideiglenesen a CCC2 csomagtól függetlenül, külön tölthető le.
    Az új változattal helyettesíteni kell a CCC2 csomag tools 
    directoryjában levő régi változatot. Egységes, új dokumentáció 
    is készült a csomaghoz: 
    \href{http://ok.comfirm.hu/ccc2/sql2.html}{SQL2 1.0 interfész}.

\label{64BIT}
\item[2005.01.24]
    Portolás 64-bites (x86-64) rendszerre.
    
    A mostani változások az Opteron, vagy Athlon-64 
    processzoron 64-bites Linuxot futtató felhasználókat érintik. 
    Egyéb 64-bites (Itanium, Sparc) rendszereken nem történt próbálkozás.  
    A régi rendszereken remélhetőleg nem romlik el semmi.
    Az új könyvtár általában futtatja a régi programokat, tehát nem kell
    feltétlenül mindent újrafordítani.
    
    32-bites rendszeren a C szintű int/long számok és pointerek 
    mindig tárolhatók egy double-ban, azaz egy Clipper számváltozóban.
    64-biten azonban már más a helyzet. Létrehoztunk ezért egy új Clipper 
    típust (jele P=pointer), amit olyan adatok tárolására lehet használni,
    mint pl. egy Windows handle, OCI handle,  thread azonosító, stb..
    
    Clipper szinten a P változókkal a tároláson, értékadáson
    kívül kevés  értelmes dolgot lehet csinálni:
    \verb!p1:=p2!,
    \verb!p1==p2!, \verb|p1!=p2|, 
    \verb|p==NIL|, \verb|p!=NIL|, 
    \verb!l2hex(p)!, 
    \verb!valtype(p)!, 
    \verb!empty(p)!, 
    \verb!xtoc(p)!, 
    \verb!qout(p)!.
    
    C szinten a \verb!pointer(void*)! függvény a veremre teszi a változót,
    a \verb!_parp(i)! makró átveszi a P tipúsú paramétert, 
    a \verb!_retp(p)! makró létrehozza a vermen a P típusú visszatérési értéket.
    
    A 64-bites port nem tartalmazza a CTREE adatbáziskezelőt. 
    Pontosabban a csomagban továbbra is benne vannak a 32-bites
    bináris könyvtárak, ezek azonban nem használhatók 64-biten.
    A BTBTX adatbáziskezelő kiválóan fordul 64-biten is. 
    
    Technikai megjegyzések: 
    
    A 64-bit érdekében nem kellett változtatni a fordítón. 
    A generált C kód 32-biten és 64-biten ugyanaz. A futtatókönyvtár 
    forrása is ugyanaz 32 és 64-biten, még \verb!#ifdef! makrókra sincs 
    szükség.  A bináris kód attól függően lesz 32 vagy 64-bites, 
    hogy a rendszeren  mi a \verb!void*! típus mérete.
    
    32-biten:
    \begin{verbatim}
    sizeof(short)==2
    sizeof(int)==4
    sizeof(long)==4
    sizeof(void*)==4
    \end{verbatim}

    64-biten:
    \begin{verbatim}
    sizeof(short)==2
    sizeof(int)==4
    sizeof(long)==8
    sizeof(void*)==8
    \end{verbatim}
    
    A mostani változtatások lényege: 
    Azokon a helyeken, ahol a C réteg 64-bites mennyiséget adott
    volna át a Clipper rétegnek számváltozóban, ott a számváltozó (N)
    helyett az új pointer (P) típusra tértünk át.
\item[2005.01.19]
    A ppo2cpp fordító hibát jelez, 
    ha a \verb!begin sequence! és \verb!recover! közül kiugrani
    akaró \verb!return/loop/exit! utasítást talál.
    
    Új szintaktika hosszú string literálok írására: 
    A \verb!<<SYM>>raw string<<SYM>>! kifejezésben SYM egy tetszőleges
    szimbólum. A raw string akármilyen hosszú lehet, bármi
    lehet benne, kivéve \verb!<<SYM>>!, mert az lezárja a stringet.

    Scrollozó Windows konzolok támogatása. Korábban a programok
    az egész képernyő buffert vették képernyőnek, most alkalmazkodnak
    a látható méretekhez.
\item[2004.10.30]
    Javítások a többszálúságban: Windowson a \verb!MUTEX_LOCK! és
    \verb!MUTEX_UNLOCK! makrók definíciója fel volt cserélve, 
    ezért fordítva működtek. Emiatt többszálú programokhoz (Windowson) 
    {\em minden objectet és minden könyvtárat\/} újra kell fordítani.
    A névtereket támogató fordító és az sql2 könyvtár 
    bekerült a ccc2.zip főcsomagba, egyúttal önálló csomagként megszűnt.
\item[2004.10.09]
    Fejlődött a  sql2 könyvtár.
    A ppo2cpp\_4\_1\_01  fordító támogatja a többszintű névtereket, 
    ez szükséges az új sql2 csomaghoz, 
    ami most az sql2.oracle és sql2.postgres névtereket használja.

\item[2004.09.28]
    A pgsql2.zip és oci2.zip csomagok megszűntek, 
    helyettük letölthető az  sql2.zip csomag,  dokumentáció az 
    \href{http://ok.comfirm.hu/ccc2/sql2.html}{SQL2 könyvtár} oldalon.

\label{NAMESPACE}
\item[2004.09.23]
    Névtér támogatás.
    A CCC névterek közvetlenül C++ névterekre képződnek le.
    Az új fordító kísérleti változata egyelőre külön csomagban tölthető le:
    ppo2cpp\_4\_1\_00.zip.
    A namespace.zip csomagban egyszerű példák vannak a névtér használatára.

\item[2004.09.05]
    A táblaobjektumban új hibaüzenet jelzi, hogy memót nem lehet indexelni.
    Áttérés (Bison helyett) Lemonra, Flexben áttérés C++ elemzőre.
    A Flex és Lemon  telepítésével külön foglalkozunk,
    egyelőre a bisonos változatok is benne vannak a csomagban.
    Új 4.0.x verziószámú (lemonos) ppo2cpp fordító. 
    Az osztály-metódus párosításban hash tábla optimalizálás.
    Az xmldom könyvtár újraírva, most reentráns és kétszer gyorsabb.
    Új Oracle és PostgreSQL interfész (külön csomagokban).
    Külső static változók szinkronizált létrehozása,
    osztályregisztrációk szinkronizálása.
    Kompatibilitás: Az új (2.1.x) CCC környezet képes
    futtatni a 2.0.x környezetben fordított programokat,
    fordítva azonban nem megy.

\item[2004.04.05]
    BTBTX pack a replikálhatóság érdekében megtartja a rekordsorrendet.
    Táblaobjektum naplózás és replikálás bővült.
    Gyorsított XML elemzés a replikációban.
    Build támogatja a Lemon fordítást.
    Az XMLRPC könyvtár DOM elemző része külön könyvtárba került (ccc2\_xmldom).
    Memóriaszivárgás javítása asize-ban.
\item[2004.03.05]
    Új fopen opciók: \verb!FO_CREATE!, \verb!FO_TRUNCATE!.
    Új fcreate opció: \verb!FC_NOTRUNCATE!.
    Táblaobjektum replikálása MySQL adatbázisba.
    Félkész MS-SQL tábla objektum.
\item[2004.02.05]
    Solaris port felfrissítve, apróbb javítások.
\item[2004.01.14]
    Hibajavítások a táblaobjektumokban:
    A 2003.11.20-as javítás elrontotta a táblaobjektumok
    file-lock protokollját. A DBFCTX táblaobjektum nem tudott
    új indexet csinálni.
\item[2004.01.10]
    SIGPIPE kezelése SIGTERM-hez hasonlóan, 
    socket\_write()-ban SIGPIPE ignorálva.
\item[2004.01.05]
    Javítás az XMLRPC kliens osztályban.
    Javítás \verb!thread_exit()!-ben.
\item[2003.12.31]
    ui\_ kiegészült az rconsole és wstatid modullal.
    Az uid-beli symbols.h frissítve.
    tbwrapper csak akkor tölti be az interaktív komponenseket, 
    ha az aktuális megjelenítő driver nem ui\_.
\item[2003.12.22]
    A socket\_write()-ban a send ismétlése kihagyva.
\item[2003.11.25]
    A borlandos dup() hibásan áttér \verb!O_BINARY!-ról \verb!O_TEXT! módra,
    ennek elkerülésére beállítva a default \verb!O_BINARY! mód.
\item[2003.11.20]
    Nem öröklődő filédescriptorok bevezetése az összes
    táblaobjektumban.
\item[2003.10.19]
    CCC alapkönyvtár portolása GCC 3.3-ra (sorvégek).
    Windows spawn-ban paraméterellenőrzés javítva.
    UNIX spawn-ban a várakozás javítva.
    fdup() bővítve, új függvény hdup (DuplicateHandle).
\item[2003.10.15]
    Az xmlrpc könyvtár portolása GCC 3.3-ra.
\item[2003.10.03]
    Null értékek kezelése az OCI könyvtárban.
\end{description}


