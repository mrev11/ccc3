
\pagetitle%
{CCC-GTK csatoló}%
{Dr. Vermes Mátyás\footnote{\ComFirm}}%
{2005. október}


\section{Áttekintés}
A CCCGTK  csatoló  elérhetővé teszi a GTK funkcióit CCC programok számára.
CCCGTK igyekszik minél kevesebbet veszíteni a GTK eredeti lehetőségeiből, 
de nem akar semmit hozzátenni:
\begin{itemize}
\item 
    Szó sincs arról, hogy ezután a régi Clipper programok
    átírás nélkül GTK köntösben tudnának megjelenni.
\item
    Nem foglalkozunk adatbázistartalom és a megjelenítés
    összekapcsolásával.
\item
    Nincsenek a Clipperben megszokott (nagyon hasznos) 
    picture-öket pótló eszközeink, stb..     
\end{itemize}
Az ilyesmit az alkalmazásfejlesztés feladatai közé soroljuk,
ezzel szemben a CCCGTK egyszerűen csak kivezeti a GTK API-t CCC szintre.

A GTK-ban kb. 200 osztály és 2000 API hívás van. 
Ez meglehetősen sok (takarékosnak semmiképp sem mondanám).
Az interfészfüggvények kb.~80\%-a automatikusan generálódik
(a PyGTK-hoz hasonlóan) defs fájlokból. A maradék 20\% esetében
nem lehetséges, vagy nem célszerű az automatikus kódgenerálás.
Az arány még változhat, ui. kiderülhet, hogy a generált
kód egy (kis) része nem jó, illetve biztosan javulni fog még 
a kódgenerátor. A kimaradó interfészfüggvények egy részére nincs 
is szükség, az eddig megtalált hiányokat viszont kézzel pótoltam.

A CCCGTK készültségi fokát úgy jellemezném, hogy valószínűleg 
jól van elkezdve. Egyelőre nem tudható több, mint hogy a szokásos 
példaprogramok futnak, és feltehető, hogy az eddig nem próbált 
widgetek többsége is működni fog. Mindenesetre a widget gallery-ben
szereplő widgetek mindegyikére működő demó van.


\section{GTK változatok}
Jelenleg Ubuntu Hoaryn dolgozom, és GTK-2.6.4 változat van a gépemen. 
Ezzel fordul a CCCGTK. Ugyanakkor tudni kell, hogy a GTK elég gyorsan 
mozgó célpont, gyorsan deprecateddé nyilvánítanak fontos osztályokat, 
amiket újakkal pótolnak, változtatgatják az osztálystruktúrákat.
A SuSE 9.0-ás gépemen (GTK-2.2.3) 
pl. hiányzik többek között a GtkCellLayout interfész
(amibe átkerült a GtkTreeViewColumn osztály funkcióinak egy része),
valamint hiányzik a GtkComboBox osztály (ami pótolja a depracteddé
vált GtkCombo-t). Kérdés, hogy érdemes-e backportolni a CCCGTK-t
régebbi GTK-ra. Windowson is van GTK, de nem kísérleteztem vele.



\section{Névtér konvenció}

Vegyük példaként a \verb!GtkWindow! osztályt. 
Ehhez az osztályhoz létezik egy sor (jónéhány) C-ből hívható
API függvény, amik így néznek ki (példaként csak néhányat veszünk):

\begin{verbatim}
    //ez C program
    gtk_window_new(type); //konstruktor
    gtk_window_set_title(GTK_WINDOW(self),title); //metódus
    gtk_window_get_title(GTK_WINDOW(self)); //metódus
    gtk_window_set_position(GTK_WINDOW(self),position); //metódus
\end{verbatim}

A CCC-ben minden GTK osztályhoz egy kétszintű névteret
társítunk, a GtkWindow osztályhoz a \verb!gtk.window! névteret.
Az osztályhoz tartozó függvényeket ebbe a névtérbe helyezzük, pl.:

\begin{verbatim}
    //ez CCC program
    gtk.window.new(type) //konstruktor
    gtk.window.set_title(self,title) //metódus
    gtk.window.get_title(self) //metódus
    gtk.window.set_position(self,position) //metódus
\end{verbatim}
Ez a névkonvenció világosan kifejezi,
hogy melyik osztály eljárásairól van szó,
világosabban, mint a C-ben használt nevek.


\section{Függvényhívási szint}

A CCCGTK csatoló az előbbi módon kivezeti a GTK API-t
CCC szintre. Ezzel már tudunk programozni, pl.:
\begin{verbatim}
    window:=gtk.window.new() //default: toplevel
    gtk.window.set_title(window,"GtkListStore Minimal Demo")
    gtk.window.set_default_size(window, 280, 250)
\end{verbatim}
Az új window-nak beállítottuk a címsorát és a méretét.
A GTK (egyszeres öröklődéses) objektumrendszerében 
a GtkWindow osztály egyik felmenője a GtkContainer osztály.
Utóbbinak van egy \verb!set_border_width! metódusa
a keret méretének beállítására. Ha konkrétan a mi előző
ablakunk keretét akarjuk beállítani, akkor ezt írjuk:
\begin{verbatim}
    gtk.container.set_border_width(window,8)
\end{verbatim}
A GtkContainer-ben
definiált \verb!set_border_width! metódus működik a 
GtkWindow dinamikus osztályú window-ra, 
mivel az egy GtkContainer leszármazott.
Ilyen egyszerűen működik a GTK-ban az öröklődés C szinten, illetve
a CCCGTK függvényhívási szintjén.


\section{Objektum szint}

Az előző pontban szereplő \verb!window! változó CCC típusa P, 
azaz egy pointer, ami közvetlenül a GTK struktúrára mutat.
A kényelmesebb és biztonságosabb használathoz a CCCGTK
a legtöbb GTK osztályhoz létrehoz egy burkoló CCC osztályt,
\begin{itemize}
\item aminek egyetlen adattagja az előbbi pointer,
\item rendelkezik a GTK-ban definiált metódusokkal,
\item leszármazás (öröklődés) terén pedig lemásolja
      a GTK eredeti osztályhierarchiáját.
\end{itemize}
A gyakorlatban ezen az objektumszinten fogjuk 
fejleszteni az alkalmazásainkat. Az előbbi példa
így néz ki objektumszinten:
\begin{verbatim}
    window:=gtkwindowNew()
    window:set_title("GtkListStore Minimal Demo")
    window:set_default_size(280, 250)
    window:signal_connect("destroy",{||gtk.main_quit()})
    window:set_border_width(8)
\end{verbatim}
Rövidebb, egyszerűbb és biztonságosabb kódot kaptunk.
Egyszerűbbet, mert a CCC az öröklődési viszonyokból ,,magától'' tudja,
hogy a GtkContainer osztály \verb!set_border_width! metódusát
kell alkalmazni a window-ra. Biztonságosabbat,
mert a CCC ellenőrzi, hogy van-e egyáltalán a window-nak ilyen
metódusa.

A két szint közötti összefüggés:
A mostani window változónk O (object) típusú.
Van neki egy \verb!gobject! nevű attribútuma,
és ez a \verb!window:gobject! éppen az előző példában
szereplő P típusú window változót tartalmazza. 
Természetesen a CCCGTK a metódusok paraméterezésében
is nyomon követi, hogy függvényhívási szinten, 
vagy objektumszinten történnek-e a dolgok.


\section{Callback függvények}

A callback függvények megvalósításának természetes módja
CCC-ben a kódblokk. Lássunk egy példát:

\begin{verbatim}
    ...
    hbox:pack_start(gtklabelNew("Second entry:"))
    hbox:pack_start(entry2:=gtkentryNew())
    entry2:signal_connect("activate",{|w|cb_activate(w,"Entry2")})
    ...
static function cb_activate(entry,entry_name)
    ? entry_name, entry:get_text
\end{verbatim}
A kódblokk nagyon rugalmas:
Mint a példa mutatja, tetszőleges adatot át tudunk adni 
általa a callback függvénynek.


\section{Interfészek}

A GTK egy egyszeres öröklődést és interfészeket
kezelő objektumrendszerre épül (a GObject-re). A CCC ezzel szemben 
többszörös öröklődést is kezel, nincs tehát szükségünk interfészekre,
ami csak szükségmegoldás az igazi többszörös öröklődéshez képest.
A GTK-beli interfészeknek a CCC-GTK-ban absztrakt osztályok
felelnek meg, amiből egyszerűen örökölnek az interfészt 
implementáló osztályok.


\section{Referenciaszámlálás, memóriakezelés}

A GTK az objektumai többségére referenciaszámlálást alkalmaz.
Ezzel kapcsolatban egyetlen szabályt kell megjegyezni:
A CCCGTK semmilyen módon nem avatkozik bele a GTK (GObject)
referenciaszámlálásába. A referenciaszámlálást befolyásoló
API-hoz akkor és csak akkor folyamodunk, ha arra a
programunk C-beli megfelelőjében is szükség volna
(és persze, ha tudjuk, hogy mit csinálunk.)

Ugyanez vonatkozik a memóriakezelésre.
A CCC pointerek által hivatkozott memóriaobjektumok
pontosan addig léteznek, mint a programunk C-beli
megfelelőjében. A memória felszabadításának szintén
ugyanazok a szabályai, a szabályok áthágásának
ugyanazok a következményei.

Szerencsére átlagos programokban elég jól működik
a referenciaszámlálás: Létrehozunk egy ablakot, 
abba beleteszünk néhány widgetet, azokba újabb widgeteket, \ldots,
minden alkatrész referenciát tart fenn az általa tartalmazott
komponensekre, így az egész megáll a saját lábán.
Végül a program kilép, és úgyis felszabadul az
összes memória. Ennél persze több kell, ha {\em sokáig futó\/} 
programot akarunk csinálni, pl. egy desktopot,
dehát az ilyet úgysem CCC-ben írjuk.

Egy típuspélda arra az esetre, 
amikor direkt kezeljük a memóriát:
\begin{verbatim}
static function dump(model)
local iter:=gtk.tree_iter.new()
local positioned:=model:get_iter_first(iter)
    while( positioned )
        ? model:get(iter)
        positioned:=model:iter_next(iter)
    end
    // clean up
    gtk.tree_iter.free(iter)
\end{verbatim}
A \verb!tree_iter! egy segédosztály, nincs burkoló osztálya.
Egy C programban az iter-t tipikusan nem new-val hoznánk létre,
hanem egyszerűen stack változóként, ami megszűnik,
amikor a rutin visszatér. CCC-ben tisztogatni kell,
ha nem akarjuk fogyasztani a memóriát.


\section{Dokumentáció}
A C és CCC API között  99\%-os megfelelés van
(figyelembe véve a névtér konvenciót, és az előbbi összefüggéseket, 
viszont figyelmen kívül hagyva az esetleges hiányokat), 
ami lehetővé teszi, hogy a CCC programozás során használjuk a meglevő 
GTK dokumentációkat. Tehát mint annyiszor, most is az a helyzet,
hogy először a C szintű API-val kell tisztába jönni, után nem
lesz probléma a CCC-vel sem. Ezért nincs szükség önálló 
dokumentációra.

