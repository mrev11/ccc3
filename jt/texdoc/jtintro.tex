
\def\htmlfigext{\png}
 
\section{Áttekintés}

\subsection{Mi a Jáva Terminál?}
 
A Jáva Terminál egy {\em alkalmazásfüggetlen} megjelenítő program,
ami más programok részére biztosít GUI-t.
A terminál alapvető feladata dialogboxok  megjelenítése.
A Jáva Terminál dialogboxai Swing elemekből építkeznek:
\begin{itemize}
\item menü, 
\item push button, 
\item rádió button, 
\item check box, 
\item get (szerkesztő sablonnal felszerelt szövegmező),
\item list box, 
\item táblázat (browse), 
\item progress indikátor, 
\item ikonok, statikus szövegek.
\end{itemize}
A Jáva Swing használata meglehetősen szép külsőt eredményez,
ami ráadásul a Jáva L\&F bevetésével tovább alakítható.


\subsection{Hogyan működik?}
 
A Jáva terminál és a CCC alkalmazás XML üzenetekkel kommunikál.
A kommunikáció kezdetekor az alkalmazás elküldi a terminálnak
a dialogbox XML leírását. Ez tartalmazza a dialogboxban megjelenítendő
komponensek paraméterezését, a bennük levő adatok kezdőértékét.
A terminál megjeleníti és működteti a dialogboxot.

A felhasználó tevékenysége közben a terminált magára hagyjuk. 
A felhasználó azt ír a szövegmezőkbe, amit akar
(és amit a szerkesztő sablonok megengednek), 
kedvére választhat a listboxokban, 
klikkelhet a check boxokban, 
választhat a browse-ban, stb.
  
Üzenetváltás csak akkor történik, amikor a felhasználó valamilyen
akciót kezdeményez: választ egy menüpontot, vagy megnyom egy pushbuttont.
Ilyenkor a terminál elküldi a szervernek az akciót azonosító adatokat 
és a dialogbox teljes aktuális adattartalmát. 
Az akció hatására a szerver végrehajtja az üzleti logikát, 
és válaszként további adatokat és utasításokat küld a terminálnak,
például:
\begin{itemize}
  \item 
    új értéket ad egyes mezőknek,
  \item
    küld egy lapnyi új adatot a browse-nak,
  \item
    utasítja a dialogboxot a kilépésre, stb.
\end{itemize}
Ez az egyszerű funkcionalitás az ügyviteli alkalmazások széles 
körének megfelelő felhasználói felületet nyújt. 

Bár a Jáva Terminált eddig csak CCC szerverek megjelenítő
moduljaként használtuk, a terminál  más nyelvekből is igénybe vehető. 
A legtöbb ma használt programnyelvben, pl.\ C++-ban, Pythonban, Jávában (!) 
könnyen megvalósítható a szerver oldali API. Az API implementálhatóságának 
demonstrálása céljából készült a 
\href{http://ok.comfirm.hu/ccc2/download/jtpython.zip}{Python interfész}.
A felsorolásban a Jáva említése nem elírás, van értelme Jáva programok 
Jáva Terminálban történő megjelenítésének, ez  ugyanis lehetővé teszi, 
hogy a szerver és a terminál egymástól  földrajzilag távoli gépeken fusson. 


Az  API  egyszerűsége abból adódik,
hogy lényegében {\em nincs belső működése}. Az alkalmazás egyszerűen
megüzeni a terminálnak, hogy mit akar, és az utasításokat a terminál
végrehajtja. Az üzenetek XML-ben (azaz szövegesen)
jönnek-mennek, mindig el lehet olvasni őket,  semmi sem marad
titokban az alkalmazásfejlesztő előtt, az esetleges hibákat ezért
hamar meg lehet találni.
Persze  nem mindenki számára tesszük lehetővé az üzenetek olvasását. 
A Jáva Terminál és a CCC szerverek képesek SSL kommunikációra, 
így a használat nyilvános hálózaton is biztonságos.

 
\subsection{Hogy néz ki?}

\subsubsection{Fix pozícionálás}
 
Alapvetően kétféle GUI-t tudunk csinálni.
A fix pozícionálású ablak a
régi Clipper readmodallal működtetett
dialogbox utóda, a lehetőségek azonban tágabbak.
Szövegmezőkön (get) kívül az új jtdialog ablak tartalmazhat browset, 
list boxot, buttonokat, progress bart stb. 
A \ref{fixpos}.~ábrán látható GUI-t a jólbevált mask
programmal rajzoltam (pdialog.msk), ebből az msk2dlg programmal
generáltam kódot (pdialog.dlg), amit végül a main.prg demonstrációs
programban használtam fel.\footnote{
  Ismerem Cs.L. véleményét  arról, hogy a maskot
  fejlettebb eszközzel kellene helyettesíteni, azonban
  szükségem volt valamire, amivel gyorsan el lehet indulni,
  az msk2dlg-ben pedig nincs több, mint félnapos munka.
  A Glade-et kipróbáltam, de egyáltalán nem tetszett. 
  Szerintem a mask+msk2dlg-vel is messzire lehet jutni, 
  és később még mindig módunk lesz a GUI tervezőt kicserélni, 
  feltéve, hogy az XML interfészen  nem változtatunk.
}
 

\xfigure{fixpos}{Fix pozícionálás}{clip,height=8cm}


Az ablak mérete változtatható, de az a kontrollok pozíciójára
nincs hatással. A statikus labelek fix (kék) fontokkal vannak
kiírva. Minden komponens ezen fix font mérete által meghatározott
rácsra illeszkedik.
A get mezők Clipper-szerű szerkesztő sablonokkal (picture) vannak felszerelve.
A getek belsejében ugyanaz a fix font van, mint a statikus 
labelekben (amik azért mégsem annyira statikusak, mert
dinamikusan lehet változtatni a szövegüket, ikonjukat).
A buttonokhoz, rádió és check boxokhoz egyedi ikon adható meg.
Minden kontrollhoz tooltipet lehet rendelni.
 
\xfigure{fixposbrw}{Fix pozícionálás browse-zal}{clip,height=8cm} 
 
A \ref{fixposbrw}.~ábra mutatja, hogy a fix pozícionálású
dialogboxba browset is rakhatunk.
A browse lapozós, ez azt jelenti, hogy  az alkalmazás a tartalmat 
laponként adja a browsenak. A lapozásra szolgálnak a nyíl ikonos buttonok.
A szokásos navigációs eszközök (Pgdn, Pgup, scrollbár)  csak
a már megkapott lapon belül mozognak.
 

\subsubsection{Rugalmas pozícionálás}

Ezzel a típussal készül a menüző browse utóda. 
A menüben sok új lehetőség van, 
rakhatunk bele ikonokat, check boxokat, rádió buttonokat.
A browse fölé/alá tehetünk toolbart, amiben lehet
push, check, rádió button, list box, get, progressbar, label (bármi).
A Swingben a komponensek (pl. egy label) szövege HTML
formátumú is lehet, amivel különböző tördelést, fontokat, 
színezést lehet megvalósítani.

\xfigure{rugpos}{Rugalmas pozícionálás (box layout) }{clip,height=8cm} 

Középen egy JTable komponens van, ez kb. megfelel a Clipper
TBrowsenak.  A JTable cellái Clipper
kompatibilis szerkesztő sablonnal vannak formázva, és editálhatók.
A logikai értékek check boxként editálódnak.
Egyszerre több sort is ki lehet választani.
Az oszlopok sorrendje egérhúzogatással felcserélhető.
 
Az alsó státusz sor szintén egy toolbar,  ebben van egy getmező,
amibe számlaszám szerint kereső  stringet lehet beírni.
A toolbárok és a JTable függőleges sorrendjét az ablakhoz való
hozzáadásuk sorrendje határozza meg.
A toolbárok egérrel áthelyezhetők.

Megjegyzem, hogy a terminál támogatja (függőleges helyett) a vízszintes
pakkolást, és más komponenseket is lehetne egymás alá (mellé)
helyezni. A toolbárok által tartalmazott komponensek például
vízszintesen vannak egymás mellé rakva.
Az ablak CCC-beli leprogramozása nem igényel GUI tervező eszközt,
nincs szükség méretek megadására. 
A fenti GUI-t a main1.prg program hozta létre.


\subsection{Hogyan indítjuk a programokat?}

\subsubsection{A szerverek indítása SSL nélkül}

A \verb!jtlisten! program feladata  szerver programok indítása.
\begin{verbatim}
  jtlisten.exe [if:]port <command>
\end{verbatim}
A program figyel az opcionálisan megadott interfészen és a megadott 
porton. Ha a portra kapcsolódnak, akkor elindítja \verb!<command>!-ot úgy, 
hogy kiegészíti azt a  \verb!-jtsocket <sck>! opcióval,
ahol \verb!<sck>! az accept-ben létrejött socket.
A Jáva terminálos programok szokás szerint a \verb!-jtsocket <sck>! opcióban 
kapják meg azt az örökölt socketet, amin a terminállal kommunikálni lehet.
A \verb!<command>!-nak spawnvp-vel (UNIX-on fork plusz execvp-vel) 
indítható  filéspecifikációval kell kezdődnie.
 


\subsubsection{A szerverek indítása SSL-lel}
 
A CCC szerver programok semmit sem tudnak 
az SSL-ről. A plain socketen kommunikáló szerver 
és az SSL-en kommunikáló terminál között egy  ,,fordító''
réteg van, hasonlóan az ssh port forwardinghoz.
Tegyük fel, hogy \verb!<command>! olyan programindító parancs,
amit \verb!jtlisten! el tud indítani az előző pontban
tárgyalt módon. Akkor a 
\begin{verbatim}
  jtlisten.exe [if:]port sslforward.exe <command>
\end{verbatim}
parancs ugyanúgy elindítja  \verb!<command>!-ot, de még
egy SSL fordító réteget is közbeiktat.


\subsubsection{Több szerver indítása xstart-tal}

Az xstart a CCC 2.x rendszer tools könyvtárában található segédprogram, 
amit a UNIX-on ismert inetd-hez hasonló módon lehet használni.
Az xstart-ot egy XML szintaktikájú szövegfilével konfiguráljuk.

\begin{verbatim}
<xstart>
<item>
    <name>Program SSL nélkül</name>
    <host>localhost</host>
    <port>46000</port>
    <workdir></workdir>
    <command>program.exe -jtsocket $(SOCKET)</command>
</item>
<item>
    <name>Program SSL-lel</name>
    <host>localhost</host>
    <port>46001</port>
    <workdir></workdir>
    <command>sslforward.exe program.exe -jtsocket $(SOCKET)</command>
</item>
</xstart>
\end{verbatim}
 
A konfigurációs filében host:port címekhez rendelünk szolgáltatásokat.
A program figyel a megadott portokon, és ha valamelyik portra
kapcsolódnak, akkor átvált a \verb!<workdir>! tagban megadott directoryba, 
és ott elindítja a \verb!<command>! tagban megadott szolgáltatást.
A \verb!<command>!-nak most is spawnvp-vel vagy execvp-vel
indítható filéspecifikációval kell kezdődnie.
A parancshoz xstart nem fűzi hozzá a -jtsocket opciót,
viszont helyettesíti a \verb!$(SOCKET)! makrót az accept-ben
kapott sockettel. Mint a példában látszik az sslforward program
most is használható SSL fordító réteg közbeiktatására.


\subsubsection{A terminál indítása}

\begin{verbatim}
  java -jar jterminal.jar <host> <port> [ssl]
\end{verbatim}

A terminál kapcsolódni fog a \verb!<host>!-ban  megadott gép
\verb!<port>!-jára. Ha az opcionális \verb!ssl!  paraméter
is meg van adva, akkor a terminál SSL-en fog kommunikálni.
Az \verb!ssl! paramétert a szerverrel összhangban kell használni,
azaz akkor és csak akkor kapcsoljuk be a terminálban az SSL-t,
ha a szerver is SSL-en kommunikál.
