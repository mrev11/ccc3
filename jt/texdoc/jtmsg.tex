
\section{Üzenetek}

Egyszer talán itt egy komplett referencia áll majd,
most azonban meg kell elégednem az üzenetek egyszerű felsorolásával,
ez legalább felhívja a figyelmet bizonyos funkciók létezésére.
Emellett megpróbálok rávilágítani a legfontosabb összefüggésekre, 
megkönnyítve ezzel a kódban való tájékozódást.  

\subsection{Az üzenetrendszer alapjai}

A szerver (CCC) és a terminál (Jáva) között SSL kapcsolat van,
és a felek XML formátumú üzenetekkel kommunikálnak egymással.

\begin{quote}\small
    Megjegyzés. Az 1.4.1-es Jáva helyből tartalmazza a szükséges 
    XML elemző csomagot és az SSL kiegészítést, az 1.3.x-es Jávához 
    ezeket külön kell telepíteni.  Régebbi Jávával nem kísérleteztem.
\end{quote}

A szerver először egy \verb!<jtdialog>! üzenetet küld a terminálnak,
ami egy dialogbox teljes leírását tartalmazza. A dialogboxot a terminál
megjeleníti és működteti. 

A terminál csak akkor üzen a szervernek, ha a felhasználó valamilyen
{\em akciót} vált ki. Akció akkor keletkezik, ha kiválasztanak egy
menüt, megnyomnak egy push buttont, entert ütnek egy get mezőben, stb.
(Szabály: a menük és push buttonok mindig jelentik az akciót, a többi
kontroll csak akkor, ha azt a programban előírtuk.)
Az akciót a terminál egy \verb!<action>! üzenettel jelenti.
Az üzenet tartalmazza az akciót kiváltó kontroll nevét (azonosítóját),
és  a dialogboxban levő  összes változtatható értékű kontroll
aktuális értékét. Az \verb!<action>! üzenet elküldése után
a dialogbox \verb!setEnabled(false)! állapotba helyezi magát, 
és vár a szerver válaszára (a reakcióra).

A szerver eközben egy messageloop-ban várakozik a terminál
üzeneteire. Amikor megérkezik az akció, azt a \verb!_dlgbase_getmessage!
függvény feldolgozza, és az akcióban jelentett új tartalmakat
betölti a dialogboxot reprezentáló szerver oldali objektumba.

Az így frissített (szerver oldali) dialog objektummal a szerver azt csinál, 
amit akar, pl. számításokat végez a kontrollok tartalmával, 
vagy éppen új értéket ad egyes kontrolloknak, ez az üzleti logika. 

Miután az üzleti logika elvégezte a dolgát,
de még mielőtt a messageloop a következő körre fordulna,
\verb!_dlgbase_getmessage! függvény egy \verb!<reaction>!
üzenetet küld a terminálnak. A reaction üzenetben egyúttal
elküldjük a (terminál oldali) dialogboxnak a megváltozott 
kontrollok új értékét.

A terminál a reaction üzenetben kapott új tartalmakat betölti
a kontrollokba, majd \verb!setEnabled(true)! állapotba helyezi
magát.

Összefoglalva:

\begin{enumerate}
\item 
   A szerver a \verb!<jtdialog>! üzenetet küld, ha meg akar jeleníteni
   egy dialogboxot.
\item
   A terminál \verb!<action>! üzenetet küld, amikor valamilyen
   akció történt. Az üzenet tartalmazza a kontrollok aktuális
   tartalmát. Az üzenet után a terminál vár a \verb!<reaction>!
   üzenetre.
\item
   Az akció feldolgozása után a szerver \verb!<reaction>! 
   üzenetet küld a terminálnak, amiben küldi a megváltozott
   kontrollok új tartalmát.
\item 
   A \verb!<reaction>!  üzenetre a terminál betölti a dialogboxba
   az új értékeket, és folytatja az akciónál felfüggesztett működést.
\end{enumerate}

A fentieken kívül vannak más üzenetek is, de az egész rendszer
vázát a fenti üzenetmechanizmus képezi, aminek megértése elengedhetetlen
rendszer használatához.
 

\subsection{A terminál üzenetei}


\subsubsection{{\tt <action>} üzenet}
Action üzenet akkor keletkezik, ha a felhasználó
\begin{itemize}
  \item 
      kiválaszt egy menüt,
  \item 
      megnyom egy pushbuttont,
  \item 
      Jáva értelemben vett akciót hajt végre egy olyan kontrollban, 
      aminek a valid attribútuma true, 
      vagy  akcióblokkal rendelkezik.
\end{itemize}
A valid attribútum elnevezése a régi Clipperre utal, ahol a
valid bővítményben lehetett megadni a get mezők postblockját.
Most ilyen postblock nincs, csak annyit mondhatunk a terminálnak,
hogyha az adott kontrollban akció történik, akkor arról azonnal
értesülni akarunk. 

\begin{verbatim}
public void action(xmlcomponent source)
{
    if( !actionenabled )
    {
        return;
    }

    String x="<action dialogid=\""+dialogid+"\">";
    x+="<source>"+source.getName()+"</source>";

    for( int i=0; i<varlist.size(); i++ )
    {
        xmlcomponent c=(xmlcomponent)varlist.get(i); 
        
        try
        {
            x+=c.xmlget();
        }
        catch( pfexception e )
        {
            if( !source.isEscape() )
            {
                //szólni kell a kontrolloknak:
                //amit küldeni akartak, nem ment el,
                //erre való az xmlreset()
 
                for( int k=0; k<i; k++  )
                {
                    ((xmlcomponent)varlist.get(k)).xmlreset();
                }

                jtalert a=new jtalert(jterm);
                a.message="Hibás adatbevitel: "+c.getName();
                a.parent=this.wnd;
                a.send=false;
                //a.beep=false;
                a.type=JOptionPane.ERROR_MESSAGE;
                a.run();
                e.getField().requestFocus();
                return;
            }
        }
    }
    x+="</action>";
    actionenabled=false;
    jterm.send(x);
}
\end{verbatim}

Az action üzenetet a jtdialog objektum készíti (és küldi) miután értesült
róla, hogy valamelyik kontrollban jelenteni való akció történt.
Mint a kódban látjuk, csak enabled állapotban küldünk akciót.
Mivel az akció eredményére várva a dialogbox disabled állapotban van,
ez egyúttal azt jelenti, hogy az akciók nem skatulyázódnak egymásba.
A \verb!<source>! tag tartalmazza az akciót kiváltó kontroll azonosítóját.
A jtdialog objektum körbekérdezi a (változtatható értékű) kontrollokat,
hogy adják meg az aktuális állapotukat (xmlget). Egyes kontrollok 
erre kivétel dobásával reagálhatnak, ha a kitöltésük nem megfelelő.
Ezt a kivételt nem vesszük figyelembe, ha az akció forrásának
escape attribútuma true. Ennek az az értelme, hogy egy ,,Kilép''
gombbal ki lehessen lépni akkor is, ha vannak érvénytelen
állapotú kontrollok.



\subsubsection{{\tt <destroy>} üzenet}
Destroy üzenet keletkezik, ha a felhasználó becsukja
a terminál valamelyik ablakát.
 
\subsubsection{Egyéb üzenetek}
Az itt felsorolt üzenetek az alkalmazás által kezdeményezett
műveletre adott válaszként fordulnak elő:
\verb!<jtversion>!,
\verb!<alert>!, 
\verb!<memoedit>!, 
\verb!<dirlist>!, 
\verb!<filechooser>!, 
\verb!<uploadbegin>!, 
\verb!<uploadend>!, 
\verb!<uploaderror>!, 
\verb!<download>!, 
\verb!<makedir>!, 
\verb!<makedirs>!, 
\verb!<delete>!, 
\verb!<exists>!, 
\verb!<isfile>!, 
\verb!<isdirectory>!, 
\verb!<rename>!.



\subsection{A szerveralkalmazás üzenetei}

\subsubsection{{\tt<jtdialog>} üzenet}

A szerver oldali jtdialog objektum xmlout metódusa készíti,
és a show metódus küldi a {\tt<jtdialog>} üzenetet. Ebben a 
szerver a dialogbox teljes leírását adja át a terminálnak.
 
\begin{verbatim}
static function _jtdialog_xmlout(this)

local x,n,c,g:={},grp,i

    x:="<jtdialog"
    x+=ATTRI("top",this:top)
    x+=ATTRI("left",this:left)
    x+=ATTRI("bottom",this:bottom)
    x+=ATTRI("right",this:right)
    x+=ATTR("name",this:name)
    x+=ATTR("dialogid",this:dialogid)
    x+=ATTR("pid",this:pid)
    x+=">"+EOL
    x+="<caption>"+cdataif(this:text)+"</caption>"+EOL
 
    if( this:layout!=NIL )
        x+="<layout>"+this:layout+"</layout>"+EOL
    end

    for n:=1 to len(this:itemlist)
        c:=this:itemlist[n] 
        if( c:classname=="jtradio" .and. c:group!=NIL )
        
            grp:=eval(c:group)

            for i:=1 to len(g)
                if( grp==g[i] )
                    exit
                end
            next

            if( i>len(g) )
                aadd(g,grp)
                x+="<jtradiogroup>"+EOL
                for i:=1 to len(grp)
                    x+=grp[i]:xmlout+EOL
                next
                x+="</jtradiogroup>"+EOL
            end

        else
            x+=c:xmlout+EOL
        end
    next

    x+="</jtdialog>"

    return x
\end{verbatim}

A metódus bejárja azt a fát, amibe a menük, kontrollok szerveződnek.
Minden komponensnek meghívódik az xmlout metódusa, ezzel minden komponens 
hozzáadja a saját járulékát a dialogboxot leíró XML dokumentumhoz.


 
\subsubsection{{\tt<reaction>} üzenet}

A szerver a reaction üzenettel jelzi a terminálnak,
hogy annak újból enabled állapotba kell helyeznie magát,
egyúttal elküldi a megváltozott kontrollok új állapotát.
 
\begin{verbatim}
static function _jtdialog_response(this)
local n,v,x:=""
    for n:=1 to len(this:varlist)
        v:=this:varlist[n]
        if( v:changed )
            x+="<"+v:name+">"+v:xmlget+"</"+v:name+">" 
            v:savestate
        end
    next
    if( empty(x) )
        x:='<reaction dialogid="'+this:dialogid+'"/>'
    else
        x:='<reaction dialogid="'+this:dialogid+'"><set>'+x+'</set></reaction>'
    end
    this:send(x)
    this:mustreact:=.f.
    return NIL
\end{verbatim}
 
Mint látjuk, a szerver oldali dialogbox objektum körbekérdezi
a megváltozott állapotú kontrollokat (xmlget), amire azok 
megadják az új értéküket.


\subsubsection{{\tt<jtexit>} üzenet}
Utasítja a terminált a kilépésre.
 

\subsubsection{{\tt<jtclose>} üzenet}
Utasítja a terminált a legfelső ablak bezárására.
 

\subsubsection{{\tt<jtalert>} üzenet}
A Clipperből ismert alertnek megfelelő üzenet.
A Jáva lehetővé teszi, hogy a messagebox szövegét HTML
jelöléseket használva különféle stílusokkal lássuk el.

\subsubsection{{\tt<jtdirlist>} üzenet}
A Clipperből ismert directory függvény kicsit okosabb
(reguláris kifejezéseket támogató) megvalósítása.  
Ezzel a szerver alkalmazás körül tud nézni a terminál gépen.


\subsubsection{{\tt<jtupload>} üzenet}
Filé másolás a terminálról a szerverre.

\subsubsection{{\tt<jtdownload>} üzenet}
Filé másolás a szerverről a terminálra.
 
\subsubsection{{\tt<jtfilechooser>} üzenet}
Filék és directoryk interaktív kiválasztása a terminálon.

\subsubsection{{\tt<jtfileutil>} üzenetcsoport}
Filé műveletek:
\begin{description}
\item[{\tt<makedir>}] directory létrehozása a terminálon,
\item[{\tt<makedirs>}] többszörös mélységű directory létrehozása,
\item[{\tt<delete>}] filé/directory törlése,
\item[{\tt<exists>}] létezeik-e a megadott filé/directory,
\item[{\tt<isfile>}] filé-e a megadott directory bejegyzés,
\item[{\tt<isdirectory>}] directory-e a megadott directory bejegyzés, 
\item[{\tt<rename>}] filé/directory átnevezése (mozgatása) a terminálon.
\end{description}

\subsubsection{{\tt<jtmessage>} üzenetcsoport}
Az üzenetben meghatározott kontrollnak kell továbbítani 
egy kontroll-specifikus üzenetet, pl. browse lapozás, progress bar léptetés,
ikon, felirat, tooltip változtatás.
 
\begin{verbatim}
static function _jtelem_changeenabled(this,v)
local x
    if( v!=NIL )
        this:enabled:=v
    end
    x:='<jtmessage'
    x+=ATTR("pid",alltrim(str(getpid())))
    x+=ATTR("dialogid",this:dialogid)
    x+='>'
    x+="<control>"+this:name+"</control>"
    x+="<enabled>"
    x+=if(this:enabled,"true","false")
    x+="</enabled>"
    x+="</jtmessage>"
    this:send(x)
    return NIL
\end{verbatim}
    
A példa érzékelteti, hogy mire számíthatunk a jtmessage
üzenetekben. Ez itt egy komponenes enabled attribútumát állítja át
a dialogbox működése közben.
    
 