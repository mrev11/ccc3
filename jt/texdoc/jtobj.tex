
\section{Szerver oldali objektumok} 

Sajnos megint csak fő összefüggések leírására kell szorítkoznom,
ez legalább megkönnyíti a kód olvasását.

\subsection{A kontrollok értékéről}

Az akció és reakció üzenetekben a kontrollok ,,értéke'' utazik
a termináltól a szerverhez és vissza. Alább felsoroljuk, hogy az egyes
kontrolloknál mit értünk érték alatt:

\begin{description}
\item[checkbox]      logikai érték, a gomb állapota, 
\item[combo box]     a kiválasztott index (1-től számozódik),
\item[get mező]      a getben editált változó, melynek típusát a picture-ből állapítja meg a rendszer,
\item[rádió button]  logikai érték, a gomb állapota,
\item[table]         a kiválasztott sorok indexéből képzett array,
\item[tabpane]       a kiválasztott fül indexe (1-től számozódik).
\end{description}

Az értéket (a régi Clipper get objektumának mintájára) a varget metódussal
lehet lekérdezni, és a varput metódussal lehet beállítani. 
A varget nem tévesztendő össze a kontroll szövegével (text),
vagy az érték XML formátumú alakjával (xmlget), bár egyes esetekben
azok megegyezhetnek.  A felsorolásban nem szereplő kontrollok esetén a
\verb!varget! metódus azonos a \verb!text! metódussal, ahogy az a jtelem 
osztályból öröklődik.


A kontrolloknak az értékükön kívül természetesen van egy sor
egyéb jellemzője, ilyen pl.\ egy checkbox felirata, ikonja, tooltipje.
Ezeknek az állítgatása lehetséges ugyan, de csak ritkán fordul elő
a programban, tipikusan csak egyszer, a kontroll létrehozásakor.


\subsection{A jtelem osztály}

A szerver oldalon az osztályok többsége a jtelem absztrakt osztályból
származik, legelőször tehát ezt kell tanulmányozni.

\paragraph{Belső használatra}

\begin{description}
\item[setdialogid]
    Összekapcsolja a kontrollt és a dialogboxot. 
\item[dialogid] 
    Ehhez a dialogboxhoz tartozik a kontroll.
\item[itemlist] 
    A beágyazott kontrollok listája. 
    Jelenleg a jtmenu, jtpanel, jtsplitpane, jttabpane, 
    jttoolbar kontrolloknak  van nem üres itemlist-je.
\item[xmlout] 
    A kontrollok ezzel adják meg saját XML leírásukat.
\item[xmlname] 
    A kontroll osztályának neve.  
\item[xmladd] 
    Egyes kontrollok xmlout metódusa úgy működik, hogy
    meghagyják ugyan a jtelemtől örökölt xmlout-ot,
    viszont felüldefiniálják xmladd-ot, amit a jtelem-beli 
    xmlout mindig meghív. Ezzel a kontrollok speciális kiegészítéseket
    fűzhetnek az általános jtelem-beli xmlout-hoz.
\item[xmlput]    
    Beállítja egy kontroll értékét. Az új értéket szövegesen, 
    XML formátumban kell megadni.
\item[xmlget]
    Kiolvassa egy kontroll értékét. Az értéket szövegesen,
    XML formátumban kapjuk.
\item[savestate]    
    Megjegyzi a kontroll aktuális értékét laststate-ben.
\item[laststate]
    A kontroll terminálnak elküldött utolsó értékét tartalmazza.
\item[changed]
    Az aktuális érték és laststate összehasonlításával megállapítja,
    hogy változott-e a kontroll értéke az utolsó küldés óta. Csak a 
    megváltozott értékeket kell küldeni a terminálnak.
\end{description}
 



\paragraph{Alkalmazási programoknak}

\begin{description}
\item[varget]
    Kiolvassa a kontroll értékét. Az értéket Clipper típusú (C, N, D, L, A) 
    változóként kapjuk.
\item[varput]
    Értéket ad egy kontrollnak. Az új értéket Clipper változóként kell átadni.
\item[top]
    A kontroll tetejének koordinátája.
\item[left]  
    A kontroll bal szélének koordinátája.
\item[bottom]  
    A kontroll aljának koordinátája.
\item[right]  
    A kontroll jobb szélének koordinátája.
\item[halign]
    Előírja, hogy a kontroll belsejében levő szöveg merre legyen
    igazítva vízszintesen.
\item[valign]
    Előírja, hogy a kontroll belsejében levő szöveg merre legyen
    igazítva függőlegesen.
\item[alignx]
    Előírja, hogy a kontroll az alatta/fölötte levő kontrollokhoz
    melyik részével illeszkedjen víszintesen (jobb/bal szélével, közepével).
\item[aligny]
    Előírja, hogy a kontroll az mellette levő kontrollokhoz
    melyik részével illeszkedjen függőlegesen (aljával, tetejével, közepével).
\item[name]   
    A kontroll azonosítója, egyedinek kell lennie.
\item[text]  
    A kontroll szövege. Egyes kontrolloknál ez egyben a kontroll 
    értéke is.
\item[tooltip]    
    A kontroll tooltipjének a szövege.
\item[icon] 
    Egyes kontrollokhoz (label, button, menü) ikon rendelhető.
    Az ikonnak a jterminal.jar-beli készletben létező jpeg, gif,
    vagy png filének kell lennie.
\item[image] 
    Olyan kép, amit futás közben a szerverről küldünk át a terminálnak.
\item[border] 
    A defaulttól eltérő keret adható itt meg.
\item[valid]
    Egy flag, ami előírja, hogy a kontrollal történt Jáva akciót
    jelenteni kell. A menük és push buttonok akcióit mindig jelenti
    a terminál.
\item[escape]  
    Ezzel a flaggel jelölt kontrollok akciójánál nem kell vizsgálni,
    hogy a kontrollok kitöltése megfelelő-e.
\item[enabled]   
    Ez a flag letiltja/engedélyezi a kontrollt.
\item[focusable]  
    Adható-e fókusz a kontrollra?
\item[actionblock]
    Ha egy kontrollnak van akcióblokkja,
    akkor azt a getmessage automatikusan végrehajtja,
    ha a kontrollból akció érkezik. Az ilyeneknek a
    valid attribútumával sem kell foglalkozni.
\item[changetext]
    Menet közben megváltoztatja a kontroll szövegét, 
    pl. egy checkbox feliratát, amit tipikus esetben csak
    a dialogbox inicializálásakor szoktunk megadni.
    (Azonnal küldi az üzenetet.)
\item[changetooltip]
    Menet közben megváltoztatja a kontroll tooltipjét. (Azonnal küldi az üzenetet.) 
\item[changeicon]
    Menet közben megváltoztatja a kontroll ikonját. (Azonnal küldi az üzenetet.) 
\item[changeimage]
    Menet közben új bináris képet küld egy kontrollnak. (Azonnal küldi az üzenetet.) 
\item[changeenabled]
    Menet közben negváltoztatja kontroll engedélyezettségét. (Azonnal küldi az üzenetet.) 
\end{description}



\subsection{Grafikus komponens osztályok jtlib-ben} 

A felsorolás képet ad a Jáva Terminál lehetőségeiről.
Az osztályok valójában a terminálon létrejövő Swing objektumok
szerver oldali reprezentációi, és többségük az előbb tárgyalt jtelem 
osztály leszármazottja.

\begin{description}
\item[jtalert]
    Clipper messageboxok létrehozásához.
\item[jtbrowse]
    Funkcionálisan a Clipper TBrowse utóda 
    (és éppenséggel a TBrowse osztály leszármazottja),
    helyette a jttable, jtbrowsearray, jtbrowsetable
    objektumok használata ajánlott.
\item[jtbrowsearray]
    A jttable osztály leszármazottja, ami fel van készítve
    array-k browseolására.
\item[jtbrowsetable]
    A jttable osztály leszármazottja, ami fel van készítve
    táblaobjektumok browseolására.
\item[jtcheck]
    Checkbox.
\item[jtcombo]
    Lista előre meghatározott elemekkel, az elemekből egyet lehet kiválasztani.
\item[jtdialog]
    Legfelső szintű grafikus elem, maga a dialogbox
\item[jtfilechooser]
    A szokásos filé kiválasztó dialogbox. A terminál filérendszerében
    lehet választani.
\item[jtget]
    Szövegmező, ami a Jáva szövegmezőnél sokkal intelligensebb,
    mivel Clipper-szerű picture-rel (formázó/szerkesztő sablonnal) van
    ellátva.
\item[jtglue]
    Olyan grafikus elem, aminek akármilyen mérete lehet,
    így alkalmas hézagok  kitöltésére.
\item[jtlabel]
    Statikus felirat.
\item[jtmenu]
    Legördülő menü. A menü többszintű lesz,
    ha a menüelemek helyére újabb menüket teszünk.
\item[jtmenuitem]
    Egy menüelem (sor) a legördülő menüben.
\item[jtmenusep]
    Elválasztó vonal (szeparátor) a menüben.
\item[jtmenucheck]
    Checkboxként működő menüitem.
\item[jtmenuradio]
    Rádió buttonként működő menüitem.
\item[jtpanel]
    Összetett grafikus elem, ami további elemeket tartamazhat.
\item[jtprogbar]
    Progressz indikátor.
\item[jtpush]
    Push button (egyszerű nyomógomb).
\item[jtradio]
    Rádió button.
\item[jttable]
    A jtbrowse olyan változata, ami nem örököl a Clipperes
    TBrowse-ból, így nem gyökerezik bele a régi könyvtárakba.
    Az egyszerűség ára, hogy a browse nem jeleníthető meg lokálisan,
    hanem csak a terminálban.
\item[jttabpane]
    Fülekkel ellátott panelek, amik egymást takarják, és
    a füllel lehet kiválasztani, hogy melyik kerüljön felülre.
\item[jttoolbar]
    A panel egy olyan speciális formája, amire bizonyos 
    egérműveleteket támogat a Jáva.
\item[jtpassw]
    Jelszavak bekérésére specializált szövegmező. Egyrészt
    "*"-ként jeleníti meg a begépelt karaktereket, másrészt nem
    a jelszót küldi vissza eredményként, hanem annak MD5 hash kódját.
\end{description}
 
Lesznek később más kontrollok is, pl. famegjelenítésre.
Jelenleg nincs még megoldva nagy szövegállományok terminál oldali
megjelenítése, ahol a ,,nagy'' akkorát jelent, amit már nem lehet
egyben átküldeni a hálózaton.

 