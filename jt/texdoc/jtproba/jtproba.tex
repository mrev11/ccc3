
\section*{Jáva Webstart próba alkalmazás}

Először telepíteni kell gépünkre a Jáva 1.4.2-et,
ehhez legjobb, ha felkeressük a
\href{http://java.sun.com/j2se/1.4.2/download.html}{Sun letöltő oldalát}.
Ha nem akarunk Jávában programozni, akkor elég a JRE, 
egyébként válasszuk az SDK-t.

\begin{quote}\small
A Jáva Terminál működik korábbi Jávával is,
ekkor azonban pótlólag be kell szerezni, és külön installálni kell
\begin{itemize}
  \item az XML elemző csomagot,
  \item az SSL csomagot és
  \item a Jáva Webstartot,
\end{itemize}
amiket az újabb Jáva kiadás már az alapcsomagban tartalmaz.
A régi Jávákkal azonban már nem fogok tesztelni, 
ezért mindenképpen a legújabb  Jáva ajánlott. 
\end{quote}

Windowson  a Jáva Webstart a standard csomagból automatikusan 
installálódik. Linuxon a csomagban benne van ugyan a Jáva Webstart, 
de azt egy külön paranccsal kell installálni. 
Így is megvan 5 percen belül.

A böngészőt úgy kell konfigurálni, 
hogy az \verb!application/x-java-jnlp-file! MIME type-ra, 
indítsa el a jws-t. A jws telepítője az ismert böngészőkkel
ezt automatikusan megteszi, ha mégsem sikerül neki, 
akkor utólag kell elvégezni a beállításokat.

A webszervert  úgy kell konfigurálni, hogy a jnlp kiterjesztésű 
filéket \verb!application/x-java-jnlp-file! MIME type-pal küldje.
A ComFirm webszerverén ez megtörtént. Ha egy idegen ISP 
webszerverén akarunk Jáva Webstart alkalmazást telepíteni, 
akkor meg kell egyezni az ottani rendszergazdával.



Mindezek után próbáljuk ki az alábbi linkeket:
\begin{itemize}
\item
  \href{http://ok.comfirm.hu/jnlp/jtproba.jnlp}{Jáva Webstart próba alkalmazás}\par
\item
  \href{http://ok.comfirm.hu/jnlp/cccdown.jnlp}{CCC csomagok letöltése}\par 
\item
  \href{http://ok.comfirm.hu/jnlp/browse.jnlp}{CCC böngésző}\par 
\item
  \href{http://ok.comfirm.hu/jnlp/ekonto.jnlp}{E-Kontó Demó}\par 
\end{itemize}

Az utóbbi demó programba \verb!vizmuvekrt! felhasználói névvel, 
a \verb!Jelmondat-123! jelszóval lehet bejelentkezni.

Ha minden rendben van, akkor  a böngésző
letölt  egy  kicsi, jnlp kiterjesztésű, XML szintaktikájú paraméterfilét.
A MIME type alapján megállapítja, hogy el kell indítania a jws-t.
A jws a paraméterfiléből kitalálja, hogy mely Jáva programokat
kell letölteni, letölti és elindítja azokat (esetünkben a jterminal.jar-t)
a jnlp filében megadott módon. Ezután a jws cache-eli az alkalmazást,
ami a jövőben letöltés nélkül is indítható. Emellett a jws
figyeli, hogy van-e az alkalmazásnak frissítése, és ha szükséges,
automatikusan elvégzi a frissítést. Szerintem ez egy jó dolog.

A jws által letöltött alkalmazás nem appletként, 
hanem teljes jogkörű, önálló programként fut. 
Indítás előtt a jws ellenőrzi a letöltött jar filék 
digitális aláírását, és első alkalommal óvatosságra int.


