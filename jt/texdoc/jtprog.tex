
\section{Szerver oldali programozás} 
 
A CCC szerverprogramokban a jtlib könyvtár osztályait használjuk 
a megjelenítéshez. A könyvtár objektumai a szerver oldalon reprezentálják 
a terminál által működtetett dialogboxot, egyúttal támogatják a két fél 
közti adatcserét. A teljes könyvtár szisztematikus leírására nincs időm, 
ezért csak néhány kiragadott kérdést tárgyalok.

\subsection{Messageloop}

Vizsgáljuk meg részletesen a main1.prg-ben található message loop-ot:

\begin{verbatim}
function msgloop(dlg)

local msg

    dlg:show  
 
    while( NIL!=(msg:=dlg:getmessage) ) 
    
        //az alábbi funkciókat lehetne
        //a kontrollok blockjába is rakni
        
        if( msg=="x"  ) //x button
            quit
 
        elseif( msg=="ok"  ) //ok button
            dlg:close

        elseif( msg=="menuitem1"  ) 
            ? "alert:", alert("Van, aki forrón szereti!",;
                             {"Válasz-1","Válasz-22","Válasz-333"})

        elseif( msg=="menuitem3"  ) 
            msgloop( makedlg(dlg:caption+"@") )

        elseif( msg=="search"  ) 

            //név szerint kikeresi a kontrollt,
            //kiolvassa a tartalmát, azzal végrehajtja a seek-et,
            //és a kapott pozíció alatti lapot megjeleníti
            //tabSeek(table,dlg:getcontrol(msg):varget)
            //dlg:getcontrol("szamla"):pagecurrent 
            
            tabSeek(table,dlg:var:search:varget)
            dlg:var:szamla:pagecurrent
 
        end
    end
    return NIL
\end{verbatim}

Először a show metódus meghívásával megjelenítjük a dialogboxot,
mint tudjuk, ez küld egy \verb!<jtdialog>! üzenetet a terminálnak. 
Ezt követően várunk a terminál \verb!<action>! üzenetére a 
\verb!dlg:getmessage! hívásban.

Ha a getmessage metódus NIL-t ad, az azt jelenti,
hogy a dialogbox ablakát becsukták, ez esetben a ciklus
befejeződik. Ha getmessage valódi akciót jelent, 
akkor annak a kontrollnak az azonosítójával tér vissza,
ami az akciót indította.

Ha az ,,x'' (Kilép) gombot nyomták meg, akkor kilépünk a programból.
Lehetne itt vacakolni egy előzőleg elküldött \verb!<jtexit>!
üzenetettel, de ez nem igazán fontos. A terminál magától
is észre fogja venni, hogy a socket kapcsolat lezáródott, 
és meg fogja tenni a szükséges lépéseket. 
Ezért az egyszerű quit is megfelelő.

Az ,,ok'' gomb megnyomására nem az egész program lép ki,
csak bezárjuk a terminál legfelső ablakát. Persze, ha éppen csak
egy ablakunk volt, akkor mégiscsak befejeződik a program.


Ha kiválasztják a ,,menuitem1''-et, elindítunk egy alertet,
ha a ,,menuitem3''-at, akkor pedig létrehozunk egy új dialogbox ablakot
a korábbihoz hasonló tartalommal. 

Az utolsó ág a legtanulságosabb. A ,,search'' annak a getmezőnek
a neve, ami az alsó toolbárban van, és célja, hogy számlaszám
szerint kereső mintát lehessen benne megadni.
Általánosságban, ha egy getben entert ütnek, 
akkor Jáva értelemben vett akció keletkezik. 
Mivel a mi ,,search'' getmezőnk valid attribútuma true-ra van állítva,
az akciót a terminál jelenti, ezért kapjuk meg a messaget. 
A getmessage metódus visszatérése után a getbe gépelt kereső 
minta már beíródott a getmezőt a szerver oldalon 
reprezenátáló jtget objektumba, ahonnan a varget metódussal
lehet azt megkapni. Van azonban egy probléma: hogyan szerezzük 
meg magát a jtget objektumot? Ez a probléma a moduláris programozás
eredményeképpen lép fel, ui. az a kód, amiben a jtget objektumot
létrehoztuk, nagy valószínűséggel nem a messageloop-ot tartalmazó
forrásmodulban van, vagy csak egy másik függvényben, mindenesetre
az a tipikus, hogy nem látszik azon a helyen, ahol hivatkoznunk 
kellene rá. Három lehetséges megoldás van:

\begin{enumerate} 
\item
   Ha a search get egy fix pozícionálású dialogboxban volna,
   és azt dialogboxot az msk2dlg kódgenerátorral készítettük
   volna, akkor a dialogboxnak volna search nevű attribútuma,
   ezen keresztül a getre így hivatkozhatnánk: dlg:search.
\item
   A második lehetőség (kikommentezve) látható a példában:
   a dialogbox név alapján meg tudja keresni a kontrolljait.
\item
   Végül, a dialogbox \verb!varinst! metódusával létre lehet hozni
   a dialogboxban egy beágyazott objektumot, 
   aminek éppen a dialogbox kontrolljai az attribútumai.
   Ennek a beágyazott objektumnak, mint attribútumnak a neve \verb!var!.
   Tehát a getet így lehet elővenni: \verb!dlg:var:search!.
\end{enumerate} 
 
Miután így sikerült megkapnunk a kereső mintát, 
végrehajtunk vele egy seek-et a browse-olt adatbázisban. 
Ezután a browse-t, melynek neve ,,szamla'', utasítjuk az 
aktuális lap frissítésére.


Remélem, idáig jutva az olvasóban már felvetődött a kérdés: 
mi van a többi akciót kiváltó kontrollal, amit egyáltalán 
nem  látunk a messageloop-ban? 
A messageloop-on kívül más lehetőség is van az akciók kezelésére.
Ha egy kontrollnak \verb!actionblock!-ot adunk, akkor azt a
getmessage automatikusan végrehajtja. Azt is megtehetjük, hogy
kizárólagosan ez utóbbi módszert alkalmazzuk, ez esetben
a messageloop belseje üres lesz (magára a ciklusra persze
akkor is szükség van).



 

\subsection{Szerkesztő sablonok (picture)}
 
A Jáva terminálnak minden adatot szövegesen küldünk, 
ezért a terminál csak akkor tudja megállapítani egy adat típusát, 
ha azt explicite megmondjuk neki.
A getek esetében a picture-ben kell kódolni a getben editálandó
adat típusát. Ha a picture function stringjében szerepel
az N, D, L betűk valamelyike, akkor a get szám, dátum, logikai
típusú adatot fog editálni. Ha az előbbi betűk egyike sem
szerepel a function stringben, akkor az adat karakteres.

A get mindig akkora, amekkorára méretezzük. Ha a méreténél
hosszabb adatot editálunk benne, akkor az külön intézkedés 
nélkül vízszintesen scrollozódik.
A getben editálható adat mindig olyan hosszú, mint amilyen
a picture template stringjének hossza. Ha a template string
rövidebb a getnél, akkor a get végére nem lehet rámenni
a kurzorral. Ha a template string hosszabb, akkor a get scrolloz.

Ha egy karakteres getnek nincs template stringje, akkor abba
akármilyen hosszú adatot be lehet gépelni, és a get scrolloz.
A többi típus mindig kap egy default template stringet:
\begin{itemize}
\item a számok "999999999999"-et,
\item a dátumok "9999-99-99"-et,
\item a logikai változók {"}A"-t, 
\end{itemize}
bár ez utóbbiakat a check boxok miatt nem fogjuk gyakran használni.

Még egy bővítmény a régi Clipperhez képest: Ha a picture function
stringjébe beírjuk az X karaktert, akkor az akció küldése előtt
a terminál ellenőrzi a get kitöltését, és saját hatáskörben
hibát jelez, ha az nem megfelelő (hogy ne keringjenek triviális
hibaüzenetek a hálózaton). 
Egyes gombokat kilépésre (escape) akarunk használni, 
kellemetlen volna, ha ezeknél a terminál nem hajtaná végre az akciót, 
hanem az X flaggel ellátott getek rossz kitöltésére figyelmeztetne.
Ezért, ha egy kontroll escape attribútuma .t., akkor az a getek
ellenőrzése nélkül jelenti az akciót.

Végül, amire programozás során feltétlenül emlékezni kell: 
A geteket {\bf mindig szereljük fel picture-rel}, 
annak a function stringjében mindig {\bf jelöljük meg az adat típusát}. 
Olyan zagyva dolgok, mint a régi Clipperben, 
ahol a get automatikusan alkalmazkodik az adat típusához, itt nincsenek.
 

\subsection{Browse (jtbrowse osztály)}

A jtbrowse osztály egyik őse a (Clipper) TBrowse. 
Ez azt jelenti, hogy egy jtbrowse objektumhoz ugyanúgy lehet
oszlopokat adni a brwColumn() függvénnyel, mint ahogy azt
korábban csináltuk. Ugyanezért a jtbrowse objektumot a brwShow()
és brwLoop() függvényekkel lokálisan is meg lehet jeleníteni.

A másik módja a jtbrowse objektum létrehozásának,
hogy a konstruktorának átadunk egy már felszerelt TBrowse objektumot, 
amit a jtbrwose objektum magába épít.

\begin{quote}\small
   Megjegyzem, hogy időközben elkészült a jtbrowsenak egy olyan 
   változata, ami nem épít a régi TBrowse osztályra, ez a jttable. 
   A Jáva terminálos programnak ui.\ nincs szüksége a TBrowse-ban 
   levő rengeteg kódra, hiszen nem kell ténylegesen megjelenítenie
   a browset, hanem csak a megjelenítéshez szükséges adatok 
   tárolása, és a terminálhoz való eljuttatása a feladat. 
   
   Az új jttable osztálynak van két közvetlen leszármazottja
   a jtbrowsearray és jtbrowsetable, amik fel vannak készítve
   array-k, illetve táblaobjektumok browseolására (a felkészítés azt 
   jelenti, hogy megfelelő kódblokkokkal vannak ellátva).
   Ezeket egyszerűbben és tisztábban lehet használni,
   mint az itt tárgyalt, először elkészült jtbrowset.

   A továbbiakban elmondottak (a TBrowse örökség kivételével) 
   mindegyik browse fajtára érvényesek. Azt is érdemes megjegyezni,
   hogy a terminál nem tesz különbséget a különféle szerver oldali
   browse fajták között, mert azok üzenetszinten egyformák, 
   így mindegyiket ugyanaz a jttable  Jáva komponenes jeleníti meg.
\end{quote}

A jtbrowse osztályban van két új kódblock, ami a browseolás
lapozós technikája miatt vált szükségessé:

\begin{description}
\item[saveposblock]
Amikor a terminálban a felhasználó le-fel mozog a browse
sávkurzorával, akkor a szerver ezt a mozgást nem követi
az adatforrás pozicionálásával. Azért, hogy később azonosíthassuk,
hogy a felhasználó melyik adatrekordot választotta ki,
a browseolandó lap átküldése előtt minden sorhoz meg kell jegyezni
az adatforrás pozícióját. Pl. tömb browsolásakor
minden browse-beli sorhoz fel kell jegyeznünk egy tömbindexet.
Ezt végzi a saveposblock. A saveposblock tehát egy (browse) 
sorindexhez olyan adatot rendel,
ami azonosítja az adatforrás sorát. \par

\item[restposblock]
Amikor a terminál egy akciót jelent, a feldolgozás során
rá kell pozícionálnunk az adatforrás azon rekordjára, 
ami a browseban ki van választva. Ehhez a restposblock-ot használjuk,
ami a kiválasztott sorindex és a saveposblock által elmentett
adat alapján pozícionálja az adatforrást. \par
\end{description}
 
A jtbrowse osztály olyan default saveposblock-kal 
és restposblock-kal  inicializálódik, ami egy minimális
működést lehetővé tesz: az adatforrás elejére lehet
pozícionálni, és onnan előrefelé lehet lapozni.
A jtbrowse-ban az alábbi metódusokkal navigálunk:

\begin{description}
\item[pageprev] 
    Betölti az előző lapot, 
    feltéve, hogy a browse előzőleg már pozícionálva volt.
\item[pagenext]
    Betölti a következő lapot, 
    feltéve, hogy a browse előzőleg már pozícionálva volt.
\item[pagefirst]
    Betölti a legelső lapot,  
    ezzel a browse pozícionált állapotba kerül.
\item[pagelast]
    Betölti a legutolsó lapot, 
    ezzel a browse pozícionált állapotba kerül.
\item[pagereload]
    Újratölti az aktuális lapot, 
    feltéve, hogy a browse előzőleg már  pozícionálva volt.
\item[pagecurrent]
    Az adatforrás aktuális pozíciójától kezdve betölt egy lapot,
    ezzel a browse pozícionált állapotba kerül.
\end{description}


A dolgok közti összefüggések szemléltetése kedvéért 
nézzük a pagereload kódját: 

\begin{verbatim}
static function _jtbrowse_pagereload(this) 

local page:="",n,ok
 
    ok:=this:restpos(1)  
    this:row2pos:=array(this:maxrow) 

    for n:=1 to this:maxrow
  
        if( n>1 )
            ok:=(1==eval(this:skipblock,1))
        end

        if( !ok )
            asize(this:row2pos,n-1)
            exit
        end

        this:savepos(n)
        page+=_jtbrowse_row_xmlout(this)
    next

    this:xmlpage(page)
    return NIL
\end{verbatim}

Először az adatforrást a browse első sorának megfelelő helyre
pozícionáljuk. Létrehozzuk a row2pos tömböt, amibe a savepos
metódus majd bejegyezi a \{browse tömbindex, adatpozícó\}
párokat. Ezután megpróbálunk végiglépkedni maxrow darab 
adatsoron, miközben a sorok pozícióit feljegyezzük,
és page-ben gyűjtjük a lap XML leírását. A savepos metódus
természetesen a saveposblock kiértékelésével állapítja meg
az adatforrás pozícióját. Végül az xmlpage metódus készít
egy \verb!<jtmessage>! üzenetet, amiben elküldi az új lapot
a terminálnak. A \verb!<jtmessage>! küldése így történik:
\begin{verbatim}
static function _jtbrowse_xmlpage(this,page)
local x:='<jtmessage'
    x+=ATTR("pid",alltrim(str(getpid())))
    x+=ATTR("dialogid",this:dialogid)
    x+='>'
    x+="<control>"+this:name+"</control>"+EOL
    x+="<setpage>"+EOL
    x+=page
    x+="</setpage>"+EOL
    x+="</jtmessage>"
    this:send(x)
    return NIL
\end{verbatim}
 
A browse használatára a main1.prg demóprogramban lehet példát
találni. A lapozásra szolgáló toolbár kódja a mkbrowsebar.prg
programban van.


 