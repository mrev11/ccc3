
\section{Web alkalmazás készítése Jáva Terminállal}


\subsection{Mi kell hozzá?}
 
Megírjuk a kívánt alkalmazást (CCC-ben) a jtlib könyvtárban 
levő megjelenítő interfészre. 
A CCC alkalmazásnak nincs szüksége web szerverre,
nem töltődik le a webes kliens gépére, 
hanem egyszerű programként fut a szolgáltatónál akár a webszerveren, 
akár a szolgáltató egy  erre kijelölt másik gépén.
Az alkalmazások a szokásos TCP (SSL) protokollt használják:
minden alkalmazáshoz elindítunk egy listenert, 
ami egy megadott porton figyeli  a kliensek konnektálását,
és minden új kliensnek elindítja az alkalmazás egy példányát.
Ahány alkalmazást akarunk közreadni, annyi listenert indítunk,
amik az alkalmazásokhoz rendelt (különböző) portokon hallgatóznak.

A webes kliensnél az alkalmazásokat a {\em Jáva Terminál\/} 
program fogja megjeleníteni. Ez egy {\em alkalmazásfüggetlen\/} 
Jáva program (nem applet), ami teljes jogkörrel (nem sandboxban)
fut a kliens gépén.  A szolgáltató által közreadott alkalmazások
futtatásához tehát a kliensnek szüksége van 
\begin{itemize}
\item  
   az 1.4.1+ Jáva környezetre,
\item  
   a jterminal.jar programra,
\item 
   folyamatos internet kapcsolatra,   
\end{itemize}
de semmi másra, pl. nincs szükség böngészőre.

\begin{quote}\small
A jterminal a Jáva régebbi (1.3.x) változatával is működik, 
ekkor azonban külön kell gondoskodni az XML csomag,
az SSL kiegészítés és a Jáva Web Start csomag letöltéséről és installálásáról. 
Az újabb Jávák előnye, hogy ezeket az alkatrészeket már 
alapértelmezésben tartalmazzák. 
\end{quote}

Olyasmivel itt nem foglalkozom, mint a Jáva környezet 
automatikus installálása. A jterminal.jar program letöltését
megkönnyíti a Jáva Web Start technológia, amiről
írok pár sort a következő pontban.


\subsection{Jáva Web Start}

A webszerveren elhelyezzük a \verb!jt.jnlp! 
filét az alábbihoz hasonló tartalommal:
\begin{verbatim}
<?xml version="1.0" encoding="utf-8"?> 
<jnlp  spec="1.0+"  
       codebase="http://1g.comfirm.ve/jterminal/" 
       href="jt.jnlp"> 

  <information> 
    <title>Jáva Terminál Demó</title> 
    <vendor>ComFirm Bt</vendor> 
    <homepage href="html/jterminal.html"/> 
    <description>CCC Download</description> 
    <offline-allowed/> 
  </information> 

  <security> 
      <all-permissions/> 
  </security> 

  <resources> 
    <j2se version="1.4.0+"/> 
    <jar href="jterminal.jar"/> 
  </resources> 

  <application-desc main-class="jterminal"> 
    <argument>1g.comfirm.ve</argument> 
    <argument>46000</argument> 
    <argument>ssl</argument> 
  </application-desc> 
</jnlp> 
\end{verbatim}

Szintén feltesszük a webszerverre a paraméterfilé által 
hivatkozott objektumokat: jar és html filéket, azok további 
alkatrészeit, stb.
Ha most a kliens a windowsos gépén kiadja az alábbi parancsot:

\begin{verbatim}
path_to_javaws\javaws "http://1g.comfirm.ve/jterminal/jt.jnlp"
\end{verbatim}

akkor a javaws program letölti, a paraméterfilében talált 
Jáva alkalmazást, jelen esetben jterminal.jar-t, és elindítja azt a 
megadott  paraméterrel, esetünkben az ip, port, ssl értékekkel,
amire viszont a szerveren automatikusan elindul a
46000-es porthoz rendelt CCC alkalmazás.

A javaws program a letöltött Jáva alkalmazásokat tárolja, 
képes azok futtatására offline módban is (a Jáva Terminál esetében
persze ennek nincs értelme), illetve ha megvan a 
hálózati kapcsolat, akkor automatikusan frissíti az alkalmazásokat.
A kliens  továbbiakban a javaws ablakában keletkező ikonokkal indíthatja 
az egyszer már letöltött és tárolt Jáva alkalmazásokat.

Ha a szolgáltató Jáva Terminálon át elérhető CCC alkalmazásokat
ad közre, akkor minden alkalmazáshoz önálló jnlp filét kell
csinálni. Ezek mind a jterminal.jar-t tartalmazzák letöltendő
Jáva programként, különbözni fognak viszont az ip:port paraméterben,
illetve az alkalmazás leírásában.

A fent leírt módszernél automatikusabb installálásra is van mód:
\begin{itemize}
\item
    A webszervert úgy konfiguráljuk, hogy a jnlp filéket
    egy speciális MIME típussal 
    (\verb!application/x-java-jnlp-file!) küldje.
\item
    A böngészőt úgy konfiguráljuk, hogy ezt a speciális 
    MIME típust felismerje, és észlelésekor indítsa el
    javaws-t az előbb leírt módon.
\end{itemize}

Az utóbbi eljárás lehetővé teszi, hogy a Jáva alkalmazás
letöltését a böngészőből indítsuk egy jnlp-re mutató linkre
kattintva.

\subsubsection*{MIME type beállítás Apache-on}
A \verb!/etc/httpd/mime.types! filébe beírjuk az alábbi sort:
\begin{verbatim}
application/x-java-jnlp-file    jnlp
\end{verbatim}
 
\subsubsection*{MIME type beállítás Netscape 4.x-ben}
Az Edit/Preferences/Navigator/Applications menüben  felvesszük 
a következő adatokat:

\begin{tabular}{l@{~:~~~}l}
Description  & Java Web Start                           \\
MIME type    & \verb!application/x-java-jnlp-file!      \\
Suffixes     & \verb!jnlp!                              \\
Application  & \verb!.../javaws %s!                     \\
\end{tabular}

Tapasztalatom szerint Linuxon a Jáva Web Start installálásakor
ez magától megtörténik. Ha viszont később máshova rakjuk
a javaws programot, akkor itt utánállításra van szükség. 

\subsubsection*{MIME type beállítás Konqueror 3.x-ban}

A Settings/Configure Konqueror/File Associations menüben 
beviszünk egy új típust:

\begin{tabular}{l@{~:~~~}l}      
Group         & \verb!application!           \\
Type name     & \verb!x-java-jnlp-file!      \\
\end{tabular}

Az új típus adatait a következőképpen adjuk meg:

\begin{tabular}{l@{~:~~~}l}
File Patterns & \verb!*.jnlp!                     \\
Description   & Java Web Start                    \\
Application   & megkeressük a javaws-t            \\
\end{tabular}
 