
\section{Áttekintés}

Az SQL2 könyvtár szabványosított adatbáziskezelő API-t nyújt  
CCC programok számára. A ,,szabványosított'' jelző azt jelenti, 
hogy ugyanaz a program ugyanazt a működést produkálja több 
adatbáziskezelővel is. Jelenleg az Oracle és PostgreSQL 
adatbáziskezelőket támogatjuk. 
Koncepciónk szerint az Oracle-t azoknak az ügyfeleknek szánjuk, 
akik nem engedhetnek meg maguknak  alacsony költségvetésű 
projekteket, a Postgrest pedig azoknak, akik bíznak a szabad szoftverekben. 
Természetesen ilyen típusú interfész készíthető volna más adatbáziskezelőkhöz 
is (DB2, Interbase, stb.).

A jelen dokumentáció programozóknak szól, számukra
leggyorsabban példaprogramokkal lehet megvilágítani
a lényeget. Lássuk tehát a lehetőségeket:
Az alkalmazások egyidejűleg belinkelhetik az Oracle és Postgres
könyvtárat. A két könyvtár párhuzamosan ugyanolyan függvényneveket 
definiál, ámde különböző  névterekben, ezért a nevek nem ütköznek.

\begin{verbatim}
    con1:=sql2.oracle.sqlconnectionNew()  //sql2.oracle névtér
    con2:=sql2.postgres.sqlconnectionNew() //sql2.postgres névtér
\end{verbatim}


Most van két adatbáziskapcsolatunk, egy Oracle és 
egy  Postgres. Hogy az \verb!sqlconnectionNew()! függvény
pontosan milyen felhasználóként, hova jelentkezik be,
az pillanatnyilag nem lényeges.

Tételezzük fel, 
hogy van egy bankszámlákat tartalmazó táblánk,
aminek van egy számlaszám és egy egyenleg oszlopa.
Lekérdezzük ezeket az sqlquery objektummal:

\begin{verbatim}
    q1:=con1:sqlqueryNew("select * from szamla") // Oracle lekérdezés
    ...
    q1:close

    q2:=con2:sqlqueryNew("select * from szamla") // Postgres lekérdezés
    while( q2:next )
        ? q2:getchar("szamlaszam"), q2:getnumber("egyenleg")
    end
    q2:close
\end{verbatim}

Ugyanaz a lekérdezés tableentity objektummal:

\begin{verbatim}
    t1:=szamla.tableEntityNew(con1) // Oracle tábla
    ...
    t2:=szamla.tableEntityNew(con2) // Postgres tábla
    rowset:=t2:select
    while( (rowentity:=rowset:next)!=NIL )
        ? rowentity:szamlaszam, rowentity:egyenleg 
    end
\end{verbatim}
    
Új adatrekord létrehozása tableentity objektummal:

\begin{verbatim}
    rowentity:=t2:instance
    rowentity:szamlaszam:="111111112222222233333333"
    rowentity:egyenleg:=0
    rowentity:insert
\end{verbatim}

Módosítás tableentity objektummal:

\begin{verbatim}
    rowentity:=t2:find("111111112222222233333333")
    if( rowentity!=NIL )
        rowentity:egyenleg+=100
        rowentity:update
    end
\end{verbatim}

A példában szereplő \verb!szamla.tableEntityNew()!
függvény adatbázisfüggetlen, azaz ilyen objektumgyártó 
függvényekből bináris könyvtárat lehet létrehozni, ami
minden adatbáziskezelővel működik. Az objektumgyártó 
függvények kódját programmal generáljuk a táblák
adatbázisfüggetlen XML leírásából.

Mint látjuk az SQL2 interfész két legfontosabb eleme az sqlquery és
tableentity osztályok. Az sqlquery körülbelül azt tudja, mint ami 
a JDBC 1.0 specifikációban van. A fejlettebb specifikációkkal 
kapcsolatban a {\em J2EE Útikalauz Java Programozóknak\/}
(szerk. Nyékyné Gaizler Judit, ELTE TTK Hallgatói Alapítvány, Budapest, 2002) 
könyv 475. oldalán olvashatjuk: 
\begin{quote}\it
    ,,JDBC 2.1 alap API: Az előző verzióhoz képest már tetszőleges
    sorrendben lehet feldolgozni az eredménytáblákat, melyeket
    már nem csak olvasni, hanem módosítani is lehet \ldots''
\end{quote}
Ehhez képest az Oracle szilárdan kitart amellett, 
hogy a fetch-ek pontosan egyszer és csakis 
előre haladhatnak végig az eredménysorokon. 
A realitáshoz alkalmazkodva az SQL2 interfészben
csak a next metódust implementáltuk sqlquery-kben és rowset-ekben 
való pozícionálásra.

Az előbbi könyv 479. oldalán a JDBC továbbfejlesztési
terveiről olvashatjuk:
\begin{quote}\it
    ,,Relációs adattáblák és Java osztályok direkt megfeleltetése:
    Egy adattábla minden sora a táblához rendelt osztály egy
    objektumpéldánya lesz, a tábla oszlopainak pedig a sorobjektumok
    adatmezői felelnek meg. A sorobjektumokkal történő bármilyen manipuláció
    esetén az azt megvalósító SQL utasítások a háttérben automatikusan
    végrehajtódnak. Ilyen típusú API például a Java Data Objects  (JDO)
    specifikáció.''
\end{quote}
A tableentity objektumok is éppen így működnek, csak Jáva helyett
CCC osztályokkal. A connection objektumból kapjuk a tableentity-ket,
ezek select metódusaival a rowset-eket, ezekből a next metódussal a 
rowentity-ket (sorokat), amiknek minden oszlophoz van egy metódusuk,
amivel kiolvasható, vagy átírható az oszlop értéke.
Ezzel nagyjából pozícionálni tudjuk, hogy mit is nyújt az SQL2 interfész.

Megemlítem még, hogy a CCC korszerű rétegeivel dolgozunk:
névtereket használunk, a felépítés teljesen objektumorientált.


%\subsubsection*{Fejlesztési tervek}
%
%Különbség van az adatbáziskezelők között abban,
%hogy melyik, mikor abortálja a tranzakciókat, 
%ennek a kiegyenlítése létfontosságú.
%A kiegyenlítés csak az lehet, 
%hogy az adatbáziskezelő alrendszer által jelzett hiba 
%esetén automatikusan rollback-eljük az egész tranzakciót.
%Ez a jövőben bele fog kerülni a kódba.
%
%Egyik adatbáziskezelő sem támogat egymásba skatulyázott
%tranzakciókat. Egy valódi alkalmazás valódi tranzakciója
%azonban meghív olyan résztranzakciókat, amikben
%szintén lehet commit, vagy elkapott kivételt követő rollback,
%méghozzá a külső hívó kód tudta nélkül.
%Ugyancsak probléma, hogy a programok több adatbáziskapcsolatra
%kiterjedő tranzakciókat végeznek (technikailag 
%minden adatbáziskapcsolat külön tranzakció). 
%Követelmény, hogy az alkalmazások átláthatóan biztosítani tudják
%a tranzakciók összehangolt commit-ját, rollback-jét. 
%Ehhez szükség lesz egy tranzakciós alrendszerre.


\subsubsection*{Verzió}

Az SQL2-1.1.x specifikáció a CCC3 környezetre épül.
Az implementáció alkalmazkodik az UTF-8/unicode kódolás követelményeihez.  

Változások történtek a tds-ben. Az oszlodefiníciókban 
az oszlop valódi tábláját és nevét nem a \verb!t=tabid! és \verb!c=colid!
opciókkal adjuk meg, hanem az új \verb!x=column_expression! opcióval.
Ez szükség esetén az \verb!x=tab_alias.real_column_name! formával
pótolja a korábbi \verb!tabid! és \verb!colid! opciókat, de sokkal
használhatóbb, mivel általánosabb SQL kifejezést is tartalmazhat.

Újabb attribútumok is megadhatók az oszlopdefiníciókban:
\begin{itemize}
    \item{\tt d=default\_value}
    \item{\tt t=tooltip}
    \item{\tt l=label}
    \item{\tt p=picture}
\end{itemize}

utóbbi három a táblafüggetlen megjelenítőprogramok érdekében.
