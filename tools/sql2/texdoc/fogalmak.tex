
\subsection{Fogalmak}

\subsubsection*{tableentity}

Egy {\em tableentity\/} objektum SQL select utasításokkal készített,
azonos struktúrájú, de különféleképpen szűrt és rendezett eredménytábla
halmazt képvisel. (Másképp: SQL select utasítások, amikben ugyanazt a joint,
különféle where és order by záradékokkal kombináljuk.)
Az SQL select tartalmazhat egy vagy több elemi táblát, vagy nézetet (view-t).

A tableentity objektumnak vannak {\em select\/} metódusai, 
ezek egy-egy {\em rowset\/} objektumot adnak. A különféle
select metódusok különféle szűréseket végeznek, ennek megfelelően
a rowsetek ugyanannak az (esetleg összetett) alaptáblának különféle 
részhalmazait tartalmazzák. A rowset objektum next metódusával lehet 
megkapni az eredménysorokat, a sorokat rowentity objektum formájában kapjuk.

A tableentity objektumnak van egy {\em find\/} metódusa, ami egy 
kulcsokkal maghatározott sort betölt egy {\em rowentity\/} objektumba. 
(A select speciális eseteként egyelemű rowsetet állít elő, 
és az egyetlen elemből rögtön rowentityt készít.)


\subsubsection*{rowentity}

Az adatbázisrekord, vagy select eredménytábla sorának 
programbeli leképezése a {\em rowentity\/} objektum. 
A rowentity objektum a kulcsmezők értékével (primary key) kapcsolódik
a neki megfelelő adatbázis rekordokhoz.
A tableentity minden oszlopához tartozik
a rowentitynek egy azonos nevű metódusa, amivel az oszlop
(mező) értéke lekérdezhető, módosítható. 
A rowentity nem egy rekordpointer vagy kurzor, hanem önállóan 
létező objektum. A program ugyananabból a táblából
egyszerre több rowentityt is készíthet, azokat egymástól
függetlenül módosíthatja, tárolhatja, törölheti.


\subsubsection*{rowset}

A {\em rowset\/} objektumokat a tableentity select metódusaival kapjuk.
Ugyanabból a tableentity objektumból származó rowsetekben közös, 
hogy  ugyanazokat az elemi táblákat tartalmazzák,
és a táblák ugyanúgy vannak összekapcsolva.
A közös alaptáblából a különféle select metódusok különféle 
filterezettségű (where) és rendezettségű (order by) rowseteket adnak, 
amikből viszont egyforma struktúrájú rowentityket kapunk. 
A \verb!rowset:next! metódushívással lehet megkapni a sorokat. 
A next a tábla elejétől a végéig haladva egyesével
adogatja a sorokat (rowentityket), ha a sorok elfogytak, akkor NIL-t.
Visszafelé haladni, vagy bármi egyéb módon pozícionálni nem lehet.


\subsubsection*{primary key}

A tableenitynek rendelkeznie kell elsődleges kulccsal (primary key).
Ez azon oszlopok felsorolásából áll, amely oszlopok értékének
megadásával egyértelműen azonosítani lehet a sort, azaz a rowentity objektumot.
A primary key-ben felsorolt oszlopoknak nem szabad null értéket megengedni,
ezenkívül az egyediséget unique indexszel ki kell kényszeríteni.


\subsubsection*{columndef}

A tableentity oszlopainak megadására  columndef objektumokat használunk.

\subsubsection*{columnref}

A columnref objektumokat olyan oszlopok leírására használjuk,
amik nem szerepelnek a tableentity oszlopai között, viszont
a tableentity valamilyen módon hivatkozik rájuk egy from, where, 
vagy  order by záradékban.


\subsubsection*{indexdef}

Az indexdef objektumokkal indexek leírását közöljük a tableentityvel.
A \verb!tableentity:create! metódus (a tábla kreálása után) létrehozza 
a megadott indexeket is. Ha az indexünk unique minősítésű, akkor azzal 
kikényszeríthetjük sorok egyediségét, mint ahogy azt a primary key esetében 
is megtesszük.

Az SQL-ben az indexek léte, nemléte, nem befolyásolja a select 
utasítások eredményét, legfeljebb a végrehajtás hatékonyságát.
Az adatbázisszerver mindig saját hatáskörben dönt egy index használatáról,
mellőzéséről, sőt létrehozásáról. 
%Az indexeléssel tehát csak
%ötletet adhatunk az SQL optimalizáló moduljának.




