
\section{Tableentityk használata}


\subsection{Fogalmak}

\subsubsection*{tableentity}

Egy {\em tableentity\/} objektum SQL select utasításokkal készített,
azonos struktúrájú, de különféleképpen szűrt és rendezett eredménytábla
halmazt képvisel. (Másképp: SQL select utasítások, amikben ugyanazt a joint,
különféle where és order by záradékokkal kombináljuk.)
Az SQL select tartalmazhat egy vagy több elemi táblát, vagy nézetet (view-t).

A tableentity objektumnak vannak {\em select\/} metódusai, 
ezek egy-egy {\em rowset\/} objektumot adnak. A különféle
select metódusok különféle szűréseket végeznek, ennek megfelelően
a rowsetek ugyanannak az (esetleg összetett) alaptáblának különféle 
részhalmazait tartalmazzák. A rowset objektum next metódusával lehet 
megkapni az eredménysorokat, a sorokat rowentity objektum formájában kapjuk.

A tableentity objektumnak van egy {\em find\/} metódusa, ami egy 
kulcsokkal maghatározott sort betölt egy {\em rowentity\/} objektumba. 
(A select speciális eseteként egyelemű rowsetet állít elő, 
és az egyetlen elemből rögtön rowentityt készít.)


\subsubsection*{rowentity}

Az adatbázisrekord, vagy select eredménytábla sorának 
programbeli leképezése a {\em rowentity\/} objektum. 
A rowentity objektum a kulcsmezők értékével (primary key) kapcsolódik
a neki megfelelő adatbázis rekordokhoz.
A tableentity minden oszlopához tartozik
a rowentitynek egy azonos nevű metódusa, amivel az oszlop
(mező) értéke lekérdezhető, módosítható. 
A rowentity nem egy rekordpointer vagy kurzor, hanem önállóan 
létező objektum. A program ugyananabból a táblából
egyszerre több rowentityt is készíthet, azokat egymástól
függetlenül módosíthatja, tárolhatja, törölheti.


\subsubsection*{rowset}

A {\em rowset\/} objektumokat a tableentity select metódusaival kapjuk.
Ugyanabból a tableentity objektumból származó rowsetekben közös, 
hogy  ugyanazokat az elemi táblákat tartalmazzák,
és a táblák ugyanúgy vannak összekapcsolva.
A közös alaptáblából a különféle select metódusok különféle 
filterezettségű (where) és rendezettségű (order by) rowseteket adnak, 
amikből viszont egyforma struktúrájú rowentityket kapunk. 
A \verb!rowset:next! metódushívással lehet megkapni a sorokat. 
A next a tábla elejétől a végéig haladva egyesével
adogatja a sorokat (rowentityket), ha a sorok elfogytak, akkor NIL-t.
Visszafelé haladni, vagy bármi egyéb módon pozícionálni nem lehet.


\subsubsection*{primary key}

A tableenitynek rendelkeznie kell elsődleges kulccsal (primary key).
Ez azon oszlopok felsorolásából áll, amely oszlopok értékének
megadásával egyértelműen azonosítani lehet a sort, azaz a rowentity objektumot.
A primary key-ben felsorolt oszlopoknak nem szabad null értéket megengedni,
ezenkívül az egyediséget unique indexszel ki kell kényszeríteni.


\subsubsection*{columndef}

A tableentity oszlopainak megadására  columndef objektumokat használunk.

\subsubsection*{columnref}

A columnref objektumokat olyan oszlopok leírására használjuk,
amik nem szerepelnek a tableentity oszlopai között, viszont
a tableentity valamilyen módon hivatkozik rájuk egy from, where, 
vagy  order by záradékban.


\subsubsection*{indexdef}

Az indexdef objektumokkal indexek leírását közöljük a tableentityvel.
A \verb!tableentity:create! metódus (a tábla kreálása után) létrehozza 
a megadott indexeket is. Ha az indexünk unique minősítésű, akkor azzal 
kikényszeríthetjük sorok egyediségét, mint ahogy azt a primary key esetében 
is megtesszük.

Az SQL-ben az indexek léte, nemléte, nem befolyásolja a select 
utasítások eredményét, legfeljebb a végrehajtás hatékonyságát.
Az adatbázisszerver mindig saját hatáskörben dönt egy index használatáról,
mellőzéséről, sőt létrehozásáról. 
%Az indexeléssel tehát csak
%ötletet adhatunk az SQL optimalizáló moduljának.





\subsection{Tds referencia}

%Előzetes megjegyzés: 
%\begin{quote}\small
%Az eredeti koncepció az volt, 
%hogy a tableentityket XML leírásból készítjük, és ez most is működik.
%A tds-ekkel teljesen azonos tartalmú, de XML szintaktikájú fájlokból
%(ted kiterjesztéssel) a ted2prg utility ugyanazt a kódot generálja,
%mint a tds2prg. Az XML azonban inkább gépi feldolgozásnál előnyös, 
%míg a humán felhasználó könnyebben elboldogul a tds szintaktikával.
%Ezért a tds2prg program (a ted2prg-hez képest) kiegészült egy előzetes 
%elemzéssel, minek során ugyanazt a DOM-ot építi fel, amit a ted2prg XML-ből. 
%Mindkét esetben a közös DOM-ból ugyanaz a modul generálja a kódot.
%\end{quote}

A Table Definition Script (tds) bekezdésekből áll.
Minden bekezdés egy sor elején álló kulcsszóval kezdődik,
amit szóköz nélkül  kettőspont követ.
Az érvényes kulcsszavak:
\verb!name!, 
\verb!version!, 
\verb!table!, 
\verb!join!, 
\verb!column!, 
\verb!colref!, 
\verb!primkey!, 
\verb!index!, 
\verb!select!, 
\verb!include!, 
\verb!type!, 
\verb!comment!.
A bekezdések a következő bekezdésig, vagy a filé végéig tartanak.
A bekezdések sorrendje nem kötött. Az elemző a comment bekezdéseket 
kihagyja.

\subsubsection*{name:}

    Névtér, amibe a tableentityNew() objektumgyártó függvény kerül.

\begin{verbatim}
    name:  multi.level.namespace
\end{verbatim}

A tds-ből generált tableentityt így tudjuk legyártani:
\begin{verbatim}
    tableentity:=multi.level.namespace.tableentityNew(con)
\end{verbatim}
    
\subsubsection*{version:}
    
    A bekezdés tartalma egy tetszőleges tartalmú string (idézőjelekkel),
    amiből a tableentity objektum version attribútumának értéke lesz.

\begin{verbatim}
    version:  "arbitrary text"
\end{verbatim}
    
\subsubsection*{table:}
    
    Legalább egy table bekezdésnek kell lennie, ezek  alakja:
\begin{verbatim}
    table:  real.qualified.tablename=aliasname
\end{verbatim}
    A script más részeiben a táblára kizárólag az alias névvel
    hivatkozunk. Ha nincs megadva join bekezdés, akkor a table
    bekezdésekben magadott táblák Descartes-féle szorzata lesz 
    a tableentity alaptáblája.
    Ha csak egy table bekezdés van, akkor triviálisan az abban
    megadott tábla lesz az alaptábla.
    
\subsubsection*{join:}
    
    Opcionális bekezdés. Ha nincs megadva, akkor a táblák
    Descartes-féle szorzata lesz  az alaptábla. Ha meg van adva, 
    akkor a join bekezdés tartalmából  készül a tableentity által 
    generált SQL select utasítások  from záradéka (tehát lényegében 
    egy from záradékot írunk a joinba).  A from záradék SQL-beli 
    szintaktikájától annyiban térünk el, hogy a táblákra és oszlopokra 
    kizárólag aliasokkal hivatkozunk. Példa:
\begin{verbatim}
    join: 
        a
        full join b on id_a=id_b
        left join c on name_a=name_c
\end{verbatim}

\subsubsection*{column:}

A legegyszerűbb példa:

\begin{verbatim}
    column: szamlaszam  C24
\end{verbatim}

    A kettőspont utáni {\em első\/} szóközökkel határolt szó egyszerre
    \begin{itemize}
    \item metódusnév a rowentity objektumokban,
    \item alias név, amivel a tds más részeiben hivatkozunk az oszlopra. 
    \end{itemize}
    
    A {\em második\/} szóközökkel határolt egység a típusleképezés.
    Ebben azt a típust kell megadni, amiben a CCC program 
    kéri/adja az oszlop adatait (függetlenül attól, hogy
    mi az oszlop tényleges SQL adattípusa). Az érvényes típusok:

    \begin{centering}    
    \begin{tabular}{l@{:~~}l}
    %\begin{tabular}{|l|l|}
        Cw   &  w hoszúságú karakter string \\
        Nw   &  w helyiértéken tárolt egész \\
        Nw.d &  w helyiértéken tárolt, d tizedesjegyet tartalmazó szám \\
        Nw,d &  w helyiértéken tárolt, d tizedesjegyet tartalmazó szám \\
        D    &  dátum \\
        L    &  logikai érték \\
        M    &  memó \\
    \end{tabular}
    \end{centering}    

    A karakter stringeket C (unicode) típusban, a memókat X (bináris)
    típusban kapja a program.

    Az oszlopnevet és típust kiegészítő adatok követhetik.    
    A kiegészítő adatok opcionálisak, és a sorrendjük nem kötött.
    

\paragraph{\tt x=column\_expression}
    
Ezzel a formával adható meg az oszlopkifejezés. 
Ha nincs megadva, akkor maga a név lesz a kifejezés:
\begin{verbatim}
    column: szamlaszam C24      ;ez a default: x=szamlaszam
\end{verbatim}

Amikor az oszlopnév nem egyértelmű, megadjuk a minősített nevét:
\begin{verbatim}
    column: szamlaszam C24      x=ugyfszl.szamlaszam
\end{verbatim}

Megadhatunk kifejezést:
\begin{verbatim}
    column: EGYENLEG   N17.2    (x=napinyito+napitartoz+napikovet)
                                (l=Aktuális egyenleg)
\end{verbatim}

Használhatunk SQL kifejezéseket:
\begin{verbatim}
    column: devnem     C3       x=lower(devnem)
\end{verbatim}

\paragraph{\tt l=label}

    Beállítja az oszlop label attribútumát.
    A labelt táblafüggetlen megjelenítőprogramok kiírhatják 
    oszlopfejlécbe, vagy az adatot megjelenítő text mező elé.

\paragraph{\tt t=tooltip}

    Beállítja az oszlop tooltip attribútumát
    (táblafüggetlen megjelenítőprogramok számára).

\paragraph{\tt p=picture}

    Beállítja az oszlop picture attribútumát
    (táblafüggetlen megjelenítőprogramok számára).

\paragraph{\tt d=default}

    Beállítja az oszlop default érték attribútumát.
    A defaultot nem az interfész, hanem az adatbázisszerver használja,
    ezért csak akkor van hatása, ha a tableentityvel kreáljuk
    a táblát (tehát ugyanaz a helyzet, mint a not null-nál).

\paragraph{\tt nn}

    Jelentése not null. Az attribútum az adatbázisszervernek szól.
    Az interfész nem tudja kikényszeríteni a not null-t, 
    hiszen az adatbázist mások is használhatják, nem feltétlenül a mi 
    SQL2 interfészünkön keresztül.
    Ha viszont a táblát a tableenity create metódusával hozzuk létre, 
    akkor az oszlop not null minősítéssel kreálódik, 
    és a szerver biztosítja a not null feltétel teljesülését.
 

Az oszlopdefiníció szóközökkel határolt egységekből áll.
Ha valamelyik paraméterérték szóközt tartalmaz, akkor
azt zárójelpárral lehet védeni. 
 
\begin{verbatim}
    column: DEVNEM  C3      l=Dev
                            (t=A számla devizaneme)
                            (x=upper( devnem ))
                            (p=@! AAA)
                            nn
\end{verbatim}

A védelemre alkalmazott zárójelpáron belül lehetnek kiegyensúlyozott
zárójelek. Ha nem kiegyensúlyozott zárójelet tartalmazó szöveget
akarunk védeni, használjunk másik zárójelfajtát: \verb![]! vagy \verb!{}!.



\subsubsection*{colref:}
    Lehetnek olyan oszlopok, amikre hivatkozni 
    akarunk a tds-ben, de nem akarjuk lekérdezni az értéküket.  
    Az ilyenek leírását helyezzük colref bekezdésbe, ami ugyanolyan, 
    mint a column, kivéve, hogy nincs benne típus információ,
    és az opciók közül csak az x (column expression) megengedett.

    
\subsubsection*{primkey:}

    A primkey bekezdés kötelező. Benne vesszőkkel elválasztott
    listában felsoroljuk azokat az oszlopokat (column bekezdésben
    definiált alias neveket), amik egyértelműen azonosítják a 
    rekordokat. Az alkalmazás felelőssége, hogy az azonosítás,
    és ezáltal az adatbázis rekord<->rowentity objektum megfeleltetés
    egyértelmű legyen.
\begin{verbatim}    
    primkey:  column_alias1,column_alias2,...
\end{verbatim}    
   
    
\subsubsection*{index:}

    Az opcionális index bekezdések formája:
\begin{verbatim}    
    index:  index_name(column_alias1,column_alias2,...) [unique]
\end{verbatim}    
    Az  index bekezdésekben indexeket definiálhatunk
    a táblára, ezzel bizonyos lekérdezéseket gyorsíthatunk,
    illetve a unique indexekkel kikényszeríthetjük rekordok
    egyediségét.

\subsubsection*{select:}
      
    Az opcionális select bekezdések első szóközökkel határolt 
    szavából a tableentity metódusneve lesz. 

    A metódusnevet követheti a \verb!where! kulcsszó, 
    majd tetszőleges SQL kifejezés, amit az SQL a where és
    order by kulcsszavak között elfogad. Az oszlopokra
    kizárólag az oszlop aliasokkal szabad hivatkozni.
    A kifejezés \verb!:1!, \verb!:2!,\ldots alakú szimbólumokat
    tartalmazhat, amik helyére az SQL select parancs generálásakor
    paramétereket fogunk helyettesíteni.
    
    A where feltételt követheti az \verb!order! kulcsszó,
    ami után zárójelek között fel kell sorolni azokat az oszlopokat, 
    amik szerint rendezni akarjuk az eredménytáblát. 
    A felsorolásban az oszlopnevek után opcionálisan megjelenhet 
    az asc/desc  kiegészítés, ami előírja, hogy az adott oszlop 
    szerint növekvő vagy csökkenő rendezettséget akarunk.
    
    A where és order záradék közül legalább az egyiknek léteznie kell.
    
    Példa:

\begin{verbatim}    
    select: select_a
        where col_a_alias like :1
        order(col_a_alias, col_b_alias desc)
\end{verbatim}

    Ezután a \verb!tableentity:select_a("a%")!
    metódushívás kiválasztja az alaptábla azon sorait,
    melyekben \verb!col_a! első karaktere {"}a", és a sorokat
    rendezi \verb!col_a!, majd csökkenő sorrendben \verb!col_b!
    szerint.

    
    
    

%
\subsection{Segédprogramok}

\subsubsection{tds2prg}
\subsubsection{mkentitylib}

%\subsubsection{bt2tds}
%\subsubsection{btimport}


\subsection{Példa egy táblával}

Nézzük az alábbi példát az sql2/test/basicdemo/tds/proba.tds-ből:

\begin{verbatim}
name: proba
version: "2,2006-07-18"

table: konto.proba=p

column: szamlaszam      C24             nn x=szamla
column: devnem          C3
column: nev             C30             x=megnevezes
column: egyenleg        N17.2           x=osszeg
column: tulmenflag      L
column: konyvkelt       D
column: megjegyzes      M

primkey: szamlaszam,devnem

index: nev(nev,szamlaszam)

select: select_kk  
    where konyvkelt<:1 or konyvkelt is null 
    order(szamlaszam)

select: select_ge  
    where szamlaszam>=:1 
    order(nev desc,szamlaszam)

select: select_tf
    where tulmenflag=:1
    order(szamlaszam)

select: select_bl
    where egyenleg<:1 
    order(szamlaszam)
\end{verbatim}

Ebből a scriptből a tds2prg kódgenerátorral programot készítünk:
\begin{verbatim}
    tds2prg.exe proba.tds
\end{verbatim}
Kapunk egy prg-t, amit a szokásos módon befordítunk a programunkba.
A prg-ben van definiálva a \verb!proba.tableEntityNew()!
objektumgyártó függvény. Nézzünk egy-két programozási mintát, 
mire használható.

Az objektumgyártó egy tableentity objektumot ad, 
amivel a con adatbáziskapcsolaton át elérhető konto.proba 
táblát manipulálhatjuk. A konto.proba név a tds script table 
bekezdéséből jön. Valójában nem tudjuk, 
hogy a név mögött tábla van-e, vagy view, mert a két eset formailag 
nem különbözik.
A tábla (view) neve lehet minősített. A jelen esetben a séma neve 
(ami a táblát tartalmazza) ,,konto'', a tábla vagy view neve ,,proba''.
A tds script name bekezdésében egy névtér van megadva, esetünkben ,,proba''. 
Az itt megadott névtérbe (ami lehetne többszintű is) kerül az objektumgyártó
tableEntityNew() függvény, amit tehát így kell meghívni:
\begin{verbatim}
    tableentity:=proba.tableEntityNew(con)
\end{verbatim}

A tableEntityNew()-nak van egy második, opcionális paramétere is.
Ha ezt megadjuk, akkor a konto.proba helyett a paraméterként
megadott masik.tabla-hoz kapunk hozzáférést. Tételezzük fel,
hogy ennek a másik táblának hasonló szerkezete van, mint a konto.proba-nak:
\begin{verbatim}
    tableentity:=proba.tableEntityNew(con,{"masik.tabla"})
\end{verbatim}
A táblanév egy lista (array) elemeként van megadva. Mint később
látni fogjuk egy tds script több táblára is hivatkozhat,
tehát több tábla paraméternek is értelme lehet, ezért a
táblákat egy arrayben felsorolva kell megadni.

Ezen a ponton nem tudjuk, hogy az adatbázisban van-e egyáltalán
ilyen tábla, egyelőre csak egy programbeli objektumunk van.
Ha emögött még nincs tábla, akkor most létre tudjuk hozni:
\begin{verbatim}
    tableentity:create
\end{verbatim}
Ez persze csak akkor értelmes, ha valóban tábláról van szó,
nem pedig view-ról, ui. az interfész mindenképpen egy
create table utasítást fog küldeni az adatbázisszervernek,
view-król mit sem tud. Természetesen a konto.proba,
vagy masik.tabla tábla fog kreálódni, ahogy a tableentity
gyártásakor paraméterként megadtuk. 

Itt egy kicsit érdemes elidőzni,
megtárgyalni, milyen lesz az új tábla. Általában olyan nevű
oszlopok lesznek benne, mint a tds scriptben az oszlopok neve,
de vannak kivételek. A  példában van egy ,,nev'' nevű oszlop,
amiből a tableentity sorobjektumainak lesz egy ,,nev'' nevű metódusa. 
A táblában azonban az \verb!x=megnevezes! előírás
alapján ennek az oszlopnak az adatbázisbeli neve ,,megnevezes''.
Nem tudjuk pontosan, hogy mi az oszlopok SQL adattípusa,
de bízhatunk benne, hogy az interfész olyan
adattípust választ, ami kompatibilis a tds scriptben
előírt Clipper típussal. A nev esetében sejthető a varchar(30),
de pl. boolean típus egyes adatbáziskezelőkben van, másokban
nincs, ezért nem egyértelmű, mi a Clipper L típus SQL megfelelője.
Az nn-nel jelölt oszlopok not null minősítést kapnak.
Az interfész a tábla elkészítésekor egyúttal indexeket kreál
a primkey és index bekezdések szerint. A unique indexeknek 
szerepük lehet sorok egyediségének kikényszerítésében.

Kissé más a helyzet, ha az adatbázistáblát nem a tableentity 
interfésszel csináljuk, hanem készen kapjuk. A megrendelő bank 
alapadatait tartalmazó táblát pl. biztosan nem mi fogjuk kreálni. 
Ilyenkor egyáltalán nem tudjuk befolyásolni az oszlopok típusát, 
azonban az interfész oda-vissza elvégzi az SQL típus és a Clipper 
típus közötti konrverziót, ha egyáltalán lehetséges. Sajnos ilyenkor
az egyediséget nem tudjuk indexekkel biztosítani, sem a not null
feltételeket betartatni, ha az eleve nincs benne az adatdefiníciókban.

Ha a táblát meg akarjuk szüntetni, megtehetjük a
\begin{verbatim}
    tableentity:drop
\end{verbatim}
metódushívással. Ez egy drop table utasítást küld a szervernek,
(view-ra nyilván nem működik).

Térjünk rá az adatok lekérdezésére. A tableentity objektumnak
mindig van egy select metódusa, ami az alaptábla összes sorát
tartalmazó rowset objektumot ad.
\begin{verbatim}
    rowset:=tableentity:select
\end{verbatim}

A rowset objektum legfontosabb metódusa a next, ezzel egyesével,
előrefelé haladva lekérhetjük a sorokat:
\begin{verbatim}
    while( (rowentity:=rowset:next)!=NIL )
        ...
    end
    rowset:close
\end{verbatim}
Amíg van újabb sor, addig rowset:next egy rowentity objektumot ad,
ha már nincs több sor, akkor NIL-t. Ügyelni kell a rowset-ek lezárásra, 
máskülönben  elfogynak bizonyos erőforrások (handlerek). 

Most már soraink is vannak, nézzük a mezőértékeket:
\begin{verbatim}
    r:=rowentity
    ? r:szamlaszam, r:devnem, r:nev, r:egyenleg
\end{verbatim}
Pontosan mit is írtunk ki? A 
konto.proba.szamla,
konto.proba.devnem,
konto.proba.megnevezes,
konto.proba.osszeg nevű adatbázismezőket.
A rowentity objektumok tehát olyan attribútumokkal rendelkeznek,
mint amilyen oszlopnevek vannak a tds scriptben. Ha nem intézkedünk
másképp, akkor ezek az adattábla azonos nevű oszlopát jelentik,
de ezt felülbírálhatjuk az \verb!x=valid_sql_expression! előírással.

A CCC programból nézve a rowentity attribútumoknak
olyan típusuk van, mint amit a tds scriptben előírtunk. Pl. az alábbi
programrészlet végrehajtása után
\begin{verbatim}
    rowentity:nev:=""
    ? valtype(rowentity:nev), strtran(rowentity:nev," ","x")
\end{verbatim}
ezt látjuk kiírva: \verb!C xxx...xxx! (30 darab x).
Az adatbázisból kaphatunk SQL null értékeket is. E tekintetben
a Clipper hagyományokat követjük, és az interfésztől
a nullok helyén nem NIL-t kapunk, hanem egy megfelelő típusú empty értéket.
Ha mindenképpen szükség van a nullok vizsgálatára, arra is van mód,
lásd a osztály referenciát és a példaprogramokat.



Tegyük fel, hogy van egy '111111112222222233333333' 
számlaszámú, 'HUF' devizanemű sorunk a táblában, ezt a
\begin{verbatim}
    rowentity:=tableentity:find({'111111112222222233333333','HUF'})
\end{verbatim}
metódushívással tudjuk beolvasni. Ha mégsincs sor a megadott
számlaszámmal, akkor tableentity:find NIL-t ad.

A tds scripttől függetlenül a tableentitynek mindig van
egy általános select metódusa, ami az alaptábla összes
sorát tartalmazó rowsetet ad, és egy find metódusa, ami
a primary key alapján kiválaszt egyetlen sort. Lehetnek
azonban más select metódusok is, ha definiálunk ilyeneket
a tds-ben. A jelen példában a
\begin{verbatim}
    rowset:=tableentity:select_kk({stod("20000101")})
\end{verbatim}
metódushívás kigyűjti azokat a sorokat, amikben a könyvelés
dátuma régebbi, mint 2000-01-01, vagy egyáltalán nem volt rajta
könyvelés, és ezért a dátum null. Az interfész a tds-ben megadott
where záradékból SQL where záradékot készít úgy, hogy a select metódus
array paraméterében felsorolt értékeket behelyettesíti az
\verb!:1!, \verb!:2!,\ldots\  szimbólumok helyére. A tds-ben 
bármit megadhatunk a where záradék helyén, amit az SQL elfogad
a ,,where'' és az ,,order by'' kulcsszavak között.
A tds-beli order záradékból  SQL order by záradék lesz.
Az interfész az order zárójelei közötti szöveget egyszerűen a
generált SQL parancs  ,,order by'' kulcsszava után írja. 
A gyakorlatban általában a zárójelek között felsoroljuk
az oszlopokat, amik szerint rendezni akarunk, esetleg az
asc/desc módosítással kiegészítve.


Rátérve a módosítások tárgyalására mégegyszer megjegyezzük,
hogy nem minden tábla módosítható. Ha az alaptábla egy view,
vagy több táblából képzett join, akkor a módosítási kísérlet
valószínűleg hibát eredményez. 
Rakjunk most be egy új sort a táblába:
\begin{verbatim}
    rowentity:=tableentity:instance
    rowentity:szamlaszam:="XXXXXXXXYYYYYYYYZZZZZZZZ"
    rowentity:devnem:="EUR"
    rowentity:egyenleg:=100000
    rowentity:insert
    con:sqlcommit
\end{verbatim}
A tableentity objektum instance metódusa gyárt nekünk egy
üres rowentityt, amit feltöltünk adatokkal. A rowentity insert
metódusa generál egy insert into utasítást, amivel kiírja a rekordot.
A kiírást a commit véglegesíti. Tegyük fel, hogy ugyanezt 
a rekordot módosítani kell:
\begin{verbatim}
    rowentity:=tableentity:find({'XXXXXXXXYYYYYYYYZZZZZZZZ','EUR'})
    rowentity:egyenleg+=1000
    rowentity:update
    con:sqlcommit
\end{verbatim}
Most az interfész generálni fog egy ilyen utasítást
\begin{verbatim}
    update "KONTO"."PROBA" set "OSSZEG"=101000 
    where "SZAMLA"='XXXXXXXXYYYYYYYYZZZZZZZZ' and "DEVNEM"='EUR'
\end{verbatim}
amivel módosul a rekord. Rendkívül fontos észrevétel, hogy
a rowentity objektum, a primary key oszlopok egyezése alapján
találja meg a hozzá tartozó rekordot. Mi történik akkor
a következők után?
\begin{verbatim}
    rowentity:szamlaszam:="valami más érték" //ROSSZ!
    rowentity:update //ROSSZ!
\end{verbatim}
Így nem az eredeti rekord módosul, hanem a szerver minden olyan
rekordot módosít, aminek a számlaszáma a "valami más érték".
Lehet, hogy semmi sem módosul, lehet, hogy egy olyan rekord,
amire nem számítottunk. A tanulság, hogy a primary key-t alkotó
 oszlopokat nem módosíthatjuk közvetlenül.
Szerencsére a gyakorlatban erre ritkán van szükség. Ha mégis,
akkor ezt csináljuk:
\begin{verbatim}
    rowentity:=tableentity:find({'XXXXXXXXYYYYYYYYZZZZZZZZ','EUR'})
    rowentity:delete
    rowentity:szamlaszam:="valami más érték"
    rowentity:insert
    con:sqlcommit
\end{verbatim}
Mivel a két művelet egy tranzakcióban van, nem fenyeget,
hogy csak az egyik hajtódik végre, a másik pedig nem.
A példa alapján világos: az alkalmazásnak létfontosságú,
hogy a primary key egyedisége megmaradjon, így az adatbázisrekordok 
és a program rowentity objektumai közötti megfeleltetés ne sérüljön.

A rowentity lényeges tulajdonsága, hogy igazi objektum,
nem pedig csak holmi rekordpointer vagy kurzor.
Az objektum az adatait  saját memóriabufferben tárolja, ami nem
szűnik meg attól, hogy végrehajtunk egy újabb rowset:next-et,
rowset:close-t, vagy akár con:rollback-et. A program változóiban
tárolhatunk egyidejűleg akárhány rowentity objektumot.










\subsection{Példa több táblával}

Ebben az alfejezetben egy olyan példát elemzünk,
amiben a tableentity alaptáblája több elemi tábla
összekapcsolásával keletkező join. Az ilyen tableentity
objektumoknak is vannak módosító metódusai
(create, drop, zap, insert, delete, update), 
ám ezek meghívása nagy valószínűséggel hibát okoz,
erre többet nem is térünk ki. 
A példa az sql2/test/basicdemo/tds/probaszerencse.tds-ből való.

\begin{verbatim}
name: probaszerencse
version: "2,2006-07-18"

table: konto.proba=p
table: konto.szerencse=q

join: p left join q on szamlaszam=qszamlaszam

column: szamlaszam      C24         x=p.szamla
column: devnem          C3
column: nev             C30         x=megnevezes
column: egyenleg        N17.2       x=osszeg
column: tulmenflag      L
column: konyvkelt       D
column: megjegyzes      M
column: kiegdata        C20

colref: qszamlaszam                 x=q.szamla

primkey: szamlaszam,devnem
\end{verbatim}

Előzőleg a konto.proba nevű táblával dolgoztunk, most ezt kiegészítjük
a konto.szerencse táblával. A két tábla a \verb!szamla! nevű oszlopon
keresztül kapcsolódik.  A szerencse nevű táblából a \verb!kiegdata! oszlopot
akarjuk hozzávenni a tableentityhez (nem túl sok, de most nem az a lényeg,
hanem maga az összekapcsolás). 

Először is megfigyeljük, hogy a tableentity objektum gyártó
függvény most a \verb!probaszerencse! névtérbe van helyezve, 
tehát így lehet meghívni:
\begin{verbatim}
    tableentity:=probaszerencse.tableEntityNew(con)
\end{verbatim}
Látjuk, hogy a tds-ben két table bekezdés is van.
Ezek megadják az alapesetben használt táblákat, és azok
alias neveit (p és q). A script más részeiben a táblákra
kizárólag az alias nevekkel hivatkozunk.

Tudjuk, hogy szükség esetén a tds-ben meghatározott táblákat
helyettesíthetjük más táblákkal, például:
\begin{verbatim}
    tablist:={"masik.proba","masik.szerencse"}
    tableentity:=probaszerencse.tableEntityNew(con,tablist)
\end{verbatim}
A helyettesítendő táblákat utólag is megadhatjuk:
\begin{verbatim}
    tableentity:=probaszerencse.tableEntityNew(con)
    tableentity:tablist:={"masik.proba","masik.szerencse"}
\end{verbatim}
A két módszer ugyanarra az eredményre vezet.  

A table definition script (tds) join bekezdésében határozzuk meg,
hogyan legyen a két tábla összekapcsolva. Ha ugyanezt a joint
közönséges SQL-ben akarnánk megcsinálni, pl. az sqlplus-ban,
akkor valami ilyesmit írnánk (rövidítésekkel):
\begin{verbatim}
    select 
        p.szamla,
        devnem,
        ...
        kiegdata 
    from 
        konto.proba p 
        left join konto.szerencse q 
        on p.szamla=q.szamla 
    order by   
        p.szamla,devnem
\end{verbatim}
Koncentráljunk a from záradékra. A tds-beli join bekezdés úgy 
vezethető le az SQL from záradékból, hogy a redundáns információt
kihagyjuk. A \verb!konto.proba p! helyett a tds joinban csak annyit 
írunk: \verb!p!, hiszen a table bekezdésből már tudjuk, hogy a p alias 
név a konto.proba táblát jelenti. 
A \verb!p.szamla! helyett azt írjuk \verb!szamlaszam!,
a \verb!q.szamla! helyett pedig \verb!qszamlaszam!,
mert a column és colref bekezdések meghatározzák, 
hogy ezek a nevek pontosan melyik oszlopot jelentik.
Látható a column bekezdések kettős szerepe.
Egyrészt metódusneveket jelentenek, másrészt oszlop alias
neveket hoznak létre, amiket a tds scriptben használunk.
A colref bekezdéseknek csak oszlop alias szerepe van.
Foglaljuk össze a table és join bekezdésekkel kapcsolatos
tudnivalókat:

1) A table bekezdések sorolják fel a tableentityben szereplő
elemi táblákat. Minden táblához rendelünk egy alias nevet. 

2) A join bekezdés opcionális (ha csak egy tábla van, akkor a joinnak
nem is volna értelme). Ha nincs join, akkor a tableentity alaptáblája 
az elemi táblák Descartes-féle szorzata, vagyis ilyenkor az interfész olyan
SQL select parancsot generál, aminek a from záradékában egyszerűen fel 
vannak sorolva a táblák.

3) Ha van join bekezdés, akkor oda egy SQL select from záradékot
írunk azzal az eltéréssel, hogy a táblanevek helyett tábla aliast, 
az oszlopnevek helyett pedig mindenhol oszlop aliast írunk.

Az a fajta from záradék, amit itt használunk az SQL92 szabvánnyal
került be az SQL-be. Korábban csak az ún. inner join volt ismert
(a where záradékban). Az újabb szabvány kiegészült az outer joinnal,
és a különféle joinok összes variációinak egységes értelmezésével,
de ezeket a from záradékba kell írni. Sajnos ebben a dokumentációban
nincs hely az SQL mélyebb ismertetésére.

Folytassuk az oszlopokkal. Az első column bekezdésben ezt látjuk:
\begin{verbatim}
    column: szamlaszam      C24     x=p.szamla
\end{verbatim}

Szó volt már a \verb!szamlaszam! név szerepéről.  
A tableentityből származó rowentity (sor)objektumoknak lesz
egy ilyen nevű attribútuma, egyúttal bevezet egy oszlop aliast,
amivel a tds-ben bárhol (join, where, order) hivatkozni tudunk
az oszlopra. 

A \verb!C24! típus értelmezése: Nem érdekel minket, hogy az
adott oszlop SQL adattípusa pontosan micsoda. Akármi is az
SQL adattípus, a mi programunk C24-re konvertálva kéri az adatot
a szervertől. Mármost a tényleges típus és a szerver tudása
dönti el, hogy ez az esetleges konverzió értelmes-e, lehetséges-e
egyáltalán. Ha nem értelmes, vagy nem lehetséges, akkor az az
alkalmazás logikai, vagy közvetlen futási hibáját fogja okozni.

A \verb!x=p.szamla! kiegészítő adat azt mondja: A \verb!szamlaszam!
metódusnévhez és oszlop aliashoz a \verb!p!  alias névvel
azonosított tábla \verb!szamla! oszlopa tartozik. Ha ez nem volna megadva, 
akkor az interfész és az adatbázis szerver \verb!szamlaszam! nevű oszlopot 
keresne, és hibához vezetne, ha az adott oszlop nem pontosan egy 
táblában volna megtalálható (esetünkben két táblában is van ilyen). 

A column adatok sorrendjéről tudni kell, hogy a kettőspont
utáni első szóközökkel határolt szó kötelezően az alias név,
a második szó kötelezően a típus, a további adatok opcionálisak,
és sorrendjük tetszőleges.

A colref bekezdésekkel olyan oszlopokhoz készítünk oszlop
aliast, amiket nem akarunk bevenni a tableentity alaptáblájába,
de hivatkoznunk kell rá a join-ban, where-ben, vagy order-ben.
A colref szintaktikája lényegében ugyanaz, mint a columné,
csak kimarad belőle a típusmeghatározás.

A további összetevőket, mint a  primkey, különféle selectek,
már ismerjük. Indexdefiníciókat helyezni egy többtáblás tds-be
értelmetlen, mivel soha nem fogjuk a táblát kreálni.
A lekérdezések ugyanúgy működnek, mint az egytáblás esetben,
például:
\begin{verbatim}
    tableentity:=probaszerencse.tableEntityNew(con)
    rowset:=tableentity:select
    while( (r:=rowset:next)!=NIL )
        ? r:szamlaszam,r:kiegdata
    end
    rowset:close
\end{verbatim}

