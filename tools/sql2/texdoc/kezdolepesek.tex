
\section{Kezdőlépések}

Itt arról lesz szó,  hogyan vegyük birtokba az SQL2 szoftvert.

\subsubsection*{Adatbázis szoftver telepítés}

Először is kell legyen tesztelésre alkalmas adatbázisszerverünk.
A függelékekben le van írva, hogyan tudunk Linuxon Oracle-t
és/vagy Postgrest installálni. Aki Windowson akar dolgozni, sajnos
magára van utalva. Az Oracle-lel Windowson sem lehet gond,
és a Postgres 8.0 már Windowson is megy. 

Akinek már eleve van adatbázisszervere, az is nézze meg
a függelékeket, ott ui. az is le van írva, hogy milyen
tablespace-t, user-t, schema-t kell létrehozni az adatbázisban,
hogy a demó programok működhessenek. A demó programok
számítanak a \verb!konto! sémára, és elszállnak, ha az nem létezik.

A CCC programokat fordító/futtató gépen szükségünk lesz 
az adatbáziskezelők fejlesztő/kliens környezetére is.

\begin{itemize}
\item 
    Oracle esetén, ha ugyanazon a gépen tesztelünk, mint amin
    az adatbázisszerver  fut, akkor ,,magától'' van kliens
    környezetünk is. Egyébként letöltjük és installáljuk az 
    instant-client  szoftvert az Oracle-től. Az SQL2 csomag tartalmazza
    azokat az include filéket, amik az Oracle interfész lefordításához 
    kellenek, tehát speciális fejlesztői készletre nincs szükség.
\item    
    Postgres esetében az SQL2 lefordításához előzőleg installálni
    kell a developer csomagot (a pontos név Linuxonként más és más),
    amivel általában a kliens könyvtárak is rákerülnek a gépünkre.
\end{itemize}

\subsubsection*{CCC környezet}

Szükségünk lesz a legfrisebb CCC környezetre, ez letölthető a webről
\href{http://ok.comfirm.hu/ccc3/download}{http://ok.comfirm.hu/ccc3/download}.
Mivel a Flex és Lemon nem olyan régen tartozik csak bele a CCC környezetbe,
külön  megemlítem, hogy a CCC-vel telepíteni kell 
a \href{http://ok.comfirm.hu/ccc3/download}{Flexet}
és a \href{http://ok.comfirm.hu/ccc3/download}{Lemont}.

\subsubsection*{Fordítás}

Le kell fordítani az egész SQL2 projektet. A fenti előkészületek
után Linuxon ez nem lehet probléma, az sql2 directoryban elindítjuk

\begin{verbatim}
    mkall.b
\end{verbatim}

Ez mindenhova benéz, és elindítja a fordítást végző Build
scripteket (Linuxon). Aki Windowson dolgozik, 
mkall.b helyett az mkall.bat scriptet indítsa el.
% megoldódott
%
%\begin{quote}\small
%  Sajnos Windowson a Postgres interfész csak a MinGW (GCC)
%  fordítóval működik. A CCC másik két támogatott fordítójával
%  (Microsoft, Borland) készült programok elszállnak.
%  Például a sequence léptetését végző demó program BCC-vel fordítva
%  minden harmadik léptetés után determinisztikusan elszáll.
%  Ugyanez a program MSC-vel fordítva minden hatodik (!)
%  léptetés után elszáll.
%  Ezekkel a hibákkal jelenleg nem tudok mit kezdeni.
%\end{quote}
Meggyőződünk arról, hogy a fordítás hibamentes. Csak a
Postgres developer csomag installálásával lehet gond
\begin{itemize}
\item 
    ha elfelejtettük installálni,
\item 
    ha az include filék egy postgresql könyvtárba kerülnek,
    miközben az SQL2 szoftver egy pgsql alkönyvtárban keresi őket,
\item 
    ha az include filék egy pgsql könyvtárba kerülnek,
    miközben az SQL2 szoftver egy postgresql alkönyvtárban keresi őket.
\end{itemize}
Az első esetben pótoljuk az installációt. A másik két esetben
csinálunk egy symlinket, ami a másik néven is elérhetővé teszi
az include-okat tartalmazó directoryt.

\subsubsection*{Tesztadatok}

Az \verb!sql2/test/testdata! directoryban van néhány bt tábla,
amik tesztadatokat tartalmaznak. A tview programmal meg tudjuk
nézni ezek tartalmát.

Az \verb!sql2/test/testdata_import! directoryban megtaláljuk
a btimport.exe programot. Ez az előbbi bt táblákat importálja
az SQL adatbázis szerverbe:
\begin{itemize}
\item
    Az \verb!so! script úgy indítja btimport-ot, 
    hogy az egy Oracle adatbázisba tölti a bt táblákat.
\item
    Az \verb!sp! script úgy indítja btimport-ot, 
    hogy az egy Postgres adatbázisba tölti a bt táblákat.
\end{itemize}
Ha esetleg valami nem működik első kipróbálásra,
akkor szükségesnek mutatkozhat az so, sp, import.b scriptek 
{\em elolvasása}. Az áttöltött adatokat szemügyre vesszük
az sqlplus (Oracle), és psql (Postgres) kliens programokból is.


\subsubsection*{Példaprogramok}

Ha már vannak tesztadataink, akkor átfáradunk az
\verb!sql2/test/basicdemo! directoryba. Itt azt fogjuk
tapasztalni, hogy a demóprogramok már mind le vannak 
fordulva, méghozzá két verzióban:
\begin{itemize}
\item 
    Az \verb!exe-ora! directoryban az Oracle-lel működő
    programok vannak.
\item 
    Az \verb!exe-pg! directoryban az Postgreszel működő
    programok vannak.
\end{itemize}
Ez a párhuzamosság érzékeltetni próbálja, hogy ugyanabból
a forrásból kétféle program készül, amik ugyanazt ugyanúgy 
csinálják, csak az egyik Oracle-en, a másik Potgresen.\footnote{
Azt is megtehetjük, hogy ugyanabba a programba egyszerre belinkeljük 
az Oracle és Postgres interfészkönyvtárat. Ekkor csak egy programváltozatunk
lesz, aminek parancssori paraméterrel, vagy környezeti változóval mondjuk
meg, hogy melyik adatbázist használja. Az is lehet, hogy egy alkalmazás
több adatbázist használ egyszerre, Oracle-t és Postgrest vegyesen.}  
Vannak programok (programcsoportok), amikhez találunk
\verb!s-*! alakú indítóscriptet, más programokat közvetlenül 
indíthatunk.

\begin{description}
\item[binarydata.exe]
    Teszteli, hogy a memó mezők képesek-e bináris adatok tárolására.
\item[connect.exe]
    Teszteli a bejelentkezést, kiírja a szerver változatszámát.
\item[memleak.exe]
    Teszteli, hogy a programból szivárog-e a memória. 
    Az Oracle kliens bizonyos változatai ebből a szempontból hibásak.
\item[s-deadlock-detect]
    Demonstrálja a deadlock kezelést.
\item[sequence0.exe]
    Létrehoz egy sequence objektumot.
\item[sequence.exe]
    Növeli az előző sequence objektumot.
\item[s-query-sqlquery]
    Lekérdezés sqlquery objektummal.
\item[s-query-tableentity]
    Lekérdezés tableentity objektummal.
\item[s-te-demo]
    Különféle lekérdezések és adatmódosítások tableentity objektummal.
\item[s-tran-autolock]
    Demonstrálja a konkurrencia kezelést automatikus lockolással.
\item[s-tran-manuallock]
    Demonstrálja a konkurrencia kezelést explicit lockolással.
\end{description}

Hangsúlyozom, hogy {\em a példaprogramok a dokumentáció
lényeges részét képezik}. Sok munkám van abban, hogy a példákból
és a közéjük írt megjegyzésekből megfelelően domborodjon a mondanivaló. 
A példaprogramokat tehát el kell olvasni, és ki kell próbálni, 
miközben a kódot összevetjük az eredményekkel. Sajnos ez időt 
és fáradságot igényel.

Felhívom a figyelmet az SQLDEBUG környezeti változóra.
Ha ez be van állítva,
\begin{verbatim}
    export SQLDEBUG=on
\end{verbatim}
akkor az SQL2 interfész listázza az adatbázisszervernek küldött
SQL utasításokat. Ezek vizsgálata rendkívül sokat segít a program
működésének megértésében, a hibák kijavításában.

\subsubsection*{Jáva terminálos browse}

A \verb!sql2/test/entitybrowse! directoryban egy olyan demó van,
ami Jáva terminálban browse-ol egy tableentity objektumot.
A program nem bonyolult, de csak akkor fog lefordulni,
ha a CCC környezetben van Jáva terminál támogatás,
ezért kezdetben kihagyjuk ennek a projektnek a fordítását.



 






