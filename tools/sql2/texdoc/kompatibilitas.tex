
\section{Kompatibilitási kérdések}

\subsection{Adatbázis objektum nevek}

Az Oracle és a Postgres is támogatja az idézett azonosítókat.
Ha Oracleben egy azonosító ilyen alakú
\begin{verbatim}
    "KONTO"."SZAMLASZAM"
    "KONTO"."UPDATE"
\end{verbatim}
akkor azokat az Oracle nem tekinti kulcsszónak. Idézőjelek nélkül
nem használhatnánk \verb!update! nevű mezőt, mert az ütközik az
update kulcsszóval. Ha az idézett azonosítókat csupa nagybetűvel írjuk,
akkor az azonosító case insensitive abban az értelemben, hogy 
\begin{verbatim}
    "KONTO"."SZAMLASZAM" <=> Konto.Szamlaszam
\end{verbatim} 
egyenértékű. Ha az idézett azonosítóban nagybetűn kívül más is van,
akkor az azonosító case sensitive lesz.
A Postgres ugyanígy nem tekinti kulcsszónak
az idézett azonosítókat, az Oraclelel ellentétben azonban
a csupa kisbetűs írásmód eredményez case insensitive azonosítókat,
azaz 
\begin{verbatim}
    "konto"."szamlaszam" <=> Konto.Szamlaszam
\end{verbatim} 
Az SQL2 interfész alkalmazkodik ehhez. Mindig idézett azonosítókat
használ, ezáltal nincs szükség az SQL kulcsszavak kerülgetésére.
Oracle esetében az idézett azonosítók nagybetűsek, Postgresnél
kisbetűsek. Ez azzal az előnnyel jár, hogy az általunk létrehozott
adatbázisokban case insensitive azonosítók lesznek, így
az sqlplus-ban vagy psql-ben érdektelen a kis/nagybetűk használata,
azaz a szokott módon dolgozhatunk. Persze egy kulcsszóval egyező
azonosító esetén interaktív módban is használnunk kell az
idézőjeleket. Hátrány, hogy a mások által létrehozott case sensitive
azonosítókat nem lehet elérni az SQL2 interfészen keresztül.


\subsection{Autocommit, tranzakcióhatárok}

Az Oracle automatikusan indítja a tranzakciókat.
Egyáltalán nincs \verb!begin transaction! (vagy hasonló) utasítás,
az Oracle mindig tranzakció közben van,
a tranzakció a következő commit/rollback utasításig tart,
ami után automatikusan indul a következő tranzakció.

A Postgres is tudott így működni a 7.4 verzió előtt,
ez volt az ún. autocommit off üzemmód. A 7.4-es verzióban azonban
megszűnt az autocommit utasítás, ezért az SQL2 interfészben
kénytelenek vagyunk minden commit/rollback után \verb!begin transaction!
utasítást küldeni a szervernek, ezzel pótolva az autocommitot.

Vannak esetek, amikor a szerver saját hatáskörben
befejezettnek nyilvánít egy tranzakciót:

Az Oracle a DDL utasítások (create table, drop table, \ldots) 
előtt és után (belsőleg) automatikusan végrehajt egy commitot. 
Az SQL2 interfész az egységes működés érdekében a Postgres 
esetében is kiadja ezeket a commitokat.

A Postgres bármilyen hiba esetén azonnal és elkerülhetetlenül
abortálja a tranzakciót, és végrehajt (belsőleg) egy rollbacket.
Mivel ilyenkor nincs módunkban automatikusan indítani a következő
tranzakciót, előfordulhat (ha nem vesszük észre a hibát),
hogy a következő utasításaink tranzakción kívülre kerülnek, 
ezáltal mind hibásak lesznek. Tulajdonképpen ez előnyös,
mert így legalább nem lehet elsiklani a hiba felett, 
úgy gondolom, éppen ennek érdekében szűnt meg az autocommit.
Az Oracle ezzel szemben úgy működik, mintha minden SQL utasítás
előtt volna egy implicit savepoint, és hiba esetén eddig a savepointig
rollbackelne. A Oracle tehát visszaáll a hibás utasítást megelőző állapotra,
és az alkalmazási programra bízza, hogy rollbackeli-e a tranzakció
egészét. Ezt a különbséget az SQL2 interfész nem egyenlíti ki.
Az alkalmazási programokat a Postgres szigorúbb feltételeihez 
alkalmazkodva kell megírni.

\subsection{Tranzakcióhatáron átnyúló fetch}

Az indexszekvenciális rekordkezelőkhöz szokott programozó
hajlamos arra, hogy az SQL select-fetch utasításokat egy iterátor
módján képzelje el: Az első fetch 1 millisec alatt behozza az első 
rekordot, a következő fetch a másodikat, stb. Ez az, amiről szó sincs.

Az SQL select még az első fetch előtt összeállítja, lerendezi
a teljes (!) eredménytáblát még akkor is, ha az GB-os nagyságrendű.
Emiatt lehet, hogy az első rekord megérkezéséig percek (órák) is eltelnek,
míg a szerver egyre csak dolgozik az eredménytáblán. 
Az alkalmazóknak tehát csínján kell bánni a nagy eredménytáblához vezető
lekérdezésekkel, mivel azok igencsak megviselik a szervert. 

Megfigyelésem szerint az Oracle az eredménytáblát a szerveren 
tartja, és csak a fetchelt rekordokat küldi át a kliensre.

A Postgres két módszert is kínál a select utasításra.
A {\em közvetlen select\/} utasítás az egész eredménytáblát előre
átküldi a kliensgépre, ebből a kliens sor- és oszlopindex alapján
(mintha csak egy mátrixot címezne) éri el az adatokat. Mondani sem
kell, milyen nehézségekkel jár egy nagy eredménytábla kezelése.
Emellett a Posgresben kurzort is definiálhatunk:
\begin{verbatim}
    declare crs_id cursor for select ...
    fetch forward 1 in crs_id
    ...
    fetch forward 1 in crs_id
\end{verbatim}
A {\em kurzoros select\/} csak a ténylegesen fetchelt rekordokat
küldi át a kliensre, ezért hatékonyabban kezeli a nagy eredménytáblákat,
viszont funkcionalitásában korlátozott, pl. nem működik vele
a lockolás (\verb!select ... for update!). Az SQL2 interfész
aszerint választ a közvetlen és kurzoros select között, hogy 
az alkalmazás akar-e lockolni. Ha lock van előírva, akkor közvetlen,
egyébként kurzoros lesz a select.

Tovább bonyolódik a helyzet a tranzakcióhatárok miatt.
Az Oracle képes a tranzakcióhatárokon átnyúló fetchelésre.
Ez a tulajdonság használható a következő típusú programnál:
\begin{verbatim}
    select 100 ezer darab számla
    while(ciklus 100 ezer darab számlára)
        fetch
        kamatszámítás egy számlán
        if(sikeres)
            commit
        else
            rollback
        end
    end
\end{verbatim}
A Postgres közvetlen select utasítása elvileg szintén tudná ezt,
a gyakorlatban viszont nagyon megkottyan neki a 100 ezres eredménytábla 
kliensen történő kezelése. A kurzoros select (új fejleményként)
{\em félig\/} képes a tranzakcióhatáron átnyúló fetchelésre,
nevezetesen a commiton átmegy, a rollback viszont
menthetetlenül lezárja. Nem könnyű tehát összhangba hozni 
az Oracle és a Postgres fetchelésre vonatkozó elképzeléseit.

Nézetem szerint azonban egyáltalán nem kell belemenni
ebbe a zsákutcába.  A tranzakcióhatárokon átnyúló fetchelés
fogalmilag zagyva dolog, jobb elkerülni, az SQL2 interfész
ezért egységesen megakadályozza.  Az előző programot egyszerűen
implementálhatjuk {\em két adatbáziskapcsolattal}:
Az egyiket használjuk a 100 ezer darab számla felsorolásához,
a másikat a kamatszámításhoz. Így a kamatszámítás commit/rollback
utasításai nincsenek hatással a számlák bejárására. Ez tiszta ügy.



