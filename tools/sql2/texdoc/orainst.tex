
\section{Oracle 10g installálása Linuxra}

\subsection{Szerver installálás}

Az itt leírt eljárással 
Debian Sarge, 
Ubuntu 4.10, 
SuSE 9.0,
SuSE 10.0
32-bites Linuxokra tudtam Oracle-t installálni.

\subsubsection*{Kezdeti állapot}

Ellenőrizzük, hogy az alábbi csomagok installálva
legyenek a rendszerünkön:

gcc, make, binutils, motif, lesstif, rpm

A hiányzó csomagokat installáljuk.
A libmotif3 csomag Ubuntun a multiverse-ből installálható.
SuSE-n csak libmotif2 van, lesstif pedig csak forrásból volna beszerezhető.
Mindenesetre nem értem, hogy minek lesstif, ha van motif, és minek
egyáltalán akármelyik, amikor a GUI megjelenítését Jáva végzi.
SuSE 9.0 esetében a linker jelezni fogja a motif hiányát, 
de attól a rendszer még használható lesz.
SuSE 10.0-án a telepítő hiányolni fogja a gcc 2.96-ot,
de az installáció a gcc 4.x-szel is működni fog.
Kezdetben (a runInstaller elindításáig) root-ként dolgozunk.


\subsubsection*{Csoportok, felhasználók}

Létrehozzuk az alábbi csoportokat és felhasználókat:

\begin{verbatim}
    id nobody  # léteznie kell
    groupadd oinstall
    groupadd dba
    useradd -g oinstall -G dba  -d /opt/oracle oracle
    passwd oracle #jelszót adunk az oracle usernek
\end{verbatim}

A továbbiakban az installációt az oracle felhasználó
nevében csináljuk.


\subsubsection*{Oracle letöltés}

Elhozzuk az Oracle-től a \verb!ship.db.cpio.gz! csomagot
és alkalmas helyen kibontjuk.

\begin{verbatim}
    gunzip ship.db.cpio.gz
    cpio -idmv < ship.db.cpio
\end{verbatim}

\subsubsection*{Base directory}

Létrehozzuk az Oracle base directoryt:

\begin{verbatim}
    mkdir /opt/oracle/base
\end{verbatim}



\subsubsection*{/etc/profile}

Beírjuk /etc/profile-ba:


\begin{verbatim}
    export ORACLE_BASE=/opt/oracle/base
    export ORACLE_HOME=$ORACLE_BASE/product/10g
    export TNS_ADMIN=$ORACLE_HOME/network/admin
    export ORA_NLS33=$ORACLE_HOME/nls/data
    export NLS_LANG=AMERICAN_AMERICA.WE8ISO8859P1
    export NLS_DATE_FORMAT="YYYY-MM-DD"

    export ORACLE_SID=test

    PATH=$PATH:$ORACLE_HOME/bin
    LD_LIBRARY_PATH=$LD_LIBRARY_PATH:$ORACLE_HOME/lib

    export PATH
    export LD_LIBRARY_PATH
\end{verbatim}


\subsubsection*{/etc/sysctl.conf}

Beírjuk /etc/sysctl.conf-ba:

\begin{verbatim}
    kernel.shmall = 2097152
    kernel.shmmax = 2147483648
    kernel.shmmni = 4096
    kernel.sem = 250 32000 100 128
    fs.file-max = 65536
    net.ipv4.ip_local_port_range = 1024 65000
\end{verbatim}

Utána futtatjuk: \verb!/sbin/sysctl -p!


\subsubsection*{/etc/security/limits.conf}

Beírjuk /etc/security/limits.conf-ba:

\begin{verbatim}
    *               soft    nproc   2047
    *               hard    nproc   16384
    *               soft    nofile  1024
    *               hard    nofile  65536
\end{verbatim}


\subsubsection*{/etc/pam.d/login}

Beírjuk /etc/pam.d/login-ba:

\begin{verbatim}
    session    required     /lib/security/pam_limits.so
\end{verbatim}


\subsubsection*{Symlinks}

Megcsináljuk a következő symlinkeket:

\begin{verbatim}
    ln -s /usr/bin/awk /bin/awk
    ln -s /usr/bin/rpm /bin/rpm
    ln -s /usr/bin/basename /bin/basename
    ln -s /etc /etc/rc.d
\end{verbatim}


\subsubsection*{RedHat emuláció}

Létrehozzuk a \verb!/etc/redhat-release!
filét a következő tartalommal:

\begin{verbatim}
    Red Hat Linux release 2.1 (drupal)
\end{verbatim}


\subsubsection*{runInstaller}

Eddig root-ként dolgoztunk, most váltunk oracle-re,
és oracle-ként elindítjuk  a runInstaller-t:

\begin{verbatim}
    cd .../Disk1
    ./runInstaller
\end{verbatim}

Itt hosszú ideig dolgozik, fordítgat,
eközben adódhatnak hibák, ezek némelyike figyelmen
kívül hagyható.

Az installer felszólít a \verb!root.sh! script
futtatására, megtesszük neki. Ezután vár 10 percet
a cssd démon elindulására. Vagy várunk 10 percet,
vagy küldünk egy HUP-ot az init-nek, amire az újraolvassa
a konfigurációs filéit. 

A root.sh script beír a /etc/inittab végére egy ilyen sort:

\begin{verbatim}
    h2:35:respawn:/etc/init.d/init.cssd run >/dev/null 2>&1 </dev/null
\end{verbatim}

A cssd démon az ASM (Automatic Storage Manager) modullal
tart fenn valamilyen kommunikációt. Nálunk nincs ASM,
tehát az inittab-ba írt sort kiszedhetjük, ha az installer
túljutott rajta (többet nem kell).


\subsubsection*{/etc/oratab}


Beírjuk /etc/oratab-ba:

\begin{verbatim}
    test:/opt/oracle/base/product/10g:Y
\end{verbatim}


\subsubsection*{/etc/init.d/oracle}

Csinálunk egy \verb!/etc/init.d/oracle! filét
a következő tartalommal:


\begin{verbatim}
#!/bin/bash
#
# Run-level Startup script for the Oracle Instance and Listener

ORA_HOME="/opt/oracle/base/product/10g"
ORA_OWNR="oracle"

# if the executables do not exist -- display error

if [ ! -f $ORA_HOME/bin/dbstart -o ! -d $ORA_HOME ]
then
        echo "Oracle startup: cannot start"
        exit 1
fi

# depending on parameter -- startup, shutdown, restart 
# of the instance and listener or usage display 

case "$1" in
    start)
        # Oracle listener and instance startup
        echo -n "Starting Oracle: "
        /sbin/sysctl -p
        su - $ORA_OWNR -c "$ORA_HOME/bin/lsnrctl start"
        su - $ORA_OWNR -c $ORA_HOME/bin/dbstart
        touch /var/lock/subsys/oracle
        echo "OK"
        ;;
    stop)
        # Oracle listener and instance shutdown
        echo -n "Shutdown Oracle: "
        su - $ORA_OWNR -c "$ORA_HOME/bin/lsnrctl stop"
        su - $ORA_OWNR -c $ORA_HOME/bin/dbshut
        rm -f /var/lock/subsys/oracle
        echo "OK"
        ;;
    reload|restart)
        $0 stop
        $0 start
        ;;
    *)
        echo "Usage: $0 start|stop|restart|reload"
        exit 1
esac
exit 0
\end{verbatim}


Ezután a szokásos módon vezérelhető a szerver:

\begin{verbatim}
    /etc/init.d/oracle start|stop|restart
\end{verbatim}


\subsubsection*{Hálózat}

A szerveren az \verb!$ORACLE_HOME/network/admin/listener.ora!
filében van leírva a listenerek paraméterezése. Az installáció
után ez általában ,,magától'' jó.

A kliens programok számára az \verb!$ORACLE_HOME/network/admin/sqlnet.ora!
filében fel vannak sorolva a protokollok, amikkel hálózati szolgáltatásokat
tudnak keresni. Nálam ez így néz ki:
\begin{verbatim}
# SQLNET.ORA Network Configuration File
NAMES.DEFAULT_DOMAIN = comfirm.x
NAMES.DIRECTORY_PATH= (TNSNAMES, ONAMES, HOSTNAME)
\end{verbatim}
Tehát elsősorban a tnsnames filét nézzük,
ha egy adatbázist keresünk a hálózaton.
A  \verb!$ORACLE_HOME/network/admin/tnsnames.ora!
filé egy kis adatbázist tartalmaz a hálózaton található Oracle
szolgáltatásokról. Ahhoz, hogy az Oracle hálózati kliens megtalálja
a ,,test''  adatbázis szolgáltatást (ez volna a SID?), el kell helyezni 
a tnsnames.ora filébe egy ilyen bekezdést:
\begin{verbatim}
test.COMFIRM.X =
  (DESCRIPTION =
    (ADDRESS_LIST =
      (ADDRESS = (PROTOCOL = TCP)(HOST = 1g.comfirm.x)(PORT = 1521))
    )
    (CONNECT_DATA =
      (SERVER = DEDICATED)
      (SERVICE_NAME = test)
    )
  )
\end{verbatim}
A domain, host és service neveket a konkrét esetnek
megfelelően kell beírni.


\subsubsection*{Tablespace, konto schema}

A jelen demóban az adatbázis objektumok
a konto schema-ban jönnek létre. Legjobb ezt a
schema-t mindjárt a telepítés után létrehozni.

Elindítjuk  (system-ként) az sqlplus-t, és
végrehajtjuk benne a következőket:

\begin{verbatim}
create tablespace konto logging
    datafile '/opt/oracle/base/oradata/test/konto.dbf'
    size 8M reuse autoextend on next 4M
    maxsize unlimited
    extent management local autoallocate
    segment space management auto;

create user konto
    identified by konto
    default tablespace konto
    quota 100000M on konto;

grant connect to konto;

quit;
\end{verbatim}

Ezután az Oracle adatbázis alkalmas a demó programok kiszolgálására.

\subsection{Kliens installálás}

Ha egy gépre Oracle szervert installálunk, azon 
lesz  Oracle kliens is. Ha azonban csak a kliensre van szükségünk,
akkor nem érdemes 100-szoros munkával szervert telepíteni.

Letöltjük az Oracle-től az {\em Oracle Instant Client\/} csomagokat.
A szoftver 4 darab zip filéből áll (egyes platformokra rpm is van).
A 4 közül a CCC-hez csak két összetevőre van szükség, a basic-re 
és az sqlplus-ra. A zip-eket kibontjuk, és a tartalmukat betesszük
a \verb!/opt/oracle/instantclient! (vagy egy tetszés szerint választott) 
directoryba.

Megcsináljuk a \verb!/opt/oracle/instantclient/network/admin!
directoryt, és abban elhelyezzük az alábbi két filét:

A \verb!sqlnet.ora! filé tartalma:
\begin{verbatim}
# SQLNET.ORA Network Configuration File
NAMES.DEFAULT_DOMAIN = comfirm.x
NAMES.DIRECTORY_PATH= (TNSNAMES)
\end{verbatim}

A \verb!tnsnames.ora! ilyen bekezdésekből áll:
\begin{verbatim}
test.COMFIRM.X =
  (DESCRIPTION =
    (ADDRESS_LIST =
      (ADDRESS = (PROTOCOL = TCP)(HOST = 1g.comfirm.x)(PORT = 1521))
    )
    (CONNECT_DATA =
      (SERVER = DEDICATED)
      (SERVICE_NAME = test)
    )
  )
\end{verbatim}
Természetesen a domain, host, service neveket a saját adatainkkal
helyettesítjük.

A \verb!/etc/profile!-ba beírjuk:
\begin{verbatim}
export ORACLE_HOME=/opt/oracle/instantclient
export PATH=$PATH:$ORACLE_HOME
export LD_LIBRARY_PATH=$LD_LIBRARY_PATH:$ORACLE_HOME
export TNS_ADMIN=$ORACLE_HOME/network/admin
export NLS_LANG=AMERICAN_AMERICA.WE8ISO8859P1
#export NLS_LANG=HUNGARIAN_HUNGARY.EE8ISO8859P2
export NLS_DATE_FORMAT="YYYY-MM-DD"
\end{verbatim}
Ezután újonnan bejelentkezve fut az sqlplus, és futnak a CCC programok.

Ha az sqlnet.ora és tnsnames.ora filék elhelyezkedése
az \verb!ORACLE_HOME!-hoz képest olyan, mint a fenti példában,
akkor a \verb!TNS_ADMIN! változó megadása felesleges (de nem árt).

Az adatbázisok általában Latin-1 (WE8ISO8859P1) kódkészlettel 
jönnek létre, mert ez a default. Ha egy ilyen adatbázissal állunk
kapcsolatban, de a kliensen Latin-2 (EE8ISO8859P2) kódkészlet van beállítva,
akkor a szövegmezőkben elromlanak a 128 feletti byteok.
(Ha kiírunk egy 128 feletti betűkből álló stringet, 
akkor a  visszaolvasás a legtöbb helyen kérdőjelet ad.)
Ezért a kliensen kötelezően ugyanazt a kódkészletet
kell beállítani, mint ami a szerveren van, akkor is, ha így
a szerver üzeneteit magyar helyett esetleg angolul kapjuk.





