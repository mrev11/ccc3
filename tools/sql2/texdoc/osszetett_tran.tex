
\section{Összetett tranzakciók}

Nézzük ezt a (pszeudo)programot:

\begin{verbatim}
    function konyveles1()
        begin
            könyvelgetünk különféle számlákra
            if( sikertelen ) 
                break(konyverrorNew())
            end
            tovább könyvelgetünk különféle számlákra
            if( sikertelen ) 
                break(konyverrorNew())
            end
            //minden rendben
            commit
        recover e <konyverror>
            //a hibát elkapjuk
            rollback
        end
\end{verbatim}
Minden príma, szépen lekezeltük a lehetséges hibákat.
Tegyük fel, hogy van egy hasonló stílusban megírt 
\verb!konyveles2()! függvényünk is, és a kétféle tranzakció
kombinálásával létre akarunk hozni egy harmadik, összetett 
(még összetettebb) tranzakciót.
\begin{verbatim}
    function osszetett_konyveles() //HIBÁS!
        begin
            konyveles1()
            konyveles2()
            commit
        recover e <konyverror>
            rollback
        end
\end{verbatim}
Rá kell jönnünk, hogy az összetett tranzakció fenti implementációja
rossz, méghozzá a résztranzakciókban levő commitok miatt.
Át kell szerveznünk a programot. Először megírjuk az 
\verb!alg_konyveles1()! függvényt, ami a könyvelés algoritmusát
(az üzleti logikát) tartalmazza.
\begin{verbatim}
    function alg_konyveles1()
        könyvelgetünk különféle számlákra
        if( sikertelen ) 
            break(konyverrorNew())
        end
        tovább könyvelgetünk különféle számlákra
        if( sikertelen ) 
            break(konyverrorNew())
        end
        //minden rendben (mégsem commitolunk)
\end{verbatim}
Azután az algoritmust tranzakciós keretek közé tesszük:
\begin{verbatim}
    function trn_konyveles1()
        begin
            alg_konyveles1()
            commit
        recover e <konyverror>
            rollback
        end
\end{verbatim}
Azaz szétválasztottuk az algoritmust és a tranzakciókezelést.
Ha ezt a módszert alkalmazzuk az összetett könyvelésünk
leprogramozására, a következő eredményhez jutunk.
\begin{verbatim}
    function alg_osszetett_konyveles()
        alg_konyveles1()
        alg_konyveles2()

    function trn_osszetett_konyveles() 
        begin
            alg_osszetett_konyveles()
            commit
        recover e <konyverror>
            rollback
        end
\end{verbatim}
Tehát minden tranzakciónak külön meg kell írni az
algoritmikus részét és a tranzakciókezelő keretét.
A tranzakciókat végrehajtó programok nyilván a
\verb!trn_*! függvényeket fogják hívni.
Valaki talán azt gondolja, hogy ez plusz munkával jár,
valójában ennek éppen az ellenkezője igaz, így ugyanis újra 
felhasználható kódot kapunk.

 
Azt is érdemes megfontolni, hogy az előző példákban
a commit/rollback helyén nem feltétlenül csak egyetlen adatbáziskapcsolatra
kiadott commit/rollback parancs áll. Az alkalmazás tudja, hogy
az általa aktivizált tranzakció milyen adatbázisokat használ,
és {\em minden\/} adatbáziskapcsolatra kiadja a commit/rollback parancsot.
Ezzel a technikával áttekinthető módon lehet kezelni
a több adatbázisra kiterjedő elosztott tranzakciókat.


