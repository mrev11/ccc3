
\section{Osztály referencia}

\subsection{Sqlconnection osztály}

Aszerint, hogy melyik névtérből hívjuk meg az sqlconnectionNew()
objektumgyártó függvényt, Oracle vagy Postgres adatbáziskapcsolathoz
jutunk:
\begin{verbatim}
    con_ora:=sql2.oracle.sqlconnectionNew(connect_string)
    con_pg:=sql2.postgres.sqlconnectionNew(connect_string)
\end{verbatim}
A \verb!connect_string! paraméter opcionális. Oracle esetében
a connect string tartalma a megszokott \verb!user@dbsid/password! alakú, 
ahogy azt pl. az sqlplus is várja. Postgres esetén a connect string
tartalma pontosan az lehet, mint amit a \verb!PQconnectdb! függvény
(a libpq klienskönyvtárból) elfogad. Ha a connect string
nincs egyáltalán megadva, akkor a program az \verb!ORACLE_CONNECT!, illetve
\verb!POSTGRES_CONNECT! környezeti változók tartalmát használja
bejelentkezéshez, például:
\begin{verbatim}
    export ORACLE_CONNECT=scott@database/tiger
    export POSTGRES_CONNECT="host=HH dbname=DD user=UU password=PP"
\end{verbatim}
A sqlconnection objektum létrehozásakor azonnal megtörténik
a bejelentkezés. Az attribútum/metódusok:

\begin{description}
\item[driver]
    Osztályváltozó, ami a konstruktor névterét tartalmazza.
\item[duplicate]
    Létrehoz egy másik sqlconnection objektumot, 
    amiben a connect infó ugyanaz, mint az elsőben, 
    tehát ugyanoda is kapcsolódik.
    Visszatérés: az új \verb!sqlconnection! objektum.
\item[sqlcommit]
    Végrehajt egy commit utasítást, egyúttal
    automatikusan indítja a következő tranzakciót.
    Visszatérés: NIL.
\item[sqldisconnect]
    Megszünteti az adatbáziskapcsolatot. 
    Az alkalmazási programnak ügyelnie kell arra,
    hogy a kapcsolatok ne halmozódjanak fel,
    mert akkor elfogynak bizonyos erőforrások (handlerek).
    Ezért minden feleslegessé váló kapcsolatot explicite
    meg kell szüntetni.
    Visszatérés: NIL.
\item[sqlexec(stmt,bind)]
    Végrehajt egy tetszőleges SQL utasítást.
    Az \verb!stmt! utásítás \verb!:1!, \verb!:2!, \ldots
    alakú formális paraméterei helyére behelyettesíti
    az (opcionális) \verb!bind! arrayből vett megfelelő indexű értékeket.
    Ha az utasítás nem hajtható végre, vagy hibás,
    akkor \verb!sqlerror!-ból származtatott futási hiba keletkezik.
    Visszatérés: az érintett (törölt/módosított) rekordok száma.
\item[sqlexecx(stmt,bind)]
    Végrehajt egy tetszőleges SQL utasítást.
    Az \verb!stmt! utásítás \verb!:1!, \verb!:2!, \ldots
    alakú formális paraméterei helyére behelyettesíti
    az (opcionális) \verb!bind! arrayből vett megfelelő indexű értékeket.
    Ha az utasítás nem hajtható végre, vagy hibás,
    akkor visszaadja a hiba leírását tartalmazó error objektumot,
    egyébként NIL-t ad.
\item[sqlisolationlevel(level,flag)]
    Beállítja a tranzakció izolációs szintet.
    Az első paraméter helyén az \verb!ISOL_SERIALIZABLE!
    vagy \verb!ISOL_READ_COMMITED! konstansokat lehet megadni. 
    A második paraméter egy opcionális flag. 
    Ha \verb!flag==.f.! (ez a default),  az izolációs szint csak ideiglenesen, 
    az adott tranzakción belül lesz átállítva. 
    Ha \verb!flag==.t.!, akkor a beállítás hatálya az egész session.
    Visszatérés: NIL.
\item[sqlquerynew(query,bind)]
    A \verb!con:sqlquerynew(query,bind)! metódushívás
    gyárt egy sqlquery objektumot (lásd alább). 
    A \verb!query! select utasítás \verb!:1!, \verb!:2!, \ldots
    alakú formális paraméterei helyére behelyettesíti
    az (opcionális) \verb!bind! arrayből vett megfelelő indexű értékeket.
    Ezzel tetszőleges  SQL lekérdezés végrehajtható.
    Visszatérés: az új \verb!sqlquery! objektum.
\item[sqlrollback]  %{|this|sqlrollback(this)})
    Végrehajt egy rollback utasítást, egyúttal
    automatikusan indítja a következő tranzakciót.
    Visszatérés: NIL.
\item[sqlsequencenew(name)]
    A \verb!con:sqlsequencenew(name)! metódushívás
    gyárt egy sqlsequence objektumot (lásd alább).
    Visszatérés: az új  \verb!sqlsequence! objektum.
\item[version]
    Megadja az adatbázisszerver verzióját (szövegesen).
\end{description}

\subsection{Sqlquery osztály}

Az sqlquery objektumok a \verb!con:sqlqueryNew(select_stmt)!
objektumgyártó metódushívással keletkeznek. A metódusok:

\begin{description}
\item[close]
    Lezárja az sqlquery objektumot.
    Az alkalmazásoknak ügyelnie kell arra,
    hogy a lezáratlan sqlquery objektumok ne halmozódjanak fel,
    másképp elfogynak bizonyos erőforrások (handlerek).
    Ezért minden feleslegessé váló sqlquery objektumot 
    explicite le kell zárni. Visszatérés: NIL.
\item[connection]
    Tárolja magában az adatbáziskapcsolatot, 
    amivel létrehozták.
\item[fcount]
    Megadja a select utasítással keletkező eredménytábla
    oszlopainak számát.
\item[findcolumn(name)]
    Neve alapján megmondja az oszlop indexét az oszlopok listájában.
    Ha a megadott nevű oszlop nem létezik, az eredmény 0.
\item[fname(x)]
    Indexe alapján megmondja az oszlop nevét.
\item[isnull(x)]
    A \verb!q:isnull(x)! metódushívás megmondja, hogy
    az x-szel azonosított oszlop utoljára fetch-elt értéke
    null volt-e, x az oszlop neve vagy indexe lehet.
\item[getbinary(x)]
    A \verb!q:getbinary(x)! metódushívás kiolvassa, 
    és binary string (X) típusban visszaadja
    az x-szel azonosított oszlop utoljára fetch-elt értékét,
    x az oszlop neve vagy indexe lehet.
\item[getchar(x)]
    A \verb!q:getchar(x)! metódushívás kiolvassa, és
    stringre (C típusra) konvertálva visszaadja
    az x-szel azonosított oszlop utoljára fetch-elt értékét,
    x az oszlop neve vagy indexe lehet.
\item[getdate(x)]
    A \verb!q:getdate(x)! metódushívás kiolvassa, és
    dátumtípusra konvertálva visszaadja 
    az x-szel azonosított oszlop utoljára fetch-elt értékét,
    x az oszlop neve vagy indexe lehet.
\item[getlogical(x)]
    A \verb!q:getlogical(x)! metódushívás kiolvassa, és
    logikai típusra konvertálva visszaadja
    az x-szel azonosított oszlop utoljára fetch-elt értékét,
    x az oszlop neve vagy indexe lehet.
\item[getnumber(x)]
    A \verb!q:getnumber(x)! metódushívás kiolvassa, és
    számtípusra konvertálva visszaadja 
    az x-szel azonosított oszlop utoljára fetch-elt értékét,
    x az oszlop neve vagy indexe lehet.
\item[next]
    Beolvassa (fetch) a lekérdezés következő eredménysorát.
    Visszatérése \verb!.t.!, ha volt még sor, egyébként (miután a 
    sorok elfogytak) \verb!.f.!.\  Az eredménytáblán
    egyszer, előrefelé lehet a \verb!next!-ekkel  végighaladni.
\end{description}


\subsection{Sqlsequence osztály}

A connection osztály sqlsequencenew metódusával kapunk új
sqlsequence objektumot: \verb!con:sqlsequenceNew(name)!,
ahol \verb!name! az adatbázisbeli sequence objektum neve. 
Ne feledjük, az objektumgyártó csak programobjektumot készít, 
az adatbázisban van, vagy nincs hozzá tartozó sequence objektum.
Az attribútum/metódusok:

\begin{description}
\item[cache]
    A cache attribútum default értéke 1.
    A \verb!seq:create! metódus hatására olyan adatbázisbeli 
    sequence keletkezik, ami egyszerre cache darabszámú
    értéket generál az alkalmazás számára. Ezzel a sequence
    használata gyorsul, viszont kihasználatlanul maradhatnak
    értékek.
\item[connection]
    Tárolja az adatbáziskapcsolatot.
\item[create]
    Létrehozza az adatbázisban a sequence objektumot.
    A sequence összes jellemzőjének figyelembevételével
    generál egy \verb!create sequence! utasítást és elküldi azt a
    szervernek. Visszatérés: NIL.
\item[cycle]
    Ha a \verb!cycle! attrubútum értéke \verb!.t.!, 
    akkor a olyan adatbázisbeli sequence keletkezik 
    (a seq:create metódus hatására),
    ami új kört kezd, ha eléri a minimális/maximális értékét.
\item[drop]
    Megszünteti a sequence-t az adatbázisban (küld egy
    \verb!drop sequence! utasítást a szervernek).
    Visszatérés: NIL.
\item[increment]
    A kreáláskor keletkező sequence növekménye.
    Az attribútum értékével nő (csökken) a sequence 
    a \verb!nextval! metódus végrehajtásakor, a default érték 1.
\item[maxvalue]
    A sequence maximális értéke.
\item[minvalue]
    A sequence minimális értéke.
\item[name]
    A sequence objektum adatbázisbeli neve. Megadhatjuk 
    az objektumgyártó függvényben, vagy utólag egyszerű értékadással.
    Az adatbázisban a sequence-ek egyedi névvel kell rendelkezzenek,
    tehát nem lehet egy sequence-nek és pl. egy adattáblának azonos neve.
    Ha a sequence már létezik az adatbázisban, akkor azt rögtön
    használhatjuk (\verb!nextval!), vagy eldobhatjuk (\verb!drop!). 
    Ha még nem létezik, akkor előbb létre kell hozni (\verb!create!).
\item[nextval]
    Lépteti a sequence-t, egyúttal szolgáltatja az új értéket.
\item[start]
    Beállítja a létrehozandó sequence kezdőértékét. 
    Növekvő  sequence-eknél a default kezdőérték \verb!minvalue!.
    Csökkenő sequence-eknél a default kezdőérték \verb!maxvalue!.
\end{description}




\subsection{Absztrakt tableentity osztály}

Az absztrakt tableentity osztály a közös őse minden
konkrét tableentitynek. Az alábbi felsorolás csak 
az alkalmazási programok számára érdekes attribútumokat/metódusokat
tartalmaza.

\begin{description}
\item[connection]
    Az adatbáziskapcsolat objektumot tartalmazza.
\item[create]
    Kreál egy adatbázistáblát, aminek a struktúrája megegyezik
    a tableentity struktúrájával. Ugyancsak kreálja a tableentityben
    definiált indexeket.
    Értelemszerűen csak akkor alkalmazható, 
    ha a tableentityben nincs join, 
    azaz a tablistnek csak egyetlen eleme van.
    Visszatérés: NIL.
\item[delete(rowentity)]
    Csak akkor alkalmazható, ha a tableentityben nincs join, 
    azaz a tablistnek egyetlen eleme van. Ha ez az egyetlen elem egy view, 
    akkor annak jellegétől függően egyes esetekben működik, más esetekben nem.
    A \verb!t:delete(r)! törli a tableentity által reprezentált
    tábla mindazon sorait, melyekben a primary key oszlopok
    az \verb!r! rowentity objektumból kiolvasható kulccsal egyeznek.
    Persze a koncepció az, hogy pontosan egy sor törlődjön,
    de ez csak akkor lesz így, ha a primary key valóban  egyedi. 
    Nem volna jó, ha az ilyen hiba rejtve maradna, ezért, ha a törölt 
    sorok száma nem pontosan 1, akkor metódus \verb!sqlrowcounterror! 
    kivételt dob.  Látható, hogy milyen fontos a primary key
    egyediségére ügyelni, máskülönben egy óvatlan update vagy delete
    könnyen nem várt eredményre vezet. Az egyediséget úgy lehet
    biztosítani, hogy kreálunk egy unique minősítésű indexet,
    ami éppen a primary key oszlopaiból áll. Ekkor az adatbázisszerver
    megakadályozza ismétlődő kulcsok bevitelét. 
    Visszatérés: a törölt sorok száma. 
\item[drop]
    Megszünteti az adattáblát, 
    amit a tableentity reprezentál. 
    Csak akkor alkalmazható, ha a tablistnek egyetlen eleme van,
    és az egy valódi tábla (nem pedig view).
    Visszatérés: NIL.
\item[getcolumn(name)]
    Az oszloplistából kikeres egy oszlopdefiníciót a neve alapján.
    Visszatérés: a \verb!columndef! objektum, vagy NIL.
\item[getprimarykey(rowentity)]
    A \verb!key:=t:getprimarykey(r)! hívás kiolvassa 
    az \verb!r! rowentity objektumból az elsődleges kulcsot, 
    és visszaad  egy arrayt, ami a klucsszegmenseket tartalmazza. 
    Ezzel a kulccsal hívható a leszármazott tableentity objektumok 
    \verb!find! metódusa.
\item[insert(rowentity)]
    A \verb!t:insert(r)! berak a tableentity által reprezentált
    adattáblába egy új sort, melyet
    az \verb!r! rowentity objektum adataival tölt fel.
    Csak akkor alkalmazható, ha a tableentityben nincs join,
    és a tablist egyetlen tagja egy elemi tábla (nem pedig view).
    Csak azok az attribútumok (mezők) jelennek meg a szervernek küldött
    insert utasításban, amiknek a null flagje \verb!.f.! 
    (ezt általában az értékadások állítják be).
    Visszatérés: a berakott sorok száma (ami normális esetben 1).
\item[instance]
    Ad egy üresre inicializált rowentity objektumot.
    Az új objektum minden attribútuma \verb!.f.! dirty flaggel
    és \verb!.t.! null flaggel rendelkezik.
\item[setdirty(rowentity,flag)]
    A \verb!rowentity! objektum összes attribútumának dirty flagjét 
    beállítja \verb!flag!-ra.
    A dirty flag jelentősége, hogy az \verb!update!
    metódus csak azokat a mezőértékeket küldi a szervernek,
    amiknek a dirty flagje \verb!.t.!-re van állítva. 
    A dirty flag nyilvántartását általában az objektumra bízhatjuk, 
    néha azonban értelme lehet kívülről belenyúlni a nyilvántartásba. 
    Visszatérés: NIL.
\item[setnull(rowentity,flag)]
    A \verb!rowentity! objektum összes attribútumának null flagjét 
    beállítja \verb!flag!-ra. A flag \verb!.t.! értéke jelenti,
    hogy az attribútumok (mezők) null értéken vannak.
    A null flag nyilvántartását általában rábízhatjuk az objektumra, 
    néha azonban értelme lehet kívülről belenyúlni a nyilvántartásba. 
    Visszatérés: NIL.
\item[show(rowentity)]
    A \verb!t:show(r)! hívás kilistázza az \verb!r! rowentity objektumot.
    Visszatérés: NIL.
\item[tablist]
    Egy array, ami a táblák (vagy view-k) felsorolását
    tartalmazza a következő alakban:
\begin{verbatim}
t:tablist:={"tab1=alias1","tab2=alias2",...}
\end{verbatim}
    ahol 
    tab1,\ldots az adattáblák valódi (minősített) neve,
    alias1,\ldots pedig a táblák hivatkozási neve.
    A tableentity objektum többi részében, ahol az szükséges
    (pl. amikor meg kell jelölnünk, hogy egy oszlop melyik táblából való)
    mindig a tábla alias nevét használjuk. Ennek eredményeként
    egy tableentity osztály invariáns a benne szereplő táblákra,
    azaz a tablist attribútum átírásával ugyanaz az objektum
    és adatstruktúra más fizikai táblákra is alkalmazható.
    
    Az alias nevek alkalmazása nem kötelező, nélkülük azonban
    elvész az invariancia. Ha a tableentity csak egyetlen táblát
    tartalmaz, akkor a tableentity definícióban sehol sincs
    szükség táblahivatkozásra, hiszen mindig
    a tablist egyetlen eleméről lehet csak szó, ilyenkor
    sem szükséges alias nevet használni.
\item[update(rowentity)]
    Csak akkor alkalmazható, ha a tableentityben nincs join, 
    azaz a tablistnek egyetlen eleme van. Ha ez az egyetlen elem egy view, 
    akkor annak jellegétől függően egyes esetekben működik, más esetekben nem.
    A \verb!t:update(r)! módosítja a tableentity által reprezentált
    tábla mindazon sorait, melyekben a primary key oszlopok
    az \verb!r! rowentity objektumból kiolvasható kulccsal egyeznek.
    Persze a koncepció az, hogy csak egy sor módosuljon,
    de ez csak akkor fog teljesülni, ha a primary key valóban egyedi.
    Nem volna jó, ha az ilyen hiba rejtve maradna, ezért, ha a módosult
    sorok száma nem pontosan 1, akkor a metódus \verb!sqlrowcounterror! 
    kivételt dob.  Csak azok az attribútumok (mezők) jelennek meg
    a szervernek küldött update utasításban,
    amiknek a dirty flagje \verb!.t.! (ezt általában az értékadások állítják be).
    Visszatérés: a módosult sorok száma. 
\item[zap]
    Törli a tableentity által reprezentált adattábla minden sorát.
    Csak akkor alkalmazható, ha a tablistnek egyetlen eleme van,
    és az egy valódi tábla (nem pedig view).
\end{description}


\subsection{Tableentity leszármazottak}

Az absztrakt tableentity osztályt  oszlopokkal és 
select metódusokkal bővítve kapjuk a konkrét tableentity
osztályokat. Ezek kódját a gyakorlatban nem kézzel írjuk, 
hanem XML leírásból, vagy tds scriptből  programmal generáljuk. 
A generált kód a tableentityre jellemző névtérbe helyezi 
az objektumgyártó függvényt, amit így hívhatunk meg:
\begin{verbatim}
    tableentity:=multi.level.namespace.tableentityNew(con)
\end{verbatim}

A \verb!__method__! alakú metódusokat
az alkalmazási programok közvetlenül nem használják, 
csak a tanulság kedvéért szerepelnek az ismertetésben.

\begin{description}

\item[\_\_primarykey\_\_]
    Osztályváltozó, egy array, ami a primary key-t alkotó
    oszlopokat sorolja fel. A koncepció fontos eleme, hogy a
    primary key egyedileg azonosítja a rowentity objektumokat.
    Ha valami miatt ez nem teljesül, akkor egy update, vagy
    delete metódus egy helyett több objektumot is módosíthat,
    vagy törölhet. A primary key egyediségét úgy lehet
    biztosítani, hogy kreálunk egy unique minősítésű indexet,
    ami éppen a primary key oszlopaiból áll. Ekkor az adatbázisszerver
    megakadályozza ismétlődő kulcsok bevitelét. 


\item[\_\_rowclassid\_\_]
    Nem módosítható osztályváltozó, a tableentityhez tartozó 
    rowentity osztály  egészszám azonosítóját tartalmazza.

\item[\_\_rowclassname\_\_]
    Nem módosítható osztályváltozó, a tableentityhez tartozó 
    rowentity osztály nevét tartalmazza.

\item[\_\_tabjoin\_\_]
    Nem módosítható osztályváltozó. Ha a tableentity több táblát
    tartalmaz, akkor ezeket általában össze kell valahogy kapcsolni.
    A tabjoin attribútum tartalmazza az SQL from záradék szövegét, 
    ami megadja az összekapcsolás módját. Például a p és q alias 
    nevű  táblák összekapcsolása a számlaszám oszlopokkal így
    nézhet ki:
\begin{verbatim}
p left join q on szamlaszam=qszamlaszam
\end{verbatim}

\item[columnlist]
    Osztályváltozó, az oszlopok leírását tartalmazó array.
    Minden oszlopot egy \verb!columndef! objektum képvisel  az arrayben.

\item[find(keylist,lock)]
    Az \verb!r:=t:find({keyseg1,keyseg2,...})! metódushívás
    veszi az alaptábla azon sorát, ami a megadott kulcshoz illeszkedik,
    és létrehoz vele egy új rowentity objektumot. Ha nincs sor
    a megadott kulccsal, akkor az eredmény NIL.
    A \verb!find! metódus használatának másik módja \verb!r1:=t:find(r0)!,
    ahol a kulcsot a mintaként használt \verb!r0! objektumból vesszük. 
    Itt r0 és r1 egymástól függetlenül létező objektumok, 
    nem csak egymás referenciái.

    A find és select metódusokkal lehet a tábla kiválasztott sorait 
    lockolni.
    \begin{itemize}
    \item   \verb!r:=t:find(key)! nem lockol.
    \item   \verb!r:=t:find(key,0)! várakozás nélkül lockol. 
    \item   \verb!r:=t:find(key,sec)! sec másodpercnyi várakozással lockol. 
    \item   \verb!r:=t:find(key,-1)! végtelen várakozással lockol. 
    \end{itemize}
    A lockolt sorokat más tranzakció nem lockolhatja, és nem
    módosíthatja. A lockot csak a tranzakció vége (commit/rollback) 
    oldja fel.  Ha a lock sikertelen, \verb!sqllockerror! vagy
    \verb!sqldeadlockerror! kivétel keletkezik.
    
    Megjegyzés: Postgresben nincs timeout támogatás, ezért
    minden lock végtelen ideig, vagy a deadlock detektálásáig vár.

\item[select(,lock)]
    A tableentityknek mindig van egy automatikus select metódusa,
    ami az alaptábla minden sorát szolgáltatja szűrés és rendezés nélkül. 
    Minden selectnek két opcionális paramétere van:
    az első  a bind értékeket felsoroló array,
    a második a lockolást módját előíró szám (mint a find-nál).
    Speciálisan az automatikus selectnek sosincsenek bind változói, 
    ezért az első paramétert sosem használjuk, és mivel ritkán célszerű 
    egy egész táblát egyszerre lockolni, a másodikat is csak ritkán.

\item[select*(bindlist,lock)]
    A kulccsal meghatározott egyetlen sort szolgáltató find 
    és az összes sort szolgáltató select metódus mindig létezik, 
    mert a programgenerátor ezeket automatikusan elkészíti.
    Emellett fel lehet szerelni a tableentityt ún.\
    opcionális selectekkel. Az opcionális selecteket egy
    where záradékkal, és egy order by záradékkal lehet megadni.
    A where záradék sorszámozott paramétereket tartalmazhat, pl.
\begin{verbatim}
szamlaszam like :1
\end{verbatim}
    Legyen az opcionális selectünk neve \verb!select_like!, 
    akkor a metódust így hívhatjuk:
\begin{verbatim}
rowset:=t:select_like({'1111111122222222%'})
\end{verbatim}
    Ez egy olyan rowsetet ad, amiben a számlaszám 1--16 jegyei
    a megadott mintához illeszkednek.

    A select* metódusok első paramétere egy array, amiben 
    a where záradék sorszámozott  paraméterei helyére
    helyettesítendő (bind) értékek vannak felsorolva (az automatikus
    selectnél ezt a paramétert üresen kell hagyni).
    
    A select* metódusok második paraméterével a lockolás
    szabályozható:
    \begin{itemize}
    \item   \verb!rowset:=t:select_x(bind)! nem lockol.
    \item   \verb!rowset:=t:select_x(bind,0)! várakozás nélkül lockol. 
    \item   \verb!rowset:=t:select_x(bind,sec)! sec másodpercnyi várakozással lockol. 
    \item   \verb!rowset:=t:select_x(bind,-1)! végtelen várakozással lockol. 
    \end{itemize}
    Ha a lock sikertelen, \verb!sqllockerror! kivétel keletkezik.
    Ha a szerver deadlockot észlel,  \verb!sqldeadlockerror! kivétel keletkezik.
    A lockolás független a sorok beolvasásától, azaz  a rowset objektum 
    létrejöttekor (még az első next előtt) az összes sor már lockolva van.
    A lockolt sorokat más tranzakció nem lockolhatja és nem módosíthatja. 
    A lockot csak a tranzakció vége (commit/rollback)  oldja fel, 
    pl. nem oldja fel a lockot a rowset lezárása.

    Megjegyzés: Postgresben nincs timeout támogatás, ezért
    minden lock végtelen ideig, vagy a deadlock detektálásáig vár.

\item[version]
    Nem módosítható osztályváltozó, tetszőleges szöveget tartalmazhat,
    általában a tableentity definíció változatszámát tároljuk benne.
\end{description}


\subsection{Rowset osztály}

A rowset objektumokat az alkalmazás sosem közvetlenül hozza létre,
hanem a tableentityk select metódusainak értékeként kapja.
Egy rowset objektum egy feldolgozás alatt álló SQL select utasítást
képvisel. A select utasítással (rowsettel) mindössze két dolgot lehet
csinálni:
\begin{description}
\item[next]
    A next metódus szolgáltatja az eredménytábla következő sorával feltöltött
    rowentity objektumot. A nextekkel a select eredménytábla elejétől
    a végéig egyesével lehet haladni, semmilyen más pozícionálás
    nem lehetséges. Ha a sorok elfogynak, akkor a rowset automatikusan
    lezáródik, és a next minden további hívása rowentity helyett NIL-t ad.
\item[close]
    Lezárja a rowsetet. A lezáratlanul hagyott rowsetek felhalmozódása
    működésképtelenné teszi a programot, mert elfogynak a handlerek,
    ezért az alkalmazásnak minden rowsetet le kell zárni.
\end{description}

\subsection{Rowentity osztályok}

Az adatbázisrekord, vagy az SQL select eredménytábla egy sorának 
programbeli képe a rowentity objektum. Háromféleképpen lehet rowentity
objektumhoz jutni:
\begin{itemize}
    \item \verb!rowentity:=tableentity:instance!
    \item \verb!rowentity:=tableentity:find(key)!
    \item \verb!rowentity:=rowset:next!
\end{itemize}
Az első esetben üresre inicializált objektumot,
a másik két esetben az adatbázisból betöltött (fetch) objektumot kapunk.
Egy új rowentity összes attribútuma (mezeje) kezdetben false dirty flaggel
rendelkezik. Az instance metódussal gyártott rowentityben minden 
attribútum null értéken van, a másik két esetben csak azok a mezők
maradnak null értéken, amiknek az adatbázisban is null értéke van.

A tableentity minden oszlopához létezik a rowentity objektumnak
egy azonos nevű metódusa, amivel az adott mező értéke lekérdezhető
és módosítható. A rowentity objektumok mind saját adatbufferrel
rendelkeznek, emiatt egyszerre több (akárhány) azonos típusú
rowentity objektumunk lehet, amikkel egymástól függetlenül
végezhetünk műveleteket.

Minden rowentity rendelkezik az alábbi metódusokkal: 
\begin{description}
\item[insert]
    Az \verb!r:insert! metódushívás új sort tesz a tableentity
    által reprezentált táblába a rowentity objektum adataival feltöltve.
    Csak azok az oszlopok jelennek meg a kapcsolódó SQL insert utasításban, 
    amiknek a program értéket adott, a többi oszlop null értéken marad.
    Ez a metódushívás csak alternatív formája a tableentitynél látott
    \verb!t:insert(r)! alaknak, ahol t az r-hez tartozó tableentity.
    Visszatérés: az inzertált rekordok száma (1).
\item[update]
    Az \verb!r:update! metódushívás módosítja a tableentity
    által reprezentált tábla azon sorát, amit az r-ből kiolvasható
    primary key azonosít. Csak azok az oszlopok jelennek meg az kapcsolódó
    SQL update utasításban, amiket a program megváltoztatott.
    Ez a metódushívás csak alternatív formája a tableentitynél látott
    \verb!t:update(r)! alaknak, ahol t az r-hez tartozó tableentity.
    Ha a módosított rekordok száma nem 1, akkor \verb!sqlrowcounterror! kivétel 
    keletkezik. Visszatérés: a módosított rekordok száma.
\item[delete]
    Az \verb!r:delete! metódushívás törli a tableentity
    által reprezentált tábla azon sorát, amit az r-ből kiolvasható
    primary key azonosít.
    Ez a metódushívás csak alternatív formája a tableentitynél látott
    \verb!t:delete(r)! alaknak, ahol t az r-hez tartozó tableentity.
    Ha a törölt sorok száma nem 1, akkor \verb!sqlrowcounterror! kivétel 
    keletkezik. Visszatérés: a törölt sorok száma.
\item[find]
    Az \verb!r1:=r0:find! metódushívás behozza azt az objektumot,
    amit a mintaként használt r0-ból kiolvasható primary key
    azonosít. Ugyanaz, mint az \verb!r1:=t:find(r0)! alak.
\item[show]
    Az \verb!r:show! metódushívás kilistázza az objektumot, ugyanaz, 
    mint a \verb!t:show(r)! alak, ahol t az r-hez tartozó tableentity.
    Visszatérés: NIL.
\end{description}

E metódusok a tableentity azonos nevű metódusaihoz továbbítva
hajtódnak végre, ezért a részletesebb leírást lásd a tableentity
osztálynál.


\subsection{Columnref osztály}

A columnref objektumok nem kerülnek bele a tableentitybe,
csak a tableentitykben levő SQL utasítások elkészítésekor
kapnak segédszerepet azáltal, hogy tárolják egyes
közvetlenül nem használt oszlopok adatbázisbeli azonosítóját.

\begin{description}
\item[name]        
    A oszlop (mező) programbeli neve.
\item[expr]
    Oszlopkifejezés. 
    Tetszőleges SQL kifejezés használható.
    A columnrefek esetében ugyan nincs értelme számított mezőknek,
    de a columndef osztály is örökli ezt az attribútumot.
    Ha nincs megadva, akkor az oszlopkifejezés egyenlő lesz
    az oszlop nevével. Tipikus eset, hogy az oszlop minősített 
    nevét adjuk  meg itt.
\item[columnid]
    Az oszlop adatbázisbeli neve, 
    az oszlopkifejezésből állapítja meg az inicializáló,
    és az update utasítások generálásához használja a rendszer.
\end{description}



\subsection{Columndef osztály}

Columndef objektumokkal az alkalmazási programokban ritkán találkozunk, 
mert az oszlopok általában csak a tableentityk belső működéséhez kellenek.
A tableentity columndef objektumokkal való feltöltését sem közvetlenül 
végezzük, mert ezt az XML definíció alapján generált kód teszi.
Ha azonban vizsgálni kell, hogy a szervertől kapott eredeti érték
null, vagy nem null, mégiscsak a columndef objektumhoz kell fordulnunk.
A columndef az előző columnref osztály leszármazottja,
tehát rendelkezik az előbbi metódusokkal. A további metódusok:

\begin{description}
\item[type]
    Az oszlop Clipper típusát azonosító string:  Cw, Nw[.d], D, L, M.
    A memókat az új M típus jelöli. A típust, szélességet, decimálisok
    számát egybe kell írni, pl. "N16.2".

    Ha egy létező Oracle adatbázis egy oszlopát le akarjuk
    képezni Clipper típusra, akkor választanunk kell a 
    Cw, Nw[.d], D, L, M típusok közül. A konverziót az adatbázisszerver
    fogja végezni, ha lehetséges, ha pedig nem lehetséges, 
    akkor runtime errort kapunk. Az ,,értelmes'' konverziók általában 
    lehetségesek.

\item[notnull]
    Egy flag, ami mutatja, hogy kötelező-e értéket adni az oszlopnak.
    A feltételt az adatbázisszerver (nem az interfész!)
    ellenőrzi az insert/update utasítások végrehajtásakor.
\item[default]
    A mező default értéke. Az értéket az adatbázisszerver 
    (nem az interfész!)
    adja a mezőnek az insert utasítások  végrehajtásakor.
\item[offset]
    A mező kezdete a rowentity adatbufferében.
\item[size] 
    A mező mérete a rowentity adatbufferében.
\item[block] 
    Kódblock, ami kiolvassa/beírja az adatot a bufferből/bufferbe.
\item[sqltype]
    Meghatározza a mező SQL típusát.
    Ha nemlétező táblát kreálunk a tableentityvel, 
    akkor (és csak akkor) kap szerepet az sqltype metódus, 
    ami SQL típust rendel az oszlop Clipper típusához.

    A típusleképezés szabályai Oracle esetében:

\begin{verbatim}
Cw     --> char(w)         ha w<=8
Cw     --> varchar(w)      ha w>8
M      --> blob
Nw     --> number(w)
Nw,d   --> number(w,d)  
Nw.d   --> number(w,d)
L      --> number(1)       a boolean-t nem ismeri
D      --> date
\end{verbatim}

    A típusleképezés szabályai Postgres esetében:

\begin{verbatim}
Cw     --> char(w)         ha w<=8
Cw     --> varchar(w)      ha w>8
M      --> bytea
Nw     --> numeric(w)
Nw,d   --> numeric(w,d)  
Nw.d   --> numeric(w,d)
L      --> boolean
D      --> date
\end{verbatim}

    Ez egyúttal behatárolja, hogy a tableentity interfésszel
    milyen struktúrájú adattáblák hozhatók létre.

\item[indvar]
    Indikátor változó.
\item[isnull(row)]
    A \verb!column:isnull(row)! metódus mutatja, 
    ha a beolvasott mező eredeti értéke null.
    A rowentityk oszlopmetódusai a null értéket nem NIL-re
    képezik, hanem egy egyező típusú empty értéket adnak.
\item[setnull(row,flag)]
    A \verb!column:setnull(row,flag)! metódus beállítja
    a rowentity objektum egy mezőjénék null flagjét.
    Visszatérés: NIL.
\item[isdirty(row)]
    A \verb!column:isdirty(row)! metódus mutatja
    hogy a program módosította-e a mezőt.
\item[setdirty(row,flag)]
    A \verb!column:setdirty(row,flag)! metódus beállítja 
    a rowentity objektum egy mezőjénék dirty flagjét.
    Visszatérés: NIL.
\item[eval]
    A \verb!col:eval(rowentity[,x])! metódushívás kiolvassa az oszlop
    értékét a rowentityből, vagy x-et értékül adja az oszlopnak. 
    Alkalmazási programban inkább  ezzel a formával találkozunk: 
    \verb!rowentity:colname[:=x]!.
\item[label]
    Szöveges attribútum, amit táblafüggetlen megjelenítőprogramok
    használhatnak, pl. oszlopfejléc kiírására.
\item[tooltip]
    Szöveges attribútum, amit táblafüggetlen megjelenítőprogramok
    használhatnak.
\item[picture]
    Szöveges attribútum, amit táblafüggetlen megjelenítőprogramok
    használhatnak.
\end{description}


\subsection{Indexdef osztály}

Index adatok tárolására használjuk az indexdef objektumokat.

\begin{description}
\item[name]     
    Az Oracle-beli név kiegészítője.
    A \verb!tableentity:create! metódus a tábla kreálásakor a tableentityben
    tárolt indexeket is létrehozza. Az indexek adatbázisbeli neve 
    \verb!tabname+'_'+i:name! lesz, ahol tabname a 
    \verb!tableentity:tablist! első eleméből van.
\item[seglist]  
    Az indexet alkotó oszlopok columndef-jeinek listája.
\item[unique]   
    Logikai érték, mutatja, hogy egyedi-e  az index.
\end{description}

