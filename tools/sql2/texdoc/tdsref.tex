\subsection{Tds referencia}

%Előzetes megjegyzés: 
%\begin{quote}\small
%Az eredeti koncepció az volt, 
%hogy a tableentityket XML leírásból készítjük, és ez most is működik.
%A tds-ekkel teljesen azonos tartalmú, de XML szintaktikájú fájlokból
%(ted kiterjesztéssel) a ted2prg utility ugyanazt a kódot generálja,
%mint a tds2prg. Az XML azonban inkább gépi feldolgozásnál előnyös, 
%míg a humán felhasználó könnyebben elboldogul a tds szintaktikával.
%Ezért a tds2prg program (a ted2prg-hez képest) kiegészült egy előzetes 
%elemzéssel, minek során ugyanazt a DOM-ot építi fel, amit a ted2prg XML-ből. 
%Mindkét esetben a közös DOM-ból ugyanaz a modul generálja a kódot.
%\end{quote}

A Table Definition Script (tds) bekezdésekből áll.
Minden bekezdés egy sor elején álló kulcsszóval kezdődik,
amit szóköz nélkül  kettőspont követ.
Az érvényes kulcsszavak:
\verb!name!, 
\verb!version!, 
\verb!table!, 
\verb!join!, 
\verb!column!, 
\verb!colref!, 
\verb!primkey!, 
\verb!index!, 
\verb!select!, 
\verb!include!, 
\verb!type!, 
\verb!comment!.
A bekezdések a következő bekezdésig, vagy a filé végéig tartanak.
A bekezdések sorrendje nem kötött. Az elemző a comment bekezdéseket 
kihagyja.

\subsubsection*{name:}

    Névtér, amibe a tableentityNew() objektumgyártó függvény kerül.

\begin{verbatim}
    name:  multi.level.namespace
\end{verbatim}

A tds-ből generált tableentityt így tudjuk legyártani:
\begin{verbatim}
    tableentity:=multi.level.namespace.tableentityNew(con)
\end{verbatim}
    
\subsubsection*{version:}
    
    A bekezdés tartalma egy tetszőleges tartalmú string (idézőjelekkel),
    amiből a tableentity objektum version attribútumának értéke lesz.

\begin{verbatim}
    version:  "arbitrary text"
\end{verbatim}
    
\subsubsection*{table:}
    
    Legalább egy table bekezdésnek kell lennie, ezek  alakja:
\begin{verbatim}
    table:  real.qualified.tablename=aliasname
\end{verbatim}
    A script más részeiben a táblára kizárólag az alias névvel
    hivatkozunk. Ha nincs megadva join bekezdés, akkor a table
    bekezdésekben magadott táblák Descartes-féle szorzata lesz 
    a tableentity alaptáblája.
    Ha csak egy table bekezdés van, akkor triviálisan az abban
    megadott tábla lesz az alaptábla.
    
\subsubsection*{join:}
    
    Opcionális bekezdés. Ha nincs megadva, akkor a táblák
    Descartes-féle szorzata lesz  az alaptábla. Ha meg van adva, 
    akkor a join bekezdés tartalmából  készül a tableentity által 
    generált SQL select utasítások  from záradéka (tehát lényegében 
    egy from záradékot írunk a joinba).  A from záradék SQL-beli 
    szintaktikájától annyiban térünk el, hogy a táblákra és oszlopokra 
    kizárólag aliasokkal hivatkozunk. Példa:
\begin{verbatim}
    join: 
        a
        full join b on id_a=id_b
        left join c on name_a=name_c
\end{verbatim}

\subsubsection*{column:}

A legegyszerűbb példa:

\begin{verbatim}
    column: szamlaszam  C24
\end{verbatim}

    A kettőspont utáni {\em első\/} szóközökkel határolt szó egyszerre
    \begin{itemize}
    \item metódusnév a rowentity objektumokban,
    \item alias név, amivel a tds más részeiben hivatkozunk az oszlopra. 
    \end{itemize}
    
    A {\em második\/} szóközökkel határolt egység a típusleképezés.
    Ebben azt a típust kell megadni, amiben a CCC program 
    kéri/adja az oszlop adatait (függetlenül attól, hogy
    mi az oszlop tényleges SQL adattípusa). Az érvényes típusok:

    \begin{centering}    
    \begin{tabular}{l@{:~~}l}
    %\begin{tabular}{|l|l|}
        Cw   &  w hoszúságú karakter string \\
        Nw   &  w helyiértéken tárolt egész \\
        Nw.d &  w helyiértéken tárolt, d tizedesjegyet tartalmazó szám \\
        Nw,d &  w helyiértéken tárolt, d tizedesjegyet tartalmazó szám \\
        D    &  dátum \\
        L    &  logikai érték \\
        M    &  memó \\
    \end{tabular}
    \end{centering}    

    A karakter stringeket C (unicode) típusban, a memókat X (bináris)
    típusban kapja a program.

    Az oszlopnevet és típust kiegészítő adatok követhetik.    
    A kiegészítő adatok opcionálisak, és a sorrendjük nem kötött.
    

\paragraph{\tt x=column\_expression}
    
Ezzel a formával adható meg az oszlopkifejezés. 
Ha nincs megadva, akkor maga a név lesz a kifejezés:
\begin{verbatim}
    column: szamlaszam C24      ;ez a default: x=szamlaszam
\end{verbatim}

Amikor az oszlopnév nem egyértelmű, megadjuk a minősített nevét:
\begin{verbatim}
    column: szamlaszam C24      x=ugyfszl.szamlaszam
\end{verbatim}

Megadhatunk kifejezést:
\begin{verbatim}
    column: EGYENLEG   N17.2    (x=napinyito+napitartoz+napikovet)
                                (l=Aktuális egyenleg)
\end{verbatim}

Használhatunk SQL kifejezéseket:
\begin{verbatim}
    column: devnem     C3       x=lower(devnem)
\end{verbatim}

\paragraph{\tt l=label}

    Beállítja az oszlop label attribútumát.
    A labelt táblafüggetlen megjelenítőprogramok kiírhatják 
    oszlopfejlécbe, vagy az adatot megjelenítő text mező elé.

\paragraph{\tt t=tooltip}

    Beállítja az oszlop tooltip attribútumát
    (táblafüggetlen megjelenítőprogramok számára).

\paragraph{\tt p=picture}

    Beállítja az oszlop picture attribútumát
    (táblafüggetlen megjelenítőprogramok számára).

\paragraph{\tt d=default}

    Beállítja az oszlop default érték attribútumát.
    A defaultot nem az interfész, hanem az adatbázisszerver használja,
    ezért csak akkor van hatása, ha a tableentityvel kreáljuk
    a táblát (tehát ugyanaz a helyzet, mint a not null-nál).

\paragraph{\tt nn}

    Jelentése not null. Az attribútum az adatbázisszervernek szól.
    Az interfész nem tudja kikényszeríteni a not null-t, 
    hiszen az adatbázist mások is használhatják, nem feltétlenül a mi 
    SQL2 interfészünkön keresztül.
    Ha viszont a táblát a tableenity create metódusával hozzuk létre, 
    akkor az oszlop not null minősítéssel kreálódik, 
    és a szerver biztosítja a not null feltétel teljesülését.
 

Az oszlopdefiníció szóközökkel határolt egységekből áll.
Ha valamelyik paraméterérték szóközt tartalmaz, akkor
azt zárójelpárral lehet védeni. 
 
\begin{verbatim}
    column: DEVNEM  C3      l=Dev
                            (t=A számla devizaneme)
                            (x=upper( devnem ))
                            (p=@! AAA)
                            nn
\end{verbatim}

A védelemre alkalmazott zárójelpáron belül lehetnek kiegyensúlyozott
zárójelek. Ha nem kiegyensúlyozott zárójelet tartalmazó szöveget
akarunk védeni, használjunk másik zárójelfajtát: \verb![]! vagy \verb!{}!.



\subsubsection*{colref:}
    Lehetnek olyan oszlopok, amikre hivatkozni 
    akarunk a tds-ben, de nem akarjuk lekérdezni az értéküket.  
    Az ilyenek leírását helyezzük colref bekezdésbe, ami ugyanolyan, 
    mint a column, kivéve, hogy nincs benne típus információ,
    és az opciók közül csak az x (column expression) megengedett.

    
\subsubsection*{primkey:}

    A primkey bekezdés kötelező. Benne vesszőkkel elválasztott
    listában felsoroljuk azokat az oszlopokat (column bekezdésben
    definiált alias neveket), amik egyértelműen azonosítják a 
    rekordokat. Az alkalmazás felelőssége, hogy az azonosítás,
    és ezáltal az adatbázis rekord<->rowentity objektum megfeleltetés
    egyértelmű legyen.
\begin{verbatim}    
    primkey:  column_alias1,column_alias2,...
\end{verbatim}    
   
    
\subsubsection*{index:}

    Az opcionális index bekezdések formája:
\begin{verbatim}    
    index:  index_name(column_alias1,column_alias2,...) [unique]
\end{verbatim}    
    Az  index bekezdésekben indexeket definiálhatunk
    a táblára, ezzel bizonyos lekérdezéseket gyorsíthatunk,
    illetve a unique indexekkel kikényszeríthetjük rekordok
    egyediségét.

\subsubsection*{select:}
      
    Az opcionális select bekezdések első szóközökkel határolt 
    szavából a tableentity metódusneve lesz. 

    A metódusnevet követheti a \verb!where! kulcsszó, 
    majd tetszőleges SQL kifejezés, amit az SQL a where és
    order by kulcsszavak között elfogad. Az oszlopokra
    kizárólag az oszlop aliasokkal szabad hivatkozni.
    A kifejezés \verb!:1!, \verb!:2!,\ldots alakú szimbólumokat
    tartalmazhat, amik helyére az SQL select parancs generálásakor
    paramétereket fogunk helyettesíteni.
    
    A where feltételt követheti az \verb!order! kulcsszó,
    ami után zárójelek között fel kell sorolni azokat az oszlopokat, 
    amik szerint rendezni akarjuk az eredménytáblát. 
    A felsorolásban az oszlopnevek után opcionálisan megjelenhet 
    az asc/desc  kiegészítés, ami előírja, hogy az adott oszlop 
    szerint növekvő vagy csökkenő rendezettséget akarunk.
    
    A where és order záradék közül legalább az egyiknek léteznie kell.
    
    Példa:

\begin{verbatim}    
    select: select_a
        where col_a_alias like :1
        order(col_a_alias, col_b_alias desc)
\end{verbatim}

    Ezután a \verb!tableentity:select_a("a%")!
    metódushívás kiválasztja az alaptábla azon sorait,
    melyekben \verb!col_a! első karaktere {"}a", és a sorokat
    rendezi \verb!col_a!, majd csökkenő sorrendben \verb!col_b!
    szerint.

    
    
    
